\documentclass{report}
%Setting margins
%\usepackage[margin = 1.75in]{geometry}
%Basic Maths
\usepackage{amsmath}
\usepackage{amssymb}
\usepackage{mathtools}
\usepackage{gensymb}
%For definining theorem-like environments
\usepackage{amsthm}
%For beautiful letters (e.g. for a partition, see $\mathscr{P}$)
\usepackage{mathrsfs}
%For importing the solution file
%\usepackage{import}
%For drawing commutative diagrams
\usepackage{quiver}
%For pretty colours
\usepackage{xcolor}
%For scaling some relations, for instance see https://tex.stackexchange.com/a/108482
\usepackage{mleftright}
%Set paragraph spacing. I believe this is close to what is used in the book.
%\usepackage[skip=.3\baselineskip, indent = 15pt]{parskip}
%To customize lists
\usepackage{enumitem}
%To strikethrough terms in equations
\usepackage{cancel}
%For bibliography
\usepackage[backend=biber]{biblatex}
%For pictures
\usepackage{tikz}
\usetikzlibrary{calc,positioning}

\usepackage[hidelinks]{hyperref}
\usepackage{soul}
\usepackage[normalem]{ulem}
\usepackage{lipsum}
\usepackage{breqn}

%\addbibresource{main.bib}


\newcommand{\myhy}[2]{\href{#1}{\color{blue}\setulcolor{blue}\ul{#2}}}

%Fix section numbering to match the book's convention
\renewcommand\thesection{\arabic{section}}

%Displays "Exercises". To put after each section.
\newcommand{\extitle}{\subsection*{Exercises}}

%For personal notes
\newcommand{\note}[1]
{\smallskip {\noindent\textbf{Note} #1}}

%Roman numerals!
\newcommand{\RNo}[1]{%
	\textup{\uppercase\expandafter{\romannumeral#1}}%
}

%San-serif for names of categories
\newcommand{\serif}[1]{{\fontfamily{cmss}\selectfont #1}}
\newcommand{\srf}{\textsf}

%Shorthands for common sets
\newcommand{\N}{\mathbb{N}}
\newcommand{\Z}{\mathbb{Z}}
\newcommand{\Q}{\mathbb{Q}}
\newcommand{\R}{\mathbb{R}}
\newcommand{\C}{\mathbb{C}}
\newcommand{\zmod}[1]{\bZ/#1\bZ}

%Miscellaneous commands
\newcommand{\defeq}{\coloneqq}
\newcommand{\divides}{\mid}
\newcommand{\legendre}[2]{\ensuremath{\left( \frac{#1}{#2} \right) }}
\newcommand{\Mod}[1]{\ (\mathrm{mod}\ #1)}
\newcommand{\mbold}[1]{\mathrm{\mathbf{#1}}}


%Useful operations and delimiters
\DeclareMathOperator{\Hom}{Hom}
\DeclareMathOperator{\End}{End}
\DeclareMathOperator{\Aut}{Aut}
\DeclareMathOperator{\Obj}{Obj}
\DeclareMathOperator{\id}{id}
\DeclareMathOperator{\lcm}{lcm}
\DeclareMathOperator{\GL}{GL}
\DeclareMathOperator{\SO}{SO}
\DeclareMathOperator{\SL}{SL}
\DeclareMathOperator{\U}{U}
\DeclareMathOperator{\SU}{SU}
\DeclareMathOperator{\Inn}{Inn}
\DeclareMathOperator{\PSL}{PSL}
\DeclareMathOperator{\im}{im}
\DeclareMathOperator{\coker}{coker}
\DeclareMathOperator{\rot}{rot}
\DeclareMathOperator{\rf}{ref}
\DeclareMathOperator{\Symm}{Symm}
\DeclareMathOperator{\vspan}{span}
\DeclareMathOperator{\ev}{ev}
\DeclareMathOperator{\Gal}{Gal}
\DeclareMathOperator{\ob}{ob}
\DeclareMathOperator{\mor}{mor}
\DeclareMathOperator{\dom}{dom}
\DeclareMathOperator{\cod}{cod}
\DeclarePairedDelimiter\abs{\lvert}{\rvert}%
\DeclarePairedDelimiter\norm{\lVert}{\rVert}%
\DeclarePairedDelimiter\innprod{\langle}{\rangle}%
\DeclarePairedDelimiter\ceil{\lceil}{\rceil}
\DeclarePairedDelimiter\floor{\lfloor}{\rfloor}
%Claim environment
\newtheorem{claim}{Claim}


%Exercise environment
\theoremstyle{definition}
\newtheorem{ex}{Exercise}

%Standard theorem-like environment
\theoremstyle{plain}
\newtheorem{thm}{Theorem}[section]

\newtheorem{prop}[thm]{Proposition}
\newtheorem{lem}[thm]{Lemma}
\newtheorem{coro}[thm]{Corollary}
\newtheorem{prob}{Problem}
\newtheorem{conj}{Conjecture}


\theoremstyle{definition}
\newtheorem{defn}[thm]{Definition}
\newtheorem{rem}[thm]{Remark}
\newtheorem{eg}[thm]{Example}
\newtheorem{egs}[thm]{Examples}
\newtheorem{fact}[thm]{Fact}
\newtheorem{task}{Task}



%Solution environment
\newenvironment{solution}
{\begin{proof}[Solution]}
	{\end{proof}}

%Function restrictions
% From https://tex.stackexchange.com/a/22255
\newcommand\restr[2]{{% we make the whole thing an ordinary symbol
		\left.\kern-\nulldelimiterspace % automatically resize the bar with \right
		#1 % the function
		\vphantom{\big|} % pretend it's a little taller at normal size
		\right|_{#2} % this is the delimiter
}}

%\newcommand\nvdash{\mkern-2mu\not\mkern2mu\vdash}

\makeatother
\begin{document}
	\title{Category Theory}
	\author{Hernán Ibarra Mejia}
	\maketitle
	\tableofcontents
	\newpage
	This is a set of lecture notes taken by me from the Part III course ``Category Theory'', lectured by Professor Peter Johnstone in Michaelmas, 2023. I take full responsibility for any mistakes in these notes. Since Professor Johnstone provides his own notes, I will be writing `skeletal lecture notes', i.e. only definitions, theorems, and proofs with little by way of motivation and almost no examples. I might expand on the proofs and statements given without rigorous proofs more than Professor Johnstone does, but that's my only contribution in terms of content.
	\section{Definitions \sout{and Examples}}
	\begin{defn}[Categories]
		A \emph{category} $\mathcal{C}$ consists of the following.
		\begin{enumerate}
			\item A collection $\ob \mathcal{C}$ of objects $A,B,C,\ldots$
			\item A collection $\mor \mathcal{C}$ of morphisms $f,g,h,\ldots$
			\item Two operations $\dom,\cod\colon \mor \mathcal{C} \to \ob\mathcal{C}$. We write $f\colon A \to B$ to mean $\dom f =A$ and $\cod f = B$.
			\item An operation $\id \colon \ob \mathcal{C} \to \mor \mathcal{C}$ so that $\dom (\id A) = \cod (\id A) = A$. We write $1_A$ for $\id A$.
			\item A partial binary operation on $\mor \mathcal{C}$ called composition. We usually denote it by concatenation i.e., $(f,g)\mapsto fg$. 
		\end{enumerate}
		Categories are also required to satisfy the following axioms.
		\begin{enumerate}
			\item The composition $fg$ is defined if and only if $\dom f = \cod g$.
			\item We have that $\dom fg = \dom g$ and $\cod fg = \cod f$.
			\item If the composites are defined, we have $f1_A = f$ and $1_Ag = g$. 
			\item Whenever $fg$ and $gh$ are defined we have $f(gh) = (fg)h$.
		\end{enumerate}
	\end{defn}
	For any category $\mathcal{C}$ there is the \emph{oppposite category} $\mathcal{C}^{\text{op}}$ with the same objects as $\mathcal{C}$ and morphisms $f\colon A\to B$ in $\mathcal{C}^{\text{op}}$ are just morphisms $f\colon B \to A$ in $\mathcal{C}$. It is easily verified that this gives a category. 
	\begin{defn}[Subcategories]
		Let $\mathcal{C}$ be a category. We say $\mathcal{D}$ is a \emph{subcategory} of $\mathcal{C}$ if $\mathcal{D}$ is a category with $\ob \mathcal{D} \subseteq \ob \mathcal{C}$ and $\mor \mathcal{D}\subseteq \mor \mathcal{D}$ so that the operations of domain, codomain, identity, and composition in $\mathcal{D}$ are just the restriction of those of $\mathcal{C}$.
	\end{defn}
	\begin{defn}[Isomorphisms]
		Let $\mathcal{C}$ be a category. An \emph{isomorphism} is a morphism $f\colon A \to B$ in $\mathcal{C}$ so that there exists some other morphism $g\colon B \to A$ with the property that 
		\[
			gf = 1_A \,\,\,\text{ and }\,\,\, fg= 1_B.
		\]
		In this case we call $g$ the \emph{inverse} of $f$. If there is an isomorphism $A\to B$ we say that $A$ and $B$ are \emph{isomorphic} and write $A\cong B$.
	\end{defn}
	\begin{prop}
		Inverses of isomorphisms are unique.
	\end{prop}
	\begin{proof}
		Suppose $f\colon A \to B$ is an isomorphism in some category, and let $g,g'\colon B \to A$ be inverses of $f$. Then,
		\[
			g = g1_B = g(fg') = (gf)g' = 1_Ag' = g'.\qedhere
		\]
	\end{proof}
	Note that in the above proof we have used all axioms for categories. Given that inverses are unique, we will denote the unique inverse of an isomorphism $f$ by $f^{-1}$.
	\begin{prop}
		In all categories $\cong$ is an equivalence relation.
	\end{prop}
	\begin{proof}
		For the rest of the proof, fix a category $\mathcal{C}$ and let $A,B,C\in \ob \mathcal{C}$ be arbitrary. Note that $A\cong A$ since the identity morphism $1_A \colon A \to A$ is an isomorphism (this is because $1_A1_A = 1_A$).
		
		Suppose $A\cong B$ and let $f\colon A \to B$ be an isomorphism. Then we have
		\[
			f^{-1}f = 1_A \,\,\,\text{ and }\,\,\, ff^{-1}= 1_B,
		\]
		which shows that $f^{-1}\colon B \to A$ is an isomorphism and hence $B\cong A$.
		
		Finally, suppose that $A\cong B$ and $B\cong C$. Let $f\colon A \to B$ and $g\colon B \to C$ be isomorphisms. We claim that $gf$ is an isomorphism. Indeed, 
		\[
			(gf)(f^{-1}g^{-1}) = g((ff^{-1})g^{-1}) = g(1_Bg^{-1}) = gg^{-1} = 1_C,
		\]
		and similarly $(f^{-1}g^{-1})(gf) = 1_A$, showing that $f^{-1}g^{-1}$ is an inverse for $gf$. Hence $A\cong C$.
	\end{proof}
	\begin{defn}[Functors]
		Let $\mathcal{C}$ and $\mathcal{D}$ be categories. A functor $F\colon \mathcal{C} \to \mathcal{D}$ consists of mappings $F\colon \ob \mathcal{C} \to \ob \mathcal{D}$ and $F\colon \mor \mathcal{C} \to \mor \mathcal{D}$ satisfying  the following.
		\begin{enumerate}
			\item Compatibility with domains and codomains, i.e., $F(\dom f) = \dom Ff$ and $F(\cod f) = \cod Ff$.
			\item Compatibility with identities, i.e., $F(1_A) = 1_{FA}$.
			\item Compatibility with composition, i.e., $F(fg) = (Ff)(Fg)$.
		\end{enumerate}
	\end{defn}
	When there is a functor $\mathcal{C}^{\text{op}} \to \mathcal{D}$ or $\mathcal{C} \to \mathcal{D}^{\text{op}}$ (these two are equivalent) we sometimes call it a \emph{contravariant} functor $\mathcal{C} \to \mathcal{D}$. We use the term \emph{covariant} to mean functors in the original sense.
	
	Note that functors can be composed. That is, if we have categories $\mathcal{C},\mathcal{D}$ and $\mathcal{E}$, together with functors $F\colon\mathcal{C}\to\mathcal{D}$ and $G\colon \mathcal{D}\to\mathcal{E}$ we can define a functor $GF\colon \mathcal{C}\to\mathcal{E}$. This is done in the obvious way: if $A,B\in\ob \mathcal{C}$ and $f\colon A \to B$ is a morphism in $\mathcal{C}$ we let 
	\[
		GF(A) \coloneqq G(F(A)) \,\,\, \text{ and }\,\,\, GF(f) \coloneqq G(F(f)).
	\]
	That this is a functor is verified immediately. Note that composition is associative as well. Therefore, if we wanted to, we can define a category $\mbold{Cat}$ whose objects are all (small) categories and its morphisms are functors. Proving that this is a category is easy though tedious.
	\begin{defn}[Natural transformations]
		Let $\mathcal{C}$ and $\mathcal{D}$ be categories and $F,G\colon \mathcal{C} \to \mathcal{D}$ be functors. A \emph{natural transformation} $\alpha \colon F \to G$ is a collection of morphisms $\alpha_A \colon FA \to FB$ of $\mor \mathcal{D}$ for all $A\in \ob \mathcal{C}$ so that if $f \colon A \to B$ is a morphism of $\mathcal{C}$ the following square (called the \emph{naturality square}) commutes.
		% https://q.uiver.app/#q=WzAsNCxbMCwwLCJGQSJdLFsxLDAsIkZCIl0sWzAsMSwiR0EiXSxbMSwxLCJHQiJdLFswLDEsIkZmIl0sWzIsMywiR2ciLDJdLFswLDIsIlxcYWxwaGFfQSIsMl0sWzEsMywiXFxhbHBoYV9CIl1d
		\[\begin{tikzcd}
			FA & FB \\
			GA & GB
			\arrow["Ff", from=1-1, to=1-2]
			\arrow["Gg"', from=2-1, to=2-2]
			\arrow["{\alpha_A}"', from=1-1, to=2-1]
			\arrow["{\alpha_B}", from=1-2, to=2-2]
		\end{tikzcd}\]
	\end{defn}
	Fix two categories $\mathcal{C}$ and $\mathcal{D}$. We can build a category $[\mathcal{C}, \mathcal{D}]$ whose objects are all functors $\mathcal{C}\to \mathcal{D}$ and the morphisms are natural transformations with $\dom$ and $\cod$ defined in the obvious way. 
	
	If $F\colon \mathcal{C} \to \mathcal{D}$ we define $1_F \colon F \to F$ to be the natural transformation consisting of morphisms $(1_F)_A \colon FA \to FA$ for each $A\in \ob \mathcal{C}$ by specifying $(1_F)_A \coloneqq 1_{FA}$. Then $1_F$ is a natural transformation since if $f\colon A \to B$ is a morphism in $\mathcal{C}$ the naturality square
	% https://q.uiver.app/#q=WzAsNCxbMCwwLCJGQSJdLFsxLDAsIkZCIl0sWzAsMSwiRkEiXSxbMSwxLCJGQiJdLFswLDEsIkZmIl0sWzIsMywiRmYiXSxbMCwyLCIoMV9GKV9BIiwyXSxbMSwzLCIoMV9GKV9CIl1d
	\[\begin{tikzcd}
		FA & FB \\
		FA & FB
		\arrow["Ff", from=1-1, to=1-2]
		\arrow["Ff", from=2-1, to=2-2]
		\arrow["{(1_F)_A}"', from=1-1, to=2-1]
		\arrow["{(1_F)_B}", from=1-2, to=2-2]
	\end{tikzcd}\]
	obviously commutes by the identity axiom of categories. 
	
	Note that if we have natural transformations $\alpha\colon F \to G$ and $\beta \colon G \to H$ for functors $F,G,H \colon \mathcal{C} \to \mathcal{D}$ we can define the composite natural transformation $\beta\alpha \colon F \to H$ by stating that for all $A \in \mathcal{C}$ we have $(\beta\alpha)_A = \beta_A\alpha_A$. This is a natural transformation since if $f\colon A \to B$ for $A,B\in\ob \mathcal{C}$ we have that in the following diagram
	% https://q.uiver.app/#q=WzAsNixbMCwwLCJGQSJdLFsxLDAsIkZCIl0sWzAsMSwiR0EiXSxbMSwxLCJHQiJdLFswLDIsIkhBIl0sWzEsMiwiSEIiXSxbMCwxLCJGZiJdLFsyLDMsIkdmIl0sWzAsMiwiXFxhbHBoYV9BIiwyXSxbMSwzLCJcXGFscGhhX0IiXSxbMiw0LCJcXGJldGFfQSIsMl0sWzMsNSwiXFxiZXRhX0IiXSxbNCw1LCJIZiIsMl1d
	\[\begin{tikzcd}
		FA & FB \\
		GA & GB \\
		HA & HB
		\arrow["Ff", from=1-1, to=1-2]
		\arrow["Gf", from=2-1, to=2-2]
		\arrow["{\alpha_A}"', from=1-1, to=2-1]
		\arrow["{\alpha_B}", from=1-2, to=2-2]
		\arrow["{\beta_A}"', from=2-1, to=3-1]
		\arrow["{\beta_B}", from=2-2, to=3-2]
		\arrow["Hf"', from=3-1, to=3-2]
	\end{tikzcd}\]
	both squares commute since both $\alpha$ and $\beta$ are natural transformations, so the outer rectangle, which is the naturality square for $(\beta\alpha)_A$, commutes. We only need to check associativity and the identity axiom (the other axioms are obvious from the construction).
	
	Let $F,G,H,I\colon \mathcal{C}\to\mathcal{C}$ be functors and let $\alpha\colon F\to G$, $\beta\colon G\to H$, and $\gamma\colon H \to I$ be natural transformations. Then for all $A\in\ob \mathcal{C}$ we have $((\alpha\beta)\gamma)_A = (\alpha_A\beta_A)\gamma_A$ and $(\alpha(\beta\gamma))_A = \alpha_A(\beta_A\gamma_A)$ by definitions; and these are equal by associativity in $\mor \mathcal{D}$. Hence, $(\alpha\beta)\gamma = \alpha(\beta\gamma)$ so associativity holds.
	
	Finally, let $F,G,H\colon\mathcal{C}\to \mathcal{D}$ be functors and let $\alpha \colon F \to G$ and $\beta\colon G \to H$ be natural transformations. Consider $1_G\alpha$. For all $A\in\ob \mathcal{C}$ we have 
	\[
		(1_G\alpha)_A = (1_G)_A\alpha_A = 1_{GA}\alpha_A = \alpha_A,
	\]
	where the last equality is the axiom of identity in $\mathcal{D}$. Therefore $1_G\alpha = \alpha$. Similarly, for $\beta1_G$ we have for all $A\in\ob \mathcal{C}$,
	\[
		(\beta1_G)_A = \beta_A(1_G)_A = \beta_A1_{GA} = \beta_A,
	\]		
	and so $\beta1_G = \beta$. This verifies the last axiom so $[C,D]$ is indeed a category. As you can see, verifying whether something that is obviously a category \emph{is} a category is quite a lot of very boring and repetitive work so I promise not to do it ever ever ever ever ever ever ever ever again.
	\begin{lem}\label{lem:nat_iso_pointw}
		Let $\alpha\colon F \to G$ be a natural transformation between functors $F,G\colon \mathcal{C}\to\mathcal{G}$. Then $\alpha$ is an isomorphism in $[\mathcal{C}, \mathcal{D}]$ if and only if $\alpha_A$ is an isomorphism in $\mathcal{D}$ for all $A\in\ob\mathcal{C}$.
	\end{lem}
	\begin{proof}
		Suppose $\alpha$ is an isomorphism in $[\mathcal{C}, \mathcal{D}]$, and let $\alpha^{-1}\colon G \to F$ be its inverse. If $A\in\ob \mathcal{C}$ we have that
		\[
		1_{FA} = (1_F)_A = (\alpha^{-1}\alpha)_A = \alpha^{-1}_A\alpha_A,
		\]
		and similarly $1_{GA} = \alpha_A\alpha^{-1}_A$. Thus $\alpha_A$ is an isomorphism.
		
		Conversely, suppose that $\alpha_A$ is an isomorphism for all $A\in\ob\mathcal{C}$. Then define $\alpha^{-1}\colon \mathcal{G} \to \mathcal{F}$ as the collection of morphisms $\alpha^{-1}_A$ for all $A$. We only need to check that this is a natural transformation. This is the case, since, if $g\colon C \to D$ is a morphism in $\mathcal{D}$ we have that the naturality square
		% https://q.uiver.app/#q=WzAsNCxbMCwwLCJHQyJdLFsxLDAsIkdEIl0sWzAsMSwiRkMiXSxbMSwxLCJGRCJdLFswLDEsIkdnIl0sWzIsMywiRmciLDJdLFswLDIsIlxcYWxwaGFeey0xfV9DIiwyXSxbMSwzLCJcXGFscGhhXnstMX1fRCJdXQ==
		\[\begin{tikzcd}
			GC & GD \\
			FC & FD
			\arrow["Gg", from=1-1, to=1-2]
			\arrow["Fg"', from=2-1, to=2-2]
			\arrow["{\alpha^{-1}_C}"', from=1-1, to=2-1]
			\arrow["{\alpha^{-1}_D}", from=1-2, to=2-2]
		\end{tikzcd}\]
		commutes. To see this, just calculate:
		\[
		(Fg)(\alpha_C^{-1}) = (\alpha^{-1}_D\alpha_D)(Fg)(\alpha_C^{-1}) \overset{!}{=} \alpha^{-1}_C(Gg)(\alpha_C)\alpha_{C}^{-1} = \alpha_C^{-1}Gg.
		\]
		Note that in $\overset{!}{=}$ we used the naturality of $\alpha$.
	\end{proof}
	So far we have studied what people call ``vertical composition'' of natural transformations. There is also \emph{horizontal composition} which is sometimes useful.
	\begin{defn}[Horizontal composition]
		Suppose we have functors as in the diagram.
		% https://q.uiver.app/#q=WzAsMyxbMCwwLCJcXG1hdGhjYWx7Q30iXSxbMSwwLCJcXG1hdGhjYWx7RH0iXSxbMiwwLCJcXG1hdGhjYWx7RX0iXSxbMCwxLCJGIiwwLHsib2Zmc2V0IjotMX1dLFsxLDIsIkgiLDAseyJvZmZzZXQiOi0xfV0sWzEsMCwiRyIsMCx7Im9mZnNldCI6LTF9XSxbMiwxLCJLIiwwLHsib2Zmc2V0IjotMX1dXQ==
		\[\begin{tikzcd}
			{\mathcal{C}} & {\mathcal{D}} & {\mathcal{E}}
			\arrow["F", shift left, from=1-1, to=1-2]
			\arrow["F'", shift left, from=1-2, to=1-3]
			\arrow["G"', shift right, from=1-1, to=1-2]
			\arrow["G'"', shift right, from=1-2, to=1-3]
		\end{tikzcd}\]
		Let $\alpha\colon F \to G$ and $\alpha'\colon F' \to G'$ be natural transformations. We define $\alpha' \ast \alpha$ to be a natural transformation $F'F \to G'G$ such that $(\alpha' \ast \alpha)_A \colon F'FA \to G'GA$ is the diagonal of the square
		% https://q.uiver.app/#q=WzAsNCxbMCwwLCJGJ0ZBIl0sWzEsMCwiRydGQSJdLFsxLDEsIkcnR0EiXSxbMCwxLCJGJ0dBIl0sWzAsMSwiXFxhbHBoYSdfe0ZBfSJdLFswLDMsIkYnKFxcYWxwaGFfQSkiLDJdLFszLDIsIlxcYWxwaGEnX3tHQX0iLDJdLFsxLDIsIkcnKFxcYWxwaGFfQSkiXV0=
		\[\begin{tikzcd}
			{F'FA} & {G'FA} \\
			{F'GA} & {G'GA}
			\arrow["{\alpha'_{FA}}", from=1-1, to=1-2]
			\arrow["{F'(\alpha_A)}"', from=1-1, to=2-1]
			\arrow["{\alpha'_{GA}}"', from=2-1, to=2-2]
			\arrow["{G'(\alpha_A)}", from=1-2, to=2-2]
		\end{tikzcd}\]
		which is commutative by naturality of $\alpha'$. We claim this is indeed natural. For a morphism $f\colon A \to B$ we need to check that the following commutes.
		% https://q.uiver.app/#q=WzAsNCxbMCwwLCJGJ0ZBIl0sWzEsMCwiRidGQiJdLFsxLDEsIkcnR0IiXSxbMCwxLCJHJ0dBIl0sWzAsMSwiRidGZiJdLFswLDMsIihcXGFscGhhXFxhc3RcXGFscGhhJylfQSIsMl0sWzMsMiwiRydHZiIsMl0sWzEsMiwiKFxcYWxwaGFcXGFzdFxcYWxwaGEnKV9CIl1d
		\[\begin{tikzcd}
			{F'FA} & {F'FB} \\
			{G'GA} & {G'GB}
			\arrow["{F'Ff}", from=1-1, to=1-2]
			\arrow["{(\alpha'\ast\alpha)_A}"', from=1-1, to=2-1]
			\arrow["{G'Gf}"', from=2-1, to=2-2]
			\arrow["{(\alpha'\ast\alpha)_B}", from=1-2, to=2-2]
		\end{tikzcd}\]
		Expanding out the definition of $\alpha'\ast\alpha$, we get the diagram
		% https://q.uiver.app/#q=WzAsOCxbMSwwLCJGJ0ZBIl0sWzQsMCwiRidGQiJdLFs0LDIsIkcnR0IiXSxbMSwyLCJHJ0dBIl0sWzAsMSwiRidHQSJdLFsyLDEsIkcnRkEiXSxbMywxLCJGJ0dCIl0sWzUsMSwiRydGQiJdLFswLDEsIkYnRmYiXSxbMywyLCJHJ0dmIiwyXSxbMCw0LCJGJ1xcYWxwaGFfQSIsMl0sWzQsMywiXFxhbHBoYSdfe0dBfSIsMl0sWzAsNSwiXFxhbHBoYSdfe0ZBfSJdLFs1LDMsIkcnKFxcYWxwaGFfQSkiXSxbMSw2LCJGJ1xcYWxwaGFfQiIsMl0sWzYsMiwiXFxhbHBoYSdfe0dCfSIsMl0sWzEsNywiXFxhbHBoYSdfe0ZCfSJdLFs3LDIsIkcnKFxcYWxwaGFfQiJdLFs0LDYsIkYnR2YiLDEseyJsYWJlbF9wb3NpdGlvbiI6MzAsImN1cnZlIjoyLCJzdHlsZSI6eyJib2R5Ijp7Im5hbWUiOiJkYXNoZWQifX19XSxbNSw3LCJHJ0ZmIiwxLHsibGFiZWxfcG9zaXRpb24iOjcwLCJjdXJ2ZSI6LTIsInN0eWxlIjp7ImJvZHkiOnsibmFtZSI6ImRhc2hlZCJ9fX1dXQ==
		\[\begin{tikzcd}
			& {F'FA} &&& {F'FB} \\
			{F'GA} && {G'FA} & {F'GB} && {G'FB} \\
			& {G'GA} &&& {G'GB}
			\arrow["{F'Ff}", from=1-2, to=1-5]
			\arrow["{G'Gf}"', from=3-2, to=3-5]
			\arrow["{F'\alpha_A}"', from=1-2, to=2-1]
			\arrow["{\alpha'_{GA}}"', from=2-1, to=3-2]
			\arrow["{\alpha'_{FA}}", from=1-2, to=2-3]
			\arrow["{G'(\alpha_A)}", from=2-3, to=3-2]
			\arrow["{F'\alpha_B}"', from=1-5, to=2-4]
			\arrow["{\alpha'_{GB}}"', from=2-4, to=3-5]
			\arrow["{\alpha'_{FB}}", from=1-5, to=2-6]
			\arrow["{G'(\alpha_B}", from=2-6, to=3-5]
			\arrow["{F'Gf}"{description, pos=0.3}, curve={height=12pt}, dashed, from=2-1, to=2-4]
			\arrow["{G'Ff}"{description, pos=0.7}, curve={height=-12pt}, dashed, from=2-3, to=2-6]
		\end{tikzcd}\]
		where I have added the extra dashed arrows. Then, if you tilt your head a little, this is a parallelepiped. If we rotate towards the page, we see that the top face and the bottom face are the naturality diagrams for $\alpha$ with respect to $f$ but with $F'$ and $G'$ applied to them respectively. The right and left faces are commutative by our previous argument. Finally, the front and the back faces are just naturality squares for $\alpha'$ with respect to $Gf$ and $Ff$ respectively. Hence the whole diagram commutes (!). This finishes the construction.
		
		In the special case when $F=G$ and $\alpha = 1_{F}$ we denote $\alpha' \ast \alpha$ by $\alpha'F$. Similarly we define $F'\alpha$.
	\end{defn}
	\begin{defn}[Equivalence of categories]
		Let $\mathcal{C}$ and $\mathcal{D}$ be categories. By an \emph{equivalence} between $\mathcal{C}$ and $\mathcal{D}$ we mean a pair of functors $F\colon \mathcal{C}\to\mathcal{D}$ and $G\colon \mathcal{D} \to \mathcal{C}$, together with natural isomorphisms $\alpha\colon 1_{\mathcal{C}} \to GF$ and $\beta \colon FG \to 1_{\mathcal{D}}$.
		
		We write $\mathcal{C}\simeq \mathcal{D}$ if there exists an equivalence between $\mathcal{C}$ and $\mathcal{D}$.
	\end{defn}
	For $A,B$ objects of some category $\mathcal{C}$ we define the notation $\mathcal{C}(A,B)$ to mean the collection of morphisms $f\in\mor\mathcal{C}$ such that $\dom f = A$ and $\cod f = B$.
	\begin{defn}
		Let $F\colon\mathcal{C}\to\mathcal{D}$ be a functor.
		\begin{enumerate}[label=(\alph*)]
			\item We say $F$ is \emph{faithful} if, given $f,g\in\mor \mathcal{C}$ the equations $\dom f = \dom g$, $\cod f = \cod g$ and $Ff = Fg$ imply $f = g$. 
			\item We say that $F$ is \emph{full} if, given $g\colon FA\to FB$ there exists $f\colon A \to B$ in $\mathcal{C}$ so that $Ff = g$.
			\item We say that $F$ is \emph{essentially surjective} if every $B\in\ob \mathcal{D}$ is isomorphic to $FA$ for some $A\in\mathcal{C}$.
		\end{enumerate}
		Equivalently, we say $F$ is faithful (respectively full) if for all $A,B\ \ob\mathcal{C}$ the assignment $\mathcal{C(A,B)} \to \mathcal{D}(F(A),F(B))$ given by $f\mapsto Ff$ is one-to-one (resp. onto).
	\end{defn}
	Note that a subcategory $\mathcal{D}$ of $\mathcal{C}$ induces a faithful functor $F\colon \mathcal{D} \to \mathcal{C}$ which is given by inclusion. If $F$ is also full, we say that $\mathcal{D}$ is a \emph{full subcategory} of $\mathcal{C}$.
	\begin{prop}
		Let $\mathcal{C}$ and $\mathcal{D}$ be categories, and suppose $F \colon\mathcal{C}\to\mathcal{D}$ is a functor. If $A,B\in\ob\mathcal{C}$ are isomorphic, then so are $F(A)$ and $F(B)$. Furthermore, if $F$ is full and faithful, then the converse holds. 
	\end{prop}
	\begin{proof}
		Suppose $A,B\in\ob\mathcal{C}$ are isomorphic, and let $f\colon A \to B$ be an isomorphism. Then, by functoriality,
		\[
			(Ff^{-1})(Ff) = F(f^{-1}f) = F1_A = 1_{FA}, 
		\]
		and similarly $(Ff)(F(f^{-1}) = 1_{FB}$. It follows that $Ff^{-1} = (Ff)^{-1}$ and $Ff$ is an isomorphism $FA\to FB$.
		
		Now suppose that $F$ is full and faithful and let us prove the converse. Assume $A,B\in\ob\mathcal{C}$ are such that $FA \cong FB$. Let $g\colon FA \to FB$ be an isomorphism. By fullness, there is a morphism $f\colon A \to B$ so that $Ff = g$. Similarly, there is a morphism $f^{*} \colon B \to A$ so that $Ff^{*} = g^{-1}$. Then, by functoriality,
		\[
			F(f^{*}f) = (Ff^{*})(Ff) = g^{-1}g = 1_{FA} = F(1_A).
		\] 
		By faithfulness, $f^{*}f = 1_A$. An analogous argument shows that $ff^{*} = 1_B$ and thus $f^{*}= f^{-1}$ and $f$ is an isomorphism.
	\end{proof}
	\begin{lem}
		Let $F\colon \mathcal{C}\to \mathcal{D}$ be a functor. Then $F$ is part of an equivalence $\mathcal{C} \simeq \mathcal{D}$ if and only if $F$ is full, faithful, and essentially surjective
	\end{lem}
	\begin{proof}
		Suppose $F$ is part of an equivalence $\mathcal{C}\simeq \mathcal{D}$ and let $G\colon \mathcal{D}\to\mathcal{C}$ be a functor so that $\alpha\colon 1_{\mathcal{C}}\to GF$ and $\beta \colon FG \to 1_{\mathcal{D}}$ are natural isomorphisms. That $F$ is essentially surjective is immediate: if $B\in\mathcal{D}$ we see that $F(G(B))$ is isomorphic to $B$ via the isomorphism $\beta_B$..
		
		Let $f,g\in\mor \mathcal{C}$ with $\dom f = \dom g$, $\cod f = \cod g$ and $Ff = Fg$. The naturality square for $g$ gives
		% https://q.uiver.app/#q=WzAsNCxbMCwwLCJBIl0sWzEsMCwiIEIiXSxbMCwxLCJHRkEiXSxbMSwxLCJHRkIiXSxbMCwxLCJnIl0sWzAsMiwiXFxhbHBoYV9BIiwyXSxbMSwzLCJcXGFscGhhX0IiXSxbMiwzLCJHRmciLDJdXQ==
		\[\begin{tikzcd}
			A & { B} \\
			GFA & GFB
			\arrow["g", from=1-1, to=1-2]
			\arrow["{\alpha_A}"', from=1-1, to=2-1]
			\arrow["{\alpha_B}", from=1-2, to=2-2]
			\arrow["GFg"', from=2-1, to=2-2]
		\end{tikzcd}.\]
		By Lemma \ref{lem:nat_iso_pointw}, $\alpha_A$ and $\alpha_B$ are isomorphisms, so $g = \alpha_B^{-1}(GFg) \alpha_A$. Similarly, from the naturality square for $f$ we get $f = \alpha_B^{-1}(GFf) \alpha_A$ (here we use that $f$ and $g$ have the same domain and codomain). But as $Ff = Fg$ we clearly have $f = g$. 
		
		Now we prove that $F$ is full, so let $g\colon FA \to FB$ be a morphism in $\mathcal{D}$. We want to obtain a map in $\mathcal{C}$ so it seems reasonable to apply $G$ to get a map $Gg\colon GFA\to GFB$. As we want a map $A\to B$ we can use the natural transformations.
		% https://q.uiver.app/#q=WzAsNCxbMCwwLCJHRkEiXSxbMSwwLCJHRkIiXSxbMCwxLCJBIl0sWzEsMSwiQiJdLFswLDEsIkdnIl0sWzIsMCwiXFxhbHBoYV9BIl0sWzMsMSwiXFxhbHBoYV9CIiwyXV0=
		\[\begin{tikzcd}
			GFA & GFB \\
			A & B
			\arrow["Gg", from=1-1, to=1-2]
			\arrow["{\alpha_A}", from=2-1, to=1-1]
			\arrow["{\alpha_B}"', from=2-2, to=1-2]
		\end{tikzcd}\]
		The only thing to do is to define $f\coloneqq \alpha^{-1}_B (Gg)\alpha_A$ and verify $Ff = g$. Consider the naturality square for $f$.
		% https://q.uiver.app/#q=WzAsNCxbMCwwLCJBIl0sWzEsMCwiQiJdLFswLDEsIkdGQSJdLFsxLDEsIkdGQiJdLFswLDEsImYiXSxbMiwzLCJHRmYiLDJdLFswLDIsIlxcYWxwaGFfQSIsMl0sWzEsMywiXFxhbHBoYV9CIl1d
		\[\begin{tikzcd}
			A & B \\
			GFA & GFB
			\arrow["f", from=1-1, to=1-2]
			\arrow["GFf"', from=2-1, to=2-2]
			\arrow["{\alpha_A}"', from=1-1, to=2-1]
			\arrow["{\alpha_B}", from=1-2, to=2-2]
		\end{tikzcd}\]
		Plugging in the definition of $f$ into this square yields the equality $Gg = GFf$. But we already proved that functors that are part of equivalences are faithful; in particular $G$ is faithful so (since $g$ and $Ff$ obviously have matching domain and codomain) we have $g = Ff$ as desired. We have finished with the first implication of lemma.
		
		Conversely, assume $F$ is a full, faithful, and essentially surjective. For all $B\in \ob\mathcal{D}$ pick some $GB \in\ob \mathcal{C}$ such that $FGB$ is isomorphic to $B$; this is possible by essential surjectivity. Furthermore, pick some isomorphism $\beta_B\colon FGB\to B$. For each morphism $g\colon B \to B'$ in $\mathcal{D}$ note that $g^{*}=\beta_{B'}^{-1}(g)\beta_B \colon FGB \to FGB'$. So, let $Gg \colon GB \to GB'$ be a morphism so that $FGg = g^{*}$; this is possible since $F$ is full. We claim that this defines a functor $G\colon \mathcal{D} \to \mathcal{C}$. It is clearly an assignment $\ob \mathcal{D} \to \mathcal{C}$ and $\mor \mathcal{D} \to \mathcal{C}$ compatible with domains and codomains.
		
		Let $B\in\ob \mathcal{D}$. Note that 
		\[
			FG(1_B) = (1_B)^{*} = \beta_B^{-1} 1_B \beta_B = 1_{FGB} = F(1_{GB}).
		\]
		Thus, by faithfulness of $F$, we have that $G(1_B) = 1_{GB}$.
		
		Now let $g\colon B \to B'$ and $h\colon B' \to B''$ be morphisms in $\mathcal{D}$. We perform a similar calculation.
		\begin{align*}
			F(G(h)G(g)) = (FGh)(FGg) = h^{*}g^{*} &= \beta_{B''}^{-1}(h)\beta_{B'}\beta_{B'}^{-1}(g)\iota_B\\
			&=\beta_{B''}^{-1}(hg)\beta_B \\
			&= (hg)^{*} = FG(hg).
		\end{align*}
		Again by faithfulness, we have $G(h)G(g) = G(hg)$. We have shown $G$ is a functor.
		
		Next, we define the natural transformations. We have already sneakily defined $\beta \colon FG \to 1_{\mathcal{D}}$ as a collection of isomorphisms, so we need to check it is indeed a natural transformation. Let $g\colon B \to B'$ be a morphism in $\mathcal{D}$. When we consider the square
		% https://q.uiver.app/#q=WzAsNCxbMCwwLCJCIl0sWzEsMCwiQiciXSxbMCwxLCJGR0IiXSxbMSwxLCJGR0InIl0sWzAsMSwiZyJdLFsyLDMsIkZHZyIsMl0sWzAsMiwiXFxiZXRhX0IiLDJdLFsxLDMsIlxcYmV0YV97Qid9Il1d
		\[\begin{tikzcd}
			B & {B'} \\
			FGB & {FGB'}
			\arrow["g", from=1-1, to=1-2]
			\arrow["FGg"', from=2-1, to=2-2]
			\arrow["{\beta_B}"', from=1-1, to=2-1]
			\arrow["{\beta_{B'}}", from=1-2, to=2-2]
		\end{tikzcd}\]
		we see that we defined $Gg$ precisely so this holds! Hence $\beta$ is a natural transformation. By Lemma \ref{lem:nat_iso_pointw}, it is also a natural isomorphism.
		
		Let $A\in\ob\mathcal{C}$. Then $\beta_{FA}^{-1}$ is a morphism $FA\to FGFA$ in $\mathcal{D}$. By fullness of $F$, there is a morphism, say $\alpha_A\colon A \to GFA$ such that $F\alpha_A = \beta_{FA}^{-1}$. We claim that this defines a natural isomorphism $\alpha 1_{\mathcal{C}}\colon GF$. 
		
		First, we check naturality. If $f\colon A \to B$ is a morphism of $\mathcal{C}$, the relevant square is:
		% https://q.uiver.app/#q=WzAsNCxbMCwwLCJBIl0sWzEsMCwiQiAiXSxbMCwxLCJHRkEiXSxbMSwxLCJHRkEiXSxbMCwxLCJmIl0sWzIsMywiR0ZmIl0sWzAsMiwiXFxhbHBoYV9BIiwyXSxbMSwzLCJcXGFscGhhX0IiXV0=
		\[\begin{tikzcd}
			A & {B } \\
			GFA & GFA
			\arrow["f", from=1-1, to=1-2]
			\arrow["GFf", from=2-1, to=2-2]
			\arrow["{\alpha_A}"', from=1-1, to=2-1]
			\arrow["{\alpha_B}", from=1-2, to=2-2]
		\end{tikzcd}.\]
	Apply $F$ to this square to get
	% https://q.uiver.app/#q=WzAsNCxbMCwwLCJGQSJdLFsxLDAsIkIgIl0sWzAsMSwiR0ZBIl0sWzEsMSwiR0ZBIl0sWzAsMSwiRmYiXSxbMiwzLCJGR0ZmIiwyXSxbMCwyLCJGXFxhbHBoYV9BPSBcXGJldGFfe0ZBfV57LTF9IiwyXSxbMSwzLCJGXFxhbHBoYV9CID0gXFxiZXRhX3tGQn1eey0xfSJdXQ==
	\[\begin{tikzcd}
		FA & {B } \\
		GFA & GFA
		\arrow["Ff", from=1-1, to=1-2]
		\arrow["FGFf"', from=2-1, to=2-2]
		\arrow["{F\alpha_A= \beta_{FA}^{-1}}"', from=1-1, to=2-1]
		\arrow["{F\alpha_B = \beta_{FB}^{-1}}", from=1-2, to=2-2]
	\end{tikzcd}\]
	 which is just the naturality square for $FA$ with respect to $\beta$! Thus it commutes and, by faithfulness, the original square commutes too. It follows that $\alpha$ is a natural transformation.
	 
	 Finally, let's check that $\alpha_A$ is an isomorphism for all $A$. Indeed, $\beta_{FA}$ is a morphism $FGFA \to FA$ in $\mathcal{D}$ so, by fullness, there is a morphism $\gamma_A \colon GFA \to A$ such that $F\gamma_A = \beta_{FA}$. Note that
	 \[
	 	F(\alpha_A\gamma_A) = (F\alpha_A)(F\gamma_A) = \beta_{FA}^{-1}\beta_{FA} = 1_{FGFA} = F(1_{GFA}).
	 \]
	 By faithfulness, $\alpha_A\gamma_A = 1_{GFA}$. A similar calculation yields $\gamma_A\alpha_A = 1_{A}$. Hence $\alpha_A$ is an isomorphism and $\gamma_A = \alpha_A^{-1}$. By Lemma \ref{lem:nat_iso_pointw} $\alpha$ is a natural isomorphism. 
	 \end{proof}
	 \begin{defn}[Skeletal]
	 	Let $\mathcal{C}$ be a category. We say $\mathcal{C}$ is \emph{skeletal} if very isomorphism class of $\ob\mathcal{C}$ has just one member, i.e. if isomorphic objects in $\mathcal{C}$ are necessarily identical.
	 	
	 	By a \emph{skeleton} of $\mathcal{C}$ we mean a full subcategory $\mathcal{C}'$ containing just one object from each isomorphism class.
	 \end{defn}
	 Note that any equivalence between skeletal categories is bijective on objects and hence an isomorphism of categories.
	 \begin{defn}[Monos and Epis]
	 	Let $f\colon A \to B$ be a morphism of a category $\mathcal{C}$. We say $f$ is a \emph{monomorphism} (or \emph{monic}) if for all $g,h\in\mor \mathcal{C}$ we have that $fg = fh$ implies $g= h$ whenever the composites are defined.
	 	
	 	Dually, $f$ is an \emph{epimorphism} (or \emph{epic}) if $gf = hf$ implies $g = h$.
	 \end{defn}
	 Note that isomorphisms are both monic and epic. We give a special name to categories where the converse holds.
	 \begin{defn}[Balanced]
	 	Let $\mathcal{C}$ be a category. We say $\mathcal{C}$ is balanced if all morphisms of $\mathcal{C}$ that are both monic and epic are also isomorphisms.
	 \end{defn}
	 \newpage
	 \section{The Yoneda Lemma}
	 \begin{defn}[Locally small]
	 	We say a category $\mathcal{C}$ is locally small if, for all $A,B\ob\mathcal{C}$ there is a set $\mathcal{C}(A,B)$ so that the class of morphisms $f\in\mor \mathcal{C}$ with $\dom f = A$ and $\cod f = B$ is in one-to-one correspondence with $\mathcal{C}(A,B)$. 
	 \end{defn}
	 Given an object $A$ of a locally small category $\mathcal{C}$ we define a functor $\mathcal{C}(A,-)\colon \mathcal{C} \to \mbold{Set}$. For $B\in\ob\mathcal{C}$ we map $B\mapsto \mathcal{C}(A,B)$, and if $f\colon B \to C$ is a morphism of $\mathcal{C}$ we give a function $f^*\colon \mathcal{C}(A,B) \to \mathcal{C}(A,C)$ defined by
	 \[
	 	f^*(g\colon A \to B) \coloneqq (fg\colon A \to C)
	 \]
	 We claim this is functorial. Compatibility with domains and codomains is immediate. If $f'\colon C \to D$ is a morphism in $\mathcal{C}$ we have $(f'f)g=f(f')g$ for all $g\colon A \to B$, so we have compatibility with composition. Also, the identity $1_B$ is sent to the map $g\mapsto 1_Bg = g$, i.e. $1_{\mathcal{C}(A,B)}$, so we have compatibility with identities.
	 
	 (From now on we use the notation $f^*$ to mean the function $g\mapsto fg$  and $f_*$ to mean $g\mapsto gf$. This is awfully useful).
	 
	 More is true, however. We have defined an assignment $\ob \mathcal{C} \to \ob[\mathcal{C}, \mbold{Set}]$ by sending $A \mapsto \mathcal{C}(A,-)$. We can extend this to a contravariant functor $\mathscr{Y}$ from $\mathcal{C}$ to $[\mathcal{C}, \mbold{Set}]$ by specifying where morphisms go. 
	 
	 Fix $f\colon B \to A$, a morphism in $\mathcal{C}$ (which corresponds to a morphism $A\to B$ in $\mathcal{C}^{\text{op}}$); our task is to define a natural transformation $\mathscr{Y}f\colon \mathcal{C}(A,-)\to \mathcal{C}(B,-)$. 
	 
	 To do this, for all $C\in\ob \mathcal{C}$ we need to define $(\mathscr{Y}f)_C\colon \mathcal{C}(A,C)\to \mathcal{C}(B,C)$, a set-function. But this is easy: 
	 \[
	 	f_*(g\colon A \to C) \coloneqq (gf \colon B\to C).
	 \]
	 
	 Now we need to verify naturality. Let $h\colon C \to D$ be a morphism in $\mathcal{C}$. The relevant square is the following.
	 \[\begin{tikzcd}[row sep=large, column sep=large]
	 	{\mathcal{C}(A,C)} & {\mathcal{C}(A,D)} \\
	 	{\mathcal{C}(B,C)} & {\mathcal{C}(B,D)}
	 	\arrow["{h^*}"', from=2-1, to=2-2]
	 	\arrow["{h^*}", from=1-1, to=1-2]
	 	\arrow["{f_*}"', from=1-1, to=2-1]
	 	\arrow["{f_*}", from=1-2, to=2-2]
	 \end{tikzcd}\]
	At a purely formal level, it is obvious that the square commutes: $g\colon A \to B$ is sent to $hgf$ along the diagonal of the square, no matter which path we take. Hence $\mathscr{Y}f$ is really a natural transformation. We have define $\mathscr{Y}$ by an assignment $\ob\mathcal{C} \to \ob[\mathcal{C} \to \mbold{Set}]$ and a contravariant assignment $\mor\mathcal{C} \to \mor [\mathcal{C} \to \mbold{Set}]$. It remains to show that $\mathscr{Y}$ is really a functor. That it respects domains and codomains is clear from the construction.
	
	Let $A\in \ob \mathcal{C}$. We would like to show that $\mathscr{Y}1_A$ is equal to the identity natural transformation $1_{\mathcal{C}(A,-)}$. To verify this, it suffices to show that each component is an identity morphism. So, let $D\in\ob\mathcal{C}$. Note that $(\mathscr{Y}1_A)_D$ sends $h\colon A\to D$ to the morphism $h1_A = h$. Hence, $\mathscr{Y}$ is compatible with identities.
	
	Next, let $f\colon A \to B$ and $g\colon B \to C$ be morphisms in $\mathcal{C}$; we aim to show $\mathscr{Y}$ respects composition. We begin by the commutative diagram
	% https://q.uiver.app/#q=WzAsMyxbMCwwLCJBIl0sWzEsMCwiQiJdLFsyLDAsIkMiXSxbMCwxLCJmIiwyXSxbMSwyLCJnIiwyXSxbMCwyLCJnZiIsMCx7ImN1cnZlIjotM31dXQ==
	\[\begin{tikzcd}
		A & B & C
		\arrow["f"', from=1-1, to=1-2]
		\arrow["g"', from=1-2, to=1-3]
		\arrow["gf", curve={height=-18pt}, from=1-1, to=1-3]
	\end{tikzcd}\]
	We ask whether this is still commutative under $\mathscr{Y}$ (remember it is contravariant):
	% https://q.uiver.app/#q=WzAsMyxbMCwwLCJcXG1hdGhjYWx7Q30oQSwtKSJdLFsxLDAsIlxcbWF0aGNhbHtDfShCLC0pIl0sWzIsMCwiXFxtYXRoY2Fse0N9KEMsLSkiXSxbMCwxLCJcXG1hdGhzY3J7WX1mIiwyXSxbMSwyLCJcXG1hdGhzY3J7WX1nIiwyXSxbMCwyLCJcXG1hdGhzY3J7WX0oZ2YpIiwwLHsiY3VydmUiOi0zfV1d
	\[\begin{tikzcd}
		{\mathcal{C}(C,-)} & {\mathcal{C}(B,-)} & {\mathcal{C}(A,-)}
		\arrow["{\mathscr{Y}g}"', from=1-1, to=1-2]
		\arrow["{\mathscr{Y}f}"', from=1-2, to=1-3]
		\arrow["{\mathscr{Y}(gf)}", curve={height=-18pt}, from=1-1, to=1-3]
	\end{tikzcd}.\]
	It suffices to verify whether it is ``pointwise'' commutative, i.e., if $D\in\ob\mathcal{C}$ that we have
	% https://q.uiver.app/#q=WzAsMyxbMCwwLCJcXG1hdGhjYWx7Q30oQSxEKSJdLFsxLDAsIlxcbWF0aGNhbHtDfShCLEQpIl0sWzIsMCwiXFxtYXRoY2Fse0N9KEMsRCkiXSxbMCwxLCIoXFxtYXRoc2Nye1l9ZilfRCIsMl0sWzEsMiwiKFxcbWF0aHNjcntZfWcpX0QiLDJdLFswLDIsIihcXG1hdGhzY3J7WX0oZ2YpKV9EIiwwLHsiY3VydmUiOi0zfV1d
	\[\begin{tikzcd}
		{\mathcal{C}(C,D)} & {\mathcal{C}(B,D)} & {\mathcal{C}(A,D)}
		\arrow["{g_*}"', from=1-1, to=1-2]
		\arrow["{f_*}"', from=1-2, to=1-3]
		\arrow["{(gf)_*}", curve={height=-18pt}, from=1-1, to=1-3].
	\end{tikzcd}\]
	This is immediate: if $h\colon C \to D$ is a morphism then $(gf)_*h = h(gf)$ while
	\[
		f_*g_*h = (hg)f,
	\] 
	and we are done by associativity.
	
	Dually, for a fixed object $B$ of $\mathcal{C}$ we define the functor
	\[
	\mathcal{C}(-,B) \coloneqq \mathcal{C}^{\text{op}}(B,-) \colon \mathcal{C^{\text{op}}} \to \mbold{Set},
	\]
	and the same constructions works in this context.
	 \begin{lem}[Yoneda Lemma]
	 	Let $\mathcal{C}$ be a locally small category, $A\in\ob \mathcal{C}$ and $F\colon\mathcal{C}\to \mbold{Set}$. Then there is a one-to-one and onto correspondence between natural transformations $\mathcal{C}(A,-) \to F$ and elements of $FA$. Moreover, this correspondence is natural in $A$ and $F$.
	 \end{lem}
	 A few words on the statement of the lemma. Note that a priori the collection of natural transformations $\mathcal{C}(A,-) \to F$ is not a set, so it doesn't make sense to say that this collection is in bijection with $FA$. Of course, during the proof we will construct such a ``bijection'', proving that this collection is indexed by a set (so it is ``small'').
	 	
	 Now to explain naturality. It is helpful to rephrase the lemma in the following form.
	 \begin{lem}[Yoneda Lemma']
	 	Let $\mathcal{C}$ be a locally small category, $A\in\ob \mathcal{C}$ and $F\colon\mathcal{C}\to \mbold{Set}$. Then $[\mathcal{C},\mbold{Set}](\mathcal{C}(A,-), F)$ is a set and
	 	\[
	 		[\mathcal{C},\mbold{Set}](\mathcal{C}(A,-), F) \cong FA,
	 	\]
	 	naturally in $A$ and $F$.
	 \end{lem}
	 This says exactly the same thing. Now, when we say a morphism $FA \to GA$ is natural in $A$ we just mean that the collection of morphisms $\alpha_A \colon FA \to GA$ define a natural transformation $F \to G$. This is easy enough, but in the statement of Yoneda Lemma' we have a map (actually an isomorphism) between two expressions with \emph{two} free variables, $A$ and $F$, rather than one. How to interpret this? 
	 
	 We will give an informal description now, and formalize later. First, define a functor from the product category $\mathcal{C}\times [\mathcal{C}, \mbold{Set}]$ (defined in the obvious way, details later) to the category of sets $\mbold{Set}$ given on objects by
	 \[
	 	(A,F) \mapsto FA
	 \]
	 (There is only one way to define the assignment on morphisms, details later). Yoneda Lemma says that there is another way to define a functor $\mathcal{C}\times [\mathcal{C}, \mbold{Set}]\to \mbold{Set}$ given on objects by
	 \[
	 	(A,F)\mapsto [\mathcal{C},\mbold{Set}](\mathcal{C}(A,-),F)
	 \]
	 and this is naturally isomorphic to the first functor. So, saying that the correspondence is natural in two variables is just stating that there is a natural isomorphism between two functors with domain a certain product category. Compare this with the following situation from linear algebra: we say a function is linear in two given variables if there is a bilinear map (with domain a product of vector spaces) representing the given function (we will come back to this analogy later, when we study functors on product categories). It is a fact that if a correspondence is natural in both variables then it is ``jointly natural'' (cf. a function $f\colon \mathbb{R}^2\to\mathbb{R}$ can be continuous in both variables without being continuous); don't worry if you don't know what this means since precise statement will be given.
	 \subsection*{Digression on Products}
	 (This subsection elaborates on some exercises from the book \cite{Leinster2014}.) Let $\mathcal{C}$ and $\mathcal{D}$ be categories. The \emph{product category} $\mathcal{C}\times \mathcal{D}$ has objects ordered pairs $(A,B)$ for $A\in\mathcal{C}$ and $B\in\mathcal{D}$ and morphisms $(A,B)\to(A',B')$ ordered pairs $(f,g)$ where $f\colon A \to A'$ in $\mathcal{C}$ and $g\colon B\to B'$ in $\mathcal{D}$. For all $(A,B)$ we define $1_{(A,B)} = (1_A,1_B)$. Let $(f,g)\colon (A,B)\to (A',B')$ and $(h,i)\colon (A',B')\to (A'',B'')$. Composition is defined by
	 \[
	 	(h,i)(f,g) \coloneqq (hf, ig).
	 \]
	 That is, business as usual.
	 \begin{prop}
	 	For all categories $\mathcal{C}$ and $\mathcal{D}$, the product category $\mathcal{C}\times\mathcal{D}$ is a category 
	 \end{prop}
	 \begin{proof}
	 	We need to verify associativity for composition and the identity axiom (the other axioms are obviously satisfied). For the rest of the proof, let $A_1,A_2,A_3,A_4$ be objects of $\mathcal{C}$ and $B_1,B_2,B_3,B_4$ be objects of $\mathcal{D}$. In addition, let $(f,g)\colon (A_1,B_1) \to (A_2,B_2)$ and $(h,i)\colon (A_2,B_2)\to (A_3,B_3)$ and $(j,k)\colon (A_3,B_3)\to (A_4,B_4)$ be morphisms in $\mathcal{C}\times \mathcal{D}$.
	 	
	 	Note that
	 	\[
	 		((j,k)(h,i))(f,g) = (jh,ki)(f,g) = ((jh)f,(ki)g),
	 	\]
	 	and similarly,
	 	\[
	 		(j,k)((h,i)(f,g)) = (j,k)(hf,ig) = (j(hf),k(ig)),
	 	\]
	 	so the results are the same by associativity of $\mathcal{C}$ and $\mathcal{D}$. 
	 	
	 	Finally, we can verify the identity axiom for $1_{(A_2,B_2)}$. We have that
	 	\[
	 		(f,g)1_{(A_2,B_2)} = (f,g)(1_{A_2},1_{B_2}) = (f1_{A_2}, g1_{B_2}) = (f,g).
	 	\]	
	 	Similarly, $1_{(A_2,B_2)} (h,i) = (h,i)$ and we are done.
	 \end{proof}
	 Now we would like to study functors $\mathcal{C}\times \mathcal{D}\to\mathcal{E}$; these are sometimes called \emph{bifunctors}. Such functors can be regarded as an interlocking pair of families of functors, much like in the bilinear/linear analogy.
	 
	 Let $F\colon\mathcal{C}\times \mathcal{D}\to\mathcal{E}$ be a functor. For each $A\in\ob\mathcal{C}$ we define the functor $F^A \colon \mathcal{D}\to\mathcal{E}$ as follows. 
	 For each $B\in\ob\mathcal{D}$ we define $F^AB = F(A,B)$.
	 \begin{itemize}
	 	\item For each $B\in\ob\mathcal{D}$ we define $F^AB \coloneqq F(A,B)$.
	 	\item For each $g\colon B\to B'$ in $\mathcal{D}$ we define $F^Ag \coloneqq F(1_A,g)$.
	 \end{itemize}
	 We claim that this is functorial. Indeed, for all $B\in\ob\mathcal{D}$ we have that 
	 \[
	 	F^A(1_B) = F(1_A,1_B) = F(1_{(A,B)}) = 1_{F(A,B)} = 1_{F^AB},
	 \]
	 so $F$ respects identities. Now let $g\colon B \to B'$ and $h\colon B' \to B''$ be maps in $\mathcal{D}$. We have that
	 \[
	 	(F^Ah)(F^Ag) = (F(1_A,h))(F(1_A,g)) = F((1_A,h)(1_A,g)) = F(1_A,hg) = F^A(hg),
	 \]
	 so composition is preserved. Hence $F^A$ is indeed a functor for each $A\in\ob\mathcal{D}$. 
	 
	 Similarly we can define $F_B$ for all $B\in\mathcal{D}$. Either go through the same construction again or note that there is a functor $R\colon \mathcal{D}\times \mathcal{C}\to\mathcal{C}\times\mathcal{D}$ that reverses the order of the factors in the obvious way; verifying  that this is a functor (and in fact an isomorphism of categories) is extremely easy. Then define $F_B \coloneq (FR)^B$.
	 
	 (Technically we don't need $F$ to be a functor to define $F^A$ and $F_B$, so we define these even when $F$ is just an assignment from objects to objects and morphisms to morphisms $\mathcal{C}\times\mathcal{D}\to\mathcal{E}$. We will need this to state the Bifunctor Criterion later.)
	 
	 So, the functor $F$ can be decomposed into two families of functors. Given the families, can we reconstruct $F$? The answer is yes.
	\begin{lem}\label{lem:prod_fam}
		Let $\mathcal{C},\mathcal{D}$ and $\mathcal{E}$ be categories. Suppose we have $(F^A)_{A\in\ob\mathcal{C}}$, a family of functors $\mathcal{D}\to \mathcal{E}$,  and $(F_B)_{B\in\ob\mathcal{D}}$, a family of functors $\mathcal{C}\to\mathcal{E}$. Furthermore, assume that the following two conditions are satisfied:
		\begin{enumerate}[label=(\alph*)]
			\item If $A\in\ob\mathcal{C}$ and $B\in\ob\mathcal{D}$ then $F^A(B) = F_B(A)$.
			\item If $f\colon A \to A'$ is a morphism in $\mathcal{C}$ and $g\colon B\to B'$ is a morphism in $\mathcal{D}$, then 
			\[
				(F^{A'}g)(F_Bf) = (F_{B'}f)(F^Ag).
			\]
		\end{enumerate}
		Then there is a unique functor $F\colon \mathcal{C}\times \mathcal{D} \to \mathcal{E}$ such that the functors $F^A$ and $F_B$ are precisely those defined earlier.
		\end{lem}
		\begin{rem}\label{rem:prod_dec_prop}
			Note that the converse is true. That is, if $F\colon \mathcal{C}\times \mathcal{D} \to \mathcal{E}$ is a functor then $F^A(B) = F_B(A) = F(A,B)$. In addition,
			\[
				(F^{A'}g)(F_Bf) = (F(1_{A'},g))(F(f,1_B))= F(1_{A'}f, g1_B) = F(f,g),
			\] 
			and similarly $(F_{B'}f)(F^{A}g)=F(f,g)$.
		\end{rem}
		\begin{proof}
			Let us define the functor $F$ by specifying $F(A,B) \coloneqq F^A(B) = F_B(A)$ for all $(A,B)\in\ob(\mathcal{C}\times\mathcal{D})$. We also specify that if $(f,g)\colon (A,B)\to (A',B')$ is a morphism in $\mathcal{C}\times \mathcal{D}$ we have that 
			\[
				F(f,g) \coloneqq (F^{A'}g)(F_Bf) = (F_{B'}f)(F^Ag).
			\]
			It takes a moment staring at the following diagrams to realize that the composites have the right domains and codomains.
			% https://q.uiver.app/#q=WzAsMyxbMCwwLCJGKEEsQikiXSxbMSwwLCJGKEEnLEIpIl0sWzIsMCwiRihBJyxCJykiXSxbMCwxLCJGX0JmIl0sWzEsMiwiRl57QSd9ZyJdXQ==
			\[\begin{tikzcd}
				{F(A,B)} & {F(A',B)} & {F(A',B')}
				\arrow["{F_Bf}", from=1-1, to=1-2]
				\arrow["{F^{A'}g}", from=1-2, to=1-3]
			\end{tikzcd}\]
			% https://q.uiver.app/#q=WzAsMyxbMCwwLCJGKEEsQikiXSxbMSwwLCJGKEEsQicpIl0sWzIsMCwiRihBJyxCJykiXSxbMCwxLCJGXkFnIl0sWzEsMiwiRl97Qid9ZiJdXQ==
			\[\begin{tikzcd}
				{F(A,B)} & {F(A,B')} & {F(A',B')}
				\arrow["{F^Ag}", from=1-1, to=1-2]
				\arrow["{F_{B'}f}", from=1-2, to=1-3]
			\end{tikzcd}\]
			By Remark \ref{rem:prod_dec_prop} if $F$ is a functor, then $F^A$ and $F_B$ are defined properly. 
			
			Let $(A,B)\in\ob(\mathcal{C}\times\mathcal{D})$ and note that
			\begin{align*}
				F(1_{(A,B)}) = F(1_A,1_B) = (F^A1_B)(F_B1_A) &= (1_{F^AB})(1_{F_BA}) \\
				&= (1_{F(A,B)})(1_{F(A,B)})\\
				&= 1_{F(A,B)}.
			\end{align*}
			Next, let there be maps in $\mathcal{C}\times\mathcal{D}$ as follows
			% https://q.uiver.app/#q=WzAsMyxbMCwwLCIoQSxCKSJdLFsxLDAsIihBJyxCJykiXSxbMiwwLCIoQScnLEInJykiXSxbMCwxLCIoZixnKSJdLFsxLDIsIihoLGkpIl1d
			\[\begin{tikzcd}
				{(A,B)} & {(A',B')} & {(A'',B'')}
				\arrow["{(f,g)}", from=1-1, to=1-2]
				\arrow["{(h,i)}", from=1-2, to=1-3]
			\end{tikzcd}.\]
			Then we have
			\begin{align*}
				F(h,i)F(f,g) &= (F^{A''}i)(F_{B'}h)(F^{A'}g)(F_Bf)\\
				&=(F^{A''}i)(F_{B'}h)(F^{B'}g)(F_Af)\\
				&= (F^{A''}i)(F_{B'}(hf))(F_Af)\\
				&= (F_{B''}(hf))(F^Ai)(F^Ag)\\
				&= F_{B''}(hf)F^A(ig)\\
				&= F(hf,ig).
			\end{align*}
			Hence $F$ is really a functor. Now we need to show that it is unique. Let $G\colon \mathcal{C}\times\mathcal{D}\to\mathcal{E}$ be a functor so that $G^A = F^A$ and $G_B = F_B$ for all $A\in\ob\mathcal{D}$ and $B\in\mathcal{D}$. By Remark \ref{rem:prod_dec_prop} it is clear that the values of $G$ on objects and morphisms are already determined, so $F =G$ as functors.
		\end{proof}
		For example, earlier we defined a functors $\mathcal{C}(A,-)\colon\mathcal{C}\to \mbold{Set}$ for any locally small category $\mathcal{C}$ and object $A\in\ob\mathcal{C}$; similarly we had a functor $\mathcal{C}(-,B)\colon\mathcal{C}^{\text{op}}\to \mbold{Set}$ for all objects $B\in\ob\mathcal{C}$. It is clear that for all $A,B\in\mathcal{C}$ we have
		\[
			\mathcal{C}(A,-)B = \mathcal{C}(A,B) = \mathcal{C}(-,B)A.
		\]
		Let $f\colon A'\to A$ and $g\colon B\to B'$ be morphisms in $\mathcal{C}$ (note the direction of $f$). Then for any $h\in \mathcal{C}(A,B)$ we have
		\[
			(\mathcal{C}(A',-)g)(\mathcal{C}(-,B)f)h = g^*(f_*h) = g(hf),
		\]
		and
		\[
			(\mathcal{C}(-,B')f)(\mathcal{C}(A,-)g)h = (gh)f,
		\]
		so the expressions match by associativity. Then Lemma \ref{lem:prod_fam} applies and gives a functor
		\[
			\mathcal{C}(-,-) \colon \mathcal{C}^{\text{op}}\times\mathcal{C} \to \mbold{Set}.
		\]
		We don't need to check that this is a functor: this is guaranteed by the lemma!
		
		Often, however, we are defining a functor $\mathcal{C}\times\mathcal{D}\to \mathcal{E}$ directly, and we would like to check functoriality. This is where the Bifunctor Criterion applies.
		\begin{coro}[Bifunctor Criterion]
			Let $\mathcal{C},\mathcal{D}$ and $\mathcal{E}$ be categories, and let $F\colon\mathcal{C}\times\mathcal{D}\to \mathcal{E}$ be an assignment $F\colon \ob(\mathcal{C}\times\mathcal{D})\to \ob(\mathcal{E})$ and $F\colon \mor(\mathcal{C}\times\mathcal{D})\to \mor(\mathcal{E})$. Then $F$ is a functor if and only if $F^A$ and $F_B$ are functors for all $A\in\ob\mathcal{C}$ and $B\in\ob\mathcal{D}$ that satisfy the following condition
			\begin{itemize}
				\item If $f\colon A \to A'$ is a morphism in $\mathcal{C}$ and $g\colon B\to B'$ is a morphism in $\mathcal{D}$, then 
				\[
				(F^{A'}g)(F_Bf) = (F_{B'}f)(F^Ag).
				\]
			\end{itemize}
		\end{coro}
		\begin{proof}
			If $F$ is a functor then $F^A$ and $F_B$ are functors: we verified this when we defined them (and they satisfy the condition by Remark \ref{rem:prod_dec_prop}). Conversely, assume $F^A$ and $F_B$ are functors for all $A\in\ob\mathcal{C}$ and $B\in\ob\mathcal{D}$ satisfying the condition. By definition we have
			\[
				F^A(B) = F(A,B) = F_B(A)
			\]
			for all $A,B$. Thus Lemma \ref{lem:prod_fam} applies and we are done.			
		\end{proof}
		Now we turn to study naturality between functors whose domain is a product category; this is the situation in the Yoneda Lemma. In particular, we will show that such transformations are natural in both variables simultaneously iff they are natural in each variable separately. 
		\begin{lem}\label{lem:nat_coord}
			Let $F,G\colon\mathcal{C}\times\mathcal{D}\to\mathcal{E}$ be functors. Suppose we have $\alpha$, a family of morphisms $\alpha_{A,B}\colon F(A,B)\to G(A,B)$ in $\mathcal{E}$ for all $(A,B)\in\ob(\mathcal{C}\times\mathcal{D})$. Then $\alpha$ is a natural transformation $F \to G$ if and only if it satisfies the following two conditions.
			\begin{itemize}
				\item For each $A\in\ob\mathcal{C}$ the family the collection of morphisms $\alpha_{A,B}\colon F^A(B) \to G^A(B)$ for $B\in\ob\mathcal{D}$ defines a natural transformation $F^A \to G^A$.
				\item For each $B\in\ob\mathcal{D}$ the family the collection of morphisms $\alpha_{A,B}\colon F_B(A) \to G_B(A)$ for $A\in\ob\mathcal{C}$ defines a natural transformation $F_B \to G_B$.
			\end{itemize}
		\end{lem}
		\begin{proof}
			First suppose $\alpha$ is a natural transformation. Let $A\in\ob\mathcal{C}$ and let $g\colon B \to B'$ be a morphism in $\mathcal{D}$. Consider the naturality square for $\alpha$ with respect to $(1_A,g)$.
			% https://q.uiver.app/#q=WzAsNCxbMCwwLCJGXkEoQikiXSxbMSwwLCJGXkEoQicpIl0sWzAsMSwiR15BKEIpIl0sWzEsMSwiR15BKEInKSJdLFswLDEsIkYoMV9BLGcpIl0sWzIsMywiRygxX0EsZykiLDJdLFswLDIsIlxcYWxwaGFfe0EsQn0iLDJdLFsxLDMsIlxcYWxwaGFfe0EsQn0iXV0=
			\[\begin{tikzcd}
				{F^A(B)} & {F^A(B')} \\
				{G^A(B)} & {G^A(B')}
				\arrow["{F(1_A,g)}", from=1-1, to=1-2]
				\arrow["{G(1_A,g)}"', from=2-1, to=2-2]
				\arrow["{\alpha_{A,B}}"', from=1-1, to=2-1]
				\arrow["{\alpha_{A,B'}}", from=1-2, to=2-2]
			\end{tikzcd}\]
			This is exactly the naturality square we needed to check, so the first condition is satisfied, and the second is checked similarly.
			
			Conversely, suppose that the conditions are satisfied. We need to check that $\alpha$ is a natural transformation. Let $(f,g)$ be a morphism $(A,B)\to(A',B')$ in $\mathcal{C}\times \mathcal{D}$. The square we need to check is
			% https://q.uiver.app/#q=WzAsNCxbMCwwLCJGKEEsQikiXSxbMSwwLCJGKEEnLEInKSJdLFswLDEsIkcoQSxCKSJdLFsxLDEsIkcoQScsQicpIl0sWzAsMSwiRihmLGcpIl0sWzIsMywiRyhmLGcpIiwyXSxbMCwyLCJcXGFscGhhX3tBLEJ9IiwyXSxbMSwzLCJcXGFscGhhX3tBLEJ9Il1d
			\[\begin{tikzcd}
				{F(A,B)} & {F(A',B')} \\
				{G(A,B)} & {G(A',B')}
				\arrow["{F(f,g)}", from=1-1, to=1-2]
				\arrow["{G(f,g)}"', from=2-1, to=2-2]
				\arrow["{\alpha_{A,B}}"', from=1-1, to=2-1]
				\arrow["{\alpha_{A',B'}}", from=1-2, to=2-2]
			\end{tikzcd}\]
			By Remark \ref{rem:prod_dec_prop} we can decompose $F(f,g)$ and $G(f,g)$ as follows.
			% https://q.uiver.app/#q=WzAsNixbMCwwLCJGKEEsQikiXSxbMiwwLCJGKEEnLEInKSJdLFswLDEsIkcoQSxCKSJdLFsyLDEsIkcoQScsQicpIl0sWzEsMCwiRihBLEInKSJdLFsxLDEsIkcoQSxCJykiXSxbMCwyLCJcXGFscGhhX3tBLEJ9IiwyXSxbMSwzLCJcXGFscGhhX3tBLEJ9Il0sWzQsNSwiXFxhbHBoYV97QSxCJ30iLDIseyJzdHlsZSI6eyJib2R5Ijp7Im5hbWUiOiJkYXNoZWQifX19XSxbMiw1LCJHXkFnIiwyXSxbNSwzLCJHX3tCJ31mIiwyXSxbMCw0LCJGXkFnIl0sWzQsMSwiRl97Qid9ZiJdXQ==
			\[\begin{tikzcd}
				{F(A,B)} & {F(A,B')} & {F(A',B')} \\
				{G(A,B)} & {G(A,B')} & {G(A',B')}
				\arrow["{\alpha_{A,B}}"', from=1-1, to=2-1]
				\arrow["{\alpha_{A,B}}", from=1-3, to=2-3]
				\arrow["{\alpha_{A,B'}}"', dashed, from=1-2, to=2-2]
				\arrow["{G^Ag}"', from=2-1, to=2-2]
				\arrow["{G_{B'}f}"', from=2-2, to=2-3]
				\arrow["{F^Ag}", from=1-1, to=1-2]
				\arrow["{F_{B'}f}", from=1-2, to=1-3]
			\end{tikzcd}\]
			The left square commutes by the first condition, and the right one commutes because of the second one. It follows that the whole rectangle commutes.
		\end{proof}
		\subsection*{End of Digression}
		Let's go back to Yoneda. Let $\mathcal{C}$ be a locally small category. We define a bifunctor called $\ev$ (for ``evaluation'')
		\[
			\ev\colon \mathcal{C}\times[\mathcal{C},\mbold{Set}] \to \mbold{Set}.
		\]
		If $A\in\ob\mathcal{C}$ and $F\colon \mathcal{C}\to\mbold{Set}$ is a functor we define,
		\[
			\ev(A,F)\coloneqq FA.
		\] 
		Now suppose $f\colon A\to B$ is a morphism in $\mathcal{C}$ and $\alpha\colon F\to G$ is a natural transformation between functors $F,G\colon \mathcal{C}\to\mbold{Set}$. We then have the naturality square for $f$:
		% https://q.uiver.app/#q=WzAsNCxbMCwwLCJGQSA9IFxcZXYoQSxGKSJdLFswLDEsIkdBIl0sWzIsMCwiRkIiXSxbMiwxLCJHQj1cXGV2KEIsRykiXSxbMCwyLCJGZiJdLFsxLDMsIkdmIiwyXSxbMCwxLCJcXGFscGhhX0EiLDJdLFsyLDMsIlxcYWxwaGFfQiJdXQ==
		\[\begin{tikzcd}
			{FA = \ev(A,F)} && FB \\
			GA && {GB=\ev(B,G)}
			\arrow["Ff", from=1-1, to=1-3]
			\arrow["Gf"', from=2-1, to=2-3]
			\arrow["{\alpha_A}"', from=1-1, to=2-1]
			\arrow["{\alpha_B}", from=1-3, to=2-3]
		\end{tikzcd}.\]
		We define $\ev(f,\alpha)\colon \ev(A,F) \to \ev(B,G)$ to be the diagonal of this square, i.e.,
		\[
			\ev(f,\alpha)\coloneqq \alpha_B(Ff) = (Gf)\alpha_A.
		\]
		Next we verify that this construction is functorial. This could be done using the Bifunctor Criterion but we choose to do it by hand. Let $A\in\ob\mathcal{C}$ and $F\colon\mathcal{C}\to \mathcal{D}$ be a functor. We have that
		\[
			\ev(1_{(A,F)}) = \ev(1_A,1_F) = (1_F)_A(F1_A) = 1_{FA}1_{FA} = 1_{\ev(A,F)},
		\]
		so identities are preserved. Next, let $f\colon A\to B$ and $g\colon B \to C$ be morphisms in $\mathcal{C}$, and let $\alpha\colon F\to G$ and $\beta\colon G\to H$ be natural transformations. We will now give our first real diagram chase!
		% https://q.uiver.app/#q=WzAsOSxbMCwwLCJGQSJdLFsxLDAsIkZCIl0sWzIsMCwiRkMiXSxbMCwxLCJHQSJdLFsxLDEsIkdCIl0sWzIsMSwiR0MiXSxbMCwyLCJIQSJdLFsxLDIsIkhCIl0sWzIsMiwiSEMiXSxbMCwxLCJGZiJdLFsxLDIsIkZnIl0sWzAsMywiXFxhbHBoYV9BIiwyXSxbMyw2LCJcXGJldGFfQSIsMl0sWzIsNSwiXFxhbHBoYV9DIl0sWzUsOCwiXFxiZXRhX0MiXSxbMSw0LCJcXGFscGhhX0IiXSxbMyw0LCJHZiJdLFs0LDUsIkdnIl0sWzQsNywiXFxiZXRhX0IiXSxbNiw3LCJIZiIsMl0sWzcsOCwiSGciLDJdXQ==
		\[\begin{tikzcd}
			FA & FB & FC \\
			GA & GB & GC \\
			HA & HB & HC
			\arrow["Ff", from=1-1, to=1-2]
			\arrow["Fg", from=1-2, to=1-3]
			\arrow["{\alpha_A}"', from=1-1, to=2-1]
			\arrow["{\beta_A}"', from=2-1, to=3-1]
			\arrow["{\alpha_C}", from=1-3, to=2-3]
			\arrow["{\beta_C}", from=2-3, to=3-3]
			\arrow["{\alpha_B}", from=1-2, to=2-2]
			\arrow["Gf", from=2-1, to=2-2]
			\arrow["Gg", from=2-2, to=2-3]
			\arrow["{\beta_B}", from=2-2, to=3-2]
			\arrow["Hf"', from=3-1, to=3-2]
			\arrow["Hg"', from=3-2, to=3-3]
		\end{tikzcd}\]
		Every small square is commutative by naturality of $\alpha$ and $\beta$; thus the big square is commutative too. Consider maps $FA\to HC$. Going along the diagonal of the big square (i.e. going along either of the sides of the big square) we get $\ev(gf,\beta\alpha)$. On the other hand, if we go along the diagonal of the upper-left small square and then along the diagonal of the lower-right small square we get $\ev(g,\beta) \ev(f,\alpha)$. Hence
		\[
			\ev((g,\beta)(f,\alpha)) =\ev(gf,\beta\alpha)= \ev(g,\beta) \ev(f,\alpha),
		\]
		and $\ev$ is indeed a functor. (Don't get discouraged if this goes too fast, just stare at the diagram for as long as you need to).
		
		As we have said before, there is another functor $\mathcal{C}\times[\mathcal{C},\mbold{Set}] \to \mbold{Set}$. The Yoneda Lemma says that this functor exists and is naturally isomorphic to $\ev$. We might as well construct the functor before proving the Yoneda Lemma. 
		
		Part of the problem in the construction is that the assignment on objects is not a priori well-defined; we need to verify that the image of the functor indeed lies in $\mbold{Set}$. Though this is part of the content of Yoneda Lemma, we will prove it separately. 
		\begin{prop}\label{prop:yone_no_nat}
			Let $\mathcal{C}$ be a locally small category, $A\in\ob\mathcal{C}$ and $F\colon\mathcal{C}\to\mbold{Set}$ a functor. Then the class
			\[
				[\mathcal{C},\mbold{Set}](\mathcal{C}(A,-),F)
			\]
			is a set isomorphic to $FA$.
		\end{prop}
		\begin{proof}
			We define a class functions 
			\begin{gather*}
				\Phi_{(A,F)} \colon [\mathcal{C},\mbold{Set}](\mathcal{C}(A,-),F) \to FA\\[6pt]
				\Psi_{(A,F)}\colon  FA\to [\mathcal{C},\mbold{Set}](\mathcal{C}(A,-),F)
			\end{gather*}
			as follows. (We drop the subscript of $(A,F)$ for simplicity but we are going to need them in the proof of the Yoneda Lemma). For $\alpha\colon \mathcal{C}(A,-)\to F$ a natural transformation define 
			\[\Phi(\alpha) \coloneqq \alpha_A(1_A)\in F_A.\]
			Fix $x\in FA$. Define, for all $B\in\ob\mathcal{C}$, a set-function $\Psi(x)_B \colon \mathcal{C}(A,B)\to FB$ by the rule
			\[
				\Psi(x)_B(f\colon A \to B) \coloneqq (Ff)x
			\]
			We claim that the collection of all such functions form a natural transformation $\Psi(x) \colon \mathcal{C}(A,-)\to F$. Indeed, let $g\colon B \to C$ be a morphism in $\mathcal{C}$. We need to check the commutativity of the square
			% https://q.uiver.app/#q=WzAsNCxbMCwwLCJcXG1hdGhjYWx7Q30oQSxCKSJdLFsxLDAsIlxcbWF0aGNhbHtDfShBLEMpIl0sWzAsMSwiRkIiXSxbMSwxLCJGQyJdLFswLDEsImdeKiJdLFsyLDMsIkZnIiwyXSxbMCwyLCJcXFBzaSh4KV9CIiwyXSxbMSwzLCJcXFBzaSh4KV9DIl1d
			\[\begin{tikzcd}
				{\mathcal{C}(A,B)} & {\mathcal{C}(A,C)} \\
				FB & FC
				\arrow["{g^*}", from=1-1, to=1-2]
				\arrow["Fg"', from=2-1, to=2-2]
				\arrow["{\Psi(x)_B}"', from=1-1, to=2-1]
				\arrow["{\Psi(x)_C}", from=1-2, to=2-2]
			\end{tikzcd}\]
			Let $f\colon A \to B$ be arbitrary. Then
			\begin{align*}
				(Fg)(\Psi(x)_B)f &= (Fg)(Ffx) \\
				&= (Fg)(Ff)x \\
				&= (F(gf))x \\
				&= \Psi(x)_C(gf) \\
				&= \Psi(x)_C)(g^*)f,
			\end{align*}
			which shows that the diagram commutes. Hence $\Psi(x)$ is a natural transformation for all $x\in FA$.
			
			Now we show that $\Phi$ and $\Psi$ are inverses of each other. Let $x\in FA$ be arbitrary. Then
			\[
				\Phi (\Psi(x)) = \Psi(x)_A(1_A) = (F1_A)x = 1_{FA}x = x.
			\] 
			Now let $\alpha\colon \mathcal{C}(A,-) \to F$ be a natural transformation. We have
			\[
				\Psi(\Phi(\alpha)) = \Psi(\alpha_A(1_A)).
			\]	
			We would like to verify that this is the same natural transformation as $\alpha$. Check the components: for $B\in\ob\mathcal{C}$ and $f\colon A \to B$ we see that  
			\[
				\Psi(\alpha_A(1_A))_Bf=(Ff)(\alpha_A(1_A)).
			\]
			By naturality of $\alpha$ in $f$ we get the following commutative square.
			% https://q.uiver.app/#q=WzAsNCxbMCwwLCJcXG1hdGhjYWx7Q30oQSxBKSJdLFsxLDAsIlxcbWF0aGNhbHtDfShBLEIpIl0sWzAsMSwiRkEiXSxbMSwxLCJGQiJdLFswLDEsImZeKiJdLFsyLDMsIkZmIiwyXSxbMCwyLCJcXGFscGhhX0EiLDJdLFsxLDMsIlxcYWxwaGFfQiJdXQ==
			\[\begin{tikzcd}
				{\mathcal{C}(A,A)} & {\mathcal{C}(A,B)} \\
				FA & FB
				\arrow["{f^*}", from=1-1, to=1-2]
				\arrow["Ff"', from=2-1, to=2-2]
				\arrow["{\alpha_A}"', from=1-1, to=2-1]
				\arrow["{\alpha_B}", from=1-2, to=2-2]
			\end{tikzcd}\]
			Hence,
			\[
				\Psi(\alpha_A(1_A))_Bf = \alpha_Bf^*1_A = \alpha_B(f1_A) = \alpha_Bf.
			\]
			Thus the natural transformations are equal and $\Psi(\Phi(\alpha)) = \alpha$.
		\end{proof}
		Thanks to Proposition \ref{prop:yone_no_nat} we can define, for any locally small category $\mathcal{C}$, a functor 
		\[\mathscr{F} \colon \mathcal{C}\times [\mathcal{C}, \mbold{Set}] \to \mbold{Set}\]
		So that for $A\in\ob\mathcal{C}$ and $\mathscr{F}\colon\mathcal{C}\to\mbold{Set}$ we define
		\[
			\mathscr{F}(A,F) \coloneqq [\mathcal{C},\mbold{Set}](\mathcal{C}(A,-),F).
		\]
		Now onto morphisms. Let $f\colon A \to B$ be a morphism in $\mathcal{C}$ and let $\alpha\colon F \to G$ be a natural transformation of functors $F,G\colon \mathcal{C}\to \mbold{Set}$. We define
		\[
			\mathscr{F}(f,\alpha) \colon [\mathcal{C},\mbold{Set}](\mathcal{C}(A,-), F) \to [\mathcal{C},\mbold{Set}](\mathcal{C}(B,-), G)
		\]
		as follows. For $\beta\colon \mathcal{C}(A,-) \to F$ a natural transformation we define $(\mathscr{F}(f,\alpha))(\beta)$ to be the natural transformation $\mathcal{C}(B,-) \to G$ whose $C$-component is, for all $C\in\ob\mathcal{C}$, the composite
		% https://q.uiver.app/#q=WzAsNCxbMCwwLCJcXG1hdGhjYWx7Q30oQixDKSJdLFsxLDAsIlxcbWF0aGNhbHtDfShBLEMpIl0sWzIsMCwiRkMiXSxbMywwLCJHQyJdLFswLDEsImZeKiJdLFsxLDIsIlxcYmV0YV9DIl0sWzIsMywiXFxhbHBoYV9DIl1d
		\[\begin{tikzcd}
			{\mathcal{C}(B,C)} & {\mathcal{C}(A,C)} & FC & GC
			\arrow["{f_*}", from=1-1, to=1-2]
			\arrow["{\beta_C}", from=1-2, to=1-3]
			\arrow["{\alpha_C}", from=1-3, to=1-4]
		\end{tikzcd}.\]
		We need to check that this is indeed a natural transformation. Let $g\colon C \to D$ be a morphism in $\mathcal{C}$. Consider the diagram:
		% https://q.uiver.app/#q=WzAsOCxbMCwwLCJcXG1hdGhjYWx7Q30oQixDKSJdLFsxLDAsIlxcbWF0aGNhbHtDfShCLEQpIl0sWzAsMywiR0MiXSxbMSwzLCJHRCJdLFswLDEsIlxcbWF0aGNhbHtDfShBLEMpIl0sWzAsMiwiRkMiXSxbMSwxLCJcXG1hdGhjYWx7Q30oQSxEKSJdLFsxLDIsIkZEIl0sWzAsMSwiZ14qIl0sWzIsMywiR2ciLDJdLFswLDQsImZeKiIsMl0sWzQsNSwiXFxiZXRhX0MiLDJdLFs1LDIsIlxcYWxwaGFfQyIsMl0sWzEsNiwiZl4qIl0sWzYsNywiXFxiZXRhX0MiXSxbNywzLCJcXGFscGhhX0QiXSxbNSw3LCJGZyJdLFs0LDYsImdeKiIsMCx7InN0eWxlIjp7ImJvZHkiOnsibmFtZSI6ImRhc2hlZCJ9fX1dXQ==
		\[\begin{tikzcd}
			{\mathcal{C}(B,C)} & {\mathcal{C}(B,D)} \\
			{\mathcal{C}(A,C)} & {\mathcal{C}(A,D)} \\
			FC & FD \\
			GC & GD
			\arrow["{g*}", from=1-1, to=1-2]
			\arrow["Gg"', from=4-1, to=4-2]
			\arrow["{f_*}"', from=1-1, to=2-1]
			\arrow["{\beta_C}"', from=2-1, to=3-1]
			\arrow["{\alpha_C}"', from=3-1, to=4-1]
			\arrow["{f_*}", from=1-2, to=2-2]
			\arrow["{\beta_C}", from=2-2, to=3-2]
			\arrow["{\alpha_D}", from=3-2, to=4-2]
			\arrow["Fg", from=3-1, to=3-2]
			\arrow["{g^*}", from=2-1, to=2-2]
		\end{tikzcd}.\]
		The outer rectangle is the naturality square for $(\mathscr{F}(f,\alpha))(\beta)$ in $g$, so to check whether it is commutative it suffices to check commutativity of the three small squares. The top square is obviously commutative. The middle one is commutative by naturality of $\beta$, and the bottom one by naturality of $\alpha$. Thus $(\mathscr{F}(f,\alpha))(\beta)$ is indeed a natural transformation.
		
		Let's remember what we are doing. We have just define the functor $\mathscr{F}$ on objects and on morphisms, but we haven't checked functoriality yet. So, we continue.
		
		We could use the Bifunctor Criterion but again we do it by hand.\footnote{Then why did we prove the Bifunctor Criterion? I don't know, unfortunately.} Let $A\in\ob\mathcal{C}$ and $F \colon \mathcal{C}\to \mbold{Set}$ a functor. We would like to show that $\mathscr{F}(1_{(A,F)}) =\mathscr{F}(1_A,1_F)$, which is a set-function from $[\mathcal{C},\mbold{Set}](\mathcal{C}(A,-),F)$ to itself, is just the identity on that set. Then let $\beta\colon \mathcal{C}(A,-)\to F$ be a natural transformation; our task is to show that we have the equality of natural transformations $\mathscr{F}(1_A,1_F)\beta = \beta$. It suffices to check this component-wise, so let $C\in\ob\mathcal{C}$ and note that $(\mathscr{F}(1_A,1_F)\beta)_C$ is the composite
		% https://q.uiver.app/#q=WzAsNCxbMCwwLCJcXG1hdGhjYWx7Q30oQSxDKSJdLFsxLDAsIlxcbWF0aGNhbHtDfShBLEMpIl0sWzIsMCwiRkMiXSxbMywwLCJGQyJdLFswLDEsIjFfQV4qIl0sWzEsMiwiXFxiZXRhX0MiXSxbMiwzLCIoMV9GKV9DIl1d
		\[\begin{tikzcd}
			{\mathcal{C}(A,C)} & {\mathcal{C}(A,C)} & FC & FC
			\arrow["{(1_A)_*}", from=1-1, to=1-2]
			\arrow["{\beta_C}", from=1-2, to=1-3]
			\arrow["{(1_F)_C}", from=1-3, to=1-4]
		\end{tikzcd},\]
		which is equal to $\beta_C$, which is what we wanted. Hence $\mathscr{F}$ respects identities.
		
		Now, let $f\colon A \to B$ and $g\colon B \to C$ be morphisms in $\mathcal{C}$. In addition, let $\alpha\colon F \to G$ and $\beta\colon G \to H$ be natural transformations between functors $F,G,H\colon \mathcal{C}\to \mbold{Set}$. We need to check the equality of set-functions
		\[
			\mathscr{F}(gf,\beta\alpha) = \mathscr{F}(g,\beta) \circ \mathscr{F}(f,\alpha). 
		\]
		Let $\gamma\colon \mathcal{C}(A,-) \to F$ be a natural transformation and $D\in\ob\mathcal{C}$. It suffices to calculate 
		\[
			(\mathscr{F}(g,\beta) \circ \mathscr{F}(f,\alpha) (\gamma))_D = ((\mathscr{F}(g,\beta))(\mathscr{F}(f,\alpha) (\gamma)))_D.
		\]
		By definition, this is the composite
		% https://q.uiver.app/#q=WzAsNCxbMCwwLCJcXG1hdGhjYWx7Q30oQyxEKSJdLFsxLDAsIlxcbWF0aGNhbHtDfShCLEQpIl0sWzIsMCwiR0QiXSxbMywwLCJIRCJdLFswLDEsImdfKiJdLFsxLDIsIihGKGYsXFxhbHBoYSkoXFxnYW1tYSkpX0QiXSxbMiwzLCJcXGJldGFfRCJdXQ==
		\[\begin{tikzcd}[column sep=huge]
			{\mathcal{C}(C,D)} & {\mathcal{C}(B,D)} & GD & HD
			\arrow["{g_*}", from=1-1, to=1-2]
			\arrow["{(F(f,\alpha)(\gamma))_D}", from=1-2, to=1-3]
			\arrow["{\beta_D}", from=1-3, to=1-4]
		\end{tikzcd}\]
		Expanding the map in the middle with the definition we get
		% https://q.uiver.app/#q=WzAsNixbMCwwLCJcXG1hdGhjYWx7Q30oQyxEKSJdLFsxLDAsIlxcbWF0aGNhbHtDfShCLEQpIl0sWzQsMCwiR0QiXSxbNSwwLCJIRCJdLFsyLDAsIlxcbWF0aGNhbHtDfShBLEQpIl0sWzMsMCwiRkQiXSxbMCwxLCJnXyoiXSxbMiwzLCJcXGJldGFfRCJdLFsxLDQsImZfKiJdLFs1LDIsIlxcYWxwaGFfRCJdLFs0LDUsIlxcZ2FtbWFfRCJdXQ==
		\[\begin{tikzcd}
			{\mathcal{C}(C,D)} & {\mathcal{C}(B,D)} & {\mathcal{C}(A,D)} & FD & GD & HD
			\arrow["{g_*}", from=1-1, to=1-2]
			\arrow["{\beta_D}", from=1-5, to=1-6]
			\arrow["{f_*}", from=1-2, to=1-3]
			\arrow["{\alpha_D}", from=1-4, to=1-5]
			\arrow["{\gamma_D}", from=1-3, to=1-4]
		\end{tikzcd}.\]
		Clearly this is the same as the composite
		% https://q.uiver.app/#q=WzAsNCxbMCwwLCJcXG1hdGhjYWx7Q30oQyxEKSJdLFszLDAsIkhEIl0sWzEsMCwiXFxtYXRoY2Fse0N9KEEsRCkiXSxbMiwwLCJGRCJdLFsyLDMsIlxcZ2FtbWFfRCJdLFswLDIsIihnZilfKiJdLFszLDEsIihcXGJldGFcXGFscGhhKV9EIl1d
		\[\begin{tikzcd}[column sep=huge]
			{\mathcal{C}(C,D)} & {\mathcal{C}(A,D)} & FD & HD
			\arrow["{\gamma_D}", from=1-2, to=1-3]
			\arrow["{(gf)_*}", from=1-1, to=1-2]
			\arrow["{(\beta\alpha)_D}", from=1-3, to=1-4]
		\end{tikzcd},\]
		which is just $(\mathscr{F}(gf,\beta\alpha) (\gamma))_D$. As $\gamma$ and $D$ were arbitrary, we are done. We have shown that $\mathscr{F}$ is a functor. Now we can state and prove the final version of the Yoneda Lemma.
		\begin{lem}[The Yoneda Lemma]
			Let $\mathcal{C}$ be a locally small category. Then $\ev$ and $\mathscr{F}$ are naturally isomorphic as functors $\mathcal{C}\times[\mathcal{C},\mbold{Set}] \to \mbold{Set}$.
		\end{lem}
		\begin{proof}
			In the proof of Proposition \ref{prop:yone_no_nat} we already constructed the relevant isomorphisms, $\Phi$ and $\Psi$, so all we need to check is that they are natural (we are using Lemma \ref{lem:nat_iso_pointw} once again to save us some work). It is only necessary to check that one of $\Phi$ and $\Psi$ is natural. We do it for $\Psi \colon \ev \to \mathscr{F}$.
			
			Let $f\colon A \to B$ be a morphism in $\mathcal{C}$ and $\alpha \colon F \to G$ a natural transformation of functors $F,G\colon\mathcal{C}\to\mbold{Set}$. We need to check the following naturality square.\footnote{I know it doesn't look like a square.}
			% https://q.uiver.app/#q=WzAsNCxbMCwwLCJGQSJdLFsxLDAsIkdCIl0sWzAsMSwiW1xcbWF0aGNhbHtDfSxcXG1ib2xke1NldH1dKFxcbWF0aGNhbHtDfShBLC0pLEYpIl0sWzEsMSwiW1xcbWF0aGNhbHtDfSxcXG1ib2xke1NldH1dKFxcbWF0aGNhbHtDfShCLC0pLEcpIl0sWzAsMSwiXFxldiBmIl0sWzIsMywiXFxtYXRoc2Nye0Z9KGYsXFxhbHBoYSkiLDJdLFswLDIsIlxcUHNpX3soQSxGKX0iLDJdLFsxLDMsIlxcUHNpX3soQixHKX0iXV0=
			\[\begin{tikzcd}[column sep=large, row sep=large]
				FA & GB \\
				{[\mathcal{C},\mbold{Set}](\mathcal{C}(A,-),F)} & {[\mathcal{C},\mbold{Set}](\mathcal{C}(B,-),G)}
				\arrow["{\ev (f,\alpha)}", from=1-1, to=1-2]
				\arrow["{\mathscr{F}(f,\alpha)}"', from=2-1, to=2-2]
				\arrow["{\Psi_{(A,F)}}"', from=1-1, to=2-1]
				\arrow["{\Psi_{(B,G)}}", from=1-2, to=2-2]
			\end{tikzcd}\]
			Fix $x\in FA$. Checking commutativity for $x$ is checking the equality of two natural transformations, so we should do so component-wise. Fix some $C\in\ob\mathcal{C}$. It is enough to show that
			\begin{equation}\label{eq:yoneda}
				(\mathscr{F}(f,\alpha) (\Psi_{(A,F)} (x)))_C = (\Psi_{(B,G)}(\ev(f,\alpha)(x)))_C.
			\end{equation}
			Start with the left-hand side. By definition, $(\mathscr{F}(f,\alpha) (\Psi_{(A,F)} (x)))_C$ is the composite
			% https://q.uiver.app/#q=WzAsNCxbMCwwLCJcXG1hdGhjYWx7Q30oQixDKSJdLFsxLDAsIlxcbWF0aGNhbHtDfShBLEMpIl0sWzIsMCwiRkMiXSxbMywwLCJHQyJdLFswLDEsImZfKiJdLFsxLDIsIihcXFBzaV97KEEsRil9KHgpKV9DIl0sWzIsMywiXFxhbHBoYV9DIl1d
			\[\begin{tikzcd}[column sep=large]
				{\mathcal{C}(B,C)} & {\mathcal{C}(A,C)} & FC & GC
				\arrow["{f_*}", from=1-1, to=1-2]
				\arrow["{(\Psi_{(A,F)}(x))_C}", from=1-2, to=1-3]
				\arrow["{\alpha_C}", from=1-3, to=1-4]
			\end{tikzcd}.\]
			Let $g\in\mathcal{C}(B,C)$. Then, following the diagram
			% https://q.uiver.app/#q=WzAsNCxbMCwwLCJnIl0sWzEsMCwiZ2YiXSxbMiwwLCJGKGdmKSh4KSJdLFszLDAsIlxcYWxwaGFfQyBGKGdmKSh4KSJdLFswLDEsIiIsMCx7InNob3J0ZW4iOnsic291cmNlIjozMCwidGFyZ2V0IjozMH0sInN0eWxlIjp7InRhaWwiOnsibmFtZSI6Im1hcHMgdG8ifX19XSxbMSwyLCIiLDAseyJzaG9ydGVuIjp7InNvdXJjZSI6MzAsInRhcmdldCI6MzB9LCJzdHlsZSI6eyJ0YWlsIjp7Im5hbWUiOiJtYXBzIHRvIn19fV0sWzIsMywiIiwwLHsic2hvcnRlbiI6eyJzb3VyY2UiOjMwLCJ0YXJnZXQiOjMwfSwic3R5bGUiOnsidGFpbCI6eyJuYW1lIjoibWFwcyB0byJ9fX1dXQ==
			\[\begin{tikzcd}
				g & gf & {F(gf)(x)} & {\alpha_C F(gf)(x)}
				\arrow[shorten <=5pt, shorten >=5pt, maps to, from=1-1, to=1-2]
				\arrow[shorten <=4pt, shorten >=4pt, maps to, from=1-2, to=1-3]
				\arrow[shorten <=4pt, shorten >=4pt, maps to, from=1-3, to=1-4]
			\end{tikzcd}.\]
			So we have just shown
			\[
				(\mathscr{F}(f,\alpha) (\Psi_{(A,F)} (x)))_C(g) = \alpha_C F(gf)(x).
			\]
			Now look at the left-hand side of equation (\ref{eq:yoneda}). We can evaluate at $g$ to get
			\begin{align*}
				(\Psi_{(B,G)}(\ev(f,\alpha)(x)))_C (g) &= (Gg)\ev(f,\alpha)(x) \\
				&= (Gg)((Gf) \alpha_A)(x)\\
				&= G(gf)\alpha_A(x)\\
				&= \alpha_CF(gf)(x),
			\end{align*}
			where the last equality is just naturality of $\alpha$ in $gf$. Thus we see that (\ref{eq:yoneda}) holds.
		\end{proof}
		\begin{coro}\label{coro:yon_emb_ff}
			For any locally small category $\mathcal{C}$ the Yonedda embedding $\mathscr{Y}$, which is a functor $\mathcal{C}^{\text{op}} \to [\mathcal{C},\mbold{Set}]$, is full and faithful.
		\end{coro}
		\begin{proof}
			Let $A,B\in\ob\mathcal{C}$ be arbitrary. Let $F = \mathcal{C}(B,-) = \mathscr{Y}(B)$ in the proof of Proposition \ref{prop:yone_no_nat}. Then $\Psi$ is a bijection
			\[
			\mathcal{C}(B,A) = \mathcal{C}^{\text{op}}(A,B) \to [\mathcal{C},\mbold{Set}](\mathcal{C}(A,-), \mathcal{C}(B,-)).
			\]
			But note that for all $f\in\mathcal{C}(B,A)$, all $C\in\ob\mathcal{C}$, and all$g\colon A \to C$ we have
			\[
				\Psi(f)_C(g) = g^*f = gf = f^*g = \mathscr{Y}(f)_C(g),
			\]
			so $\Psi$ and $\mathscr{Y}$ coincide in this case.
		\end{proof}
		This says that any locally small category is equivalent to a full subcategory of a functor category $[\mathcal{C}^{\text{op}},\mbold{Set}]$.
		\begin{defn}[Representables]
			If $\mathcal{C}$ is a locally small category, we say a functor $F\colon \mathcal{C}\to\mbold{Set}$ is \emph{representable} if it is naturally isomorphic to $\mathcal{C}(A,-)$ for some $A\in\mathcal{C}$. By a \emph{representation} of $F$ we mean a pair $(A,x)$ where $A\in\ob\mathcal{C}$ and $x\in FA$ is such that $\Psi(x)$ is a natural isomorphism $\mathcal{C}(A,-)\to F$.
		\end{defn}
		\begin{coro}
			Let $\mathcal{C}$ be locally small. If $(A,x)$ and $(B,y)$ are both representations of a functor $F\colon\mathcal{C}\to \mbold{Set}$, then there is a unique isomorphism $f\colon A \to B$ such that $Ff(x) = y$.
		\end{coro}
		\begin{proof}
			Note that $\Psi(x)^{-1} \Psi(y)$ is an isomorphism $\mathcal{C}(B,-) \to \mathcal{C}(A,-)$. By Corollary \ref{coro:yon_emb_ff}, which says that $\mathscr{Y}$ is full and faithful, there is a unique $f\colon A \to B$ (necessarily an isomorphism) such that the following diagram commutes.
			% https://q.uiver.app/#q=WzAsMyxbMCwwLCJcXG1hdGhjYWx7Q30oQiwtKSJdLFsyLDAsIlxcbWF0aGNhbHtDfShBLC0pIl0sWzEsMSwiRiJdLFswLDEsIlxcbWF0aHNjcntZfShBLEIpIl0sWzAsMiwiXFxQc2koeSkiLDJdLFsxLDIsIlxcUHNpKHgpIl1d
			\[\begin{tikzcd}
				{\mathcal{C}(B,-)} && {\mathcal{C}(A,-)} \\
				& F
				\arrow["{\mathscr{Y}f}", from=1-1, to=1-3]
				\arrow["{\Psi(y)}"', from=1-1, to=2-2]
				\arrow["{\Psi(x)}", from=1-3, to=2-2]
			\end{tikzcd}\]
			Specialize to the object $B$.
			% https://q.uiver.app/#q=WzAsMyxbMCwwLCJcXG1hdGhjYWx7Q30oQixCKSJdLFsyLDAsIlxcbWF0aGNhbHtDfShBLEIpIl0sWzEsMSwiRkIiXSxbMCwxLCJmXyoiXSxbMCwyLCJcXFBzaSh5KV9CIiwyXSxbMSwyLCJcXFBzaSh4KV9CIl1d
			\[\begin{tikzcd}
				{\mathcal{C}(B,B)} && {\mathcal{C}(A,B)} \\
				& FB
				\arrow["{f_*}", from=1-1, to=1-3]
				\arrow["{\Psi(y)_B}"', from=1-1, to=2-2]
				\arrow["{\Psi(x)_B}", from=1-3, to=2-2]
			\end{tikzcd}\]
			Further specialize to the identity $1_B$ and note that one path gives
			\begin{equation*}
				(\Psi(x)_Bf_*)1_B = \Psi(x)_B(1_Bf) = \Psi(x)_B f = Ff(x)
			\end{equation*}
			while the other gives 
			\[\Psi(y)_B1_B = F1_B(y) = 1_{FB} (y) = y.\]
			Thus $Ff(x) = y$. Now let $f'\colon A \to B$ be an isomorphism such that $Ff'(x) = y$. Then for any $C\in\ob\mathcal{C}$ and $g\colon B \to C$ we have
			\begin{align*}
				(\Psi(x)_C (f')_*)g &= \Psi(x)_C (gf')\\
				&= F(gf')x \\
				&= (Fg)(Ff')(x)\\
				&= Fg(y)\\
				&= \Psi(y)_C(g).
			\end{align*}
			This implies that the diagram we drew when we defined $f$ commutes when we replace $f$ by $f'$. Hence, by uniqueness of $f$, we have $f = f'$.
		\end{proof}
		Let $A,B\in \ob\mathcal{C}$ be objects in some category. A \emph{categorical product} of $A$ and $B$ is an object $P$ with morphisms 
		% https://q.uiver.app/#q=WzAsMyxbMSwwLCJQIl0sWzAsMSwiQSJdLFsyLDEsIkIiXSxbMCwxLCJcXHBpX0EiLDJdLFswLDIsIlxccGlfQiJdXQ==
		\[\begin{tikzcd}
			& P \\
			A && B
			\arrow["{\pi_A}"', from=1-2, to=2-1]
			\arrow["{\pi_B}", from=1-2, to=2-3]
		\end{tikzcd}\]
		so that if $C$ is any object and $f_A \colon C\to A$ and $f_B \colon C \to B$ are morphisms, then there exists a unique morphism $\tau \colon C\to P$ so that the following diagram commutes.
		% https://q.uiver.app/#q=WzAsNCxbMSwxLCJQIl0sWzIsMCwiQSJdLFsyLDIsIkIiXSxbMCwxLCJDIl0sWzMsMSwiZl9BIiwwLHsiY3VydmUiOi0yfV0sWzMsMiwiZl9CIiwyLHsiY3VydmUiOjJ9XSxbMywwLCJcXGV4aXN0cyEgXFx0YXUiLDAseyJzdHlsZSI6eyJib2R5Ijp7Im5hbWUiOiJkYXNoZWQifX19XSxbMCwxLCJcXHBpX0EiLDJdLFswLDIsIlxccGlfQiJdXQ==
		\[\begin{tikzcd}
			&& A \\
			C & P \\
			&& B
			\arrow["{f_A}", curve={height=-12pt}, from=2-1, to=1-3]
			\arrow["{f_B}"', curve={height=12pt}, from=2-1, to=3-3]
			\arrow["{\tau}", dashed, from=2-1, to=2-2]
			\arrow["{\pi_A}"', from=2-2, to=1-3]
			\arrow["{\pi_B}", from=2-2, to=3-3]
		\end{tikzcd}\]
		It is easy to show that a categorical product, if it exists, is unique up to a unique isomorphism.
		
		Now let $f,g\colon A \to B$ be a pair of parallel morphisms. An \emph{equalizer} for $(f,g)$ is an object $E$ together with a morphism $e\colon E \to A$ such that $fe =ge$,  and with the property that, if $h\colon C \to A$ satisfies $fh = gh$ then there exists a unique $\sigma\colon C \to E$ such that 
		% https://q.uiver.app/#q=WzAsMyxbMCwwLCJDIl0sWzEsMCwiRSJdLFsyLDAsIkEiXSxbMSwyLCJlIiwyXSxbMCwxLCJcXGV4aXN0cyFcXHNpZ21hIiwyXSxbMCwyLCJoIiwwLHsiY3VydmUiOi0yfV1d
		\[\begin{tikzcd}
			C & E & A
			\arrow["e"', from=1-2, to=1-3]
			\arrow["{\sigma}"', dashed, from=1-1, to=1-2]
			\arrow["h", curve={height=-12pt}, from=1-1, to=1-3]
		\end{tikzcd}\]
		commutes. Again, equalizers are unique up to unique isomorphism. Both products and equalizers can be defined as representations of certain functors but this only works for locally small categories.
		
		Note in particular that equalizers are monomorphisms. Indeed, suppose $(E,e)$ is an equalizer for $(f,g)$ as above. Let $\alpha_1,\alpha_2\colon C \to E$ be such that $e\alpha_1  = e\alpha_2 $. 
		
		Define $h\coloneqq  e\alpha_1  = e\alpha_2 $. Then
		\[
			fh = fe\alpha_1 = ge \alpha_1 = gh.
		\]
		Hence, by the universal property of equalizers, there is a unique morphism $\sigma\colon C \to E$ so that $e\sigma = h$. But the definition of $h$ implies that $\sigma= \alpha_1 = \alpha_2$ as desired.
		
		We say a monomorphism is \emph{regular} if it occurs as an equalizer. There are dual notions of coproducts and coequalizers we do not elaborate on right now.
		
		\begin{defn}[Separating and detecting families]
			Let $\mathcal{C}$ be a locally small category and $\mathcal{G}$ a class of objects of $\mathcal{C}$.
			\begin{enumerate}
				\item We say $\mathcal{G}$ is a \emph{separating family} of $\mathcal{C}$ if the functors $\mathcal{C}(G,-)$ for $G \in \mathcal{G}$ are collectively faithful, i.e. given morphisms $f,g\colon A\to B$ in $\mathcal{C}$ the equations $fh = gh$ for all $h\colon G \to A$ with $G\in \mathcal{G}$ imply $f = g$.
				\item We say $\mathcal{G}$ is a \emph{detecting family} of $\mathcal{C}$ if the functors $\mathcal{C}(G,-)$ collectively reflect isomorphisms, i.e. if given some $f\colon A \to B$ with the property that for all $h\colon G \to B$ with $G\in \mathcal{G}$ there exists some unique $\sigma \colon G \to A$ such that the following diagram commutes
				% https://q.uiver.app/#q=WzAsMyxbMCwxLCJBIl0sWzEsMSwiQiJdLFsxLDAsIkciXSxbMiwxLCJoIl0sWzAsMSwiZiJdLFsyLDAsIlxcc2lnbWEiLDJdXQ==
				\[\begin{tikzcd}
					& G \\
					A & B
					\arrow["h", from=1-2, to=2-2]
					\arrow["f"', from=2-1, to=2-2]
					\arrow["\sigma"', from=1-2, to=2-1]
				\end{tikzcd},\]
				then $f$ is an isomorphism.
			\end{enumerate}
			If $\mathcal{G} = \{G\}$ for some object $G$ we call $G$ a \emph{separator} or a \emph{detector} respectively.
		\end{defn}
		Recall that a category is \emph{balanced} if all morphisms that are both monic and epic are also isomorphisms.
		\begin{lem}\leavevmode
			\begin{enumerate}
				\item If $\mathcal{C}$ has equalizers then any detecting family is separating.
				\item If $\mathcal{C}$ is balanced then any separating family is detecting
			\end{enumerate}
		\end{lem}
		\begin{proof}\leavevmode
			\begin{enumerate}
				\item First suppose that $\mathcal{C}$ has equalizers and let $\mathcal{G}$ be a detecting family. Let $f,g\colon A \to B$ be morphisms such that for all $h\colon G \to A$ with $G\in \mathcal{G}$ we have $fh = gh$. We aim to show that $f = g$. First we form the equalizer of $(f,g)$, say a morphism $e\colon E \to A$. The strategy is to prove that $e$ is an isomorphism using the detecting property. Then we will be done because $fe=ge$ will imply $f =g$ by composing with $e^{-1}$.
				
				To use the detecting property, let $h\colon G \to A$ be a morphism with $G\in \mathcal{G}$. By assumption, $fh = gh$. Using the universal property for equalizers we get a unique $\sigma G \to E$ such that $e\sigma = h$. As $\mathcal{G}$ is a detecting family, this implies that $e$ is an isomorphism.
				\item Suppose $\mathcal{C}$ is balanced and let $\mathcal{G}$ be a separating family for it. Let $f\colon A \to B$ have the property that for all $h\colon G \to B$ with $G\in \mathcal{G}$ there is some unique $\sigma \colon G \to A$ such that $f\sigma = h$. We wish to show that $f$ is an isomorphism. By the balanced hypothesis, it is enough to show that $f$ is both epic and monic.
				
				Let $\alpha_1,\alpha_2 \colon C \to A$ be morphisms such that $f\alpha_1 = f\alpha_2$. We would like to show that $\alpha_1 = \alpha_2$ using the separating property. For that, let $h\colon G \to C$ be a morphism; we need to show that $\alpha_1 h = \alpha_2 h$. But note that the composites $f\alpha_1 h$ and $f\alpha_2 h$ are equal morphisms $G \to B$. Applying the hypothesis we get a morphism $\sigma\colon G \to A$ unique in satisfying that $f\sigma$ is equal to both $f\alpha_1 h$ and $f\alpha_2 h$. But then uniqueness of $\sigma$ implies that $\alpha_1h = \alpha_2 h$ as required. As $G$ is a separating family we get that $\alpha_1 = \alpha_2$ as required. This all shows that $f$ is a monomorphism.
				% https://q.uiver.app/#q=WzAsNCxbMCwwLCJDIl0sWzEsMCwiQSJdLFsyLDAsIkIiXSxbMCwxLCJHIl0sWzAsMSwiXFxhbHBoYV8xIiwwLHsib2Zmc2V0IjotMX1dLFswLDEsIlxcYWxwaGFfMiIsMix7Im9mZnNldCI6MX1dLFsxLDIsImYiXSxbMywxLCJcXHNpZ21hIiwyLHsic3R5bGUiOnsiYm9keSI6eyJuYW1lIjoiZGFzaGVkIn19fV0sWzMsMCwiaCJdXQ==
				\[\begin{tikzcd}
					C & A & B \\
					G
					\arrow["{\alpha_1}", shift left, from=1-1, to=1-2]
					\arrow["{\alpha_2}"', shift right, from=1-1, to=1-2]
					\arrow["f", from=1-2, to=1-3]
					\arrow["\sigma"', dashed, from=2-1, to=1-2]
					\arrow["h", from=2-1, to=1-1]
				\end{tikzcd}\]
				Similarly, suppose $\beta_1,\beta_2\colon B \to D$ are morphisms satisfying $\beta_1 f = \beta_2 f$. Let $h\colon G \to B$ be a morphism with $G \in \mathcal{D}$. Then, by hypothesis, there is some $\sigma G \to A$ unique in satisfying $f\sigma = h$.
				
				Now it is clear that $\beta_1 f\sigma = \beta_2 f \sigma$. Hence $\beta_1 h = \beta_2 h$. As $\mathcal{G}$ is separating we have $\beta_1 = \beta_2$ and thus $f$ is epic.
				
				% https://q.uiver.app/#q=WzAsNCxbMCwwLCJBIl0sWzEsMCwiQiJdLFsyLDAsIkQiXSxbMSwxLCJHIl0sWzAsMSwiZiJdLFsxLDIsIlxcYmV0YV8xIiwwLHsib2Zmc2V0IjotMX1dLFsxLDIsIlxcYmV0YV8yIiwyLHsib2Zmc2V0IjoxfV0sWzMsMSwiaCIsMl0sWzMsMCwiXFxzaWdtYSIsMCx7InN0eWxlIjp7ImJvZHkiOnsibmFtZSI6ImRhc2hlZCJ9fX1dXQ==
				\[\begin{tikzcd}
					A & B & D \\
					& G
					\arrow["f", from=1-1, to=1-2]
					\arrow["{\beta_1}", shift left, from=1-2, to=1-3]
					\arrow["{\beta_2}"', shift right, from=1-2, to=1-3]
					\arrow["h"', from=2-2, to=1-2]
					\arrow["\sigma", dashed, from=2-2, to=1-1]
				\end{tikzcd}\qedhere\]
			\end{enumerate}
		\end{proof}
		\begin{defn}[Projective and injective objects]
			We say an object $P$ of a category $\mathcal{C}$ is \emph{projective} if given morphisms
			% https://q.uiver.app/#q=WzAsMyxbMSwwLCJQIl0sWzEsMSwiUiJdLFswLDEsIlEiXSxbMiwxLCJnIiwyXSxbMCwxLCJmIl1d
			\[\begin{tikzcd}
				& P \\
				Q & R
				\arrow["g"', from=2-1, to=2-2]
				\arrow["f", from=1-2, to=2-2]
			\end{tikzcd}\]
			with $g$ epic, there exists some $h\colon P \to Q$ making the diagram commute. (If $\mathcal{C}$ is locally small then this is the same as saying that $\mathcal{C}(P,-)$ preserves epimorphisms).
			
			If this holds this doesn't hold for all epic $g$ but for all $g$ in some class $\xi$ of epimorphisms we say $P$ is $\xi$-projective. (Dually, $P$ is \emph{injective} if it's projective in $\mathcal{C}^{\text{op}}$). 
		\end{defn}
		We will now consider the class of pointwise epimorphisms in $[\mathcal{C},\textbf{Set}]$, i.e., those natural transformations $\alpha\colon F \to G$ where $F,G\colon\mathcal{C} \to \textbf{Set}$ are functors and for all $A\in \ob\mathcal{C}$ we have that $\alpha_A\colon FA\to GA$ is an epimorphism in $\text{Set}$, i.e., it is surjective. It turns out that $\alpha$ as is a pointwise epimorphism if and only if it is an epimorphism in $[\mathcal{C},\textbf{Set}]$.
		
		One direction is immediate. Let $\alpha$ be a pointwise epimorphism as above, and suppose $\beta_1,\beta_2\colon G \to H$ are natural transformations such that $\beta_1\alpha = \beta_2 \alpha$. Then for all $A\in \ob\mathcal{C}$ we have
		\[
			(\beta_1)_A\alpha_A = (\beta_1\alpha)_A = (\beta_2\alpha)_A = (\beta_2)_A\alpha_A,
		\]
		and as $\alpha_A$ is epic we have $(\beta_1)_A = (\beta_2)_A$ and it follows that $\alpha$ is epic.
		
		The converse, as mentioned, is true, but we won't be able to prove it presently. In any case, we can prove some things about pointwise epimorphisms.
		\begin{coro}
			Let $\mathcal{C}$ be a locally small category, and let $\xi$ be the class of all pointwise epimorphisms in $[\mathcal{C},\textbf{Set}]$. Then, for all $A\in \ob\mathcal{C}$, the functor $\mathcal{C}(A,-)$ is $\xi$-projective as an object of $[\mathcal{C},\textbf{Set}]$.
		\end{coro}
		\begin{rem}
			When $\xi$ is the class of all pointwise epimorphisms we call $\xi$-projective objects \emph{pointwise projective}. Then the above is saying that objects of the form $\mathcal{C}(A,-)$ are pointwise projective.
		\end{rem}
		\begin{proof}
			Suppose we are given functors $F,G\colon \mathcal{C}\to \textbf{Set}$, and let $\alpha\colon F \to G$ and $\beta \colon \mathcal{C}(A,-) \to G$ be natural transformations with $\alpha$ a pointwise epimorphism. We need to show that there is some natural transformation $\gamma\colon \mathcal{C}(A,-) \to F$ making the diagram 
			% https://q.uiver.app/#q=WzAsMyxbMSwwLCJcXG1hdGhjYWx7Q30oQSwtKSJdLFsxLDEsIkciXSxbMCwxLCJGIl0sWzIsMSwiXFxhbHBoYSIsMl0sWzAsMSwiXFxiZXRhIl0sWzAsMiwiXFxnYW1tYSIsMix7InN0eWxlIjp7ImJvZHkiOnsibmFtZSI6ImRhc2hlZCJ9fX1dXQ==
			\[\begin{tikzcd}
				& {\mathcal{C}(A,-)} \\
				F & G
				\arrow["\alpha"', from=2-1, to=2-2]
				\arrow["\beta", from=1-2, to=2-2]
				\arrow["\gamma"', dashed, from=1-2, to=2-1]
			\end{tikzcd}\]
			commute. We know that natural transformations $\mathcal{C}(A,-)\to F$ correspond to elements of $FA$ by Yoneda. We have a function $\beta_A\colon \mathcal{C}(A,A) \to GA$ which gives us $\beta_A(1_A) \in GA$. As $\alpha$ is a pointwise epimorphism there is some $x\in FA$ such that $\alpha_A(x) = \beta_A(1_A)$. Then $\gamma \coloneqq \Psi(x)$ is a natural transformation $\mathcal{C}(A,-)\to F$ and we claim it makes the diagram commute.
			
			Let $B \in \ob\mathcal{C}$ and $f\colon A \to B$ a morphism. We need to check that 
			\[(\alpha\gamma)_B(f) = \beta_B(f).\]
			
			We evaluate:
			\begin{align*}
				\alpha_B\gamma_B(f) &= \alpha_B (\Psi(x))_Bf\\
				&= \alpha_B((Ff)x)\\
				&= (Gf)\alpha_A(x)\\
				&= (Gf)\beta_A(1_A)\\
				&= \beta_B f^*(1_A)\\
				&= \beta_B (f),
			\end{align*}
			where we used naturality of $\alpha$ and $\beta$.
			% https://q.uiver.app/#q=WzAsNixbMCwwLCJGQSJdLFsxLDAsIkZCIl0sWzAsMSwiR0EiXSxbMSwxLCJHQiJdLFswLDIsIlxcbWF0aGNhbHtDfShBLEEpIl0sWzEsMiwiXFxtYXRoY2Fse0N9KEEsQikiXSxbMCwxLCJGZiJdLFswLDIsIlxcYWxwaGFfQSIsMl0sWzEsMywiXFxhbHBoYV9CIl0sWzIsMywiR2YiLDJdLFs0LDUsImZeKiIsMl0sWzQsMiwiXFxiZXRhX0EiXSxbNSwzLCJcXGJldGFfQiIsMl1d
			\[\begin{tikzcd}
				FA & FB \\
				GA & GB \\
				{\mathcal{C}(A,A)} & {\mathcal{C}(A,B)}
				\arrow["Ff", from=1-1, to=1-2]
				\arrow["{\alpha_A}"', from=1-1, to=2-1]
				\arrow["{\alpha_B}", from=1-2, to=2-2]
				\arrow["Gf"', from=2-1, to=2-2]
				\arrow["{f^*}"', from=3-1, to=3-2]
				\arrow["{\beta_A}", from=3-1, to=2-1]
				\arrow["{\beta_B}"', from=3-2, to=2-2]
			\end{tikzcd}\]
		\end{proof}
		\begin{lem}\label{lem:coprod_proj_proj}
			Let $\mathcal{C}$ be a category and $\xi$ a class of epimorphisms in $\mathcal{C}$. Further, let $\mathcal{P}$ be a class of $\xi$-projective objects in $\mathcal{C}$. If the coproduct of all objects of $\mathcal{P}$ exists, then it is itself $\xi$-projective.
		\end{lem}
		\begin{proof}
			Let $g\colon Q \to R$ be an epimorphism in $\xi$ and $f\colon T \to R$ be a morphism where $T$ is the coproduct of all elements of $\xi$. Our task is to find a map $h \colon T \to Q$ such that $gh = f$. 
			
			For all $P\in \mathcal{P}$ we have morphisms $\iota_P \colon P \to T$ so $f\iota_P$ is a morphism $P \to R$ and so, as $P$ is $\xi$-projective, we get some $h_P \colon P \to Q$ satisfying $gh_P = f\iota_P$. By the universal property of coproducts there is a morphism $h\colon T \to Q$ unique in satisfying $h\iota_P = h_P$ for all $P\in\mathcal{P}$. Composing with $g$ yields 
			\[gh\iota_P = gh_P = f\iota_P,\]
			for all $P\in\mathcal{P}$. 
			
			Now define for each $P\in\mathcal{P}$ the morphism $f_P\coloneqq f\iota_P = gh\iota_P$, which is a map $P \to R$. Again, by the universal property of coproducts, there is a map $T \to R$ which is unique in that its composition with $\iota_P$ is exactly $f_P$ for all $P\in \mathcal{P}$. But by definition both $f$ and $gh$ satisfy this requirement so they must be equal by uniqueness.
		\end{proof}
		We note the category $[\mathcal{C},\textbf{Set}]$ has (small) coproducts. Indeed, if $\{F_i\}_{i\in I}$ is a set of functors $\mathcal{C}\to \textbf{Set}$, we define the functor
		\[
			F \coloneqq \coprod_{i\in I} F_i \colon \mathcal{C}\to \textbf{Set}
		\]
		by defining for each $A\in \ob \mathcal{C}$ 
		\[
			FA \coloneqq \coprod_{i\in I} F_iA.
		\]
		We still need to define $F$ on morphisms. Note that for each arrow $f\colon A\to B$ in $\mathcal{C}$ we have a set-function $F_if\colon F_iA \to F_iB$ and so, by the universal property of coproducts in $\textbf{Set}$, there is a unique set-function $FA \to FB$, which we take as our definition of $Ff$, so that $(Ff)(x) = (F_if)(x)$ for all $x\in F_iA$. It is immediate from this that $F$ respects identities and composition, i.e., $F$ is a functor.
		
		For each $i\in I$ there is a natural transformation $\iota_i \colon F_i \to F$. Indeed, for each $A\in\ob C$ we take $(\iota_i)_A \colon F_iA \to FA$ to be the inclusion on coproducts in $\textbf{Set}$. Naturality follows by definition of $Ff$.
		
		Now let $\alpha_i \colon F_i \to G$ be a natural transformation for all $i\in I$, where $G\colon \mathcal{C} \to \textbf{Set}$. Then for all $A\in \ob\mathcal{C}$ we obtain set-functions $(\alpha_i)_A\colon F_iA \to GA$ and so, by the universal property of coproducts in $\textbf{Set}$, there is a set-function $\alpha_A\colon FA\to GA$ unique in satisfying $\alpha_A(x) = (\alpha_i)_A(x)$ for all $i\in I$ and $x\in F_iA$. This defines a natural transformation $\alpha \colon F \to G$. Indeed, if $f\colon A \to B$ is a morphism in $\mathcal{C}$ and $x\in F^{i}A$ we have
		\begin{align*}
			(Gf)(\alpha_A)(x) &= (Gf)(\alpha_i)_A (x)\\
			&= (\alpha_i)_B(F^if)(x)\\
			&= (\alpha)_B(Ff)(x),
		\end{align*}
		so naturality is satisfied. Furthermore, $\alpha \iota_i = \alpha_i$ for all $i\in I$ by definition of $\alpha$. Conversely, suppose $\beta\colon F \to G$ satisfies $\beta\iota_i = \alpha_i$ for all $i\in I$. Then for all $A\in \ob\mathcal{C}$ we have $\beta_A(x) = (\alpha_i)_A(x)$ for all $x\in F^iA$ and hence $\beta_A = \alpha_A$ by uniqueness. Thus we see that $F$ has the universal property of coproducts in $[\mathcal{C}, \textbf{Set}]$. This construction comes up in the next proof.
		\begin{prop}
			Let $\mathcal{C}$ is a small category then $[\mathcal{C},\textbf{Set}]$ `has enough pointwise projectives', i.e., for any object $F$ of $[\mathcal{C},\textbf{Set}]$ there exists a pointwise epimorphism $P \to F$ with $P$ pointwise projective.
		\end{prop}
		\begin{proof}
			Let $F\colon \mathcal{C}\to \textbf{Set}$ be a functor. Define
			\[
				P\coloneqq \coprod_{A\in \ob\mathcal{C}}\coprod_{x\in FA} \mathcal{C}(A,-).
			\]
			This is a coproduct of pointwise projective objects and hence pointwise projective by Lemma \ref{lem:coprod_proj_proj} (the coproduct exists precisely because, by assumption, $\ob\mathcal{C}$ is a set).
			
			For $A\in\ob\mathcal{C}$ and $x\in FA$ we have a natural transformation $\Psi_{(A,F)}(x)\colon \mathcal{C}(A,-) \to F$ by the Yoneda Lemma. By the universal property of coproducts, there is a natural transformation $\alpha\colon P \to F$ which is unique in satisfying $\alpha\iota_{(A,x)} = \Psi_{(A,F)}(x)$ for all $A\in\ob\mathcal{C}$ and $x\in FA$, where $\iota_{(A,x)}$ is the usual coprojection. We now check that $\alpha$ is a pointwise epimorphism.
			
			For each $A\in \ob\mathcal{C}$ we need to check that the set-function $\alpha_A \colon PA \to FA$ is surjective. Let $x\in FA$. Then
			\[
				(\Psi_{(A,F)}(x))_A(1_A)  = (F1_A)(x) = 1_{FA} (x) = x,
			\] 
			so it follows that $\alpha_A(\iota_{(A,x)})_A(1_A) = x$ and thus $\alpha_A$ is surjective.			
		\end{proof}
		\newpage
		\section{Adjunctions}
		\begin{defn}[Adjunctions]
			Let $\mathcal{C}$ and $\mathcal{D}$ be two categories. An \emph{adjunction} between $\mathcal{C}$ and $\mathcal{D}$ is a pair of functors $F\colon \mathcal{C}\to \mathcal{D}$ and $G\colon \mathcal{D}\to \mathcal{C}$ together with bijections for each $A\in \mathcal{C}$ and $B\in \mathcal{D}$
			\[
				\mathcal{D}(FA,B) \cong \mathcal{C}(A,GB)
			\]
			which are natural in both $A$ and $B$.
			
			In this case we say that $F$ is \emph{left adjoint} to $G$ and $G$ is \emph{right adjoint} to $F$ and write $(F \dashv G)$.
		\end{defn}
		Like in the Yoneda lemma, if we assume local smallness, we can express this naturality in two variables as the existence of a natural isomorphism between bifunctors. However, unlike in the Yoneda lemma, we do not assume local smallness for the definition to make sense. We explore the case when both $\mathcal{C}$ and $\mathcal{D}$ are locally small to figure out what exactly does naturality entail.
		
		Recall that in this case we defined functors $\mathcal{C}(-,-)\colon \mathcal{C}^{\text{op}} \times \mathcal{C} \to \textbf{Set}$ and $\mathcal{D}(-,-)\colon \mathcal{D}^{\text{op}} \times \mathcal{D} \to \textbf{Set}$. Then there is a functor $\mathcal{C}^{\text{op}} \times \mathcal{D} \to \textbf{Set}$ given by the following commutative diagram.
		% https://q.uiver.app/#q=WzAsNixbMSwxLCJcXG1hdGhjYWx7RH1ee1xcdGV4dHtvcH19XFx0aW1lcyBcXG1hdGhjYWx7RH0iXSxbMiwxLCJcXHRleHRiZntTZXR9Il0sWzIsMCwiXFxtYXRoY2Fse0Ree1xcdGV4dHtvcH19fSJdLFsyLDIsIlxcbWF0aGNhbHtEfSJdLFsxLDAsIlxcbWF0aGNhbHtDfV57XFx0ZXh0e29wfX0iXSxbMCwxLCJcXG1hdGhjYWx7Q31ee1xcdGV4dHtvcH19XFx0aW1lc1xcbWF0aGNhbHtEfSJdLFs1LDQsIiIsMCx7ImN1cnZlIjotMn1dLFs0LDIsIkZee1xcdGV4dHtvcH19Il0sWzUsMywiIiwyLHsiY3VydmUiOjR9XSxbMCwyXSxbMCwzXSxbNSwwLCJGXntcXHRleHR7b3B9fVxcdGltZXMgMV97XFxtYXRoY2Fse0R9fSIsMCx7InN0eWxlIjp7ImJvZHkiOnsibmFtZSI6ImRhc2hlZCJ9fX1dLFswLDEsIlxcbWF0aGNhbHtEfSgtLC0pIl1d
		\[\begin{tikzcd}[column sep=large,row sep=2.25em]
			& {\mathcal{C}^{\text{op}}} & {\mathcal{D^{\text{op}}}} \\
			{\mathcal{C}^{\text{op}}\times\mathcal{D}} & {\mathcal{D}^{\text{op}}\times \mathcal{D}} & {\textbf{Set}} \\
			&& {\mathcal{D}}
			\arrow[curve={height=-12pt}, from=2-1, to=1-2]
			\arrow["{F^{\text{op}}}", from=1-2, to=1-3]
			\arrow[curve={height=24pt}, from=2-1, to=3-3]
			\arrow[from=2-2, to=1-3]
			\arrow[from=2-2, to=3-3]
			\arrow["{F^{\text{op}}\times 1_{\mathcal{D}}}", dashed, from=2-1, to=2-2]
			\arrow["{\mathcal{D}(-,-)}", from=2-2, to=2-3]
		\end{tikzcd}\]
		Here the unlabelled arrows are the usual projections and the dashed arrow is induced by the universal property of products (in ``the category of categories'' which is not a thing but one can define this map directly anyway). We denote this functor by $\mathcal{D}(F-,-)$.
		
		 Similarly there is another functor $\mathcal{C}^{\text{op}} \times \mathcal{D} \to \textbf{Set}$ given by the following diagram.
		% https://q.uiver.app/#q=WzAsNixbMSwxLCJcXG1hdGhjYWx7Q31ee1xcdGV4dHtvcH19XFx0aW1lcyBcXG1hdGhjYWx7Q30iXSxbMiwxLCJcXHRleHRiZntTZXR9Il0sWzIsMiwiXFxtYXRoY2Fse0N9Il0sWzIsMCwiXFxtYXRoY2Fse0N9XntcXHRleHR7b3B9fSJdLFsxLDIsIkQiXSxbMCwxLCJcXG1hdGhjYWx7Q31ee1xcdGV4dHtvcH19XFx0aW1lc1xcbWF0aGNhbHtEfSJdLFs1LDQsIiIsMCx7ImN1cnZlIjoyfV0sWzQsMiwiRyIsMl0sWzUsMywiIiwwLHsiY3VydmUiOi00fV0sWzAsMl0sWzAsM10sWzUsMCwiIDFfe1xcbWF0aGNhbHtDfV57XFx0ZXh0e29wfX19XFx0aW1lcyBHIiwwLHsic3R5bGUiOnsiYm9keSI6eyJuYW1lIjoiZGFzaGVkIn19fV0sWzAsMSwiXFxtYXRoY2Fse0N9KC0sLSkiXV0=
		\[\begin{tikzcd}[column sep=large,row sep=2.25em]
			&& {\mathcal{C}^{\text{op}}} \\
			{\mathcal{C}^{\text{op}}\times\mathcal{D}} & {\mathcal{C}^{\text{op}}\times \mathcal{C}} & {\textbf{Set}} \\
			& D & {\mathcal{C}}
			\arrow[curve={height=12pt}, from=2-1, to=3-2]
			\arrow["G"', from=3-2, to=3-3]
			\arrow[curve={height=-24pt}, from=2-1, to=1-3]
			\arrow[from=2-2, to=3-3]
			\arrow[from=2-2, to=1-3]
			\arrow["{ 1_{\mathcal{C}^{\text{op}}}\times G}", dashed, from=2-1, to=2-2]
			\arrow["{\mathcal{C}(-,-)}", from=2-2, to=2-3]
		\end{tikzcd}\]
		We denote this functor by $\mathcal{C}(-,G-)$. Then $(F\dashv G)$ is just the statement that these two functors are naturally isomorphic. Let
		\[
			\alpha \colon \mathcal{D}(F-,-) \to \mathcal{C}(-,G-)
		\]
		be a transformation which is a collection of bijections. Then this gives an adjunction iff $\alpha$ is natural, which happens iff for all $f\colon A'\to A$ in $\mathcal{C}$ and $g\colon B \to B'$ in $\mathcal{D}$ the diagram
		% https://q.uiver.app/#q=WzAsNCxbMCwwLCJcXG1hdGhjYWx7RH0oRkEsQikiXSxbMCwxLCJcXG1hdGhjYWx7Q30oQSxHQikiXSxbMSwwLCJcXG1hdGhjYWx7RH0oRkEnLEInKSJdLFsxLDEsIlxcbWF0aGNhbHtDfShBJyxHQicpIl0sWzAsMiwiKEZmKV97XFxhc3R9Z15cXGFzdCJdLFsxLDMsImZfXFxhc3QoR2cpXlxcYXN0Il0sWzAsMSwiXFxhbHBoYV97QSxCfSIsMl0sWzIsMywiXFxhbHBoYV97QScsQid9Il1d
		\[\begin{tikzcd}[column sep=large,row sep=2.25em]
			{\mathcal{D}(FA,B)} & {\mathcal{D}(FA',B')} \\
			{\mathcal{C}(A,GB)} & {\mathcal{C}(A',GB')}
			\arrow["{(Ff)_{\ast}g^\ast}", from=1-1, to=1-2]
			\arrow["{f_\ast(Gg)^\ast}", from=2-1, to=2-2]
			\arrow["{\alpha_{A,B}}"', from=1-1, to=2-1]
			\arrow["{\alpha_{A',B'}}", from=1-2, to=2-2]
		\end{tikzcd}\]
		commutes. And \emph{this} condition can be stated without local smallness (the bijections may be class functions but who cares), hence is equivalent to our definition of adjunction. By Lemma \ref{lem:nat_coord} we can perform this check in two stages by verifying naturality in each coordinate separately; this is sometimes easier, or at least less overwhelming.
		
		Given a functor $G \colon \mathcal{D} \to \mathcal{C}$ and $A\in \ob\mathcal{C}$ let $(A\,\,{\downarrow}\,\, G)$ be the category whose objects are pairs $(B,f)$ with $B\in \ob \mathcal{D}$ and $f\colon A \to GB$ in $\mathcal{C}$, and whose morphisms $(B,f) \to (B',f')$ are morphisms $g\colon B \to B'$ such that the diagram
		% https://q.uiver.app/#q=WzAsMyxbMCwwLCJBIl0sWzEsMCwiR0IiXSxbMSwxLCJHQiciXSxbMCwyLCJmJyIsMl0sWzAsMSwiZiJdLFsxLDIsIkdnIl1d
		\[\begin{tikzcd}[column sep=large,row sep=2.25em]
			A & GB \\
			& {GB'}
			\arrow["{f'}"', from=1-1, to=2-2]
			\arrow["f", from=1-1, to=1-2]
			\arrow["Gg", from=1-2, to=2-2]
		\end{tikzcd}\]
		commutes. 
		\begin{thm}\label{thm:ladj_init}
			Let $G \colon \mathcal{D}\to\mathcal{C}$ be a functor. Then there is a bijective correspondence between left adjoints of $G$ and families $(I_{A})_{A\in\ob\mathcal{C}}$, where $I_A$ is an initial object of $(A\,\,{\downarrow}\,\, G)$ for each $A\in\ob\mathcal{C}$.
		\end{thm}
		\begin{proof}
			Let $F$ be a left-adjoint of $G$ and $\alpha_{X,Y} \colon \mathcal{D}(FX,Y) \to \mathcal{C}(X,GY)$ be the corresponding bijections for all $X\in\ob\mathcal{C}$ and $Y\in\ob\mathcal{D}$. Fix some $A\in\ob\mathcal{A}$. Then $\alpha_{A,FA}(1_{FA})$ is a map $A \to GFA$ in $\mathcal{C}$. Define $I_{A} \coloneqq (FA, \alpha_{A,FA}(1_{FA}))$; we claim that $I_{A}$ is initial in $(A\,\,{\downarrow}\,\, G)$ for all $A\in\ob\mathcal{C}$.
			
			Let $B\in\ob\mathcal{D}$ and $f\colon A \to GB$ be a morphism so that $(B,f)$ is an object of $(A\,\,{\downarrow}\,\, G)$. Suppose $g\colon FA \to B$ is such that the following diagram commutes.
			% https://q.uiver.app/#q=WzAsMyxbMCwwLCJBIl0sWzEsMCwiR0ZBIl0sWzEsMSwiR0IiXSxbMCwyLCJmIiwyXSxbMCwxLCJcXGFscGhhX3tBLEZBfSgxX3tGQX0pIl0sWzEsMiwiR1xcYWxwaGFeey0xfV97QSxCfShmKSJdXQ==
			\[\begin{tikzcd}[sep=huge]
				A & GFA \\
				& GB
				\arrow["f"', from=1-1, to=2-2]
				\arrow["{\alpha_{A,FA}(1_{FA})}", from=1-1, to=1-2]
				\arrow["{Gg}", from=1-2, to=2-2]
			\end{tikzcd}.\]
			By naturality of $\alpha$ we have the following commutative diagram (this is naturality in the second variable).
			% https://q.uiver.app/#q=WzAsNCxbMCwwLCJcXG1hdGhjYWx7RH0oRkEsRkEpIl0sWzEsMCwiXFxtYXRoY2Fse0R9KEZBLEIpIl0sWzAsMSwiXFxtYXRoY2Fse0N9KEZBLEdCKSJdLFsxLDEsIlxcbWF0aGNhbHtDfShBLEdCKSJdLFswLDIsIlxcYWxwaGFfe0EsRkF9IiwyXSxbMSwzLCJcXGFscGhhX3tBLEJ9Il0sWzAsMSwiZ14qIl0sWzIsMywiKEdnKV4qIiwyXV0=
			\[\begin{tikzcd}[sep=large]
				{\mathcal{D}(FA,FA)} & {\mathcal{D}(FA,B)} \\
				{\mathcal{C}(FA,GB)} & {\mathcal{C}(A,GB)}
				\arrow["{\alpha_{A,FA}}"', from=1-1, to=2-1]
				\arrow["{\alpha_{A,B}}", from=1-2, to=2-2]
				\arrow["{g^*}", from=1-1, to=1-2]
				\arrow["{(Gg)^*}"', from=2-1, to=2-2]
			\end{tikzcd}.\]
			Applying this to $1_{FA}$ we get that 
			\[
				\alpha_{A,B}(g1_{FA}) = (Gg)\alpha_{A,FA}(1_{FA}).
			\]
			This implies, by the other commutative diagram, that $\alpha_{A,B}(g) = f$, i.e. $g = \alpha_{A,B}^{-1}(f)$. So, $g$ is uniquely determined, and reversing the argument shows that this value of $g$ does indeed work. Hence $I_{A}$ is indeed initial.
			
			Conversely, suppose $(FA,\eta_A)$ is initial in $(A\,\,{\downarrow}\,\, G)$. We make $F$ into a functor so that given some $f\colon A \to A'$, we declare $Ff$ to be the unique functor making the diagram
			% https://q.uiver.app/#q=WzAsNCxbMCwwLCJBIl0sWzEsMCwiR0ZBIl0sWzAsMSwiQSciXSxbMSwxLCJHRkEnIl0sWzAsMiwiZiIsMl0sWzAsMSwiXFxldGFfQSJdLFsyLDMsIlxcZXRhX3tBJ30iLDJdLFsxLDMsIkdGZiJdXQ==
			\[\begin{tikzcd}[sep=large]
				A & GFA \\
				{A'} & {GFA'}
				\arrow["f"', from=1-1, to=2-1]
				\arrow["{\eta_A}", from=1-1, to=1-2]
				\arrow["{\eta_{A'}}"', from=2-1, to=2-2]
				\arrow["GFf", from=1-2, to=2-2]
			\end{tikzcd}\]
			commute (this is possible since $(A,\eta_A)$ is initial). Note that the diagram 
			% https://q.uiver.app/#q=WzAsNCxbMCwwLCJBIl0sWzEsMCwiR0ZBIl0sWzAsMSwiQSJdLFsxLDEsIkdGQSciXSxbMCwyLCIxX0EiLDJdLFswLDEsIlxcZXRhX0EiXSxbMiwzLCJcXGV0YV97QSd9IiwyXSxbMSwzLCJHKDFfe0ZBfSkgPSAxX3tHRkF9Il1d
			\[\begin{tikzcd}[sep=large]
				A & GFA \\
				A & {GFA}
				\arrow["{1_A}"', from=1-1, to=2-1]
				\arrow["{\eta_A}", from=1-1, to=1-2]
				\arrow["{\eta_{A}}"', from=2-1, to=2-2]
				\arrow["{G(1_{FA}) = 1_{GFA}}", from=1-2, to=2-2]
			\end{tikzcd}\]
			commutes, so by uniqueness $F(1_{A}) = 1_{FA}$. Similarly, commutativity of the diagram
			% https://q.uiver.app/#q=WzAsNixbMCwwLCJBIl0sWzEsMCwiR0ZBIl0sWzAsMSwiQSciXSxbMSwxLCJHRkEnIl0sWzAsMiwiQScnIl0sWzEsMiwiR0ZBJyJdLFswLDIsImYiLDJdLFswLDEsIlxcZXRhX0EiXSxbMiwzLCJcXGV0YV97QSd9IiwyXSxbMSwzLCJHRmYiLDJdLFsyLDQsImciLDJdLFs0LDUsIlxcZXRhX3tBJyd9IiwyXSxbMyw1LCJHRmciLDJdLFsxLDUsIkdGKGdmKSIsMCx7ImN1cnZlIjotNX1dXQ==
			\[\begin{tikzcd}[sep=large]
				A & GFA \\
				{A'} & {GFA'} \\
				{A''} & {GFA'}
				\arrow["f"', from=1-1, to=2-1]
				\arrow["{\eta_A}", from=1-1, to=1-2]
				\arrow["{\eta_{A'}}"', from=2-1, to=2-2]
				\arrow["GFf"', from=1-2, to=2-2]
				\arrow["g"', from=2-1, to=3-1]
				\arrow["{\eta_{A''}}"', from=3-1, to=3-2]
				\arrow["GFg"', from=2-2, to=3-2]
				\arrow["{GF(gf)}", curve={height=-30pt}, from=1-2, to=3-2]
			\end{tikzcd}\]
			together with uniqueness proves that $F$ respects composition and hence is a functor.
			
			For $A\in\ob\mathcal{C}$ and $B\in\ob\mathcal{D}$ define $\alpha_{A,B}\colon \mathcal{D}(FA,B) \to \mathcal{C}(A,GB)$ so that for each $f\colon FA \to B$ we have $\alpha_{A,B}(f)$ to be the composite
			% https://q.uiver.app/#q=WzAsMyxbMCwwLCJBIl0sWzEsMCwiR0ZBIl0sWzIsMCwiR0IiXSxbMCwxLCJcXGV0YV9YIl0sWzEsMiwiR2YiXV0=
			\[\begin{tikzcd}
				A & GFA & GB
				\arrow["{\eta_A}", from=1-1, to=1-2]
				\arrow["Gf", from=1-2, to=1-3]
			\end{tikzcd}\]
			Then $\alpha_{A,B}$ is a bijection. Indeed, given some $g\colon A \to GB$ we apply the initial property of $(FA,\eta_A)$ to find a unique $f\colon FA \to B$ such that $g = (Gf)\eta_A = \alpha_{A,B}(f)$. Now we check naturality of $\alpha$. Let $f\colon A' \to A$ be a morphism in $\mathcal{C}$ and let $g\colon B \to B'$ be a morphism in $\mathcal{D}$. Consider the diagram
			\[\begin{tikzcd}[column sep=large,row sep=2.25em]
				{\mathcal{D}(FA,B)} & {\mathcal{D}(FA',B')} \\
				{\mathcal{C}(A,GB)} & {\mathcal{C}(A',GB')}
				\arrow["{(Ff)_{\ast}g^\ast}", from=1-1, to=1-2]
				\arrow["{f_\ast(Gg)^\ast}", from=2-1, to=2-2]
				\arrow["{\alpha_{A,B}}"', from=1-1, to=2-1]
				\arrow["{\alpha_{A',B'}}", from=1-2, to=2-2]
			\end{tikzcd}\]
			Let $\varphi\colon FA \to B$ be a morphism. Following the diagram both ways we get
			% https://q.uiver.app/#q=WzAsOSxbMSwwLCJcXHZhcnBoaSJdLFswLDEsIihBIl0sWzEsMSwiR0ZBIl0sWzIsMSwiR0IpIl0sWzMsMSwiKEEnIl0sWzQsMSwiQSJdLFs1LDEsIkdGQSJdLFs2LDEsIkdCIl0sWzcsMSwiR0InKSJdLFsxLDIsIlxcZXRhX0EiXSxbMiwzLCJHXFx2YXJwaGkiXSxbMCwyLCIiLDAseyJzaG9ydGVuIjp7InNvdXJjZSI6MjAsInRhcmdldCI6MjB9LCJzdHlsZSI6eyJ0YWlsIjp7Im5hbWUiOiJtYXBzIHRvIn19fV0sWzQsNSwiZiJdLFs1LDYsIlxcZXRhX0EiXSxbNiw3LCJHXFx2YXJwaGkiXSxbNyw4LCJHZyJdLFszLDQsIiIsMCx7InNob3J0ZW4iOnsic291cmNlIjoyMCwidGFyZ2V0IjoyMH0sInN0eWxlIjp7InRhaWwiOnsibmFtZSI6Im1hcHMgdG8ifX19XV0=
			\[\begin{tikzcd}[column sep=scriptsize]
				& \varphi \\
				{(A} & GFA & {GB)} & {(A'} & A & GFA & GB & {GB')}
				\arrow["{\eta_A}", from=2-1, to=2-2]
				\arrow["G\varphi", from=2-2, to=2-3]
				\arrow[shorten <=2pt, shorten >=2pt, maps to, from=1-2, to=2-2]
				\arrow["f", from=2-4, to=2-5]
				\arrow["{\eta_A}", from=2-5, to=2-6]
				\arrow["G\varphi", from=2-6, to=2-7]
				\arrow["Gg", from=2-7, to=2-8]
				\arrow[shorten <=3pt, shorten >=3pt, maps to, from=2-3, to=2-4]
			\end{tikzcd}\]
			and the following.
			% https://q.uiver.app/#q=WzAsMTEsWzAsMCwiXFx2YXJwaGkiXSxbMSwwLCIoRkEnIl0sWzIsMCwiRkEiXSxbNCwwLCJCIl0sWzUsMCwiQicpIl0sWzEsMSwiKEEnIl0sWzIsMSwiR0ZBJyJdLFszLDEsIkdGQSJdLFs0LDEsIkdCIl0sWzUsMSwiR0InKSJdLFszLDBdLFsxLDIsIkZmIl0sWzIsMywiXFx2YXJwaGkiXSxbMyw0LCJnIl0sWzUsNiwiXFxldGFfe0EnfSIsMl0sWzYsNywiR0ZmIiwyXSxbNyw4LCJHXFx2YXJwaGkiLDJdLFs4LDksIkdnIiwyXSxbMTAsNywiIiwyLHsic2hvcnRlbiI6eyJzb3VyY2UiOjIwLCJ0YXJnZXQiOjIwfSwic3R5bGUiOnsidGFpbCI6eyJuYW1lIjoibWFwcyB0byJ9fX1dLFswLDEsIiIsMCx7InNob3J0ZW4iOnsic291cmNlIjoyMCwidGFyZ2V0IjoyMH19XV0=
			\[\begin{tikzcd}[column sep=scriptsize]
				\varphi & {(FA'} & FA & {} & B & {B')} \\
				& {(A'} & {GFA'} & GFA & GB & {GB')}
				\arrow["Ff", from=1-2, to=1-3]
				\arrow["\varphi", from=1-3, to=1-5]
				\arrow["g", from=1-5, to=1-6]
				\arrow["{\eta_{A'}}"', from=2-2, to=2-3]
				\arrow["GFf"', from=2-3, to=2-4]
				\arrow["G\varphi"', from=2-4, to=2-5]
				\arrow["Gg"', from=2-5, to=2-6]
				\arrow[shorten <=2pt, shorten >=2pt, maps to, from=1-4, to=2-4]
				\arrow[shorten <=3pt, shorten >=3pt, from=1-1, to=1-2]
			\end{tikzcd}\]
			These two results coincide by definition of $Ff$.
		\end{proof}
		\begin{rem}\label{rem:unit_nat}
			Using the notation of the previous proof, let $\eta_A \coloneqq \alpha_{A,FA}(1_{FA})$. We mention that the collection of morphisms $\eta_A\colon A \to GFA$ actually defines a natural transformation $\eta \colon 1_{\mathcal{C}}\to GF$. Indeed, let $f\colon A \to B$ be a morphism in $\mathcal{C}$. We need to show that the following square commutes.
			% https://q.uiver.app/#q=WzAsNCxbMCwwLCJBIl0sWzAsMSwiQiJdLFsxLDAsIkdGQSJdLFsxLDEsIkdGQiJdLFswLDEsIlxcZXRhX0EiLDJdLFsyLDMsIlxcZXRhX0IiXSxbMCwyLCJmIl0sWzEsMywiR0ZmIiwyXV0=
			\[\begin{tikzcd}
				A & B \\
				GFA & GFB
				\arrow["{\eta_A}"', from=1-1, to=2-1]
				\arrow["{\eta_B}", from=1-2, to=2-2]
				\arrow["f", from=1-1, to=1-2]
				\arrow["GFf"', from=2-1, to=2-2]
			\end{tikzcd}\]
			Consider the following diagram.
			% https://q.uiver.app/#q=WzAsNixbMCwwLCJcXG1hdGhjYWx7RH0oRkEsRkEpIl0sWzAsMSwiXFxtYXRoY2Fse0N9KEEsR0ZBKSJdLFsxLDAsIlxcbWF0aGNhbHtEfShGQSxGQikiXSxbMiwwLCJcXG1hdGhjYWx7RH0oRkIsRkIpIl0sWzIsMSwiXFxtYXRoY2Fse0N9KEIsR0ZCKSJdLFsxLDEsIlxcbWF0aGNhbHtDfShBLEdGQikiXSxbMCwyLCIoRmYpXioiXSxbMywyLCIoRmYpXyoiLDJdLFsxLDUsIihHRmYpXioiLDJdLFs0LDUsImZfKiJdLFswLDEsIlxcYWxwaGFfe0EsRkF9IiwyXSxbMiw1LCJcXGFscGhhX3tBLEZCfSJdLFszLDQsIlxcYWxwaGFfe0IsRkJ9Il1d
			\[\begin{tikzcd}
				{\mathcal{D}(FA,FA)} & {\mathcal{D}(FA,FB)} & {\mathcal{D}(FB,FB)} \\
				{\mathcal{C}(A,GFA)} & {\mathcal{C}(A,GFB)} & {\mathcal{C}(B,GFB)}
				\arrow["{(Ff)^*}", from=1-1, to=1-2]
				\arrow["{(Ff)_*}"', from=1-3, to=1-2]
				\arrow["{(GFf)^*}"', from=2-1, to=2-2]
				\arrow["{f_*}", from=2-3, to=2-2]
				\arrow["{\alpha_{A,FA}}"', from=1-1, to=2-1]
				\arrow["{\alpha_{A,FB}}", from=1-2, to=2-2]
				\arrow["{\alpha_{B,FB}}", from=1-3, to=2-3]
			\end{tikzcd}\]
			Both subsquares are commutative by naturality of $\alpha$ (in single variables rather than both variables at the same time), so the whole diagram is commutative. Hence we can have
			% https://q.uiver.app/#q=WzAsNixbMCwwLCIxX3tGQX0iXSxbMSwwLCJGZiJdLFsyLDAsIjFfe0ZCfSJdLFsyLDEsIlxcZXRhX0IiXSxbMSwxLCJcXGFscGhhX3tBLEZCfShGZikgPSAoR0ZmKVxcZXRhX0EgPSBcXGV0YV9CZiJdLFswLDEsIlxcZXRhX0EiXSxbMCwxLCIiLDAseyJzaG9ydGVuIjp7InNvdXJjZSI6MjAsInRhcmdldCI6MjB9LCJzdHlsZSI6eyJ0YWlsIjp7Im5hbWUiOiJtYXBzIHRvIn19fV0sWzIsMSwiIiwyLHsic2hvcnRlbiI6eyJzb3VyY2UiOjIwLCJ0YXJnZXQiOjIwfSwic3R5bGUiOnsidGFpbCI6eyJuYW1lIjoibWFwcyB0byJ9fX1dLFszLDQsIiIsMCx7InNob3J0ZW4iOnsic291cmNlIjoyMCwidGFyZ2V0IjoyMH0sInN0eWxlIjp7InRhaWwiOnsibmFtZSI6Im1hcHMgdG8ifX19XSxbMSw0LCIiLDAseyJzaG9ydGVuIjp7InNvdXJjZSI6MjAsInRhcmdldCI6MjB9LCJzdHlsZSI6eyJ0YWlsIjp7Im5hbWUiOiJtYXBzIHRvIn19fV0sWzIsMywiIiwwLHsic2hvcnRlbiI6eyJzb3VyY2UiOjIwLCJ0YXJnZXQiOjIwfSwic3R5bGUiOnsidGFpbCI6eyJuYW1lIjoibWFwcyB0byJ9fX1dLFs1LDQsIiIsMix7InNob3J0ZW4iOnsic291cmNlIjoyMCwidGFyZ2V0IjoyMH0sInN0eWxlIjp7InRhaWwiOnsibmFtZSI6Im1hcHMgdG8ifX19XSxbMCw1LCIiLDIseyJzaG9ydGVuIjp7InNvdXJjZSI6MjAsInRhcmdldCI6MjB9LCJzdHlsZSI6eyJ0YWlsIjp7Im5hbWUiOiJtYXBzIHRvIn19fV1d
			\[\begin{tikzcd}
				{1_{FA}} & Ff & {1_{FB}} \\
				{\eta_A} & {\alpha_{A,FB}(Ff) = (GFf)\eta_A = \eta_Bf} & {\eta_B}
				\arrow[shorten <=12pt, shorten >=12pt, maps to, from=1-1, to=1-2]
				\arrow[shorten <=12pt, shorten >=12pt, maps to, from=1-3, to=1-2]
				\arrow[shorten <=3pt, shorten >=3pt, maps to, from=2-3, to=2-2]
				\arrow[shorten <=2pt, shorten >=2pt, maps to, from=1-2, to=2-2]
				\arrow[shorten <=2pt, shorten >=2pt, maps to, from=1-3, to=2-3]
				\arrow[shorten <=3pt, shorten >=3pt, maps to, from=2-1, to=2-2]
				\arrow[shorten <=2pt, shorten >=2pt, maps to, from=1-1, to=2-1]
			\end{tikzcd}\]
			which is what we wanted to show.
		\end{rem}
		\begin{coro}
			If $F$ and $F'$ are both left adjoint to a functor $G\colon \mathcal{D} \to \mathcal{C}$ then $F\cong F'$ in $[\mathcal{C},\mathcal{D}]$. 
		\end{coro}
		\begin{proof}
			Let $\alpha\colon \mathcal{D}(F-,-) \to \mathcal{C}(-,G-)$ and $\alpha'\colon \mathcal{D}(F'-,-) \to \mathcal{C}(-,G-)$ be a natural isomorphisms. We can apply Theorem \ref{thm:ladj_init}. This implies that there are initial objects of $(A\,\,{\downarrow}\,\, G)$ of the form $(FA,\eta_A)$ and $(F'A,\eta'_{A})$ for all $A\in\ob\mathcal{C}$, where $\eta_A \coloneqq \alpha_{A,FA}(1_{FA})$ and $\eta'_{A}\coloneqq \alpha_{A,F'A}(1_{F'A})$. Hence there are isomorphisms $\beta_A \colon FA \to F'A$ unique in making the following diagram commute.
			% https://q.uiver.app/#q=WzAsMyxbMCwwLCJBIl0sWzEsMCwiR0ZBIl0sWzEsMSwiR0YnQSJdLFswLDEsIlxcZXRhX0EiXSxbMCwyLCJcXGV0YSdfe0F9IiwyXSxbMSwyLCJHXFxiZXRhX0EiXV0=
			\[\begin{tikzcd}
				A & GFA \\
				& {GF'A}
				\arrow["{\eta_A}", from=1-1, to=1-2]
				\arrow["{\eta'_{A}}"', from=1-1, to=2-2]
				\arrow["{G\beta_A}", from=1-2, to=2-2]
			\end{tikzcd}\]
			We need to show that $\beta$ is natural and then we are done.
			
			Let $f\colon A\to B$ be a morphism in $\mathcal{C}$. We would like to show that the following diagram commutes.
			% https://q.uiver.app/#q=WzAsNCxbMCwwLCJGQSJdLFsxLDAsIkZCIl0sWzAsMSwiRidBIl0sWzEsMSwiRidCIl0sWzAsMSwiRmYiXSxbMCwyLCJcXGJldGFfQSIsMl0sWzIsMywiRidmIiwyXSxbMSwzLCJcXGJldGFfQiJdXQ==
			\[\begin{tikzcd}
				FA & FB \\
				{F'A} & {F'B}
				\arrow["Ff", from=1-1, to=1-2]
				\arrow["{\beta_A}"', from=1-1, to=2-1]
				\arrow["{F'f}"', from=2-1, to=2-2]
				\arrow["{\beta_B}", from=1-2, to=2-2]
			\end{tikzcd}\]
			Our strategy is to prove that both ways around the square are morphisms $(FA,\eta_A) \to (F'B,\eta'_B f)$ and hence they must be equal since $(FA,\eta_A)$ is initial.
			
			First we show that $F'f\beta_A$ is a morphism. We need to show that the following commutes.
			% https://q.uiver.app/#q=WzAsNSxbMCwwLCJBIl0sWzAsMSwiQiJdLFsyLDIsIkdGJ0IiXSxbMiwxLCJHRidBIl0sWzIsMCwiR0ZBIl0sWzAsNCwiXFxldGFfQSJdLFswLDMsIlxcZXRhJ19BIl0sWzQsMywiR1xcYmV0YV9BIl0sWzAsMSwiZiIsMl0sWzEsMiwiXFxldGEnX0IiLDJdLFszLDIsIkdGJ2YiXV0=
			\[\begin{tikzcd}
				A && GFA \\
				B && {GF'A} \\
				&& {GF'B}
				\arrow["{\eta_A}", from=1-1, to=1-3]
				\arrow["{\eta'_A}", from=1-1, to=2-3]
				\arrow["{G\beta_A}", from=1-3, to=2-3]
				\arrow["f"', from=1-1, to=2-1]
				\arrow["{\eta'_B}"', from=2-1, to=3-3]
				\arrow["{GF'f}", from=2-3, to=3-3]
			\end{tikzcd}\]
			The top triangle is commutative by definition of $\beta_A$ and the bottom rectangle is commutative by naturality of $\eta'$. Hence $F'f\beta_A$ is a morphism. The proof that $\beta_B Ff$ is a morphism is essentially the same.
		\end{proof}
		\begin{lem}
			Suppose we are given functors as in the diagram below.
			% https://q.uiver.app/#q=WzAsMyxbMCwwLCJcXG1hdGhjYWx7Q30iXSxbMSwwLCJcXG1hdGhjYWx7RH0iXSxbMiwwLCJcXG1hdGhjYWx7RX0iXSxbMCwxLCJGIiwwLHsib2Zmc2V0IjotMX1dLFsxLDIsIkgiLDAseyJvZmZzZXQiOi0xfV0sWzEsMCwiRyIsMCx7Im9mZnNldCI6LTF9XSxbMiwxLCJLIiwwLHsib2Zmc2V0IjotMX1dXQ==
			\[\begin{tikzcd}
				{\mathcal{C}} & {\mathcal{D}} & {\mathcal{E}}
				\arrow["F", shift left, from=1-1, to=1-2]
				\arrow["H", shift left, from=1-2, to=1-3]
				\arrow["G", shift left, from=1-2, to=1-1]
				\arrow["K", shift left, from=1-3, to=1-2]
			\end{tikzcd}\]
			If $(F \dashv G)$ and $(H \dashv K)$, then $(HF \dashv GK)$.
		\end{lem}
		\begin{proof}
			There are natural isomorphisms\footnote{By the remarks made when adjoints were defined, these aren't really natural transformations. However, we will stop worrying about size issues from now on. The worried reader will be somewhat relieved to know that these issues can be dealt with via a more rigorous approach, look up \emph{Grothendieck Universe}.} $\alpha\colon \mathcal{D}(F-,-) \to \mathcal{C}(-,G-)$ and also $\beta\colon \mathcal{E}(H-,-) \to \mathcal{D}(-,K-)$. We will define a natural isomorphism
			\[
				\mathcal{E}(HF-,-)\to \mathcal{C}(-,GK-)
			\]
			given by
			% https://q.uiver.app/#q=WzAsNCxbMCwxLCJcXG1hdGhjYWx7Q31ee1xcdGV4dHtvcH19XFx0aW1lc1xcbWF0aGNhbHtFfSJdLFsxLDAsIlxcbWF0aGNhbHtEXntcXHRleHR7b3B9fX1cXHRpbWVzXFxtYXRoY2Fse0V9Il0sWzIsMSwiXFx0ZXh0YmZ7U2V0fSJdLFsxLDIsIlxcbWF0aGNhbHtDfV57XFx0ZXh0e29wfX1cXHRpbWVzIFxcbWF0aGNhbHtEfSJdLFswLDIsIlxcbWF0aGNhbHtEfShGLSxLLSkiLDFdLFswLDEsIkZee1xcdGV4dHtvcH19XFx0aW1lczFfe1xcbWF0aGNhbHtFfX0iLDFdLFswLDMsIjFfe1xcbWF0aGNhbHtDfV57XFx0ZXh0e29wfX19XFx0aW1lcyBLIiwxXSxbMywyLCJcXG1hdGhjYWx7RH0oRi0sLSkiLDAseyJjdXJ2ZSI6LTJ9XSxbMSwyLCJcXG1hdGhjYWx7RX0oSC0sLSkiLDAseyJjdXJ2ZSI6LTJ9XSxbMCwyLCJcXG1hdGhjYWx7RX0oSEYtLC0pIiwwLHsiY3VydmUiOi01LCJjb2xvdXIiOlswLDAsNjBdfV0sWzMsMiwiXFxtYXRoY2Fse0N9KC0sRy0pIiwyLHsiY3VydmUiOjJ9XSxbMSwyLCJcXG1hdGhjYWx7RH0oLSxLLSkiLDIseyJjdXJ2ZSI6Mn1dLFswLDIsIlxcbWF0aGNhbHtDfSgtLEdLLSkiLDIseyJjdXJ2ZSI6NSwiY29sb3VyIjpbMCwwLDYwXX1dLFs4LDExLCJcXGJldGEiLDIseyJzaG9ydGVuIjp7InNvdXJjZSI6MjAsInRhcmdldCI6MjB9fV0sWzcsMTAsIlxcYWxwaGEiLDAseyJzaG9ydGVuIjp7InNvdXJjZSI6MjAsInRhcmdldCI6MjB9fV1d
			\[\begin{tikzcd}[sep=huge]
				& {\mathcal{D^{\text{op}}}\times\mathcal{E}} \\
				{\mathcal{C}^{\text{op}}\times\mathcal{E}} && {\textbf{Set}} \\
				& {\mathcal{C}^{\text{op}}\times \mathcal{D}}
				\arrow["{\mathcal{D}(F-,K-)}"{description}, from=2-1, to=2-3]
				\arrow["{F^{\text{op}}\times1_{\mathcal{E}}}"{description}, from=2-1, to=1-2]
				\arrow["{1_{\mathcal{C}^{\text{op}}}\times K}"{description}, from=2-1, to=3-2]
				\arrow[""{name=0, anchor=center, inner sep=0}, "{\mathcal{D}(F-,-)}", curve={height=-12pt}, from=3-2, to=2-3]
				\arrow[""{name=1, anchor=center, inner sep=0}, "{\mathcal{E}(H-,-)}", curve={height=-12pt}, from=1-2, to=2-3]
				\arrow["{\mathcal{E}(HF-,-)}", draw={rgb,255:red,153;green,153;blue,153}, curve={height=-100pt}, from=2-1, to=2-3]
				\arrow[""{name=2, anchor=center, inner sep=0}, "{\mathcal{C}(-,G-)}"', curve={height=12pt}, from=3-2, to=2-3]
				\arrow[""{name=3, anchor=center, inner sep=0}, "{\mathcal{D}(-,K-)}"', curve={height=12pt}, from=1-2, to=2-3]
				\arrow["{\mathcal{C}(-,GK-)}"', draw={rgb,255:red,153;green,153;blue,153}, curve={height=100pt}, from=2-1, to=2-3]
				\arrow["\beta"', shorten <=4pt, shorten >=4pt, Rightarrow, from=1, to=3]
				\arrow["\alpha", shorten <=4pt, shorten >=4pt, Rightarrow, from=0, to=2]
			\end{tikzcd}\]
			where the diagram is commutative except where a natural transformation appears. This defines the required natural isomorphism from top to bottom by horizontally composing with the identity when required.
		\end{proof}
		\begin{coro}
			Suppose we have a commutative square of functors as follows.
			% https://q.uiver.app/#q=WzAsNCxbMCwwLCJcXG1hdGhjYWx7fUMiXSxbMCwxLCJcXG1hdGhjYWx7RX0iXSxbMSwwLCJcXG1hdGhjYWx7RH0iXSxbMSwxLCJcXG1hdGhjYWx7Rn0iXSxbMCwyLCJGIl0sWzEsMywiSyIsMl0sWzAsMSwiRyIsMl0sWzIsMywiSCJdXQ==
			\[\begin{tikzcd}
				{\mathcal{}C} & {\mathcal{D}} \\
				{\mathcal{E}} & {\mathcal{F}}
				\arrow["F", from=1-1, to=1-2]
				\arrow["K"', from=2-1, to=2-2]
				\arrow["G"', from=1-1, to=2-1]
				\arrow["H", from=1-2, to=2-2]
			\end{tikzcd}\]
			If all functors have left adjoints then the commutative square of left adjoints commutes up to natural isomorphism.
		\end{coro}
		\begin{proof}
			Just apply uniqueness of left adjoints and composition of left adjoints.
		\end{proof}
		Given an adjunction $(F \dashv G)$ so that $\alpha\colon \mathcal{D}(F-,-) \to \mathcal{C}(-,G-)$ is a natural isomorphism, we have defined a natural transformation $\eta\colon 1_{\mathcal{C}} \to GF$. We call $\eta$ the \emph{unit} of the adjunction. Dually we can define the \emph{counit} $\varepsilon\colon {FG}\to 1_{\mathcal{D}}$ given by $\varepsilon_B \coloneqq \alpha^{-1}_{GB,B}(1_{GB})$. This is a dual definition: $\varepsilon$ is the unit of the adjunction $(G^{op} \dashv F^{op})$ given by $\alpha^{-1}$. By duality, $\varepsilon$ is a natural transformation.
		\begin{thm}
			Given functors $F\colon \mathcal{C} \to \mathcal{D}$ and $G\colon \mathcal{D} \to \mathcal{C}$ there is a bijective correspondence between adjunctions $(F \dashv G)$ and natural transformations $\eta\colon 1_{\mathcal{C}} \to GF$, $\varepsilon\colon {FG}\to 1_{\mathcal{D}}$ satisfying the `triangular identities'
			% https://q.uiver.app/#q=WzAsNixbMCwwLCJGIl0sWzEsMCwiRkdGIl0sWzEsMSwiRiJdLFszLDAsIkciXSxbNCwwLCJHRkciXSxbNCwxLCJHIl0sWzAsMSwiRlxcZXRhIl0sWzAsMiwiMV9GIiwyXSxbMSwyLCJcXHZhcmVwc2lsb24gRiJdLFszLDQsIlxcZXRhIEciXSxbMyw1LCIxX0ciLDJdLFs0LDUsIkdcXHZhcmVwc2lsb24iXV0=
			\[\begin{tikzcd}
				F & FGF && G & GFG \\
				& F &&& G
				\arrow["F\eta", from=1-1, to=1-2]
				\arrow["{1_F}"', from=1-1, to=2-2]
				\arrow["{\varepsilon F}", from=1-2, to=2-2]
				\arrow["{\eta G}", from=1-4, to=1-5]
				\arrow["{1_G}"', from=1-4, to=2-5]
				\arrow["G\varepsilon", from=1-5, to=2-5]
			\end{tikzcd}\]
		\end{thm}
		\begin{proof}
			Given an adjunction we know how to define the unit and the counit, so we just verify that they satisfy the triangular identities. Let $A \in \ob\mathcal{C}$. Then the following diagram commutes by naturality of $\alpha$.
			% https://q.uiver.app/#q=WzAsNCxbMCwwLCJcXG1hdGhjYWx7RH0oRkdGQSxGQSkiXSxbMCwxLCJcXG1hdGhjYWx7Q30oR0ZBLEdGQSkiXSxbMSwwLCJcXG1hdGhjYWx7RH0oRkEsRkEpIl0sWzEsMSwiXFxtYXRoY2Fse0N9KEEsR0ZBKSJdLFswLDIsIihGXFxldGFfe0F9KV8qIl0sWzEsMywiKFxcZXRhX0EpXyoiXSxbMSwwLCJcXGFscGhhXnstMX1fe0dGQSxGQX0iXSxbMywyLCJcXGFscGhhXnstMX1fe0EsRkF9IiwyXV0=
			\[\begin{tikzcd}
				{\mathcal{D}(FGFA,FA)} & {\mathcal{D}(FA,FA)} \\
				{\mathcal{C}(GFA,GFA)} & {\mathcal{C}(A,GFA)}
				\arrow["{(F\eta_{A})_*}", from=1-1, to=1-2]
				\arrow["{(\eta_A)_*}", from=2-1, to=2-2]
				\arrow["{\alpha^{-1}_{GFA,FA}}", from=2-1, to=1-1]
				\arrow["{\alpha^{-1}_{A,FA}}"', from=2-2, to=1-2]
			\end{tikzcd}\]
			Hence we have
			% https://q.uiver.app/#q=WzAsNCxbMCwwLCJcXHZhcmVwc2lsb25fe0ZBfSJdLFswLDEsIjFfe0dGQX0iXSxbMSwwLCJcXHZhcmVwc2lsb25fe0ZBfUZcXGV0YV9BID0gMV97RkF9Il0sWzEsMSwiXFxldGFfQSJdLFswLDIsIiIsMCx7InNob3J0ZW4iOnsic291cmNlIjoyMCwidGFyZ2V0IjoyMH0sInN0eWxlIjp7InRhaWwiOnsibmFtZSI6Im1hcHMgdG8ifX19XSxbMSwzLCIiLDAseyJzaG9ydGVuIjp7InNvdXJjZSI6MjAsInRhcmdldCI6MjB9LCJzdHlsZSI6eyJ0YWlsIjp7Im5hbWUiOiJtYXBzIHRvIn19fV0sWzEsMCwiIiwwLHsic2hvcnRlbiI6eyJzb3VyY2UiOjIwLCJ0YXJnZXQiOjIwfSwic3R5bGUiOnsidGFpbCI6eyJuYW1lIjoibWFwcyB0byJ9fX1dLFszLDIsIiIsMix7InNob3J0ZW4iOnsic291cmNlIjoyMCwidGFyZ2V0IjoyMH0sInN0eWxlIjp7InRhaWwiOnsibmFtZSI6Im1hcHMgdG8ifX19XV0=
			\[\begin{tikzcd}
				{\varepsilon_{FA}} & {\varepsilon_{FA}F\eta_A = 1_{FA}} \\
				{1_{GFA}} & {\eta_A}
				\arrow[shorten <=3pt, shorten >=3pt, maps to, from=1-1, to=1-2]
				\arrow[shorten <=6pt, shorten >=6pt, maps to, from=2-1, to=2-2]
				\arrow[shorten <=2pt, shorten >=2pt, maps to, from=2-1, to=1-1]
				\arrow[shorten <=2pt, shorten >=2pt, maps to, from=2-2, to=1-2]
			\end{tikzcd}\]
			which is the first triangular identity. By duality, the second triangular identity holds.
			
			Conversely, suppose $\eta\colon 1_{\mathcal{C}}\to GF$ and $\varepsilon\colon FG \to 1_{\mathcal{D}}$ satisfy the triangular identities. We have to show that there is a natural isomorphism $\alpha\colon \mathcal{D}(F-,-)\to \mathcal{C}(-,G-)$ that gives rise to an adjunction whose unit and counit are precisely $\eta$ and $\varepsilon$.
			
			For $A\in\ob\mathcal{C}$, $B\in\ob\mathcal{D}$, and $f\colon FA \to B$ define
			\[
				\alpha_{A,B}(f) \coloneqq (Gf)\eta_A\colon A \to GB.
			\]
			Clearly $\alpha_{A,FA}(1_{FA}) = \eta_A$ so we are in good shape. Now define $\beta\colon \mathcal{C}(-,G-) \to \mathcal{D}(F-,-)$ such that for $g\colon A \to GB$ we have
			\[
				\beta_{A,B}(g) \coloneqq \varepsilon_{B}(Fg) \colon FA \to B
			\]
			We have that $\beta_{GB,B}(1_{GB}) = \varepsilon_{B}$. Thus, if we can show that $\beta = \alpha^{-1}$ and these are natural we can conclude that these form an adjunction with unit and counit $\eta$ and $\varepsilon$ respectively.
			
			But
			\begin{align*}
				\beta_{A,B}\alpha_{A,B}(f) &= \beta_{A,B}((Gf)\eta_A)\\
				&= \varepsilon_B(F((Gf)\eta_A))\\
				&= (\varepsilon_B(FGf))F\eta_A\\
				&= f\varepsilon_{FA}F\eta_A\\
				&=f
			\end{align*}
			where we have used naturality of $\varepsilon$
			% https://q.uiver.app/#q=WzAsNCxbMCwwLCJGR0ZBIl0sWzEsMCwiRkdCIl0sWzAsMSwiRkEiXSxbMSwxLCJCIl0sWzIsMywiZiIsMl0sWzAsMiwiXFx2YXJlcHNpbG9uX3tGQX0iLDJdLFsxLDMsIlxcdmFyZXBzaWxvbl9CIl0sWzAsMSwiRkdmIl1d
			\[\begin{tikzcd}
				FGFA & FGB \\
				FA & B
				\arrow["f"', from=2-1, to=2-2]
				\arrow["{\varepsilon_{FA}}"', from=1-1, to=2-1]
				\arrow["{\varepsilon_B}", from=1-2, to=2-2]
				\arrow["FGf", from=1-1, to=1-2]
			\end{tikzcd}\]
			and the first triangular identity. Dually, we have $\alpha_{A,B}\beta_{A,B}(g) = g$. Hence $\beta = \alpha^{-1}$. It remains to verify that $\alpha$ is natural. Let $f\colon A' \to A$ be a morphism in $\mathcal{C}$ and $g\colon B \to B'$ a morphism in $\mathcal{D}$. We need to check that the following commutes.
			\[\begin{tikzcd}[column sep=large,row sep=2.25em]
				{\mathcal{D}(FA,B)} & {\mathcal{D}(FA',B')} \\
				{\mathcal{C}(A,GB)} & {\mathcal{C}(A',GB')}
				\arrow["{(Ff)_{\ast}g^\ast}", from=1-1, to=1-2]
				\arrow["{f_\ast(Gg)^\ast}", from=2-1, to=2-2]
				\arrow["{\alpha_{A,B}}"', from=1-1, to=2-1]
				\arrow["{\alpha_{A',B'}}", from=1-2, to=2-2]
			\end{tikzcd}\]
			If $\varphi\colon FA \to B$ then following the diagram both ways we get the composites
			% https://q.uiver.app/#q=WzAsNSxbMCwwLCJBJyJdLFsxLDAsIkEiXSxbMiwwLCJHRkEiXSxbMywwLCJHQiJdLFs0LDAsIkdCJyJdLFswLDEsImYiXSxbMSwyLCJcXGV0YV9BIl0sWzIsMywiR1xcdmFycGhpIl0sWzMsNCwiR2ciXV0=
			\[\begin{tikzcd}
				{A'} & A & GFA & GB & {GB'}
				\arrow["f", from=1-1, to=1-2]
				\arrow["{\eta_A}", from=1-2, to=1-3]
				\arrow["G\varphi", from=1-3, to=1-4]
				\arrow["Gg", from=1-4, to=1-5]
			\end{tikzcd}\]
			and
			% https://q.uiver.app/#q=WzAsNSxbMCwwLCJBJyJdLFsxLDAsIkdGQSciXSxbMiwwLCJHRkEiXSxbMywwLCJHQiJdLFs0LDAsIkdCJyJdLFswLDEsIlxcZXRhX3tBJ30iXSxbMSwyLCJHRmYiXSxbMiwzLCJHXFx2YXJwaGkiXSxbMyw0LCJHZyJdXQ==
			\[\begin{tikzcd}
				{A'} & {GFA'} & GFA & GB & {GB'}
				\arrow["{\eta_{A'}}", from=1-1, to=1-2]
				\arrow["GFf", from=1-2, to=1-3]
				\arrow["G\varphi", from=1-3, to=1-4]
				\arrow["Gg", from=1-4, to=1-5]
			\end{tikzcd}\]
			which only differ in the first two arrows. But these are equal by naturality of $\eta$ so we are done.
		\end{proof}
		We note that an equivalence of categories does not always satisfy the triangular identities. However, the following is true.
		\begin{lem}
			Suppose $(F,G,\alpha,\beta)$ is an equivalence of the categories $\mathcal{C}$ and $\mathcal{D}$. Then there are natural isomorphisms $\alpha'\colon 1_{\mathcal{C}}\to GF$ and $\beta'\colon FG\to 1_{\mathcal{D}}$ which satisfy the triangular identities. In particular, $F$ is both left and right adjoint to $G$ (and viceversa).
		\end{lem}
		\begin{proof}
			Take $\alpha' \coloneqq \alpha$ and $\beta'$ to be the composite
			% https://q.uiver.app/#q=WzAsNCxbMCwwLCJGRyJdLFsxLDAsIkZHRkciXSxbMiwwLCJGRyJdLFszLDAsIjFfe1xcbWF0aGNhbHtEfX0iXSxbMCwxLCIoRkdcXGJldGEpXnstMX0iXSxbMSwyLCIoRlxcYWxwaGEgRyleey0xfSJdLFsyLDMsIlxcYmV0YSJdXQ==
			\[\begin{tikzcd}
				FG & FGFG & FG & {1_{\mathcal{D}}}
				\arrow["{(FG\beta)^{-1}}", from=1-1, to=1-2]
				\arrow["{(F\alpha G)^{-1}}", from=1-2, to=1-3]
				\arrow["\beta", from=1-3, to=1-4]
			\end{tikzcd}.\]
			
		\end{proof}
\end{document} 