	\section{Appendix: Proof of Lemma \ref{lem:emb_iff_uha}}
We assume some basic familiarity with first-order logic and model theory, in particular with the compactness theorem and the method of diagrams (the Part III Model Theory course will suffice).
\subsection{Model-theoretic preliminaries}

We work in first-order logic with equality. Let $\mathcal{L}$ be a language. If $\mathcal{T}$ is any $\mathcal{L}$-theory then we write $\mathcal{T}_{\forall}$ for the set of universal sentences implied by $\mathcal{T}$. If $M$ is an $\mathcal{L}$-structure then $\mathcal{L}_M$ denotes the language obtained by adding $|M|$ constants to the signature of $\mathcal{L}$.

Recall that for any $\mathcal{L}$-structure $M$ we can define the diagram of $M$, written $\Diag(M)$, to be an $\mathcal{L}_M$ theory: the set of all quantifier-free $\mathcal{L}$-sentences with parameters in $M$ that are satisfied by $M$. Furthermore, models of $\Diag(M)$ correspond to extensions (i.e. superstructures) of $M$.

We say a theory $\mathcal{S}$ \emph{axiomatizes} a class $C$ if	the class of models of $\mathcal{S}$ is equal to $C$. 
\begin{lem}\label{lem:t_univ_ax_subs}
	Let $\mathcal{T}$ be an $\mathcal{L}$-theory. The class of substructures of models of $\mathcal{T}$ is axiomatized by $\mathcal{T}_{\forall}$.
\end{lem}
\begin{proof}
	Since universal sentences are preserved under substructures, it is clear that a substructure of a model of $\mathcal{T}$ is a model of $\mathcal{T}_{\forall}$. Conversely, suppose $M$ is a model of $\mathcal{T}_{\forall}$. We will prove that $\mathcal{T}\cup\Diag(M)$ is a consistent theory. If this is true then this theory has a model, which would be a model of $\mathcal{T}$ which is also an extension of $M$, giving the result.
	
	Assume, for the sake of contradiction, that $\mathcal{T}\cup\Diag(M)$ is inconsistent. By the compactness theorem, there is some finite subset of $\mathcal{T}\cup \Diag(M)$ that is inconsistent. As $\mathcal{T}$ is consistent, there must be a finite subset of $\Diag(M)$ that is inconsistent with $\mathcal{T}$.
	
	Take the conjunction of all of these sentences and call it $\varphi(\bar{m})$. Then $\mathcal{T}\cup\{\varphi(\bar{m})\}$ is inconsistent, so $\mathcal{T}\vdash \neg \varphi(\bar{m})$ by the deduction theorem. But $\mathcal{T}$ is an $\mathcal{L}$-theory, so it contains none of the constants $\bar{m}$. Hence, by generalization, $\mathcal{T} \vdash (\forall x)\neg \varphi(\bar{x})$. However $M$ is a model of $\mathcal{T}_{\forall}$ so $M \models (\forall x)\neg \varphi(\bar{x})$, and thus $M \models \neg \varphi(\bar{m})$, a contradiction.
\end{proof}

This lemma is not entirely satisfactory because, a priori, we know very little about the structure of $\mathcal{T}_{\forall}$. To improve our results to universal Horn axioms, we restrict to classes which are closed under finite direct products. Recall that for a collection of structures $(M_i)_{i\in I}$ we can consider their Cartesian product $\prod_{i\in I} M_i$ as an structure in such a way that the following proposition holds.
\begin{prop}\label{prop:prod_str_atom}
	Suppose $\phi(x_1,\ldots,x_n)$ is an atomic formula and consider some  $\bar{\alpha} = (\alpha_1,\ldots,\alpha_n)\in \left(\prod_{i\in I} M_i\right)^n$ where for each $j$
	\[
	\alpha_j \coloneqq \left(m^j_i\right)_{i\in I} \in \prod_{i\in I} M_i.
	\]
	Then $\prod_{i\in I} M_i \models \phi(\bar{\alpha})$ if and only if for all $i\in I$ we have $M_i \models \phi(m_i^1, \ldots,m_i^n)$.
\end{prop}
\begin{rem}
	In essence, Proposition \ref{prop:prod_str_atom} says that an atomic formulae is true in a product structure if and only if it is true ``componentwise''. We note that this is not in general the case for non-atomic formulae.
	
	It follows from this proposition that $\prod_{i\in I} M_i \models \neg\phi(\bar{\alpha})$ iff  there is at least one $i\in I$ with $M_i \models \neg\phi(m_i^1, \ldots,m_i^n)$. We will use this fact in what follows.
	
	We additionally remark that we do not consider the empty product (when $I=\emptyset$) to be defined; we will deal with this (surprisingly important) case separately later on.
\end{rem}
For the rest of this subsection we largely follow \cite{cohn1981universal}. We want to find a set of sentences in $\mathcal{T}_{\forall}$ that imply all sentences in $\mathcal{T}_{\forall}$; such a set of sentences is said to \emph{generate} $\mathcal{T}_{\forall}$. Consider a universal sentence $\forall \bar{x}. \varphi(\bar{x})$ in $\mathcal{T}_{\forall}$. By putting $\varphi(\bar{x})$ in conjunctive normal form and distributing the universal quantifier over the conjunction, we see that $\forall \bar{x}. \varphi(\bar{x})$ is implied by sentences of the form
\begin{equation}\label{eq:conj_uni}
	\forall \bar{x}. \left(\bigvee_{j=1}^k \psi_j(\bar{x})\right) 
\end{equation}
where all the $\psi_j$ are literals (i.e. an atomic formula or the negation of an atomic formula). A sentence of the form (\ref{eq:conj_uni}) is said to be a \emph{universally disjunctive} sentence, and the $\psi_j$ are called the \emph{components} of such a sentence. If $\psi_j$ is an atomic formula we say it is a \emph{positive} component, while if it is the negation of an atomic formula we say it is a \emph{negative} component. We have just seen that $\mathcal{T}_{\forall}$ is generated by universally disjunctive sentences.

Let $\Phi\in \mathcal{T}_{\forall}$ be a universally disjunctive sentence and for $i$ with $1\leq i \leq k$ define
\[
\Phi^i \coloneqq \forall \bar{x}.\, \left(\,\bigvee_{\mathclap{\substack{j=1;\\
			j\neq i}}}^k \psi_j(\bar{x})\right) 
\]
Say that $\Phi$ is \emph{irreducible} (in $\mathcal{T}_{\forall}$) if $k = 1$ or if for all $i$ with $1\leq i \leq k$ there is a $\mathcal{T}_{\forall}$-model $M_i$ such that $M_i \nvDash\Phi^i$; otherwise we say $\Phi$ is \emph{reducible}. 

Note that if $\Phi$ is reducible in $\mathcal{T}_{\forall}$ then it has more than one component and there is some $i$ such that all models $M$ of $\mathcal{T}_{\forall}$ model $\Phi^i$; in other words $\mathcal{T}_{\forall} \models \Phi^i$, from which it follows that $\Phi^i \in \mathcal{T}_{\forall}$. Also, it is clear from the definition that $\Phi^i \models \Phi$ for any $i$, regardless of whether $\Phi$ is reducible or not. Hence $\Phi$ is implied by $\Phi^i$, a universally disjunctive sentence in $\mathcal{T}_{\forall}$ with fewer components. By repeating this argument if necessary, we see that $\Phi$ is implied by an irreducible universally disjunctive sentence in $\mathcal{T}_{\forall}$. We have just shown the following
\begin{prop}
	Let $\mathcal{T}$ be a theory. Then $\mathcal{T}_{\forall}$ is generated by irreducible universally disjunctive sentences.
\end{prop}
Now we are ready to prove the result from which Lemma \ref{lem:emb_iff_uha} will immediately follow.
\begin{thm}\label{thm:th_univ_ax_horn}
	Let $\mathcal{T}$ be a consistent theory and suppose that the class of $\mathcal{T}_{\forall}$-models is closed under finite products. Then the irreducible universally disjunctive sentences of $\mathcal{T}_{\forall}$ have at most one positive component.
\end{thm}
\begin{proof}
	Let 
	\[
	\Phi = \forall \bar{x}. \left(\bigvee_{j=1}^k \psi_j(\bar{x})\right) 
	\]
	be an irreducible universally disjunctive sentence in $\mathcal{T}_{\forall}$. If $k= 1$ the claim is trivial so assume $k>1$. By definition of irreducible, there are, for all $i$ with $1\leq i \leq k$, models $M_i$ of $\mathcal{T}_{\forall}$ with $M_i \nvDash \Phi^i$.
	
	For the sake of contradiction, assume $\Phi$ has at least two positive components. Without loss of generality, suppose there is some $k'$ with $2\leq k' \leq k$ such that $\psi_i$ is a positive component for $i$ with $1\leq i\leq k'$, and a negative component with $i$ in the range $k'< i \leq k$. For all $i$ we have that $M_i\models \Phi$ but $M_i\nvDash \Phi^i$; in other words
	\[
	M_i \models \forall \bar{x}. \left(\bigvee_{j=1}^k \psi_j(\bar{x})\right) 
	\]
	and
	\[
	M_i \models \exists \bar{x}.\, \left(\,\bigwedge_{\mathclap{\substack{j=1;\\
				j\neq i}}}^k \neg\psi_j(\bar{x})\right) 
	\]
	It follows that 
	\begin{equation}\label{eq:pr_pos_comp_mi}
		M_i \models \exists \bar{x}.\left(\psi_i(\bar{x}) \wedge  \bigwedge_{\mathclap{\substack{j=1;\\
					j\neq i}}}^k \neg\psi_j(\bar{x})\right).
	\end{equation}
	If we have $\bar{x} = (x_1,\ldots,x_n)$ then for each $i$ with $1\leq i \leq k'$ we choose a witness $\bar{m}_i \coloneqq (m^1_i,\ldots,m^n_i)\in (M_i)^n$ to (\ref{eq:pr_pos_comp_mi}). Consider the product $M\coloneqq \prod_{i=1}^{k'} M_i$. By assumption, $M\models \mathcal{T}_{\forall}$. We define for $1\leq j \leq n$
	\[
	\alpha_j \coloneqq (m^j_1,m^j_2\ldots,m^j_{k'}) \in M
	\]
	and we let $\bar{\alpha}\coloneqq (\alpha_1,\ldots,\alpha_n)\in M^n$.
	Now we make the following observations.
	\begin{itemize}
		\item Let $l$ be such that $1\leq l \leq k'$. So, $\psi_l(\bar{x})$ is an atomic formula. Let $l'$ be any number satisfying $1\leq l'\leq k'$ and $l\neq l'$ (here we use the assumption that $k'\geq 2$). Then it follows from (\ref{eq:pr_pos_comp_mi}) that $M_{l'} \models \neg\psi_{l}(\bar{m}_{l'})$. Hence, by the remark following Proposition \ref{prop:prod_str_atom}, we have $M \models \neg \psi_l(\bar{\alpha})$.
		\item Let $l$ be such that $k'< l\leq k$. So, $\neg\psi_l(\bar{x})$ is (equivalent to) an atomic formula. Then it follows from (\ref{eq:pr_pos_comp_mi}) that for all $i$ with $1\leq i\leq k'$ we have that $M_i\models \neg\psi_{l}(\bar{m}_{i})$. Thus, by Proposition \ref{prop:prod_str_atom}, we have that $M \models \neg \psi_l(\bar{\alpha})$.
	\end{itemize}
	So we see that $\bar{\alpha}$ is a witness in $M$ of the negation of all $\psi_j$'s. Hence
	\[
	M \models \exists \bar{x}. \left(\bigwedge_{j=1}^k \neg\psi_j(\bar{x})\right).
	\]
	In other words, $M \models \neg \Phi$, contradicting the fact that $M\models \mathcal{T}_{\forall}$ and that $\Phi\in \mathcal{T}_{\forall}$ (here we use the assumption that $\mathcal{T}$ is consistent).
\end{proof}
We will use the above theorem only in the special case where the class of $\mathcal{T}_{\forall}$-models includes a singleton structure---playing the role of the empty product---which satisfies all existential closures of atomic formulae. 
\begin{coro}\label{coro:ax_horn_ax_gen}
	Let $\mathcal{T}$ be a consistent theory and let $C$ be the class of substructures of models of $\mathcal{T}$.  Suppose that $C$ is closed under finite products. Further suppose that there is some singleton structure $S$ in $C$ with the property that for all atomic formulae $\varphi(\bar{x})$ in the language we have
	\[
	S \models \exists \bar{x}. \,\varphi(\bar{x})
	\]
	Then $C$ is axiomatized by formulae of the form
	\begin{equation}\label{eq:horn_form_gen}
		\forall\bar{x}. \left(\bigwedge_{j=1}^k\psi_j(\bar{x}) \Rightarrow \psi(\bar{x})\right) 
	\end{equation}
	where $\psi$ and all the $\psi_i$'s are atomic formulae.
\end{coro}
\begin{rem}
	As for universal Horn axioms, we allow for $k=0$, in which case (\ref{eq:horn_form_gen}) reduces to the form $\forall \bar{x}. \,\psi(\bar{x})$ for $\psi$ atomic.
\end{rem}
\begin{proof}
	By Lemma \ref{lem:t_univ_ax_subs}, we have that $C$ is just the class of all $\mathcal{T}_{\forall}$-models, so it is enough to show that $\mathcal{T}_{\forall}$ is generated by formulae of form (\ref{eq:horn_form_gen}). By Theorem \ref{thm:th_univ_ax_horn}, $\mathcal{T}_{\forall}$ is generated by irreducible universally disjunctive sentences with at most one positive component. But note that these types of sentences cannot, in this case, have zero positive components, since by assumption 
	\[
	S \nvDash \forall \bar{x}. \left(\bigvee_{j=1}^k \neg\psi_j(\bar{x})\right).
	\]
	for any atomic $\psi_1,\ldots,\psi_j$ (note that it is crucial that $S$ is a singleton). Hence $\mathcal{T}_{\forall}$ is generated by universally disjunctive sentences with exactly one positive component, which are easily seen to be equivalent to formulae of the form (\ref{eq:horn_form_gen}).
\end{proof}
\subsection{Proof of the characterization and further examples}
Now we check the conditions of Corollary \ref{coro:ax_horn_ax_gen} in a few different theories of interest.
\begin{eg}
	Let $\mathcal{L}$ be the language of monoids, the signature being a single binary operation, denoted by concatenation. Let $\mathcal{T}$ be the $\mathcal{L}$-theory of groups. Obviously $\mathcal{T}$ is consistent. If $M_1,\ldots, M_n$ are monoids embedding in the groups $G_1,\ldots,G_n$ respectively, then $M_1\times\cdots\times M_n$ embeds in the group $G_1\times\cdots \times G_n$, as one can easily verify. Hence the class of embeddable monoids is closed under finite products. 
	
	Furthermore, the trivial monoid $\{\ast\}$ is embeddable (in fact, it is a group). Atomic formulae in this language are just equalities between products of variables, and in the trivial monoid these equalities are always true when the variables are all set to $\ast$. In other words $\{\ast\}\models \exists \bar{x}. E$ for all equations $E$ in $\mathcal{L}$. So, Corollary \ref{coro:ax_horn_ax_gen} applies and gives us the desired characterization, Lemma \ref{lem:emb_iff_uha} for groups.
\end{eg}
\begin{eg}
	In a similar manner, we can change $\mathcal{T}$ to be the theory of abelian groups, which is consistent. The previous argument shows that the class of monoids that embed into abelian groups is closed under finite products, and the trivial monoid belongs to this class (it being an abelian group). Again the conditions of Corollary \ref{coro:ax_horn_ax_gen} are satisfied. The same argument works for Boolean groups (since the trivial monoid is a Boolean group), so that we get Lemma \ref{lem:emb_iff_uha} for abelian and Boolean groups.
\end{eg}
\begin{eg}
	With some work, we can also apply Corollary \ref{coro:ax_horn_ax_gen} to characterize categories which embed into groupoids, a direction explored by Johnstone in \cite{johnstone2008embedding}. We only give a sketch on how this is done. Firstly, we need to define categories as a first-order theory. This is done formally in \cite[Section 3]{nlab:fully_formal_etcs} and we follow the construction therein. We define a language $\mathcal{L}$ with signature two unary function symbols $s,t$ and one ternary predicate symbol $c$. We intend for the models to consists of morphisms of a category, rather than having objects and morphisms. 
	
	So, the intended interpretation of $c(f,g,h)$ is $h = f\circ g$. Also $s(f)$ is supposed to be the identity morphism of the domain of $f$; similarly $t(f)$ is the identity morphism of the codomain of $f$. Having this in mind, it is not hard to write down an $\mathcal{L}$-theory $\mathcal{T}$ such that $\mathcal{T}$-models are simply categories (or rather the morphisms of a category, which amounts to the same thing).
	
	The class of categories which embed into groupoids is closed under finite products, and the trivial category (one object and one morphism) is a groupoid, so it is not hard to see that the conditions of Corollary \ref{coro:ax_horn_ax_gen} are satisfied. The characterization is somewhat harder to state precisely since atomic formulae are not only equations between products, but, as $c(f,g,h)$ is really the equation $h = f\circ g$, it basically boils down to universal Horn axioms, as in the previous cases.
\end{eg}