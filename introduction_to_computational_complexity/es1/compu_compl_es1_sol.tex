\documentclass{article}
%Setting margins
%\usepackage[margin = 1.75in]{geometry}
%Basic Maths
\usepackage{amsmath}
\usepackage{amssymb}
\usepackage{mathtools}
\usepackage{gensymb}
%For definining theorem-like environments
\usepackage{amsthm}
%For beautiful letters (e.g. for a partition, see $\mathscr{P}$)
\usepackage{mathrsfs}
%For importing the solution file
%\usepackage{import}
%For drawing commutative diagrams
\usepackage{quiver}
%For pretty colours
\usepackage{xcolor}
%For scaling some relations, for instance see https://tex.stackexchange.com/a/108482
\usepackage{mleftright}
%Set paragraph spacing. I believe this is close to what is used in the book.
%\usepackage[skip=.3\baselineskip, indent = 15pt]{parskip}
%To customize lists
\usepackage{enumitem}
%To strikethrough terms in equations
\usepackage{cancel}
%For bibliography
\usepackage[backend=biber]{biblatex}
%For pictures
\usepackage{tikz}
\usetikzlibrary{calc,positioning}

\usepackage[hidelinks]{hyperref}
\usepackage{soul}
\usepackage[normalem]{ulem}
\usepackage{lipsum}
\usepackage{breqn}

%\addbibresource{main.bib}


\newcommand{\myhy}[2]{\href{#1}{\color{blue}\setulcolor{blue}\ul{#2}}}

%Fix section numbering to match the book's convention
\renewcommand\thesection{\arabic{section}}

%Displays "Exercises". To put after each section.
\newcommand{\extitle}{\subsection*{Exercises}}

%For personal notes
\newcommand{\note}[1]
{\smallskip {\noindent\textbf{Note} #1}}

%Roman numerals!
\newcommand{\RNo}[1]{%
	\textup{\uppercase\expandafter{\romannumeral#1}}%
}

%San-serif for names of categories
\newcommand{\serif}[1]{{\fontfamily{cmss}\selectfont #1}}
\newcommand{\srf}{\textsf}

%Shorthands for common sets
\newcommand{\N}{\mathbb{N}}
\newcommand{\Z}{\mathbb{Z}}
\newcommand{\Q}{\mathbb{Q}}
\newcommand{\R}{\mathbb{R}}
\newcommand{\C}{\mathbb{C}}
\newcommand{\zmod}[1]{\bZ/#1\bZ}

%Miscellaneous commands
\newcommand{\defeq}{\coloneqq}
\newcommand{\divides}{\mid}
\newcommand{\legendre}[2]{\ensuremath{\left( \frac{#1}{#2} \right) }}
\newcommand{\Mod}[1]{\ (\mathrm{mod}\ #1)}
\newcommand{\mbold}[1]{\mathrm{\mathbf{#1}}}


%Useful operations and delimiters
\DeclareMathOperator{\Hom}{Hom}
\DeclareMathOperator{\End}{End}
\DeclareMathOperator{\Aut}{Aut}
\DeclareMathOperator{\Obj}{Obj}
\DeclareMathOperator{\id}{id}
\DeclareMathOperator{\lcm}{lcm}
\DeclareMathOperator{\GL}{GL}
\DeclareMathOperator{\SO}{SO}
\DeclareMathOperator{\SL}{SL}
\DeclareMathOperator{\U}{U}
\DeclareMathOperator{\SU}{SU}
\DeclareMathOperator{\Inn}{Inn}
\DeclareMathOperator{\PSL}{PSL}
\DeclareMathOperator{\im}{im}
\DeclareMathOperator{\coker}{coker}
\DeclareMathOperator{\rot}{rot}
\DeclareMathOperator{\rf}{ref}
\DeclareMathOperator{\Symm}{Symm}
\DeclareMathOperator{\vspan}{span}
\DeclareMathOperator{\ev}{ev}
\DeclareMathOperator{\Gal}{Gal}
\DeclareMathOperator{\ob}{ob}
\DeclareMathOperator{\mor}{mor}
\DeclareMathOperator{\dom}{dom}
\DeclareMathOperator{\cod}{cod}
\DeclareMathOperator{\Cone}{Cone}
\DeclarePairedDelimiter\abs{\lvert}{\rvert}%
\DeclarePairedDelimiter\norm{\lVert}{\rVert}%
\DeclarePairedDelimiter\innprod{\langle}{\rangle}%
\DeclarePairedDelimiter\ceil{\lceil}{\rceil}
\DeclarePairedDelimiter\floor{\lfloor}{\rfloor}
%Claim environment
\newtheorem{claim}{Claim}


%Exercise environment
\theoremstyle{definition}
\newtheorem{ex}{Exercise}

%Standard theorem-like environment
\theoremstyle{plain}
\newtheorem{thm}{Theorem}[section]

\newtheorem{prop}[thm]{Proposition}
\newtheorem{lem}[thm]{Lemma}
\newtheorem{coro}[thm]{Corollary}
\newtheorem{prob}{Problem}
\newtheorem{conj}{Conjecture}


\theoremstyle{definition}
\newtheorem{defn}[thm]{Definition}
\newtheorem{rem}[thm]{Remark}
\newtheorem{eg}[thm]{Example}
\newtheorem{egs}[thm]{Examples}
\newtheorem{fact}[thm]{Fact}
\newtheorem{task}{Task}



%Solution environment
\newenvironment{solution}
{\begin{proof}[Solution]}
	{\end{proof}}

%Function restrictions
% From https://tex.stackexchange.com/a/22255
\newcommand\restr[2]{{% we make the whole thing an ordinary symbol
		\left.\kern-\nulldelimiterspace % automatically resize the bar with \right
		#1 % the function
		\vphantom{\big|} % pretend it's a little taller at normal size
		\right|_{#2} % this is the delimiter
}}

%\newcommand\nvdash{\mkern-2mu\not\mkern2mu\vdash}

\makeatother
\setlist{parsep=0pt,listparindent=\parindent}
\begin{document}
\title{Introduction to Computational Complexity\\ ES1 Solutions}
\date{}
\maketitle
\begin{enumerate}
	\item First suppose $f$ is in NP and let $T$ be a nondeterministic Turing machine computing $f$ in polynomial time. We can construct a deterministic Turing machine $T'$ such that it takes input $x$ and $y$ and outputs what $T$ would've outputted with $x$ as an input, with choices of transition functions encoded in $y$ (we don't need two input tapes for this since we can, for example, agree that even positions are supposed to be $x$ and odd positions are $y$). Let $g$ be the function computed by $T'$. Then it is not hard to see that $g\in P$ and we can pick $y$ so that $g(x,y) = 1$ iff $f(x) = 1$ subject to the conditions in the size of $y$.
	
	On the other direction, suppose we are given $g$ and $p$. Let $T'$ be the Turing machine computing $g$ in polynomial time. We can reverse the above process by constructing a nondeterministic Turing machine that, given $x$, tries to write down the corresponding $y$ randomly and then computing $g(x,y)$. As $|y| = p(|x|)$ and $g\in \text{P}$ we see that this new Turing machine computes $f$ in polynomial time.
	
	\item Just because we can solve the decision problem in polynomial time, it doesn't mean we can solve the corresponding search problem in polynomial time; though this would follow from P=NP as was shown in lectures.
	\item We need to check that the problem ``Is $n$ prime?'' is in NP. We can assume $n>2$ because we can check the case $n=2$ separately. Let $g(n,a, p_1,\ldots,p_k)$ be the Boolean function defined to be 1 if and only if
	\begin{enumerate}[label=(\arabic*)]
		\item $a^{n-1} \equiv 1 \pmod{n}$; and
		\item $p_1,\ldots, p_k$ are the prime factors of $n-1$; and 
		\item $a^{\frac{n-1}{p_i}} \not\equiv 1 \pmod{n}$ for all $i$.
	\end{enumerate}
	First we show that $n$ is prime iff there is $a,p_1,\ldots,p_k$ such that $g(n,a, p_1,\ldots,p_k) = 1$. If $n$ is prime then take $a$ to a generator of the cyclic group $(\mathbb{Z}/n\mathbb{Z})^*$ and the $p_i$'s to be the prime factors of $n-1$. Then (1) is satisfied by Fermat's little theorem, (2) is obviously satisfied, and (3) is satisfied since the order of $a$ is $n-1$.
	
	Conversely, suppose we have found $a,p_1,\ldots,p_k$ such that $g(n,a, p_1,\ldots,p_k) = 1$. As $n>2$, we see that $aa^{n-2}\equiv 1 \pmod{n}$ so $a$ has a multiplicative inverse and thus $a\in (\mathbb{Z}/n\mathbb{Z})^*$, i.e., $a$ is coprime to $n$. In particular, the order of $a$ divides the order of the group, which is $n-1$. So, if $|a|<n-1$, we see that $|a|$ divides $\frac{n-1}{p_i}$ for some $i$, contradicting (3); so by the hint we see that $n$ is prime. \textcolor{red}{Missing}
	\item 
	\item Suppose $f$
	
	
\end{enumerate}
\end{document}