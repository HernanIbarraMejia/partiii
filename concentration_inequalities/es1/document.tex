\documentclass{article}
%Setting margins
%\usepackage[margin = 1.75in]{geometry}
%Basic Maths
\usepackage{amsmath}
\usepackage{amssymb}
\usepackage{mathtools}
\usepackage{gensymb}
%For definining theorem-like environments
\usepackage{amsthm}
%For beautiful letters (e.g. for a partition, see $\mathscr{P}$)
\usepackage{mathrsfs}
%For importing the solution file
%\usepackage{import}
%For drawing commutative diagrams
\usepackage{quiver}
%For pretty colours
\usepackage{xcolor}
%For scaling some relations, for instance see https://tex.stackexchange.com/a/108482
\usepackage{mleftright}
%Set paragraph spacing. I believe this is close to what is used in the book.
%\usepackage[skip=.3\baselineskip, indent = 15pt]{parskip}
%To customize lists
\usepackage{enumitem}
%To strikethrough terms in equations
\usepackage{cancel}
%For bibliography
\usepackage[backend=biber]{biblatex}
%For pictures
\usepackage{tikz}
\usetikzlibrary{calc,positioning}

\usepackage[hidelinks]{hyperref}
\usepackage{soul}
\usepackage[normalem]{ulem}
\usepackage{lipsum}
\usepackage{breqn}

%\addbibresource{main.bib}


\newcommand{\myhy}[2]{\href{#1}{\color{blue}\setulcolor{blue}\ul{#2}}}

%Fix section numbering to match the book's convention
\renewcommand\thesection{\arabic{section}}

%Displays "Exercises". To put after each section.
\newcommand{\extitle}{\subsection*{Exercises}}

%For personal notes
\newcommand{\note}[1]
{\smallskip {\noindent\textbf{Note} #1}}

%Roman numerals!
\newcommand{\RNo}[1]{%
	\textup{\uppercase\expandafter{\romannumeral#1}}%
}

%San-serif for names of categories
\newcommand{\serif}[1]{{\fontfamily{cmss}\selectfont #1}}
\newcommand{\srf}{\textsf}

%Shorthands for common sets
\newcommand{\N}{\mathbb{N}}
\newcommand{\Z}{\mathbb{Z}}
\newcommand{\Q}{\mathbb{Q}}
\newcommand{\R}{\mathbb{R}}
\newcommand{\C}{\mathbb{C}}
\newcommand{\zmod}[1]{\bZ/#1\bZ}

%Miscellaneous commands
\newcommand{\defeq}{\coloneqq}
\newcommand{\divides}{\mid}
\newcommand{\legendre}[2]{\ensuremath{\left( \frac{#1}{#2} \right) }}
\newcommand{\Mod}[1]{\ (\mathrm{mod}\ #1)}
\newcommand{\mbold}[1]{\mathrm{\mathbf{#1}}}


%Useful operations and delimiters
\DeclareMathOperator{\Hom}{Hom}
\DeclareMathOperator{\End}{End}
\DeclareMathOperator{\Aut}{Aut}
\DeclareMathOperator{\Obj}{Obj}
\DeclareMathOperator{\id}{id}
\DeclareMathOperator{\lcm}{lcm}
\DeclareMathOperator{\GL}{GL}
\DeclareMathOperator{\SO}{SO}
\DeclareMathOperator{\SL}{SL}
\DeclareMathOperator{\U}{U}
\DeclareMathOperator{\SU}{SU}
\DeclareMathOperator{\Inn}{Inn}
\DeclareMathOperator{\PSL}{PSL}
\DeclareMathOperator{\im}{im}
\DeclareMathOperator{\coker}{coker}
\DeclareMathOperator{\rot}{rot}
\DeclareMathOperator{\rf}{ref}
\DeclareMathOperator{\Symm}{Symm}
\DeclareMathOperator{\vspan}{span}
\DeclareMathOperator{\ev}{ev}
\DeclareMathOperator{\Gal}{Gal}
\DeclareMathOperator{\ob}{ob}
\DeclareMathOperator{\mor}{mor}
\DeclareMathOperator{\dom}{dom}
\DeclareMathOperator{\cod}{cod}
\DeclareMathOperator{\Cone}{Cone}
\DeclarePairedDelimiter\abs{\lvert}{\rvert}%
\DeclarePairedDelimiter\norm{\lVert}{\rVert}%
\DeclarePairedDelimiter\innprod{\langle}{\rangle}%
\DeclarePairedDelimiter\ceil{\lceil}{\rceil}
\DeclarePairedDelimiter\floor{\lfloor}{\rfloor}
%Claim environment
\newtheorem{claim}{Claim}


%Exercise environment
\theoremstyle{definition}
\newtheorem{ex}{Exercise}

%Standard theorem-like environment
\theoremstyle{plain}
\newtheorem{thm}{Theorem}[section]

\newtheorem{prop}[thm]{Proposition}
\newtheorem{lem}[thm]{Lemma}
\newtheorem{coro}[thm]{Corollary}
\newtheorem{prob}{Problem}
\newtheorem{conj}{Conjecture}


\theoremstyle{definition}
\newtheorem{defn}[thm]{Definition}
\newtheorem{rem}[thm]{Remark}
\newtheorem{eg}[thm]{Example}
\newtheorem{egs}[thm]{Examples}
\newtheorem{fact}[thm]{Fact}
\newtheorem{task}{Task}



%Solution environment
\newenvironment{solution}
{\begin{proof}[Solution]}
	{\end{proof}}

%Function restrictions
% From https://tex.stackexchange.com/a/22255
\newcommand\restr[2]{{% we make the whole thing an ordinary symbol
		\left.\kern-\nulldelimiterspace % automatically resize the bar with \right
		#1 % the function
		\vphantom{\big|} % pretend it's a little taller at normal size
		\right|_{#2} % this is the delimiter
}}

%\newcommand\nvdash{\mkern-2mu\not\mkern2mu\vdash}

\makeatother
\setlist{parsep=0pt,listparindent=\parindent}
\begin{document}
	\title{Concentration Inequalities\\ ES1 Solutions}
	\date{}
	\maketitle
	\begin{enumerate}[wide, labelwidth=!, labelindent=0pt]
		\item \leavevmode \begin{enumerate}
			\item Note that 
			\begin{align*}
				\mathbb{E}(e^{\lambda Y}) &= \sum_{k=0}^\infty e^{\lambda k}\mathbb{P}(Y=k)\\
				&= \sum_{k=0}^\infty e^{\lambda k}\frac{e^{-v}v^k}{k!}\\
				&= e^{-v}\sum_{k=0}^\infty \frac{(e^{\lambda} v)^k}{k!}\\
				&= \exp(e^\lambda v - v).
			\end{align*}
			Hence,
			\[
				\mathbb{E}(e^{\lambda Z}) = e^{-\lambda v}\mathbb{E}(e^{\lambda Y}) = \exp(e^{\lambda}v - v - \lambda v).
			\]
			It follows that $\psi_Z(\lambda) = ve^{\lambda} - v - \lambda v$. Hence, as $t>0 = \mathbb{E}(Z)$ we have
			\begin{align*}
				\psi_Z^*(t) &= \sup_{\lambda} \lambda t - ve^{\lambda} + v + \lambda v\\
				&= \sup_{\lambda} \lambda(t+ v) -ve^{\lambda} + v.
			\end{align*}
			Let $f(x)\coloneqq x(t+v) -ve^{x} + v$. Then $f'(x) = t+v -ve^{x}$. By setting the derivative equal to zero, we get that $x = \ln(\frac{t}{v} + 1)$ yields a maximum of $f$ (we can check this is a maximum by computing the second derivative). And
			\[
				\psi_Z^*(t) = f\left(\ln\left(\frac{t}{v} + 1\right)\right) = (t + v)\ln\left(\frac{t}{v} + 1\right) -t.
			\]
			Then the Chernoff bound yields 
			\begin{align*}
			\mathbb{P}(Z\geq t) &\leq \exp\left(-(t + v)\ln\left(\frac{t}{v} + 1\right) +t\right)\\
			 &= e^t\left(\frac{t}{v} + 1\right)^{-(t+v)}\\
			 &= e^{-v}\left(\frac{ev}{t+v}\right)^{t+v}.
			\end{align*}
			\item We clearly have $\psi_{-Z}(\lambda) = \psi_Z(-\lambda) = ve^{-\lambda} - v + \lambda v$. Hence
			\begin{align*}
				\psi^*_{-Z}(t) &= \sup_{\lambda }\lambda t -ve^{-\lambda} + v - \lambda v\\
				&=\sup_{\lambda} \lambda(t-v) - ve^{-\lambda} + v
			\end{align*}
			If $t > v$ then the $\psi^*_{-Z}(t) = +\infty$ and the Chernoff bound gives a probability of zero. Assume then that $t < v$. It then follows by taking derivatives that the supremum is achieved when $\lambda = \ln(\frac{v}{v-t})$ and so $\psi^*_{-Z}(t) = (t-v)\ln(\frac{v}{v-t}) + t$. The Chernoff bound yields
			\begin{align*}
				\mathbb{P}(Z\leq -t) &\leq \exp\left((v-t)\ln\left(\frac{v}{v-t}\right) - t\right)\\
				&= e^{-t}\left(\frac{v}{v-t}\right)^{v-t}\\
				&= e^{-v}\left(\frac{ev}{v-t}\right)^{v-t}.\\
			\end{align*}
		\end{enumerate}
		
		
	\end{enumerate}
\end{document}