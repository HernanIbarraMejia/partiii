\documentclass{report}
%Setting margins
%\usepackage[margin = 1.75in]{geometry}
%Basic Maths
\usepackage{amsmath}
\usepackage{amssymb}
\usepackage{mathtools}
\usepackage{gensymb}
%For definining theorem-like environments
\usepackage{amsthm}
%For beautiful letters (e.g. for a partition, see $\mathscr{P}$)
\usepackage{mathrsfs}
%For importing the solution file
%\usepackage{import}
%For drawing commutative diagrams
\usepackage{quiver}
%For pretty colours
\usepackage{xcolor}
%For scaling some relations, for instance see https://tex.stackexchange.com/a/108482
\usepackage{mleftright}
%Set paragraph spacing. I believe this is close to what is used in the book.
%\usepackage[skip=.3\baselineskip, indent = 15pt]{parskip}
%To customize lists
\usepackage{enumitem}
%To strikethrough terms in equations
\usepackage{cancel}
%For bibliography
\usepackage[backend=biber]{biblatex}
%For pictures
\usepackage{tikz}
\usetikzlibrary{calc,positioning}

\usepackage{hyperref}
\usepackage{soul}

\usepackage{lipsum}
\usepackage{xfrac}

%\addbibresource{main.bib}
\usepackage{ stmaryrd }

\newcommand{\myhy}[2]{\href{#1}{\color{blue}\setulcolor{blue}\ul{#2}}}

%Fix section numbering to match the book's convention
\renewcommand\thesection{\arabic{section}}

%Displays "Exercises". To put after each section.
\newcommand{\extitle}{\subsection*{Exercises}}

%For personal notes
\newcommand{\note}[1]
{\smallskip {\noindent\textbf{Note} #1}}

%Roman numerals!
\newcommand{\RNo}[1]{%
	\textup{\uppercase\expandafter{\romannumeral#1}}%
}

%San-serif for names of categories
\newcommand{\serif}[1]{{\fontfamily{cmss}\selectfont #1}}
\newcommand{\srf}{\textsf}

%Shorthands for common sets
\newcommand{\N}{\mathbb{N}}
\newcommand{\Z}{\mathbb{Z}}
\newcommand{\Q}{\mathbb{Q}}
\newcommand{\R}{\mathbb{R}}
\newcommand{\C}{\mathbb{C}}
\newcommand{\zmod}[1]{\bZ/#1\bZ}

%Miscellaneous commands
\newcommand{\defeq}{\coloneqq}
\newcommand{\divides}{\mid}
\newcommand{\legendre}[2]{\ensuremath{\left( \frac{#1}{#2} \right) }}
\newcommand{\Mod}[1]{\ (\mathrm{mod}\ #1)}


%Useful operations and delimiters
\DeclareMathOperator{\Hom}{Hom}
\DeclareMathOperator{\End}{End}
\DeclareMathOperator{\Aut}{Aut}
\DeclareMathOperator{\Obj}{Obj}
\DeclareMathOperator{\id}{id}
\DeclareMathOperator{\lcm}{lcm}
\DeclareMathOperator{\GL}{GL}
\DeclareMathOperator{\SO}{SO}
\DeclareMathOperator{\SL}{SL}
\DeclareMathOperator{\U}{U}
\DeclareMathOperator{\SU}{SU}
\DeclareMathOperator{\Inn}{Inn}
\DeclareMathOperator{\PSL}{PSL}
\DeclareMathOperator{\im}{im}
\DeclareMathOperator{\coker}{coker}
\DeclareMathOperator{\rot}{rot}
\DeclareMathOperator{\rf}{ref}
\DeclareMathOperator{\Symm}{Symm}
\DeclareMathOperator{\vspan}{span}
\DeclareMathOperator{\ev}{ev}
\DeclareMathOperator{\Gal}{Gal}
\DeclareMathOperator{\Diag}{Diag}
\DeclareMathOperator{\Th}{Th}
\DeclarePairedDelimiter\abs{\lvert}{\rvert}%
\DeclarePairedDelimiter\norm{\lVert}{\rVert}%
\DeclarePairedDelimiter\innprod{\langle}{\rangle}%
\DeclarePairedDelimiter\ceil{\lceil}{\rceil}
\DeclarePairedDelimiter\floor{\lfloor}{\rfloor}
%Claim environment
\newtheorem{claim}{Claim}


%Exercise environment
\theoremstyle{definition}
\newtheorem{ex}{Exercise}

%Standard theorem-like environment
\theoremstyle{plain}
\newtheorem{thm}{Theorem}[section]

\newtheorem{prop}[thm]{Proposition}
\newtheorem{lem}[thm]{Lemma}
\newtheorem{coro}[thm]{Corollary}
\newtheorem{prob}{Problem}
\newtheorem{conj}{Conjecture}


\theoremstyle{definition}
\newtheorem{defn}[thm]{Definition}
\newtheorem{rem}[thm]{Remark}
\newtheorem*{rem*}{Remark}
\newtheorem{eg}[thm]{Example}
\newtheorem{egs}[thm]{Examples}
\newtheorem{fact}[thm]{Fact}
\newtheorem{task}{Task}



%Solution environment
\newenvironment{solution}
{\begin{proof}[Solution]}
	{\end{proof}}

%Function restrictions
% From https://tex.stackexchange.com/a/22255
\newcommand\restr[2]{{% we make the whole thing an ordinary symbol
		\left.\kern-\nulldelimiterspace % automatically resize the bar with \right
		#1 % the function
		\vphantom{\big|} % pretend it's a little taller at normal size
		\right|_{#2} % this is the delimiter
}}


\makeatother

\setenumerate[1]{label=(\alph*)}	
\renewcommand{\thesection}{\thechapter.\arabic{section}}
\begin{document}
	\title{Model Theory and Non-Classical Logic}
	\author{Hernán Ibarra Mejia}
	\maketitle
	This is a set of lecture notes taken by me from the Part III course ``Model Theory and Non-Classical Logic'', lectured by Dr J. Siqueira in Michaelmas, 2023. I take full responsibility for any mistakes in these notes. Chapter 0 is my summary/expansion of \cite{NOLAST}
	
	
	\setcounter{chapter}{-1}
	\chapter{Logic Background (INCOMPLETE)}
	\begin{defn}[Signature]
		A \emph{signature} $\Sigma$ is a triplet $(\Omega, \Pi, \alpha)$, where $\Omega$ and $\Pi$ are disjoint sets and $\alpha\colon \Omega \cup \Pi \to \mathbb{N}$. We call the elements of $\Omega$ \emph{function symbols}, those of $\Pi$ we call \emph{predicate symbols}, and if $s\in \Omega \cup \Pi$ we call $\alpha(s)$ the \emph{arity} of $s$.
	\end{defn}
	For the rest of this chapter, assume $\Sigma = (\Omega, \Pi, \alpha)$ is an arbitrary signature and that we are given a countable set $X=\{x_1,x_2\ldots\}$, which we call the set of \emph{variables}. This set does not contain any symbols in our signature (nor in the set of strings on our signature, see below).
	\section{Terms, formulae, and structures}
	\begin{defn}[Terms]
		The set of $\Sigma$-\emph{terms} is a subset of the set of strings on $\Omega \cup X$, defined inductively as follows. 
		\begin{enumerate}
			\item If $x\in X$ then $x$ is a term
			\item If $t_1,\ldots t_n$ are terms, and $\omega\in \Omega$ with $\alpha(\omega) = n$ then $\omega (t_1,\ldots, t_n)$ is a term.
			\item That is all.
		\end{enumerate}
	\end{defn}
	\begin{rem}
		Now assume that $X\cup \Omega \cup \Pi$ do not contain the symbols `$=$', `(',`)', `$\bot$', `$\forall$' nor `$\Rightarrow$' (nor commas).
	\end{rem} 
	\begin{defn}[Atomic formulae]
		Let $T$ be the set of $\Sigma$-terms. We define the \emph{atomic formulae} of $\Sigma$ as certain strings on $T \cup \Pi \cup \{(, ), =, ,\}$ (note that the last comma is not a typo) according to the following rules.
		\begin{enumerate}
			\item If $s$ and $t$ are terms then $(s = t)$ is an atomic formula.
			\item If $\phi \in \Pi$, $\alpha(\phi) = n$ and $t_1, \ldots, t_n$ are terms then $\phi(t_1,\ldots, t_n)$ is an atomic formula.
			\item That is all.
		\end{enumerate}
	\end{defn}
	\begin{defn}[Pre-formulae]
		Let $T$ be the set of terms of $\Sigma$. We inductively define the set of $\Sigma$-\emph{pre-formulae} as a subset of the set of strings on $T \cup \Pi\cup \{=,\bot, \forall, \Rightarrow, (, )\}$ satisfying the following.
		\begin{enumerate}
			\item Atomic formulae are pre-formulae
			\item $\bot$ is a pre-formula.
			\item If $p$ and $q$ are pre-formulae then so is $(p\Rightarrow q)$.
			\item If $p$ is a pre-formula and $x\in X$ is a variable then $(\forall x)p$ is a pre-formula.
			\item That's all.
		\end{enumerate}
	\end{defn}
	
	Now we can define a function $\textnormal{PFV}$ (for pre-free variables) on the set of terms union with the set of pre-formulae by the following rules
	\begin{align*}
		&\textnormal{PFV}(x) = \{x\}\\
		&\textnormal{PFV}(\omega t_1 \cdots t_n) = \bigcup_{i = 1}^{n}\textnormal{PFV}(t_i)\\
		&\textnormal{PFV}(s=t) = \textnormal{PFV}(s) \cup \textnormal{PFV}(t)\\
		&\textnormal{PFV}(\phi(t_1, \ldots, t_n)) = \bigcup_{i = 1}^{n}\textnormal{PFV}(t_i)\\
		&\textnormal{PFV}(\bot) = \emptyset\\
		&\textnormal{PFV}(p \Rightarrow q) = \textnormal{PFV}(p) \cup \textnormal{PFV}(q)\\
		&\textnormal{PFV}((\forall x)p) = \textnormal{PFV}(p)\setminus\{x\} 
	\end{align*}
	
	Finally, we can define $\Sigma$-\emph{formulae} to be all pre-formulae of $\Sigma$ except those of the form $(\forall x)p$ where $x\notin \textnormal{PFV}(p)$. Define $\textnormal{FV}$ to be the restriction of $\textnormal{PFV}$ so that it only applies to terms and formulae.
	
	By the \emph{language} $\mathcal{L}$ of a signature $\Sigma$ we mean the set of all terms and formulae of $\Sigma$. Instead of saying $\Sigma$-terms and $\Sigma$-formulae we say $\mathcal{L}$-terms and  $\mathcal{L}$-formulae to mean the same thing.
	\begin{defn}[Language structures]
		An $\mathcal{L}$-\emph{structure} is a set $A$ together with functions $\omega_A\colon A^{\alpha(w)} \to A$ for each $\omega \in \Omega$ and relations $\phi_A\subseteq A^{\alpha(\phi)}$ for each $\phi \in \Pi$. We use the convention that $S^0$ is a singleton set (say $\{0\}$) for all sets $S$.
	\end{defn}
	\section{Derived symbols}
	\begin{defn}[Derived operations]
		Let $A$ be an $\mathcal{L}$-structure and $t$ a term. In addition, suppose $n$ is an integer with $\textnormal{FV}(t) \subseteq \{x_1,\ldots,x_n\}$. We define, $t_A(n)$ to be a function $A^{n} \to A$ as follows.
		\begin{enumerate}
			\item If $t\in X$ then $t = x_i$ for some $i\leq n$. Let $t_A\colon A^n \to A$ be the $i$-th projection function.
			\item Suppose $t = \omega t_1\ldots t_m$ where $\omega \in \Omega$ with $\alpha(\omega) = m$, and the $t_i$'s are terms for which we have defined $(t_i)_A(n)$. Then $t_A$ is the composite 
			% https://q.uiver.app/#q=WzAsMyxbMywwLCJBXm0iXSxbNCwwLCJBIl0sWzAsMCwiQV5uIl0sWzAsMSwiXFxvbWVnYV9BIl0sWzIsMCwiKCh0XzEpX0EobiksKHRfMilfQShuKSxcXGxkb3RzLCAodF9tKV9BKG4pKSJdXQ==
			\[\begin{tikzcd}[column sep=huge]
				{A^n} &&& {A^m} & A
				\arrow["{\omega_A}", from=1-4, to=1-5]
				\arrow["{((t_1)_A(n),(t_2)_A(n),\ldots, (t_m)_A(n))}", from=1-1, to=1-4]
			\end{tikzcd}\]		
		\end{enumerate}
	\end{defn}
	Note that in the empty structure all derived operations are the empty function (there can't be any constant symbols).
	\begin{lem}[Variable redundancy in terms]\label{lem:redun_terms}
		Let $A$ be a structure and $t$ a term with $\textnormal{FV}(t) \subseteq \{x_1,\ldots,x_n\}$ for some $n$. Suppose we have two sequences of elements of $A^n$, say $a$ and $b$, such that $a$ and $b$ agree on free variables, i.e. $a_k = b_k$ whenever $x_k\in \textnormal{FV}(t)$. Then $t_A(n)(a) = t_A(n)(b)$.
	\end{lem}
	\begin{proof}
		Induction on $t$. Suppose $t = x_i\in X$. Then $t_A(n)(a)$ and $t_A(n)(b)$ are $a_i$ and $b_i$ respectively. We have assumed these are the same, as $x_i\in \textnormal{FV}(t)$. Thus $t_A(n)(a)= t_A(n)(b)$.
		
		Now suppose $t = \omega t_1 \cdots t_m$ where $\omega \in \Omega$ with $\alpha(\omega) = m$, and the $t_i$'s are terms. By inductive hypothesis, we may assume that for all $1\leq i \leq n$ we have
		\[
		(t_i)_A(n) (a) =  (t_i)_A(n) (b). 
		\]
		It follows that 
		\[
		((t_1)_A(n),\ldots, (t_m)_A(m))(a) = ((t_1)_A(n),\ldots, (t_m)_A(n))(b).
		\]
		Applying $\omega_A$ to both sides gives the result.
	\end{proof}
	\begin{defn}(Derived formulae)
		Let $A$ be an $\mathcal{L}$-structure and $p$ a formula. In addition, suppose $n$ is an integer with $\textnormal{FV}(p) \subseteq \{x_1,\ldots,x_n\}$. We define $p_A(n)$ to be subset of $A^{n}$, or equivalently a function $A^{n} \to 2$, as follows.
		\begin{enumerate}
			\item If $p$ is the formula $(s=t)$ for terms $s$ and $t$ then
			%
			\[
				p_A(n) \coloneqq \{\, a \in A^{n} \mid s_A(n)(a) = t_A(n)(a)\,\}
			\]
			\item Suppose $p = \phi(t_1,\ldots, t_m)$ for some $\phi \in \Pi$, with  $\alpha(\phi) = m$, and terms $t_1,\ldots, t_m$. Then (the characteristic function of) $p_A(n)$ is defined by
			% https://q.uiver.app/#q=WzAsMyxbMywwLCJBXm0iXSxbNCwwLCIyIl0sWzAsMCwiQV5uIl0sWzAsMSwiXFxwaGlfQSJdLFsyLDAsIigodF8xKV9BKG4pLCh0XzIpX0EobiksXFxsZG90cywgKHRfbSlfQShuKSkiXV0=
			\[\begin{tikzcd}[column sep=huge]
				{A^n} &&& {A^m} & 2
				\arrow["{\phi_A}", from=1-4, to=1-5]
				\arrow["{((t_1)_A(n),(t_2)_A(n),\ldots, (t_m)_A(n))}", from=1-1, to=1-4]
			\end{tikzcd}\]
			\item If $p = \bot$ then $p_A(n)$ is the empty set (i.e. its characteristic function is constant with value zero).
			\item Suppose $p$ is $(q \Rightarrow r)$ for formulas $q$ and $r$, where $q_A(n)$ and $r_A(n)$ have already been defined. Then we define $p_A$ by the composition
			% https://q.uiver.app/#q=WzAsMyxbMiwwLCIyXFx0aW1lcyAyIl0sWzMsMCwiMiJdLFswLDAsIkFebiJdLFswLDEsIlxcUmlnaHRhcnJvd18yIl0sWzIsMCwiKHFfQShuKSxyX0EobikpIl1d
			\[\begin{tikzcd}[column sep=huge]
				{A^n} && {2\times 2} & 2
				\arrow["{\Rightarrow_2}", from=1-3, to=1-4]
				\arrow["{(q_A(n),r_A(n))}", from=1-1, to=1-3]
			\end{tikzcd}\]
			\item Suppose $p = (\forall x_m)q$ for some formula $q$ with $x_m \in \textnormal{FV}(q)$. Define $N\coloneqq \max(n,m)$. We can assume that $q_A(N)$ is defined. Let $a\in A^{n}$. We say that $a\in p_A(n)$ if and only if for all $a'\in A^N$ so that $a'$ agrees with $a$ in the first $n$ terms---except possibly on the $m$-th term---we have that $a'\in q_{A}(N)$. 
		\end{enumerate}
	\end{defn}
	\begin{lem}[Variable redundancy in formulae]\label{lem:redun_form}
		Let $A$ be a structure and $p$ a formula with $\textnormal{FV}(p) \subseteq \{x_1,\ldots,x_n\}$ for some $n$. Suppose we have two elements of $A^n$, say $a$ and $b$ such that $a$ and $b$ agree on free variables, i.e. $a_k = b_k$ whenever $x_k\in \textnormal{FV}(t)$. Then $a\in p_A(n)$ if and only if $b\in p_A(n)$.
	\end{lem}
	\begin{proof}
		By induction on $p$. If $p$ is the formula $(s=t)$ for terms $s$ and $t$ we have
		\begin{align*}
			a\in p_A(n) &\iff s_A(n)(a) = t_A(n)(a)\\
			&\iff s_A(n)(b) = t_A(n)(b)\\
			&\iff b\in p_A(n),
		\end{align*}
		where we have used Lemma \ref{lem:redun_terms}.
		
		Now suppose $p = \phi (t_1,\ldots,t_m)$ for some $\phi \in \Pi$ with $\alpha(\phi) = m$, and terms $t_1,\ldots, t_m$. Again by Lemma \ref{lem:redun_terms} we have
		\[
		((t_1)_A(n),\ldots, (t_m)_A(m))(a) = ((t_1)_A(n),\ldots, (t_m)_A(n))(b).
		\]
		Applying $\phi_A$ to both sides gives the result. This finishes the induction in the case that $p$ is an atomic formula.
		
		If $p = \bot$ then $a,b\notin p_A(n)$ so the claim holds. Now let $q,r$ be formulae with $p$ being $(q\Rightarrow r)$. By inductive hypothesis
		\[
		(q_A(n),r_A(n))(a) = (q_A(n),r_A(n))(b),
		\]
		and applying $\Rightarrow_2$ to both sides gives the result.
		
		Finally, we consider the case where $p = (\forall x_m)q$ for some formula $q$ and some variable $x_m\in \textnormal{FV}(q)$. Write $a = (a_1,\ldots,a_n)$ and $b= (b_1,\ldots,b_n)$. By symmetry, we only need to show that $a\in p_A(n)$ implies $b\in p_A(n)$. So, suppose $a \in p_A$. 
		
		Let $N= \max(n,m)$ and let $b'\in A^N$ be a sequence agreeing with $b$ in the first value except possibly on the $m$-th value, and call this value $c_m$. If we show that $b' \in q_A(N)$ then we are done.
		
		Define $a'\in A^N$ to be the sequence $b'$ but with its first $n$ values replaced by $a$ \emph{except} the $m$-th value, which remains $c_m$. As $a\in p_A(n)$ it is clear by definition that $a'\in q_A(N)$. But $a'$ and $b'$ agree on free variables of $q$, which are $\textnormal{FV}(p)\cup\{x_m\}$: the first $n$ variables are just $a$ and $b$, which agree on $\textnormal{FV}(p)$, only that we have specified that they agree on the $m$-th value $c_m$, and otherwise $a'$ and $b'$ are identical. Hence, by the inductive hypothesis, $b'\in q_A(N)$.
	\end{proof}
	Variable redundancy implies that free variables are the only thing that affects the values of $t_A(n)$ and $p_A(n)$. Hence we will write $t_A$ and $p_A$, without specifying $n$, we take $n$ to be the minimum so that $\textnormal{FV}(t)$ (resp. $\textnormal{FV}(p)$) is a subset of $\{x_1,\ldots,x_n\}$. And in any case, this only function only requires inputs $a_k$ where $x_k$ is a free variable of $t$ (resp. $p$).
	\section{First-order theories}
	\begin{defn}[Satisfying a formula]
		Let $A$ be an $\mathcal{L}$-structure, and let $p$ be a formula. We say that $p$ is \emph{satisfied} in $A$ if the indicator function of $p_A$ is constant with value 1. In this case we write $A\models p$. 
	\end{defn}
	\begin{defn}[Sentences and universal closure]
		Let $p$ be a formula. We say that $p$ is a \emph{sentence} if it has no free variables. (In this case $p_A$ is a constant function since $n=0$). In any case, we can obtain a sentence $\bar{p}$, called the \emph{universal closure} of $p$, by prefixing $p$ with universal quantifiers for each of the free variables of $p$ (say, in decreasing order of subscripts).
	\end{defn}
	We remark that if $A$ is empty and $p$ is not a sentence then the indicator $p_\emptyset \colon \emptyset^{n} \to 2$ is constant with value 1 since it sends all elements of $\emptyset^{\mathbb{N}}$ (which there are none) to 1 (this is a vacuous truth).
	\begin{prop}
		For all $\mathcal{L}$-structures $A$ and formulas $p$,
		\[
		A \models p \,\,\,\textnormal{ if and only if }\,\,\, A\models \bar{p}.
		\]
	\end{prop}
	\begin{proof}
		We prove a weaker statement first. Let $q$ be a formula with a free variable $x_m$. Then we claim that $A \models q$ if and only if $A\models (\forall x_m) q$. Indeed, $A\models \forall x_n q$ if and only if for all $a\in A^{n}$ (where $n$ is the minimum so that $\textnormal{FV}((\forall x_m) q)\subseteq \{x_1,\ldots,x_n\}$) we have $a\in ((\forall x_m) q)_A(n)$. And this happens iff for all $a\in A^n$ and for all $a'\in A^{N}$ (where $N\coloneqq \max(n,m)$) that agrees with $a$ on the first $n$ values except possibly in the $m$-th value we have $a'\in q_A(N)$.
		
		This is clearly equivalent to saying that for all $a' \in A^N$ we have $a \in q_A(N)$, i.e. $A \models q_A$.
		
		Now if $p$ is a formula with $k$ free variables we can use induction on $k$, together with the above result, to deduce the claim about $\bar{p}$. 
	\end{proof}
	\begin{defn}[First-order theory]
		Let $T$ be a set of $\mathcal{L}$-formulae and $A$ a structure. We write $A\models T$ if $A\models p$ for all $p\in T$.
		
		In the special case where $T$ is a set of sentences we call it a \emph{first-order theory}, and its formulae are called \emph{axioms}. If $A\models T$ in this case we would say that $A$ \emph{models} the theory $T$.
	\end{defn}
	\section{Semantics and syntax}
	Recall that we had an arbitrary signature $\Sigma$ that generated a language $\mathcal{L}$. We would like to add things to the signature (which will generate a difference language) from time to time. Here I will give some notation for a typical situation. Let $S$ be a set. we denote by $\Sigma_S$ the signature $\Sigma$ but with $|S|$ new constant symbols (i.e. function symbols of arity zero) added. Similarly, we denote by $\mathcal{L}_S$ the language generated by $\Sigma_S$. Note that in the special case that $S$ is an $\mathcal{L}$-structure we clearly have that $S$ is an $\mathcal{L}_S$-structure: just interpret the new constant symbols as the elements of $S$. 
	\begin{defn}[Semantic entailment]
		 Let $T$ be a theory and let $p$ be a sentence. We say that $T$ \emph{semantically entails} $p$, written as $T\models p$, to mean that every model of $T$ satisfies $p$.
		 
		 In the case where $T$ and $p$ are not sentences simply consider the language 
		 \[\mathcal{L}'\coloneqq \mathcal{L}_{\textnormal{FV}(T)\cup\textnormal{FV}(p)},\]
		 where $\textnormal{FV}(T)$ is just the set of all free variables appearing in a formulae of $T$. Let $T'$ and $p'$ be the same formulae but with free variables replaced by the corresponding constants in $\mathcal{L}'$. Then $T'\cup\{p'\}$ is just a set of sentences in $\mathcal{L}'$, so declare that $T\models p$ in $\mathcal{L}$ if and only if $T'\models p'$ in $\mathcal{L}'$.
	\end{defn}
	This seems roundabout: why not define semantic entailment $T\models p$ as ``for all $A\models T$ we have $A\models p$''? This certainly makes sense when $T\cup\{p\}$ is not a set of sentences. The problem is that if we adopted this alternate definition we would have undesired consequences with the empty structure. For example, if $T=\{\neg(x_1 = x_1)\}$ and $p = \{\bot\}$ then the only model for $T$ is the empty structure but $\emptyset \nvDash p$. However, since $T$ is clearly a contradictory statement we would like to have $T\models p$ in this case, which is guaranteed by the real definition since the addition of constants invalidate the empty structure.
	
	Now we turn to our system of deduction. If $w$ is a formula, $t$ is a term and $x$ a variable, we define $w[t/x]$ to be the formula obtained from $w$ on replacing each free occurrence of $x$ by $t$, \emph{provided} no free variable of $t$ occurs bound in $w$. More formally, we define
	\begin{align*}
		&y[t/x] = \begin{cases}
			y & \textnormal{ if }x \neq y\\
			t & \textnormal{ if }x = y
		\end{cases}\\[6pt]
		&(\omega t_1 \ldots t_n)[t/x] = \omega (t_1[t/x]) \cdots (t_n[t/x])\\[6pt]
		&(s=s')[t/x] = (s[t/x] = s'[t/x])\\[6pt]
		&\phi(t_1, \ldots ,t_n)[t/x] = \phi((t_1[t/x]), \ldots ,(t_n[t/x]))\\[6pt]
		&\bot[t/x] = \bot\\[6pt]
		&(p \Rightarrow q)[t/x] = (p[t/x] \Rightarrow q[t/x])\\[6pt]
		&((\forall y)p)[t/x]= \begin{cases}
			(\forall y) (p[t/x]) & \textnormal{ if } x \neq y\\
			(\forall y)p & \textnormal{ if }x =y.
		\end{cases} 
	\end{align*}
	\begin{lem}\label{lem:subs_op}
		Let $w$ be a term or a formula, let $x_m$ be a variable and let $t$ be a term such that all free variables in $t$ do not appear bound in $w$. Suppose $A$ is an $\mathcal{L}$-structure and $a\in A^{n}$ where $n$ is the minimum nonnegative integer so that $\textnormal{FV}(w)\subseteq \{x_1,\ldots,x_n\}$. Denote by $a'$ the sequence $a$ but with the $m$-th value replaced by $t_A(a)$. Then we have
		\[
		(w[t/x_n])_A(a) = w_A(a'). 
		\]
		If $A = \emptyset$ then $(w[t/x_n])_A = w_A$
	\end{lem}
	\begin{proof}
		Suppose $A = \emptyset$. If $w$ does not have $x_n$ as a free variable then $w[t/x_n] = w$. It follows that $(w[t/x_n])_A = w_A$.  
		
		Now assume that $x_n$ is a free variable of $w$. Clearly $w$ is not a sentence. We claim that neither is $w[t/x_n]$. This is easily seen from the fact that $t$ is not a constant (since otherwise the empty set could not be an $\mathcal{L}$-structure) and thus has free variables and \emph{in addition} we assumed that no free variables of $t$ are being bound in $w$. Thus, as neither $w$ nor $w[t/x_n]$ are sentences, they are indicators $\emptyset^{\mathbb{N}} \to 2$ and thus equal. This proves the claim for the empty structure, so from now on assume $A\neq \emptyset$.
		
		First suppose that $w$ is a term. We use induction, so assume $w = x_m$ for some $m$. If $m \neq n$ then $w[t/x_n] = w$ and so we only need to show that $w_A(a) = w_A(a')$. This is immediate by variable redundancy: $a$ and $a'$ agree on the free variable $x_m$. Now suppose $m = n$. Then $w[t/x_n] = t$ and we need to show that $t_A(a) = (x_n)_A(a')$. Again, this is obvious: the right-hand side of the equation is the $n$-th value of $a'$, which we assumed is $t_A(a)$. This closes the base case.
		
		Now, suppose $w = \omega t_1 t_2\cdots t_m$ for some $\omega \in \Omega$ with $\alpha(\omega) = m$ and where the $t_i$'s are terms. Clearly 
		\[
		w[t/x_n] = \omega (t_1[t/x_n]) (t_2[t/x_n])\cdots (t_m[t/x_n]).
		\]
		It follows that
		\begin{align*}
			(w[t/x_n])_A(a) &= \omega_A ((t_1[t/x_n])_A(a),(t_2[t/x_n])_A(a),\ldots,(t_m[t/x_n])_A(a))\\
			&= \omega_A((t_1)_A(a'), (t_2)_A(a'),\ldots, (t_m)_A(a'))\\
			&= w_A(a'),
		\end{align*}
		where we have used the inductive hypothesis. This closes the induction and proves the statement when $w$ is a term.
		
		Suppose now that $w$ is a formula. We again use induction. If $w$ is the formula $(s = s')$ for terms $s$ and $s'$ we have that 
		\[
		w[t/x_n] = (s[t/x_n] = s'[t/x_n]).
		\]
		Then,
		\begin{align*}
			a \in (s[t/x_n] = s'[t/x_n])_A &\iff (s[t/x_n])_A(a) = (s'[t/x_n])_A(a)\\
			&\iff s_A(a') = t_A(a')\\
			&\iff a'\in w_A,
		\end{align*}
		where we used the result for terms. Now suppose $w$ is $\phi (t_1,\ldots,t_m)$ for some $\phi \in \Pi$ with $\alpha(\phi) = m$, and terms $t_1,\ldots, t_m$. Again, using the claim for terms, we have 
		\[
		((t_1[t/x_n])_A(a),\ldots,(t_m[t/x_n])_A(a)) = ((t_1)_A(a'),\ldots,(t_m)_A(a'))
		\]
		and applying $\phi_A$ to both sides gives the result. This closes the base case, i.e. the case where $w$ is an atomic formula.
		
		Clearly $(\bot[t/x_n])_A(a) = \bot_A (a) = 0 = \bot_A(a')$. Now, if $w$ is $(p\Rightarrow q)$ then, by the inductive hypothesis
		\[
		((p[t/x_n])_A(a), (q[t/x_n])_A(a)) = (p_A(a'),q_A(a')).
		\]
		Applying $\Rightarrow_2$ to both sides gives the result.
		
		Finally, suppose $w = (\forall x_m) p$. Then we have two cases. If $m = n$ then $w[t/x_n] = w$ and so we need to show that $w_A(a) = w_A(a')$. But in this case clearly $x_n$ is not a free variable of $w$, so $a$ and $a'$ agree on free variables, and the claim follows by variable redundancy.
		
		Now assume $m\neq n$. Then $w[t/x_n] = (\forall x_m) (p[t/x_n])$. First, note that that 
		
		\[a = (a_1,a_2,\ldots)\in ((\forall x_m) (p[t/x_n]))_A\]
		if and only if
		\[
		(a_1,\ldots, a_{m-1}, c,a_{m+1},\ldots)\in (p[t/x_n])_A \textnormal{ for all }c\in A. 
		\]
		For $c\in A$ let $\alpha(c)$ be the sequence above, i.e. $a$ but the $m$-th value replaced by $c$. Similarly, let $\alpha'(c)$ be the sequence $a'$ but replacing the $m$-th value with $c$. Finally, let $\alpha^*(c)$ be the sequence $\alpha(c)$ but with the $n$-th value replaced by $t_A(\alpha(c))$ Then we can reformulate our statement as so:
		\[
		\alpha(c) \in (p[t/x_n])_A \textnormal{ for all }c\in A.
		\]
		By the inductive hypothesis, this new statement is equivalent to
		\[
		\alpha^*(c) \in p_A \textnormal{ for all }c\in A.
		\]
		Note that, for all $c$, we have that $\alpha^*(c)$ and $\alpha'(c)$ agree on all values (including the $m$-th) except possibly on the $n$-th value, where we have $t_A(\alpha(c))$ and $t_A(a)$ fro $\alpha^*(c)$ and $\alpha'(c)$ respectively. We claim that in fact \emph{they do} agree on the $n$-th value, i.e. $t_A(\alpha(c)) = t_A(a)$.
		
		Indeed, by definition, $a$ and $\alpha(c)$ agree on all values except possibly on the $m$-th value. However, we assumed (and this is the first and only time we use the assumption when $A\neq \emptyset$) that the free variables of $t$ do not appear bound in $w$. Clearly $x_m$ is bound in $w$ (recall that we insist that variables that are being bound appear in the formula). Thus $x_m$ cannot be a free variable of $t$, which implies that $a$ and $\alpha(c)$ agree on free variables; hence $t_A(\alpha(c)) = t_A(a)$ by variable redundancy. Thus $\alpha^*(c)$ is the same sequence as $\alpha'(c)$. Therefore we can, once again, reformulate our statement:
		\[
		\alpha'(c) \in p_A \textnormal{ for all }c\in A.
		\]
		This is manifestly equivalent to $a'\in((\forall x_m)p)_A$, as desired.
	\end{proof}
	
	We now postulate our axioms to be substitution instances of these propositions.
	\begin{enumerate}
		\item $(p\Rightarrow (q\Rightarrow p))$
		\item $((p\Rightarrow (q\Rightarrow r)) \Rightarrow ((p \Rightarrow q) \Rightarrow (p\Rightarrow r)))$
		\item $(\neg\neg p \Rightarrow p)$\\ (here $p,q,r$ may be any formulae of $\mathcal{L}$)
		\item $((\forall x)p \Rightarrow p[t/x])$\\ (here $p$ is any formula with $x\in \textnormal{FV}(p)$, $t$ any term whose free variables don't occur bound in $p$)
		\item $((\forall x) (p\Rightarrow q) \Rightarrow (p \Rightarrow (\forall x)q))$\\ ($p,q$ formulae, $x\notin \textnormal{FV}(p)$)
		\item $(\forall x)(x = x)$
		\item $(\forall x,y)((x = y) \Rightarrow (p \Rightarrow p[y/x]))$\\ ($p$ any formula with $x\in \textnormal{FV}(p)$, $y$ not bound in $p$ and distinct from $x$)
	\end{enumerate}
	\begin{prop}
		All the axioms above are tautologies.
	\end{prop}
	\begin{proof}
		Let $p,q,r$ be formulae in $\mathcal{L}$ and let $A$ be an $\mathcal{L}$-structure.
		\begin{enumerate}
			\item First suppose that $A=\emptyset$. Then, if there are free variables in $p$ or $q$ then it is clear that $\emptyset\models (p\Rightarrow (q\Rightarrow p))$. Otherwise, $p$ and $q$ are sentences and so they have a truth value. Case-by-case analysis reveals that $\emptyset\models (p\Rightarrow (q\Rightarrow p))$. Now assume that $A$ is nonempty.
			
			Note that for all $a\in A^{\mathbb{N}}$ we have
			\[
			(p\Rightarrow (q\Rightarrow p))_A(a) = (\Rightarrow_2) (p_A(a), (\Rightarrow_2)(p_A(a),q_A(a)))
			\]
			as elements of $2 = \{0,1\}$. Plugging in the possible values for $p_A(a)$ and $q_A(a)$ we conclude that in all cases $(p\Rightarrow (q\Rightarrow p))_A(a) = 1$.
			\item Similar to (a).
			\item Similar to (a).
			\item Suppose $x\in \textnormal{FV}(p)$ and $t$ is any term whose free variables don't occur bound in $p$. It is easy to see that the axiom is never a sentence, so $\emptyset$ models it. Assume now that $A\neq \emptyset$.
			
			Let $a =(a_1,a_2,\ldots)\in A^{\mathbb{N}}$ and consider 
			\[
			(\Rightarrow_2)(((\forall x)p)_A(a), p[t/x]_A(a))
			\]
			If $((\forall x)p)_A(a) = 0$ then the above equals 1, clearly. Now suppose $((\forall x)p)_A(a) = 1$ and let $x = x_n$ for some $n$. This means that, for all $a_n'\in A$ we have
			\[
			(a_1,\ldots, a_{n-1},a_n',a_{n+1},\ldots)\in p_A.
			\]
			Set $a_n' \coloneqq t_A(a)$. By Lemma \ref{lem:subs_op}, the above implies that $p[t/x]_A(a) = 1$, as desired.
			\item Let $x\notin \textnormal{FV}(p)$. If the axiom is not a sentence then it has $\emptyset$ as a model. Suppose now that the axiom is a sentence; this is easily seen to imply that $q$ has $x$ as its only free variable. Clearly $(\forall x)(p\Rightarrow q)$ is satisfied in $\emptyset$. Note that $(\forall x) q$ is also a satisfied sentence in $\emptyset$. Therefore the whole axiom is seen to be satisfied in $\emptyset$. Now assume $A\neq \emptyset$.
			
			Let $a =(a_1,a_2,\ldots)\in A^{\mathbb{N}}$. If $((\forall x) (p \Rightarrow q))_A(a) = 0$ then the formula is true for $a$. So, assume that $a\in ((\forall x) (p \Rightarrow q))_A$. Let $x = x_n$ for some $n$. We have that, for all $a_n'\in A$:
			\[
			(a_1,\ldots, a_{n-1},a_n',a_{n+1},\ldots) \in (p\Rightarrow q)_A.
			\]
			In other words, for all $a_n'\in A$:
			\[
			(\Rightarrow_2)(p_A,q_A)(a_1,\ldots, a_{n-1},a_n',a_{n+1},\ldots) = 1
			\]
			But, as $x_n\notin \textnormal{FV}(p)$, the value $p_A(a_1,\ldots, a_{n-1},a_n',a_{n+1},\ldots)$ does not depend on $a_n'$ by variable redundancy. Thus we conclude that for all $a_n'\in A$.
			\[
			(\Rightarrow_2)(p_A(a),q_A((a_1,\ldots, a_{n-1},a_n',a_{n+1},\ldots))) = 1.
			\]
			From this, it is easy to deduce that $a\in (p \Rightarrow (\forall x)q)_A$, as desired.
			
			\item The empty set is easily seen to model this axiom. Let $a =(a_1,a_2,\ldots)\in A^{\mathbb{N}}$ and let $x = x_n$l Then $a\in ((\forall x)(x=x))_A$ iff for all $a_n'\in A$ we have
			\[
			(a_1, \ldots, a_{n-1}, a_{n}',a_{n+1},\ldots) \in (x=x)_A.
			\]	
			This happens iff for all $a_n'$ we have
			\[
			x_A(a_1, \ldots, a_{n-1}, a_{n}',a_{n+1},\ldots) = x_A(a_1, \ldots, a_{n-1}, a_{n}',a_{n+1},\ldots),
			\]
			which is manifestly true. 
			\item The empty set is easily seen to model this axiom. Let $x = x_n\in \textnormal{FV}(p)$ and $y = x_m$ be not bound in $p$ with $n\neq m$. Have some $a =(a_1,a_2,\ldots)\in A^{\mathbb{N}}$. We need to show that 
			\[
			a\in ((\forall x_n)(\forall x_m) ((x_n= x_m) \Rightarrow (p \Rightarrow p[x_m/x_n])))_A
			\]
			For $c_n,c_m \in A$ define $\alpha(c_n,c_m)$ to be the sequence $a$ but with the $i$-th value replaced by $c_i$ for $i\in \{n,m\}$ (recall that $n\neq m$). Then the above proposition is equivalent to
			\[
			\alpha(c_n,c_m) \in ((x_n= x_m) \Rightarrow (p \Rightarrow p[x_m/x_n]))_A \textnormal{ for all }c_n,c_m\in A
			\]
			We need to prove the above. To that end, let $c_n,c_m\in A$ be arbitrary. If $\alpha(c_n, c_m)\notin (x_n = x_m)_A$ then we do have the inclusion above. So, assume $\alpha(c_n,c_m)\in (x_n = x_m)_A$; this clearly implies that $c \coloneqq c_n = c_m$. Now, we need to show that 
			\[
			\alpha(c,c) \in (p \Rightarrow p[x_m/x_n])_A.
			\]
			If $\alpha(c,c)\notin p_A$ then the above is true. Therefore we can suppose $\alpha(c,c)\in p_A$. We want to prove that $\alpha(c,c)\in (p[x_m/x_n])_A$. As $x_m$ is not bound in $p$ we can apply Lemma \ref{lem:subs_op} which tells us that it suffices to show that $\alpha(c,c)' \in p_A$, where $\alpha(c,c)'$ denotes the sequence $\alpha(c,c)$ but replacing the $n$-th value by $(x_m)_A(\alpha(c,c)) = c$. Clearly $\alpha(c,c)' = \alpha(c,c)$ and we supposed at the start that $\alpha(c,c)\in A$. Thus we are done.
		\end{enumerate}
	\end{proof}
	To our deductive system we add the following rules of inference.
	\begin{enumerate}
		\item[(MP)] From $p$ and $(p\Rightarrow q)$, we may infer $q$, \emph{provided} either $q$ has a free variable or $p$ is a sentence.
		\item [(Gen)] From $p$ we may infer $(\forall x) p$, \emph{provided} $x$ does not occur free in any premiss which has been used in the proof of $p$ (but is a free variable of $p$).
	\end{enumerate}
	Formally, we define our concept of deduction as follows.
	\begin{defn}[Deduction sequence]
		Let $S$ be a set of formulae. A \emph{deduction sequence} on $S$ is a finite sequence on the set of formulae of $\mathcal{L}$, defined inductively below.
		\begin{enumerate}[label=(\roman*)]
			\item The empty sequence is a deduction sequence.
			\item If $(p_1,\ldots,p_n)$ is a deduction sequence and $p$ is an axiom or an element of $S$, then $(p_1,\ldots,p_n,p)$ is a deduction sequence.
			\item Let $(p_1,\ldots,p_n)$ be a deduction sequence. Suppose there are $1\leq i,j\leq n$ so that $p_j$ is the formula $(p_i\Rightarrow p)$ for some $p$, and, in addition either $p_i$ is a sentence or $p$ has a free variable. Then $(p_1,\ldots,p_n,p)$ is a deduction sequence.
			\item Suppose $(p_1,\ldots, p_n)$ is a deduction sequence so that $p_n$ has a free variable $x$ but $x$ is not a free variable of $p_i$ for $i<n$, and $p_n\notin S$. Then, if $p = (\forall x) p_n$, we have that  $(p_1,\ldots, p_n,p)$ is a deduction sequence.
			\item That is all.
		\end{enumerate}
	\end{defn}
	\begin{defn}[Syntactic entailment]
		Let $S$ be a set of formulae and $p$ a formula. We say that $S$ \emph{syntactically entails} $p$, written as $S\vdash p$, if there is a deduction sequence terminating at $p$.
	\end{defn}
	\section{Properties of first-order languages}
	Again, we fix a language $\mathcal{L}$ with a set of variables $X=\{x_1,x_2,\ldots\}$
	\section{Completeness}
	The aim of this section is to prove the Completeness theorem. Before that, we need to prove the Soundness theorem. First, a couple of lemmata.
	\begin{lem}\label{lem:sem_mp}
		Let $S$ be a set of formulae and let $p$ and $q$ be formulae so that either $q$ has a free variable or $p$ is a sentence. If $S\models p$ and $S\models (p\Rightarrow q)$, then $S\models q$.
	\end{lem}
	\begin{proof}
		Let $A$ be an $\mathcal{L}$-structure. We want to show that one of the two following statements holds:
		\begin{enumerate}
			\item[(I)] $A$ is empty and there is a non-sentence in $S\cup \{q\}$ .
			\item[(II)] $\bigcap_{r\in S} r_A \subseteq q_A$.
		\end{enumerate}
		We know that one of these two statements holds:
		\begin{enumerate}
			\item[(a)] $A$ is empty and there is a non-sentence in $S\cup \{p\}$ .
			\item[(b)] $\bigcap_{r\in S} r_A \subseteq p_A$.
		\end{enumerate}
		Suppose (a) holds. If there is a non-sentence in $S$ then there is a non-sentence in $S\cup \{q\}$ and $A$ is empty, i.e. (I) holds. Otherwise, if $p$ is a non-sentence, then, by the premiss of the statement $q$ is a non-sentence and again (I) holds. So, from now on, assume (b) holds	
		
		Similarly, we also know that one of the two statements below holds:
		\begin{enumerate}
			\item[(a')] $A$ is empty and there is a non-sentence in $S\cup \{p,q\}$ .
			\item[(b')] $\bigcap_{r\in S} r_A \subseteq (p\Rightarrow q)_A$.
		\end{enumerate}
		Suppose (a') holds. If $q$ is a non-sentence then (I) holds, and if instead the non-sentence is in $S\cup \{p\}$ we have reduced to case (a). So, we can assume (b') holds. But (b) and (b') are easily seen to imply (II), even when $A$ is empty.	
	\end{proof}
	\begin{lem}\label{lem:sem_gen}
		Let $S$ be a set of formulae, $p$ a formula, and $x$ a variable so that $x$ does not occur free in any formulae of $S$. If $S\models p$ then $S\models (\forall x)p$.
	\end{lem}
	\begin{proof}
		Let $A$ be an $\mathcal{L}$-structure. As before, we want to show that one of the two following statements holds:
		\begin{enumerate}
			\item[(I)] $A$ is empty and there is a non-sentence in $S\cup \{(\forall x) p\}$ .
			\item[(II)] $\bigcap_{r\in S} r_A \subseteq ((\forall x)p)_A$.
		\end{enumerate}
		The hypothesis is that one of the two following statements holds. (We know that $p$ is a non-sentence already)
		\begin{enumerate}
			\item[(a)] $A$ is empty
			\item[(b)] $\bigcap_{r\in S} r_A \subseteq p_A$.
		\end{enumerate}
		Suppose (a) is true. If there is a non-sentence in $S\cup \{(\forall x) p\}$ then (I) holds, so assume that $S$ is a set of sentences, and that $(\forall x) p$ is a sentence. But then, as $A$ is empty, we have that the indicator of $((\forall x) p)_A$ is constant with value 1, implying that (II) holds.
		
		Now suppose (a) is not true. Then $A$ is nonempty and (b) holds. Let $a= (a_1,a_2,\ldots)\in r_A\subseteq A^{\mathbb{N}}$ for all $r\in S$.  We would like to show that $a\in ((\forall x)p)_A$, which, if $x= x_n$, is equivalent to the statement
		\[
		(a_1,\ldots, a_{n-1},a_n',a_{n+1},\ldots) \in p_A \textnormal{ for all }a_n'\in A.
		\]
		So, fix some $a_n'\in A$ and set $a' \coloneqq (a_1,\ldots, a_{n-1},a_n',a_{n+1},\ldots)$. By assumption $x_n$ is not a free variable of $r$ for all $r\in S$. By variable redundancy, we conclude that $a'\in r_A$ for all $r\in S$. Finally, (b) implies that $a'\in p_A$, as desired. 
	\end{proof}
	\begin{prop}[the Soundness Theorem]
		Let $S$ be a set of formulae and $p$ a formula. If $S\vdash p$ then $S\models p$.
	\end{prop}
	\begin{proof}
		It is enough to show that, for all deduction sequences $\sigma$, all formulae of $\sigma$ are semantically entailed by $S$. We use induction on the set of deduction sequences.
		
		The claim is vacuously true when $\sigma$ is the empty sequence. Suppose $\sigma = (p_1,\ldots, p_n,p)$, where $S\models p_i$ for all $i$, and $p$ is an axiom or an element of $S$. It easily follows that $S\models p$ (recall axioms are tautologies).
		
		Now suppose that $\sigma = (p_1,\ldots,p_n,p)$, where $S\models p_i$ for all $i$, and there are $1\leq i,j, \leq n$ so that $p_j$ is the formula $(p_i \Rightarrow p)$, and, in addition, either $p_i$ is a sentence or $p$ has a free variable. Then Lemma \ref{lem:sem_mp} says that $S\models p$.
		
		Finally, suppose that $\sigma = (p_1,\ldots,p_n,p)$, where $S\models p_i$ for all $i$, and that $p_n\notin S$ has a free variable $x$ but $x$ is not a free variable of $p_i$ for $i<n$. In addition, we suppose $p = (\forall x)p_n$. Let $S' = \{p_1,\ldots, p_{n-1}\}$. We claim that $S'\models p_n$	
	\end{proof}
	\chapter{Model Theory}
	\section{Substructures and diagrams}
	\begin{defn}[$\mathcal{L}$-homomorphism]
		Let $M$ and $N$ be $\mathcal{L}$-structures. An $\mathcal{L}$-\emph{homomorphism} is a map $\eta\colon M \to N$ such that given $\bar{a} = (a_1,\ldots,a_n)\in M^n$:
		\begin{itemize}
			\item for all function symbols $f$ of arity $n$ we have that 
			\[
				\eta(f^{M}(\bar{a})) = f^N(\eta^n(\bar{a})),
			\]
			in other words the diagram
			% https://q.uiver.app/#q=WzAsNCxbMCwwLCJNXm4iXSxbMCwxLCJNIl0sWzEsMSwiTiJdLFsxLDAsIk5ebiJdLFsxLDIsIlxcZXRhIiwyXSxbMCwxLCJmXk0iLDJdLFszLDIsImZeTiJdLFswLDMsIlxcZXRhXm4iXV0=
			\[\begin{tikzcd}
				{M^n} & {N^n} \\
				M & N
				\arrow["\eta"', from=2-1, to=2-2]
				\arrow["{f^M}"', from=1-1, to=2-1]
				\arrow["{f^N}", from=1-2, to=2-2]
				\arrow["{\eta^n}", from=1-1, to=1-2]
			\end{tikzcd}\]
			commutes;
			\item for all relation symbols $R$ of arity $n$ we have that
			\[
				\bar{a}\in R^M \text{ iff }\eta^n(\bar{a})\in  R^N.
			\]
		\end{itemize}
		An injective $\mathcal{L}$-homomorphism is an $\mathcal{L}$-\emph{embedding} and an invertible one is an $\mathcal{L}$-\emph{isomorphism}. If $M$ and $N$ are isomorphic we write $M \cong N$. If $M\subseteq N$ and the inclusion map is an $\mathcal{L}$-homomorphism we say that $M$ is a \emph{substructure} of $N$, and $N$ is an \emph{extension} of $M$.
	\end{defn}
	We are going to stop writing $\bar{m}\in M^n$ where $n$ is the length of $\bar{m}$ and just write $\bar{m}\in M$ when $n$ can be inferred or its unimportant.
	\begin{egs}\label{egs:substructures}\leavevmode
		\begin{enumerate}
			\item Let $\mathcal{L}$ be the language of groups. Then $(\mathbb{N},+,0)$ is a subset of the the integers $(\mathbb{Z},+,0)$, but it is not a substructure.
			\item If $M$ is an $\mathcal{L}$-structure and $X\subseteq M$ then $X$ is the domain of a substructure of $M$ iff it is closed under the interpretation of all function symbols.
			
			Indeed, the inclusion $\iota\colon X \to M$ clearly preserves relations. But if it is not closed under some function $f^M$ then there is no way to interpret $f^X$. 
			\item It follows from the previous point that the intersection of a family of substructures is a substructure: indeed, applying a function $f^M$ to anything in the intersection will land on all substructures (since these are closed under function symbols) and thus in the intersection.
			
			The substructure generated by $X\subseteq M$ is defined to be the intersection of all substructures of $M$ containing $X$; it is denoted by $\langle X\rangle_{M}$. Again, by the previous point, $\langle X\rangle_M$ is also the intersection of all subsets of $M$ that are closed under function symbols.
			
			Hence
			\[
				\langle X\rangle_{M} = X \cup \{t^M(\bar{m})\mid t\text{ a term and }\bar{m}\in X\}.
			\]
			Indeed, the RHS is obviously closed under function symbols and no strict subset of it could possibly be. Therefore $|\langle X \rangle_{M}| \leq |X| + |\mathcal{L}|$.
			
			We say a structure $M$ is \emph{finitely generated} if $M = \langle X \rangle_M$ for some finite $X\subseteq M$.
		\end{enumerate}
	\end{egs}
	What kind of sentence is preserved under substructures?
	\begin{prop}\label{prop:qf_preser_substr}
		Let $\varphi(\bar{x})$ be a quantifier-free $\mathcal{L}$-formula with $n$ variables, $M$ be an $\mathcal{L}$-structure and $\bar{a}\in M$. For every extension $N$ of $M$ we have $M \models \varphi(\bar{a})$ iff $N\models \varphi(\bar{a})$.
	\end{prop}
	\begin{proof}
		First we show that if $t(\bar{x})$ is a term with $k$ free variables then $t^M(\bar{b}) = t^N(\bar{b})$ for all $\bar{b}\in M$.
		
		This is clearly the case if $t = x_i$ is a variable since then $t^M(\bar{b}) = b_i = t^N(\bar{b})$. Now suppose $t = f(q_1,\ldots, q_l)$ for a function symbol $f$ of arity $l$ and the $q_i$'s are terms. By the inductive hypothesis we can assume $q_i^M(\bar{b}) = q_i^N(\bar{b})$ for all $i$. Then,
		\[
			t^M(\bar{b}) = f^M(q_1^M(\bar{b}),\ldots, q_l^M(\bar{b})) = f^N(q_1^N(\bar{b}), \ldots, q_l^N(\bar{b}))) = t^N(\bar{b})
		\]
		where we have used the fact that $M$ is a substructure of $N$.
		
		Now onto the main result. Let $t_1$ and $t_2$ be terms with at most $n$ free variables. Then
		\[
			M\models (t_1(\bar{a}) = t_2(\bar{a}))
		\] 
		if and only if $t^{M}(\bar{a}) = t^M(\bar{a})$. But this happens iff $t^{N}(\bar{a}) = t^N(\bar{a})$, and this is equivalent to $N \models (t_1(\bar{a}) = t_2(\bar{a}))$.
		
		Next, let $R(t_1,\ldots,t_l)$ be an $l$-ary relation, where all the $t_i$'s have at most $n$ free variables. We have that $M\models R(t_1(\bar{a}),\ldots,t_l(\bar{a}))$ iff $(t_1^M(\bar{a}),\ldots,t_l^M(\bar{a}))\in R^M$. As $N$ is an extension (and by the result for terms) the latter happens iff $(t_1^N(\bar{a}),\ldots, t_l^N(\bar{a}))\in R^N$. And this is of course the same as $N \models R(t_1(\bar{a}),\ldots,t_l(\bar{a}))$.
		
		This finishes the induction for atomic formulae. Now if $\varphi$ is a formula satisfying the claim then clearly 
		\[
			M\models \neg \varphi(\bar{a})\,\,\, \text{ iff }\,\,\, M \nvDash \varphi(\bar{a}) \,\,\,\text{ iff }\,\,\, N \nvDash\varphi(\bar{a})\,\,\, \text{ iff }\,\,\, N \models \varphi(\bar{a}). 
		\]
		If the proposition is true for $\varphi$ and $\psi$ then $M \models \varphi \vee \psi$ iff $M \models \varphi$ or $M\models \psi$, which happens iff $N \models \varphi$ or $N \models \psi$, which is clearly equivalent to $N \models \varphi \vee \psi$.
	\end{proof}
	A \emph{universal formula} is one of the form $\forall \bar{x} \varphi(\bar{x},\bar{y})$ where $\varphi$ is quantifier free. A \emph{universal theory} is one whose axioms are universal sentences.
	\begin{defn}
		Structures $M$ and $N$ are \emph{elementary equivalent} if for every $\mathcal{L}$-sentence $\varphi$ we have $M\models \varphi$ iff $N\models \varphi$. 
		
		A homomorphism $f\colon M \to N$ is an \emph{elementary embedding} if it is injective and for all $\mathcal{L}$-formulae $\varphi(x_1,\ldots,x_n)$ and elements $m_1,\ldots,m_n\in M$ we have 
		\[
			M \models \varphi(m_1,\ldots,m_n) \,\,\,\text{ iff }\,\,\, N \models \varphi(f(m_1),\ldots, f(m_n)).
		\] 
		
		We denote `$M$ and $N$ are elementary equivalent' by $M \equiv N$.
	\end{defn}
	\begin{rem*}
		If $M$ and $N$ are $\mathcal{L}$-structures and we have some tuples of the same size $\bar{m}\in M$ and $\bar{n}\in N$ then by $(M,\bar{m}) \equiv (N,\bar{n})$ we mean that the expanded structures are elementary equivalent as $\mathcal{L}_{\bar{c}}$-structures, where $\bar{c}$ has the same size as $\bar{m}$ and $\bar{n}$.
	\end{rem*}
	\begin{prop}\label{prop:iso_imp_el_equiv}
		If $M\cong N$ then $M \equiv N$.
	\end{prop}
	\begin{proof}
		Let $F\colon M \to N$ be an isomorphism. By symmetry, we only need to show that all formulaes modelled by $M$ are also modelled by $N$.
		
		
		First we would like to show that if $t$ is a term then for all $\bar{m} \in M$ we have $t^M(\bar{m}) = t^N(F(\bar{m}))$.
		
		We use induction over the structure of formulae.
		
		If $t_1(\bar{x})$ and $t_2(\bar{x})$ are terms with the same free variables, then $M \models (t_1(\bar{x}) = t_2(\bar{x}))$ iff for all $\bar{m}\in M$ we have $t_1^M(\bar{m}) = t_2^M(\bar{m})$. But then for $\bar{n}\in N$ we have that
		\begin{align*}
			t_1^N(\bar{n}) &= t_1^N(FF^{-1}\bar{n})\\
			&= F(t_1^M(F^{-1}(\bar{n})))\\
			&= F(t_2^M(F^{-1}(\bar{n})))\\
			&= t_2^N(FF^{-1}\bar{n})\\
			&= t_2^N(\bar{n}).
		\end{align*}
		Now suppose that $R(x_1,\ldots,x_l)$ is a relation so that $M \models R(x_1,\ldots,x_l)$. Then $\bar{m}\in R^M$ for all $\bar{m}\in M$. But then if $\bar{n}\in N$ we have that $\bar{n}\in R^N$ iff $F^{-1}(n)\in R^M$ which is clearly true.
		
		Let $\varphi$ be a formulae for which this holds. Then $M\models \neg\varphi$ iff $M \nvDash \varphi$ iff $N \nvDash \varphi$ iff $N \models \neg\varphi$. Similarly for disjunction.
		
		Now suppose $M \models (\forall x) \varphi(x,\bar{y})$. That means that for all $\bar{m},m'\in M$ we have that  $\varphi^M(m',\bar{m})$ is true which means that $M\models \varphi(x,\bar{y})$. Apply inductive hypothesis and we are done.
		\end{proof}
		Recall that a theory $\mathcal{T}$ is \emph{complete} if $T\models \varphi$ or $T \models \neg \varphi$ for every sentence $\varphi$. Any two models of the same complete theory are elementary equivalent: indeed the formulae satisfied by them are completely determined by the theory. However, the models can have different cardinalities (and hence be non-isomorphic), see Examples \ref{egs:vaught_test}.
		
		\begin{defn}
			A substructure $M\subseteq N$ is an \emph{elementary substructure} if the inclusion map $M \hookrightarrow N$ is an elementary embedding.
		\end{defn}
		\begin{defn}
			A theory $\mathcal{T}$ is \emph{model-complete} if every embedding between models of $\mathcal{T}$ is elementary.
		\end{defn}
		\begin{defn}
			Let $\kappa$ be an infinite cardinal. We say that a theory $\mathcal{T}$ is $\kappa$-categorical if all models of $\mathcal{T}$ of cardinality $\kappa$ are isomorphic.
		\end{defn}
		\begin{prop}[Vaught's Test]
			Let $\mathcal{T}$ be a consistent $\mathcal{L}$-theory with no finite models. If $\mathcal{T}$ is $\kappa$-categorical for some infinite $\kappa\geq |\mathcal{L}|$ then $\mathcal{T}$ is a complete theory. 
		\end{prop}
		\begin{proof}
			For the sake of contradiction, suppose $\mathcal{T}$ is not complete, so that there is a sentence $\varphi$ with $T \nvDash \varphi$ and $T\nvDash \neg\varphi$. It follows (by the Deduction Theorem) that $T\cup\{\varphi\}$ and $T\cup\{\neg\varphi\}$ are consistent, i.e. they have a model.
			These two models cannot be finite since then they would be finite models of $\mathcal{T}$. Thus they are infinite.
			
			By the Upwards L{\"o}wenheim-Skolem Theorem (together with the Downwards version if necessary) we get models of $T\cup\{\varphi\}$ and $T\cup\{\neg\varphi\}$ of cardinality $\kappa$. In particular, these are models of $\mathcal{T}$ so, as $\mathcal{T}$ is $\kappa$-categorical, they must be isomorphic. But they disagree on the valuation of $\varphi$, contradicting Proposition \ref{prop:iso_imp_el_equiv}.
 		\end{proof}
 		\begin{egs}\label{egs:vaught_test}\leavevmode
 			\begin{enumerate}
 				\item Any two countable dense linear orders without endpoints are isomorphic to $\mathbb{Q}$. So the theory is $\aleph_0$-categorical and hence complete.
 				\item For every field $F$ the theory of infinite $F$-vector spaces is $\kappa$-categorical for some $\kappa > |F|$ ({\color{red} Exercise}), hence complete.
 			\end{enumerate}
 		\end{egs}
 		\begin{prop}[Tarski-Vaught test]
 			Let $N$ be an $\mathcal{L}$-structure and $M\subseteq N$. Then $M$ is the domain of an elementary substructure of $N$ if and only if the following condition is satisfied.
 			\begin{itemize}
 				\item Let $\varphi(x,\bar{t})$ be a formula and $\bar{m}\in M$. If there is some $n\in N$ such that $N\models \varphi(n,\bar{m})$, then there is $\hat{m}\in M$ such that $N \models \varphi(\hat{m},\bar{m})$.
 			\end{itemize}
 		\end{prop}
 		\begin{proof}
 			Suppose $M$ is an elementary substructure of $N$. Let $\varphi(x,\bar{t})$ be a formula and $\bar{m}\in M$. Furthermore, assume that there is some $n\in N$ so that $N\models \varphi(n,\bar{m})$. It follows that $N\models (\exists x)\varphi(x,\bar{m})$. Thus, as $M$ is an elementary substructure, we have $M \models (\exists x)\varphi(x,\bar{m})$. Hence, there is some $\hat{m}\in M$ such that $M \models \varphi(\hat{m},\bar{m})$. But then clearly $N \models \varphi(\hat{m},\bar{m})$.
 			
 			Conversely, assume that the property holds. Consider the formulae $\varphi_f(x,\bar{t}) \coloneqq (x = f(\bar{t}))$ for each function symbol $f$ in $\mathcal{L}$.
 			
 			For any $\bar{m}\in M$ there is $n\in N$ such that $N\models (n = f(\bar{m}))$ (say $n\coloneqq f^N(\bar{m})$). By hypothesis there is $\hat{m}$ such that $N \models (\hat{m} = f(\bar{m}))$. It follows that $M$ is closed under function symbols. Interpreting relation symbols as $R^M \coloneqq R^N \cap M$ we turn $M$ into an $\mathcal{L}$-structure that is clearly a substructure of $N$. 			
 			
 			We need to show that $N$ and $M$ satisfy the same formulae when the parameters are in $M$. Let $\varphi(\bar{x})$ be a formula and $\bar{m}\in M$. Note that if $N \models \varphi(\bar{m})$ iff $M \models \varphi(\bar{m})$ whenever $\varphi$ is quantifier free by Proposition \ref{prop:qf_preser_substr}. So, by induction, we only need to check that the case when $\varphi$ is an existential formula, say $\varphi(\bar{x}) = (\exists t)\psi(t, \bar{x})$.
 			
 			If $M \models (\exists t)\psi(t, \bar{m})$ then clearly $N \models (\exists t)\psi(t, \bar{x})$ since a witness in $M$ is also a witness in $M$. Conversely, if $N \models (\exists t)\psi(t, \bar{m})$ then there is some $n\in N$ so that $N\models \psi(n,\bar{m})$. By the condition, there is some $\hat{m}$ such that $N \models \psi(\hat{m},\bar{m})$. Using the inductive hypothesis, we get that $M \models \psi(\hat{m},\bar{m})$ and thus $M \models \varphi(\bar{m})$ as desired. Hence $M$ is an elementary substructure of $N$.
 		\end{proof}
 		\begin{defn}
 			Let $N$ be an $\mathcal{L}$-structure. We define the \emph{diagram} of $N$ as 
 			\[
 				\Diag(N) \coloneqq \{\varphi(n_1,\ldots,n_k) \mid \varphi \text{ is a q.f. $\mathcal{L}_{N}$-formula and }N \models \varphi(n_1,\ldots,n_k)\}.
 			\]
 			The \emph{elementary diagram} of $N$ is defined as 
 			\[
 				\Diag_{\text{el}}(N) \coloneqq \{\varphi(n_1,\ldots,n_k) \mid \varphi \text{ is an $\mathcal{L}_{N}$-formula and }N \models \varphi(n_1,\ldots,n_k)\}
 			\]
 		\end{defn}
 		Basically, diagrams are all (q.f.) formulas the structure believes in. For the difference between diagrams of structures and the theory of a structure (the latter we'll discuss later) see \href{https://math.stackexchange.com/a/3644376}{this link}.
 		
 		The most important thing about diagrams is their models. Indeed, let $M$ be a model of $\Diag(N)$. In particular, $M$ is an $\mathcal{L}$-structure with a constant $m_n\in M$ for each element $n\in N$. Hence we have a function $\eta \colon N \to M$ given by $\eta(n) \coloneqq m_n$. But $N\models \neg(n = n')$ for any distinct $n,n'\in N$. Thus, $M\models \neg(m_n = m_{n'})$ for distinct $n,n'\in N$, i.e. $\eta$ is injective. 
 		
 		Let $f$ be a function symbol. Then (bare with me on this) for all $\bar{n}\in N$ we have $N \models (f(\bar{n}) = f^{N}(\bar{n}))$ where the LHS is the function symbol $f$ applied to the constants $\bar{n}$ and the RHS is just the constant $f^{N}(\bar{n})$. Hence, $M\models (f(\eta(\bar{n})) = \eta(f^{N}(\bar{n})))$, i.e. we have
 		\[
 			f^M(\eta(\bar{n})) = \eta(f^N(\bar{n})).
 		\]
 		Also, if $R$ is a relation symbol, then for all $\bar{n}\in N$, if we have  $N \models R(\bar{n})$ then $M \models R(\eta(\bar{n}))$ then we have $N \models R(\bar{n})$. Conversely, if $N \nvDash R(\bar{n})$ then $N\models \neg R(\bar{n})$ and we apply the same argument. This all shows that $f$ is an injective homomorphism, so we can see $M$ as an extension of $N$. In summary, models of $\Diag(N)$ are just extensions of $N$.
 		
 		Similarly, models $\Diag_{\text{el}}(N)$ are elementary extensions of $N$. Indeed, if $M$ is such a model then $M$ is an extension of $N$ by the previous argument. Let $\varphi(\bar{x})$ be a formula and  let $\bar{m}\in M$. If $N \models \varphi(\bar{m})$ then $M \models \varphi(\bar{m})$ by definition of $\Diag_{\text{el}}(N)$. Conversely, if $M \models \varphi(\bar{m})$ we have that $N \models \varphi(\bar{m})$ by Proposition \ref{prop:qf_preser_substr} if $\varphi$ is quantifier free, and otherwise it is obvious for universal formulae.
 		
 		\begin{lem}\label{lem:uni_consis_diag}
 			Let $\mathcal{T}$ be a consistent theory and let $\mathcal{T}_{\forall}$ be the theory of all universal sentences that follow from $\mathcal{T}$. If $N$ is a model of $\mathcal{T}_{\forall}$, then $\mathcal{T}\cup \Diag(N)$ is consistent.
 		\end{lem}
 		\begin{proof}
 			For the sake of contradiction, suppose it is inconsistent. By Compactness, there is some finite subset of $\mathcal{T}\cup \Diag(N)$ that is inconsistent. As $\mathcal{T}$ is consistent, there must be a finite subset of $\Diag(N)$ that is inconsistent with $\mathcal{T}$.
 			
 			Take the conjunction of all of these sentences and call it $\varphi(\bar{n})$. Then $\mathcal{T}\cup\{\varphi(\bar{n})\}$ is inconsistent, i.e. $\mathcal{T}\vdash \neg \varphi(\bar{n})$. But $\mathcal{T}$ is an $\mathcal{L}$-theory, so it contains none of the constants $\bar{n}$. Hence, by generalization, $\mathcal{T} \vdash (\forall x)\neg \varphi(\bar{x})$. However $N$ is a model of $\mathcal{T}_{\forall}$ so $N \models (\forall x)\neg \varphi(\bar{x})$, and thus $N \models \neg \varphi(\bar{n})$, a contradiction.
 		\end{proof}
 		We say a theory $\mathcal{T}$ has an \emph{universal axiomatization} if there is an universal theory that has exactly the same models as $\mathcal{T}$
 		\begin{thm}[Tarski, \L o\'s]
 			An $\mathcal{L}$-theory $\mathcal{T}$ has a universal axiomatization iff whenever $N$ is a substructure of $M$ and $M \models T$, then $N \models T$.
 		\end{thm}
 		\begin{proof}
 			One direction is obvious (see Proposition \ref{prop:qf_preser_substr}). For the converse, suppose $\mathcal{T}$ is preserved under taking substructures. We would like to say that $\mathcal{T}_{\forall}$ is an universal axiomatization of $\mathcal{T}$. To prove this we need to show that if $N \models T_{\forall}$ then $N \models\mathcal{T}$ (since the converse is obvious).
 			
 			By Lemma \ref{lem:uni_consis_diag}, we have that $\mathcal{T} \cup \Diag(N)$ is consistent, so let $M$ be a model for it. Then $M$ is a model of $\Diag(N)$ and hence an extension of $N$. But $M$ is a model of $\mathcal{T}$ so taking substructures we conclude that $N \models \mathcal{T}$.
 		\end{proof}
 		The method of diagrams is powerful. We can show much more with the same method.
 		\subsubsection*{1. Finding a common elementary extension to given structures}
 		\begin{thm}[Elementary amalgamation]
 			Let $M$ and $N$ be $\mathcal{L}$-structures, $\bar{m}\in M$ and $\bar{n}\in N$ be of the same length such that $(M, \bar{m}) \equiv (N, \bar{n})$.  Then there is an $\mathcal{L}$-structure $K$ and elementary embeddings $g\colon N \hookrightarrow K$ and $h\colon M \hookrightarrow K$ such that $g(\bar{n}) = h(\bar{m})$.
 		\end{thm}
 		\begin{proof}
 			Form the disjoint union of $M$ and $N$ and quotient it out by the smallest equivalence relation containing $(m_i,n_i)$ for all $i$. Then the resulting set is basically copies of the two sets $M$ and $N$ that only intersect at $\bar{m} = \bar{n}$. Hence we may assume, without loss of generality that $\bar{m} = \bar{n}$ and otherwise $M$ and $N$ are disjoint.
 			
 			We would like to show that $\mathcal{T}\coloneqq \Diag(N)_{\text{el}} \cup \Diag_{\text{el}}(M)$ is consistent; of course, we do so by Compactness. Let $\Phi$ be a finite subset of $\mathcal{T}$, which of course contains only finitely many sentences of $\Diag_{\text{el}}(N)$.
 			
 			Let $\varphi(\bar{n},\bar{k})$ be the conjunction of all these sentences, where $\bar{k}$ does not contain any elements of $\bar{n}$ and its elements are pairwise disjoint. Define $\varphi(\bar{x},\bar{y})$ to be the corresponding $\mathcal{L}_N$ formula. If $\Phi$ is inconsistent then $\Diag_{\text{el}} \vdash \neg \varphi(\bar{m},\bar{k})$. Since the elements in $\bar{k}$  are distinct and not in $M$ we in fact have $\Diag_{\text{el}}(M)\vdash (\forall \bar{y}) \neg \varphi(\bar{m},\bar{y})$ by generalization.
 			
 			In particular, $(M,\bar{m})\models (\forall \bar{y}) \neg \varphi(\bar{m},\bar{y})$ and so by hypothesis  $(N,\bar{n})\models (\forall \bar{y}) \neg \varphi(\bar{m},\bar{y})$, contradicting the fact that $\varphi(\bar{m},\bar{k})\in \Diag_{\text{el}}(N)$.
 			
 			By compactness $T$ must be consistent and a model for it would be an elementary extension of both $N$ and $M$.
 		\end{proof}
 		\subsubsection*{2. Controlling the size of a model}
 		\begin{thm}[Löwenheim–Skolem]
 			Let $M$ be an infinite $\mathcal{L}$-structure and $\kappa \geq |\mathcal{L}|$ be an infinite cardinal.
 			\begin{enumerate}
 				\item[(\textdownarrow)] If $\kappa < |M|$ then $M$ admits an elementary substructure of size $\kappa$.
 				\item[(\textuparrow)] If $\kappa > |M|$ then $M$ admits an elementary extension of size $\kappa$.
 			\end{enumerate}
 		\end{thm}
 		\begin{proof}[Proof of (\textuparrow)]
 			Expand the language $\mathcal{L}$ by adding one constant symbol for each $m\in M$ and $c\in \kappa$. Let $\mathcal{T} \coloneqq \Diag_{\text{el}}(M)\cup \{\neg(c = c')\}_{c,c'\in \kappa, c\neq c'}$ be a theory in such a language. By compactness $\mathcal{T}$ has a model, i.e. an elementary extension of size at least $\kappa$. By (\textdownarrow), there must be one of size exactly $\kappa$.
 		\end{proof}
 		\section{Existentially closed structures and quantifier-elimination}
 		\begin{defn}
 			Let $\mathcal{T}$ be an $\mathcal{L}$-theory and $\varphi(\bar{x},y)$ be an $\mathcal{L}$-formula with $\bar{x}$ nonempty. A \emph{Skolem function} for $\varphi$ is an $\mathcal{L}$-term $t(\bar{x})$ such that
 			\[
 				\mathcal{T} \models \forall \bar{x}:((\exists y: \varphi(\bar{x}, y)) \Rightarrow \varphi(\bar{x},t(\bar{x}))).
 			\]
 			A \emph{skolemization} of an $\mathcal{L}$-theory $\mathcal{T}$ is a language $\mathcal{L}^+\supseteq \mathcal{L}$ together with an $\mathcal{L}^+$-theory $\mathcal{T}^+\supseteq \mathcal{T}$ such that :
 			\begin{enumerate}[label=(\arabic*)]
 				\item Every $\mathcal{L}$-structure that models $\mathcal{T}$ can be expanded to a model of $\mathcal{T}^+$.
 				\item The $\mathcal{L}^+$-theory $\mathcal{T}^+$ admits Skolem functions for every $\mathcal{L}^+$-formula $\varphi(\bar{x},y)$ with $\bar{x}\neq \emptyset$.
 			\end{enumerate}
 			Finally, a theory $\mathcal{T}$ is a \emph{Skolem theory} if it is a skolemization of itself.
 		\end{defn}
 		We say two $\mathcal{L}$-formulae $\varphi$ and $\psi$ are \emph{equivalent modulo} $\mathcal{T}$, where $\mathcal{T}$ is an $\mathcal{L}$-theory, iff $T \vdash (\varphi \iff \psi)$.
 		\begin{prop}\label{prop:fam_eq_mod}
 			Let $\mathcal{T}$ be an $\mathcal{L}$-theory and $F$ be a collection of $\mathcal{L}$-formulae that includes all atomic formulae and is closed under Boolean combinations. If for every formula $\psi(\bar{x},y)$ in $F$ we have $\varphi(\bar{x})$ in $F$ such that 
 			\[
 				\mathcal{T} \vdash \forall \bar{x}: ((\exists y: \psi(\bar{x}, y))\iff \varphi(\bar{x})),
 			\]
 			then every $\mathcal{L}$-formula is equivalent modulo $\mathcal{T}$ to one in $F$ with the same free variables.
 		\end{prop}
 		\begin{proof}
 			By induction on the structure of formulae. Atomic formulae are in $F$ by hypothesis and $F$ is closed under Boolean combinations, so we only need to check the case for existential statements. But that's exactly the hypothesis.
 		\end{proof}
 		If $M$ is an elementary substructure of $N$ we denote this by $M \preccurlyeq N$.
 		\begin{prop}\label{prop:skolem_qf}
 			Let $\mathcal{T}$ be an Skolem $\mathcal{L}$-theory. Then
 			\begin{enumerate}[label=(\arabic*)]
 				\item Every $\mathcal{L}$-formula $\varphi(\bar{x})$ with $\bar{x}\neq \emptyset$ is equivalent to some quantifier-free $\psi(\bar{x})$ modulo $\mathcal{T}$.
 				\item If $N \models \mathcal{T}$ and $X$ is a subset of $N$, then either $\langle X \rangle_N = \emptyset$ or $\langle X \rangle_N\preccurlyeq N$.
 			\end{enumerate}
 		\end{prop}
 		\begin{proof}\leavevmode
 			\begin{enumerate}[label=(\arabic*)]
 				\item This just follows from Proposition \ref{prop:fam_eq_mod} by taking $F$ to be the set of all quantifier-free formulae, and by the definition of Skolem theory.
 				\item Assume $X$ is nonempty (otherwise this is trivial). Let $M \coloneqq \langle X \rangle_N$; we will use the Tarski-Vaught test. Let $\varphi(x,\bar{y})$ be an $\mathcal{L}$-formula and $\bar{m}\in M$. Suppose that there is some $n\in N$ with $N \models \varphi(n,\bar{m})$. Then $N \models (\exists y)(\varphi(y,\bar{m}))$. Hence there is a Skolem function $t(\bar{x})$ such that $N \models \varphi(t(\bar{m}), \bar{m})$. As $M$ is a substructure of $N$ it is closed under interpretations  of terms, hence $t^N(\bar{m}) \in M$. We are done.\qedhere
 			\end{enumerate}
 		\end{proof}
 		\begin{thm}[Skolemization Theorem]\label{thm:skolem}
 			Every (first-order) language $\mathcal{L}$ can be expanded to some $\mathcal{L}^+ \supseteq \mathcal{L}$ that includes an $\mathcal{L}^+$-theory $\Sigma$ such that
 			\begin{enumerate}[label=(\arabic*)]
 				\item $\Sigma$ is a Skolem $\mathcal{L}^+$-theory.
 				\item Every $\mathcal{L}$-structure can be expanded to an $\mathcal{L}^+$-structure that models $\Sigma$.
 				\item $|\mathcal{L}| = |\mathcal{L}^+|$.
 			\end{enumerate}
 		\end{thm}
 		\begin{proof}
 			For an arbitrary language $\mathcal{L}$ and for every $\mathcal{L}$-formula of the form $\chi(\bar{x},y)$ with $\bar{x}\neq \emptyset$ we create an function symbol $F_{\chi}$ of arity $|\bar{x}|$. By adding all such function symbols to $\mathcal{L}$ we get a new language $\mathcal{L}^*$. We define $\Sigma(\mathcal{L})$ to be the $\mathcal{L}^*$-theory:
 			\[
 				\Sigma(\mathcal{L}) \coloneqq \{\forall \bar{x} ((\exists y: \chi(\bar{x}, y))\Rightarrow \chi(\bar{x}, F_{\chi}(\bar{x}))) \mid \chi(\bar{x},y) \text{ is an $\mathcal{L}$-formula and $\bar{x}\neq\emptyset$}\}.
 			\]
 			Intuitively, $\Sigma$ says that $\mathcal{L}^*$ has Skolem functions \emph{for all $\mathcal{L}$-formulae}. We would be basically done if this included $\mathcal{L}^*$ formulae too. The way to fix this is to iterate the construction.
 			
 			Let $\mathcal{L}$ be a language. Define a sequence of languages and theories as follows. Start with $\mathcal{L}_0 \coloneqq \mathcal{L}$ and $\Sigma_0 \coloneqq \emptyset$. For $n\geq 1$ define
 			\[
 				\mathcal{L}_{n} \coloneqq \mathcal{L}_{n-1}^* \,\,\,\text{ and }\,\,\,\Sigma_n \coloneqq \Sigma(\mathcal{L}_{n-1}) \cup \Sigma_{n-1}.
 			\]
 			Set $\mathcal{L}^+ \coloneqq \bigcup_{n \in \mathbb{N}} \mathcal{L}_n$ and $\Sigma \coloneqq \bigcup_{n\in \mathbb{N}} \Sigma_n$.
 			
 			It is easy to see that $\Sigma$ is an $\mathcal{L}^+$-theory. Furthermore, it is Skolem since every $\mathcal{L}^+$-formula of the required form is in some $\mathcal{L}_n$ and thus there is an Skolem function for it by $\Sigma_{n+1}$. Also $|\mathcal{L}| = |\mathcal{L}^+|$ by cardinal multiplication (countable cardinalities are absorbed).
 					
 			We now check the structure expansion property. We first check it step-by-step down the chain. Let $M$ be a nonempty $\mathcal{L}$-structure. Say we have $\chi(\bar{x},y)$ with $\bar{x}\neq\emptyset$ and a tuple $\bar{m}\in M$. If there is a $b$ such that $M \models \chi(\bar{m}, b)$, choose one and define $F_{\chi}^M(\bar{m}) \coloneqq b$. If there is no such $b$ interpret it as $m_0$ (or whatever). By construction, $M$ models $\Sigma(\mathcal{L})$ as an $\mathcal{L}^\ast$-structure. Repeating this process, we can add interpretations for all function symbols in $\mathcal{L}^+$ to $M$ so that $M \models \Sigma$.
 		\end{proof}
 		\begin{coro}
 			Any $\mathcal{L}$-theory $\mathcal{T}$ admits a Skolemization $\mathcal{T}^+$ in a language of the same cardinality as $\mathcal{L}$.
 		\end{coro}
 		\begin{proof}
 			Take $\mathcal{T}^{+}\coloneqq \mathcal{T} \cup \Sigma$ as in Theorem \ref{thm:skolem}. Clearly $\mathcal{T}^{+}$ has Skolem functions. Moreover any $M \models T$ can be expanded to some $M^{+}\models \Sigma$ so $M^{+}\models \mathcal{T}^+$.
 		\end{proof}
 		\begin{coro}
 			Let $M$ be an $\mathcal{L}$-structure and $X\subseteq M$ a subset such that 
 			\[
 				|\mathcal{L}| + |X| \leq \kappa \leq |M|,
 			\]
 			where $\kappa$ is an infinite cardinal. Then $M$ admits an elementary substructure $N$ with $X\subseteq N$ and $|N| = \kappa$.
 		\end{coro}
 		\begin{proof}
 			First, if necessary, add elements to $X$ until we get a set $X'\subseteq M$ such that $|X'| = \kappa$; this is possible by the size restrictions. Apply Theorem \ref{thm:skolem} to get a language $\mathcal{L}^{+}$ and a theory $\Sigma$; also, expand $M$ to an $\mathcal{L}^{+}$-structure that models $\Sigma$. By Proposition \ref{prop:skolem_qf} $N \coloneqq \langle X'\rangle_M$ is an elementary substructure of $M$ containing $X$. Furthermore we have by Example \ref{egs:substructures} (d) that
 			\[
 				\kappa = |X'|\leq |N| \leq |X'| + |\mathcal{L}^+| = \kappa + |\mathcal{L}|  = \max\{\kappa, |\mathcal{L}|\} = \kappa.\qedhere
 			\]
 		\end{proof}
 		Now we define what it means for a theory to have quantifier-elimination.
 		\begin{defn}[Quantifier-elimination]
 			Let $\mathcal{T}$ be an $\mathcal{L}$-theory. We say that a set $F$ of $\mathcal{L}$-formulae is an \emph{elimination set} for $\mathcal{T}$ is for all $\mathcal{L}$-formulae $\varphi$ there is a Boolean combination $\varphi^*$ of formulae in $F$ such that
 			\[
 				\mathcal{T} \vdash (\varphi \iff \varphi^*). 
 			\]
 			A theory $\mathcal{T}$ has \emph{quantifier-elimination} if the family of all quantifier-free $\mathcal{L}$-formulae is an elimination set for $\mathcal{T}$. 
 		\end{defn}
 		\begin{eg}
 			Consider a polynomial 
 			\[
 				p(x) \coloneqq x^3 - 31x^2 +6.
 			\]
 			To determine whether the sentence $\exists x. p(x)$ holds over an algebraically closed field we can equivalently check the quantifier-free sentence 
 			\[
 				(1\neq 0)\vee (-31 \neq 0).
 			\]
 			Similarly, for a real matrix $M$, the sentence saying that $M$ is invertible is equivalent to $\neg(\det(M) = 0)$.
 		\end{eg}
 		Why should we care about elimination sets and quantifier-elimination?
 		\begin{enumerate}[label=\arabic*.]
 			\item Deciding whether two models of $\mathcal{T}$ are elementary equivalent reduces to verifying whether they satisfy the same sentences in the elimination set.
 			\item In particular, completeness of $\mathcal{T}$ is reduced to checking decidability of sentences in the elimination set.
 			\item If the language $\mathcal{L}$ is recursive and the procedure of quantifier elimination is computable, then the question of whether there is an algorithm to decide whether a given sentence is provable from $\mathcal{T}$ or not boils down to checking sentences in the elimination set.
 			\item The elementary embeddings $M\hookrightarrow N$ are precisely those embeddings that preserve formulae in the elimination set and their negations.
 			\item Definable subsets in a model $M$ are Boolean combinations of subsets definable by formulae in the elimination set. 
 		\end{enumerate}
 		In what follows we will use the notation $\neg F$ for $\{\neg \varphi \colon \varphi \in F\}$.
 		\begin{prop}[Syntactic quantifier-elimination]\label{prop:syn_el}
 			Let $\mathcal{T}$ be an $\mathcal{L}$-theory and $F$ be a family of $\mathcal{L}$-formulae that includes all atomic formulae. Suppose that for every $\mathcal{L}$-formulae of the form
 			\[
 				\Theta(\bar{x})\coloneqq \exists y.\bigwedge_{i} \varphi_i(\bar{x},y)
 			\]
 			with $\varphi_i\in F\cup \mathcal{T}$ for all $i$, there is a Boolean combination $\Theta^*(\bar{x})$ for $F$-formulae such that
 			\[
 				\mathcal{T} \vdash \forall \bar{x}. (\Theta(\bar{x}) \iff \Theta^*(\bar{x})).
 			\]
 			Then $F$ is an elimination set for $\mathcal{T}$.
 		\end{prop}
 		\begin{proof}
 			Induction over the structure of formulae. All atomic formulae are in $F$ so the base case is trivial. If $\varphi,\psi$ are equivalent modulo $\mathcal{T}$ to formulae $\varphi',\psi'$ which are Boolean combinations of $F$-formulae, then $\varphi\wedge \psi$ is equivalent modulo $\mathcal{T}$ to $\varphi'\wedge \psi'$ which is still a Boolean combination of $F$-formulae. A similar argument works for negation.
 			
 			\textcolor{red}{I can't quite finish this argument. Isn't $F$ supposed to be closed under Boolean combinations?}
 		\end{proof}
 		\begin{eg}
 			The theory $\mathcal{T}_{\infty}$ of infinite sets (in the language with empty signature) has quantifier-elimination. See Example Sheet 2 for a proof of a more difficult statement.
 		\end{eg}
 		If $M$ and $N$ are $\mathcal{L}$-structures we write $M \Rightarrow_{1} N$ if every existential sentence modelled by $M$ is also modelled by $N$.
 		\begin{thm}[Existential amalgamation]
 			Let $M$ and $N$ be $\mathcal{L}$-structures, $S$ a subset of $M$, and $f\colon \langle S\rangle_{M}\to N$ a homomorphism. If $(N, f(S))\Rightarrow_1 (M, S)$, then there is an elementary extension $K$ of $M$ and an embedding $g\colon N \hookrightarrow K$ making the diagram commute.
 			% https://q.uiver.app/#q=WzAsNCxbMSwyLCJcXGxhbmdsZSBTXFxyYW5nbGVfTSJdLFswLDEsIk0iXSxbMiwxLCJOIl0sWzEsMCwiSyJdLFswLDEsIiIsMCx7InN0eWxlIjp7InRhaWwiOnsibmFtZSI6Imhvb2siLCJzaWRlIjoidG9wIn19fV0sWzAsMiwiZiIsMl0sWzEsMywiIiwwLHsic3R5bGUiOnsidGFpbCI6eyJuYW1lIjoiaG9vayIsInNpZGUiOiJ0b3AifX19XSxbMiwzLCJnIiwyLHsic3R5bGUiOnsidGFpbCI6eyJuYW1lIjoiaG9vayIsInNpZGUiOiJib3R0b20ifX19XV0=
 			\[\begin{tikzcd}
 				& K \\
 				M && N \\
 				& {\langle S\rangle_M}
 				\arrow[hook, from=3-2, to=2-1]
 				\arrow["f"', from=3-2, to=2-3]
 				\arrow[hook, from=2-1, to=1-2]
 				\arrow["g"', hook', from=2-3, to=1-2]
 			\end{tikzcd}\] 
 		\end{thm}
 		\begin{proof}
 			We can assume $M$ an $N$ are disjoint. Consider the $\mathcal{L}_{M\cup N}$-theory 
 			\[
 				\mathcal{T}\coloneqq \Diag_{\text{el}}(M)\cup\Diag(N)\cup\{\,s = f(s)\,\mid\, s\in S\,\}.
 			\]
 			Let $\mathcal{T}'\subseteq \mathcal{T}$ be finite. In particular it contains only finitely many of the sentences of $\Diag(N)$, so let $\varphi(\bar{n},\bar{k})$ be the conjunction of these quantifier-free sentences, where $\bar{n}\in N\setminus f(S)$ and $\bar{k}\in f(S)$. Suppose, for the sake of contradiction, that $\mathcal{T}'$ is inconsistent. Then $\Diag_{\text{el}}(M)\cup\{s = f(s) \mid s\in S\}$ must entail $\neg\varphi(\bar{n},\bar{k})$. But the elements $\bar{n}$ clearly do not appear in this theory so we must have 
 			\[
 				\Diag_{\text{el}}(M)\cup\{s = f(s) \mid s\in S\} \vdash \forall \bar{x}. \neg \varphi(\bar{x},\bar{k}).
 			\]
 			Equivalently,
 			\[
 				\Diag_{\text{el}}(M)\cup\{s = f(s) \mid s\in S\} \vdash \neg\exists \bar{x}. \varphi(\bar{x},\bar{k})
 			\]
 			By the sentences saying that $s= f(s)$, we can assume $\bar{k}\in S$, so that in particular we have that $(M,S) \nvDash \exists \bar{x}. \varphi(\bar{x},\bar{k})$. By assumption, this means that $(N,f(S))\nvDash \exists \bar{x}. \varphi(\bar{x},\bar{k})$ contradicting the fact that $N \models \varphi(\bar{n},\bar{k})$. So by Compactness $\mathcal{T}$ has a model $K$ which is an elementary extension of $M$, an extension of $N$, and which identifies $s$ with its image under $f$. The claim follows.
 		\end{proof}
 		We say that a class of $\mathcal{L}$-structures has the \emph{amalgamation property} when, given $\mathbb{K}$-structures $A,B,C$ and embeddings $C\hookrightarrow A$ and $C\hookrightarrow B$, then there is a structure $D$ in $\mathbb{K}$ and embeddings $A\hookrightarrow D$ and $B \hookrightarrow D$ making the diagram
 		% https://q.uiver.app/#q=WzAsNCxbMSwyLCJDIl0sWzAsMSwiQSJdLFsyLDEsIkIiXSxbMSwwLCJEIl0sWzAsMSwiIiwwLHsic3R5bGUiOnsidGFpbCI6eyJuYW1lIjoiaG9vayIsInNpZGUiOiJ0b3AifX19XSxbMCwyLCIiLDIseyJzdHlsZSI6eyJ0YWlsIjp7Im5hbWUiOiJob29rIiwic2lkZSI6ImJvdHRvbSJ9fX1dLFsxLDMsIiIsMCx7InN0eWxlIjp7InRhaWwiOnsibmFtZSI6Imhvb2siLCJzaWRlIjoidG9wIn19fV0sWzIsMywiIiwyLHsic3R5bGUiOnsidGFpbCI6eyJuYW1lIjoiaG9vayIsInNpZGUiOiJib3R0b20ifX19XV0=
 		\[\begin{tikzcd}
 			& D \\
 			A && B \\
 			& C
 			\arrow[hook, from=3-2, to=2-1]
 			\arrow[hook', from=3-2, to=2-3]
 			\arrow[hook, from=2-1, to=1-2]
 			\arrow[hook', from=2-3, to=1-2]
 		\end{tikzcd}\]
 		commute.
 		\begin{defn}[Existentially closed]
 			Let $\mathbb{K}$ be a class of $\mathcal{L}$-structures and $M\in \mathbb{K}$. We say that $M$ is \emph{existentially closed} in $\mathbb{K}$ iff for every existential formula $\psi(\bar{x})$ and tuple $\bar{m}\in M$ the existence of an extension $M\hookrightarrow N$ with $N\models \psi(\bar{m})$ and $N \in \mathbb{K}$ forces $M \models \psi (\bar{m})$.
 		\end{defn} 
 		\begin{egs}
 			\begin{enumerate}
 				\item Every existentially closed field (in the class of all fields) is algebraically closed. Indeed, let $A$ be an existentially closed field and let $f(y)$ be a non-trivial polynomial with coefficients in $A$. For each $n\in \mathbb{N}$ define the $\mathcal{L}$-term (where $\mathcal{L}$ is the language of rings)
 				\[
 					p_n(\bar{x},y) \coloneqq x_ny^n + x_{n-1}y^{n-1} + \ldots + x_1 y + x_0.
 				\]
 				Then $f(y) = p_n(\bar{a},y)$ for some $n$ and some choices of $\bar{a}$.
 				
 				If $f$ is not irreducible it suffices to find a root for one of its factors, so assume $f$ is irreducible. Then the quotient $A[y]/(f)$ is a field extension of $A$ in which $f$ has a root, i.e.,
 				\[
 					A[y]/(f) \models \exists y. p(\bar{a},y) = 0.
 				\]
 				As $A$ is existentially closed, this implies that $f$ has a root in $A$. In fact, the existentially closed fields are precisely the algebraically closed ones. This is a version of the Nullstellensatz.
 				
 				\item The existentially closed linear orders are the dense total orders without endpoints. Indeed, for any linear order $L$ and two elements $x<y$ in $L$ we can add a new element $z$ such that $x<z<y$ and similarly add an element that is greater (or lesser) than any given element. This shows that existentially closed fields are DLOs without endpoints.
 				
 				Conversely, \textcolor{red}{what follows?}
 				\item The existentially closed ordered fields are the real closed fields (i.e. the ordered fields which are elementarily equivalent to the real numbers). \textcolor{red}{how to begin...}
 			\end{enumerate}
 		\end{egs}
 		\begin{thm}\label{thm:ax_ex_cl_imp_qe}
 			Let $\mathbb{K}$ be a class of $\mathcal{L}$-structures closed under isomorphisms and such that the class of all the substructures of $\mathbb{K}$-structures has the amalgamation property. Then every existential $\mathcal{L}$-formula is equivalent to a quantifier-free formula in all existentially closed structures in $\mathbb{K}$.
 			
 			In particular, if $\mathcal{T}$ is a theory axiomatising existentially closed structures in $\mathbb{K}$, then $\mathcal{T}$ has quantifier elimination
 		\end{thm}
 		\begin{proof}
 			Let $\varphi(\bar{x})$ be an existential formula. Call a pair $(M,\bar{m})$ a \emph{witnessing pair} if $M$ is existentially closed in $\mathbb{K}$ and $M\models \varphi(\bar{m})$. Let $\Theta_{(M,\bar{m})}(\bar{x})$ be the conjunction of all literals $\psi(\bar{x})$ such that $M\models \psi(\bar{m})$. Now, let $\chi(\bar{x})$ be the disjunction of all $\Theta_{(M,\bar{m})}(\bar{x})$. Here $\chi$ will be our ``infinite formula''.
 			
 			It is enough to show that if $N$ is existentially closed in $\mathbb{K}$, then, for all $\bar{n}\in N$, we have $N\models \varphi(\bar{n})$ if and only if $N \models \chi(\bar{n})$. One direction is clear: if $\bar{n}\in N$ is such that $N \models \varphi(\bar{n})$ then $(N,\bar{n})$ is a witnessing pair so $N \models \Theta_{(N,\bar{n})}(\bar{n})$ and thus $N \models \chi(\bar{n})$.
 			
 			Conversely, if $N \models \chi(\bar{n})$, then there must be a witnessing pair $(M,\bar{m})$ such that $N\models \Theta_{(M,\bar{m})}$. This means that if $\psi(\bar{x})$ is a literal and $M \models \psi(\bar{m})$ then $N \models \psi(\bar{n})$. This implies that there is an embedding $e\colon \langle \bar{m}\rangle_M \hookrightarrow N$ mapping $\bar{m}$ to $\bar{n}$ (\textcolor{red}{why?}).
 				
 			The amalgamation property provides an structure $\mathcal{C}$ together with some embeddings making the diagram
 			% https://q.uiver.app/#q=WzAsNCxbMSwyLCJcXGxhbmdsZSBcXGJhcnttfVxccmFuZ2xlX3tNfSJdLFswLDEsIk0iXSxbMiwxLCJOIl0sWzEsMCwiXFxtYXRoY2Fse0N9Il0sWzAsMSwiIiwwLHsic3R5bGUiOnsidGFpbCI6eyJuYW1lIjoiaG9vayIsInNpZGUiOiJ0b3AifX19XSxbMCwyLCJlIiwyLHsic3R5bGUiOnsidGFpbCI6eyJuYW1lIjoiaG9vayIsInNpZGUiOiJib3R0b20ifX19XSxbMSwzLCIiLDAseyJzdHlsZSI6eyJ0YWlsIjp7Im5hbWUiOiJob29rIiwic2lkZSI6InRvcCJ9fX1dLFsyLDMsIiIsMix7InN0eWxlIjp7InRhaWwiOnsibmFtZSI6Imhvb2siLCJzaWRlIjoiYm90dG9tIn19fV1d
 			\[\begin{tikzcd}
 				& {\mathcal{C}} \\
 				M && N \\
 				& {\langle \bar{m}\rangle_{M}}
 				\arrow[hook, from=3-2, to=2-1]
 				\arrow["e"', hook', from=3-2, to=2-3]
 				\arrow[hook, from=2-1, to=1-2]
 				\arrow[hook', from=2-3, to=1-2]
 			\end{tikzcd}\]
 			commute. \textcolor{red}{OK... I give up from here. How does the proof continue? It makes no sense to me. See Theorem 8.1.3 in Hodge's book}.
 		\end{proof}
 		\begin{eg}
 			\begin{enumerate}
 				\item The theory ACF of algebraically closed fields has quantifier elimination. Indeed, recall that this theory axiomatized the existentially closed fields. By Theorem \ref{thm:ax_ex_cl_imp_qe}, we just need to check that the class of all substructures of fields (i.e. integral domains) has the amalgamation property. Some field theory tells us that the class of fields has the amalgamation property, and by taking fraction fields we can extend this to integral domains. (\textcolor{red}{what's the exact argument? use uniqueness of algebraic closures somehow... })
 				
 				\item The theory of dense linear orders without endpoints has quantifier elimination. (\textcolor{red}{taking a coequalizer probably works but I don't want to work out the details})
 			\end{enumerate}
 		\end{eg}
 		\begin{defn}[Inductive classes]
 			A class $\mathbb{K}$ of $\mathcal{L}$-structures is \emph{inductive} if it is closed under isomorphisms and unions of embedding chains.
 		\end{defn}
 		\begin{thm}
 			Let $M$ be as structure in an inductive class $\mathbb{K}$. Then then there is some existentially closed extension $N$ of $M$ with $N \in \mathbb{K}$.
 		\end{thm}
 		\begin{proof}
 			We show that $M$ can be extended to a structure $M^*$ in $\mathbb{K}$ with the following property: for $\bar{m}\in M$ and existential formula $\varphi(\bar{x})$, if $\varphi(\bar{m})$ holds in some extension of $M^*$ in $\mathbb{K}$, then it holds in $M^*$.
 			
 			This is enough. Indeed, we could then recursively define a chain of $\mathcal{K}$-structures by setting $M^0\coloneqq M$ and
 			\[
 				M^{j+1} \coloneqq (M^{(j)})^*.
 			\]
 			Then let $N \coloneqq \bigcup_{j<\omega} M^{j}$. It follows that $N$ is in $\mathbb{K}$ as $\mathbb{K}$ is inductive. Furthermore, $N$ is existentially closed in $\mathbb{K}$. To see this, suppose that $\varphi(\bar{x})$ is an existential formula, $\bar{n}\in N$, and $D$ is a structure in $\mathbb{K}$ such that $D\models \varphi(\bar{n})$. By definition of $N$, $\bar{n}\in M^k$ for some $k<\omega$. Note that $D$ extends $M^{k+1} = (M^k)^*$ so, by the $\ast$-property, $M^{k+1}\models \varphi(\bar{n})$. But existential sentences are preserved under extensions, hence $N \models \varphi(\bar{n})$ as desired.
 			
 			To build $M^*$ we first list all pairs $(\varphi_\beta, \bar{m}_\beta)_{\beta < \delta}$ where $\varphi$ is existential and $\bar{m}\in M$. We construct a chain of $\mathbb{K}$-structures by transfinite induction. 
 			\begin{itemize}
 				\item $M_0 \coloneqq M$;
 				\item $M_{\beta + 1}\coloneqq$ some $\mathbb{K}$-structure $\mathcal{D}$ with $M_{\beta}\subseteq \mathcal{D} \models \varphi_{\beta}(\bar{m}_\beta)$ if this exists, else $M_\beta$;
 				\item $M_{\lambda} \coloneqq \bigcup_{\beta < \lambda} M_{\beta}$ for $\lambda \leq \delta$ a limit ordinal. 
 			\end{itemize} 
 			Finally, define $M^* \coloneqq M_{\delta}$. If $\varphi(\bar{x})$ is existential, $\bar{m}\in M$, and $\mathcal{D}\in \mathbb{K}$ is such that $M^* \subseteq \mathcal{D}\models \varphi(\bar{m})$ then $(\varphi,\bar{m})$ is $(\varphi_\beta,\bar{m}_\beta)$ for some $\beta < \delta$. But then the clause in the definition of $M_{\beta + 1}$ holds true and we must have $M_{\beta + 1} \models \varphi(\bar{m})$. We are done since $M_{\beta + 1} \subseteq M^*$ and existential formulae are preserved under extensions.
 		\end{proof}
 		Whenever two structures $M,N$ satisfy the same quantifier free sentences (this is weaker than elementary equivalence) we write $M \equiv_0 N$.
 		\begin{thm}\label{thm:qe_char}
	 		Let $\mathcal{T}$ be an $\mathcal{L}$-theory. The following are equivalent.
	 		\begin{enumerate}
	 			\item $\mathcal{T}$ has quantifier elimination.
	 			\item If $A,B \models \mathcal{T}$ and $\bar{a}\in A$ and $\bar{b}\in B$ are tuples of the same size such that $(A,\bar{a}) \equiv_0 (B,\bar{b})$ then in fact $(A,\bar{a}) \Rightarrow_{1} (B,\bar{b})$.
	 			\item If $A,B\models \mathcal{T}$, $S$ is a subset of $A$, and $e\colon \langle S \rangle_{A} \hookrightarrow B$ is an embedding, then there is an elementary extension $D$ of $B$ and an embedding $f\colon A \to D$ extending $e$.
	 			\item $\mathcal{T}$ is model-complete and $\mathcal{T}_{\forall}$ has the amalgamation property.
	 			\item For every quantifier-free $\mathcal{L}$-formula $\varphi(\bar{x},y)$ the formula $\exists y. \varphi(\bar{x}, y)$ is $\mathcal{T}$-equivalent to a quantifier free formula $\psi(\bar{x})$.
	 		\end{enumerate}
	 		\begin{proof}\leavevmode
	 			\begin{enumerate}
	 				\item[$(a)\Rightarrow (b)$] This is obvious: just replace the existential sentences by quantifier-free ones.
	 				\item[$(b)\Rightarrow (c)$]  By existential amalgamation, it suffices to show that $(A,S) \Rightarrow_{1} (B,e(S))$. But an $\mathcal{L}_S$-formula can only mention finitely many constants of $S$; thus it suffices to check $(A,\bar{a}) \Rightarrow_{1} (B,e(\bar{a}))$ for all $\bar{a}\in \langle S\rangle_{A}$. But existence of the embedding $e$ implies that $(A,\bar{a}) \equiv_0 (B,e(\bar{a}))$ by Proposition \ref{prop:qf_preser_substr}. So we are done by $(b)$.
	 				\item[$(c)\Rightarrow (d)$] First we show that $\mathcal{T}$ is model complete using the result of Example Sheet 1, question 11 (c). Fix an embedding $h\colon M \hookrightarrow N$ between models of $\mathcal{T}$. We need to show that there's some $D$ model of $\mathcal{T}$ elementarily extending $M$ and an embedding $g\colon N \hookrightarrow D$  such that the diagram
	 				% https://q.uiver.app/#q=WzAsMyxbMCwxLCJNIl0sWzIsMSwiTiJdLFsxLDAsIkQiXSxbMCwxLCJoIiwyLHsic3R5bGUiOnsidGFpbCI6eyJuYW1lIjoiaG9vayIsInNpZGUiOiJ0b3AifX19XSxbMCwyLCJcXHByZWNjdXJseWVxIiwwLHsic3R5bGUiOnsidGFpbCI6eyJuYW1lIjoiaG9vayIsInNpZGUiOiJ0b3AifX19XSxbMSwyLCJnIiwwLHsic3R5bGUiOnsidGFpbCI6eyJuYW1lIjoiaG9vayIsInNpZGUiOiJib3R0b20ifX19XV0=
	 				\[\begin{tikzcd}
	 					& D \\
	 					M && N
	 					\arrow["h"', hook, from=2-1, to=2-3]
	 					\arrow["\preccurlyeq", hook, from=2-1, to=1-2]
	 					\arrow["g", hook', from=2-3, to=1-2]
	 				\end{tikzcd}\]
	 				commutes.
	 				 
	 				In (c) let $S = h(M)\subseteq N$ and define $e\coloneqq h^{-1}\colon h(M)\to M$. We then get an extension $D$ of $M$ and an embedding $g\colon N \hookrightarrow D$ such that the following diagram commutes.
	 				% https://q.uiver.app/#q=WzAsNCxbMCwxLCJNIl0sWzIsMSwiTiJdLFsxLDAsIkQiXSxbMSwyLCJoKE0pIl0sWzAsMiwiXFxwcmVjY3VybHllcSIsMCx7InN0eWxlIjp7InRhaWwiOnsibmFtZSI6Imhvb2siLCJzaWRlIjoidG9wIn19fV0sWzMsMSwiIiwyLHsic3R5bGUiOnsidGFpbCI6eyJuYW1lIjoiaG9vayIsInNpZGUiOiJ0b3AifX19XSxbMywwLCJlIiwwLHsic3R5bGUiOnsidGFpbCI6eyJuYW1lIjoiaG9vayIsInNpZGUiOiJ0b3AifX19XSxbMSwyLCJnIiwyLHsic3R5bGUiOnsidGFpbCI6eyJuYW1lIjoiaG9vayIsInNpZGUiOiJib3R0b20ifX19XV0=
	 				\[\begin{tikzcd}
	 					& D \\
	 					M && N \\
	 					& {h(M)}
	 					\arrow["\preccurlyeq", hook, from=2-1, to=1-2]
	 					\arrow[hook, from=3-2, to=2-3]
	 					\arrow["e", hook, from=3-2, to=2-1]
	 					\arrow["g"', hook', from=2-3, to=1-2]
	 				\end{tikzcd}\]
	 				This clearly makes the first diagram commute so we have shown that $\mathcal{T}$ is model-complete. 
	 				
	 				To see that $\mathcal{T}_{\forall}$ has the amalgamation property, let $C\hookrightarrow A'$ and $C\hookrightarrow B'$ be embeddings between models of $\mathcal{T}_{\forall}$. By Lemma \ref{lem:uni_consis_diag} $A'$ and $B'$ are substructures of models $A$ and $B$ of $\mathcal{T}$ respectively.
	 				
	 				Taking the instance of (c) where $S\coloneqq C$ and $e$ is the embedding $C\hookrightarrow B$ we have an elementary extension $D$ of $B$ and an embedding $f\colon A \hookrightarrow D$ making the diagram
	 				% https://q.uiver.app/#q=WzAsNixbMCwyLCJCJyJdLFsyLDIsIkEnIl0sWzEsMywiQyJdLFswLDEsIkIiXSxbMiwxLCJBIl0sWzEsMCwiRCJdLFsyLDEsIiIsMix7InN0eWxlIjp7InRhaWwiOnsibmFtZSI6Imhvb2siLCJzaWRlIjoiYm90dG9tIn19fV0sWzIsMCwiZSIsMCx7InN0eWxlIjp7InRhaWwiOnsibmFtZSI6Imhvb2siLCJzaWRlIjoiYm90dG9tIn19fV0sWzAsMywiIiwwLHsic3R5bGUiOnsidGFpbCI6eyJuYW1lIjoiaG9vayIsInNpZGUiOiJib3R0b20ifX19XSxbMSw0LCIiLDAseyJzdHlsZSI6eyJ0YWlsIjp7Im5hbWUiOiJob29rIiwic2lkZSI6ImJvdHRvbSJ9fX1dLFszLDUsIlxccHJlY2N1cmx5ZXEiLDAseyJzdHlsZSI6eyJ0YWlsIjp7Im5hbWUiOiJob29rIiwic2lkZSI6ImJvdHRvbSJ9fX1dLFs0LDUsImYiLDIseyJzdHlsZSI6eyJ0YWlsIjp7Im5hbWUiOiJob29rIiwic2lkZSI6ImJvdHRvbSJ9fX1dXQ==
	 				\[\begin{tikzcd}
	 					& D \\
	 					B && A \\
	 					{B'} && {A'} \\
	 					& C
	 					\arrow[hook', from=4-2, to=3-3]
	 					\arrow["e", hook', from=4-2, to=3-1]
	 					\arrow[hook', from=3-1, to=2-1]
	 					\arrow[hook', from=3-3, to=2-3]
	 					\arrow["\preccurlyeq", hook', from=2-1, to=1-2]
	 					\arrow["f"', hook', from=2-3, to=1-2]
	 				\end{tikzcd}\]
	 				commute. As $B \preccurlyeq D$ and $B \models \mathcal{T}$ we have $D\models \mathcal{T}$ and thus $D\models \mathcal{T}_{\forall}$. Hence $\mathcal{T}_{\forall}$ has the amalgamation property.
	 				\item[$(d)\Rightarrow (e)$] Since the models of $\mathcal{T}_{\forall}$ are precisely the substructures of $\mathcal{T}$ and by Example Sheet 1 question 11 (b), we can apply Theorem \ref{thm:ax_ex_cl_imp_qe} thanks to (d).
	 				\item[$(e)\Rightarrow (a)$] Immediate from Proposition \ref{prop:syn_el}.\qedhere
	 			\end{enumerate}
	 		\end{proof}
 		\end{thm}
 		\begin{coro}
 			Let $A$ be a finite $\mathcal{L}$-structure. Then $\text{Th}(A)$ has quantifier elimination iff every isomorphism between finitely generated substructures of $A$ can be extended to an automorphism of $A$.
 		\end{coro}
 		\begin{proof}
 			Suppose that $\text{Th}(A)$ has quantifier elimination and let $e\colon \langle \bar{a} \rangle_A \stackrel{\cong}{\to} \langle \bar{b}\rangle_A \hookrightarrow A$. Let $A = B$ in Theorem \ref{thm:qe_char} so that we get a structure $D$, an elementary embedding $h\colon A \to D$ and an embedding $f\colon A \to D$.
 			
 			As $|A| = n$ for some $n<\omega$ we have a sentence saying so and $D$ must satisfy it. Thus the embeddings $h,f\colon A \hookrightarrow D$ are also surjective, i.e., they are isomorphisms. We also had the commutative diagram
 			% https://q.uiver.app/#q=WzAsNSxbMCwxLCJcXGxhbmdsZVxcYmFye2F9XFxyYW5nbGVfQSJdLFsyLDEsIlxcbGFuZ2xlXFxiYXJ7Yn1cXHJhbmdsZV9BIl0sWzIsMCwiQSJdLFswLDAsIkEiXSxbMSwwLCJEIl0sWzAsMSwiXFxjb25nIiwyXSxbMCwzXSxbMSwyXSxbMyw0LCJmIl0sWzIsNCwiaCIsMl1d
 			\[\begin{tikzcd}
 				A & D & A \\
 				{\langle\bar{a}\rangle_A} && {\langle\bar{b}\rangle_A}
 				\arrow["\cong"', from=2-1, to=2-3]
 				\arrow[from=2-1, to=1-1]
 				\arrow[from=2-3, to=1-3]
 				\arrow["f", from=1-1, to=1-2]
 				\arrow["h"', from=1-3, to=1-2]
 			\end{tikzcd}.\]
 			 So $h^{-1}f$ is an automorphism of $A$ extending the isomorphism.
 			 
 			 Conversely, we now assume that the isomorphisms can be extended and prove quantifier elimination of $\text{Th}(A)$. We show the condition in Theorem \ref{thm:qe_char} (b). So let $\bar{b}\in B\models \text{Th}(A)$ and $\bar{c}\in C\models \text{Th}(A)$ be such that $(B,\bar{b}) \equiv_0 (C,\bar{c})$. Note that $B$ and $C$ are elementarily equivalent to $A$ and by elementary amalgamation, together with the fact the structures are finite, we see that $A$, $B$, and $C$ are all isomorphic. Without loss of generality, assume $A=B=C$.
 			 
 			 As $(A,\bar{b}) \equiv_0 (A,\bar{c})$ in particular we have $\langle \bar{b} \rangle_A \models \Diag(\langle \bar{c} \rangle_A)$ so in particular there is an embedding $\langle \bar{b} \rangle_A$ into $\langle \bar{c} \rangle_A$ taking $b_i$ to $c_i$. Similarly there is an embedding the other way so, as the structures are finite, the embeddings are actually isomorphisms. By assumption we can extend this to an automorphism $f\colon A \to A$.
 			 
 			 Suppose $(A,\bar{b})\models \exists \bar{y}. \varphi(\bar{b},\bar{y})$ with witness $\bar{M}\in A$. Then $f(\bar{m})$ witnesses $\exists \bar{y}.\varphi(\bar{c},\bar{y})$ and hence $(A,\bar{b})\Rightarrow_1 (A,\bar{c})$.
 		\end{proof}
 		\begin{eg}
 			Let $V$ be a finite vector space (over a necessarily finite field). Any isomorphism between subspaces can be extended to an automorphism of $V$ by Steinitz lemma, so $\textnormal{Th}(V)$ has quantifier elimination.
 		\end{eg}
 		\begin{coro}\label{coro:qf_iim}
 			Let $\mathcal{T}$ be an $\mathcal{L}$-theory such that:
 			\begin{enumerate}
 				\item If $A,B\models \mathcal{T}$ with $A\subseteq B$, $\varphi(\bar{x},y)$ is a quantifier free formula, and $\bar{a}\in A$ is such that $B\models \exists y. \varphi(\bar{a}, y)$, then $A\models \exists y. \varphi(\bar{a},y)$.
 				\item For any $A\models \mathcal{T}$ and $C\subseteq A$, there is an initial intermediate model $A'\models \mathcal{T}$. This means that $C\subseteq A'\subseteq A$ and if $B\models \mathcal{T}$ with $C\subseteq B$ then there is an embedding $A' \hookrightarrow B$.
 			\end{enumerate}
 			Then $\mathcal{T}$ has quantifier elimination.
 		\end{coro}
 		\begin{proof}
 			Let $A,B\models \mathcal{T}$ and $\bar{a}\in A,\bar{b}\in B$ be such that $(A,\bar{a})\equiv_0 (B,\bar{b})$. It is enough to show that $(A,\bar{a})\Rightarrow_1 (B,\bar{b})$ by Theorem \ref{thm:qe_char} (b). Suppose $\varphi(\bar{x},\bar{y})$ is a quantifier-free $\mathcal{L}$-formula such that $A\models \exists \bar{y}. \varphi(\bar{x},\bar{y})$. We can have a witness $\bar{c} = (c_0,\ldots,c_{k-1})\in A$ such that $A\models \varphi(\bar{a},\bar{c})$.
 			
 			We claim that there is some elementary extension $B_0$ of $B$ and $d_0\in B_0$ such that $(A,\bar{a},c_0) \Rightarrow_0 (B_0,\bar{b},d_0)$, i.e. all quantifier-free formulae $\psi(\bar{x},y)$ such that $A\models \psi(\bar{a},c_0)$ also have $B_0\models \psi(\bar{b},d_0)$. Suppose, for a moment, that we can do this. Then iterate the procedure to get a sequence of extensions
 			\[
 				B \preccurlyeq B_0 \preccurlyeq \cdots \preccurlyeq B_{k-1}
 			\]
 			and elements $d_j\in B_j$ for all $j<k$ such that $(A,\bar{a},\bar{c}) \Rightarrow_0 (B_{k-1},\bar{b},\bar{d})$. Since $\varphi$ is quantifier free we must have $B_{k-1}\models \varphi(\bar{b},\bar{d})$. But then $B_{k-1} \models \exists \bar{y}.\varphi(\bar{b},\bar{y})$ and, as $B \preccurlyeq B_{k-1}$, we see that $B\models \exists \bar{y}.\varphi(\bar{b},\bar{y})$ and we are done.
 			
 			It remains to show that we can construct $B_0$. As $(A,\bar{a})\equiv (B,\bar{b})$ we have $(\langle \bar{a}\rangle_A,\bar{a}) \models \Diag (\langle \bar{b}\rangle_B,\bar{b})$ so there is an embedding $(\langle \bar{a}\rangle_A,\bar{a})$ into $(\langle \bar{b}\rangle_B,\bar{b})$, i.e. an embedding $\langle \bar{a}\rangle_A$ into $\langle \bar{b}\rangle_B$ mapping $a_i$ to $b_i$. But then the map is clearly surjective since $\bar{b}$ generates $\langle \bar{b}\rangle_B$. This all shows that there is an isomorphism $\langle \bar{a}\rangle_A \stackrel{\cong}{\to} \langle \bar{b}\rangle_B$.
 			
 			By assumption (b) there is an initial intermediate model $A'$ with respect to the extension $\langle\bar{a}\rangle_A \subseteq A$. But $\langle \bar{a}\rangle_A \stackrel{\cong}{\to} \langle \bar{b}\rangle_B \to A$ is an intermediate extension, so there is an embedding $A' \hookrightarrow \bar{b}\rangle_B$ fixing $\langle \bar{a}\rangle_A$. Without loss of generality, assume that this embedding is an inclusion.
 			
 			Let $\mathcal{L}^+$ be the language $\mathcal{L}$ with a constant for each element of $B$, separate constants for each element of $\bar{b}$, and an additional constant $d$. Consider the $\mathcal{L}^+$-theory:
 			\[
 				\Psi \coloneqq \Diag_{\text{el}}(B,\bar{b}) \cup \{\psi(\bar{b},d) \colon \psi \text{ is quantifier-free and } A \models \psi(\bar{a}, c_0) \}.
 			\]
 			A model of this theory is a structure $(B_0,\bar{b}, d_0)$ so that $(B_0,\bar{b})$ is an elementary extension of $(B,\bar{b})$ and there is $d_0\in B_0$ so that $A\models \psi(\bar{a},c_0)$ implies $B_0\models \psi(\bar{b},d_0)$ for $\psi$ quantifier-free, i.e. $(A,\bar{a},c_0) \Rightarrow_0 (B_0,\bar{b},d_0)$ as desired. So we are reduced to finding a model of $\Psi$. We do so by Compactness.
 			
 			We claim that $(B,\bar{b})$ is the domain of a model of any finite subset of $\Psi$. Indeed, it clearly satisfies $\Diag_{\text{el}}(B,\bar{b})$ so, as the conjunction of finitely many quantifier-free formulae is a single quantifier-free formula, we will be done if, given some $\psi\in \Psi$, we can find $d\in B$ with $\psi(\bar{b},d)$.
 	 			
 			As $\bar{a}\in A'$ we have that $A'\models \exists y. \psi(\bar{a},y)$ for all $\psi \in \Psi$ (this is by hypothesis (a)). Now $A'\subseteq B$ and existential formulae are preserved under extensions so $B\models \exists y. \psi(\bar{b},y)$, which is exactly what we wanted. 			
 		\end{proof}
 		\begin{eg}
 			The theory RCF of real closed fields (with signature $(+,\times, 0,1,<)$) has quantifier elimination. We will assume the following facts.
 			\begin{itemize}
 				\item Every ordered field admits a real closure, i.e. an ordered field extension that is real closed. Furthermore, this closure is unique up to unique isomorphism fixing the ground field.
 				\item All real closed fields satisfy the Intermediate Value Theorem (IVT) for polynomials. This means that if $R\models \text{RCF}$ and $f\in R[x]$ then if $f(a) < 0$ and $f(b) > 0$ for some $a<b \in R$ we have $f(c) = 0$ for some $a<c<b$.
 			\end{itemize}
 			We verify the conditions in Corollary \ref{coro:qf_iim}. Suppose we are given $A\subseteq B$ where $A,B$ are real closed fields, $\bar{a}\in A$, and a quantifier-free $\varphi(\bar{x},y)$ such that $B\models \exists y. \varphi(\bar{a},y)$. Let $b\in B$ be such that $B\models \varphi(\bar{a},b)$. By considering the Disjunction Normal Form (Example Sheet 1, question 1) we may assume that $\varphi(\bar{a,y})$ is a disjunction of conjunctions of literals.
 			
 			Pushing the existential inside the disjunction we can assume $\varphi$ isa conjunction of literals. Moreover $y\neq z$ and $\neg(y<z)$ can be written in terms of equality and $<$ so we can assume $\varphi$ is of the form
 			\[
 				\left(\bigwedge_{i<r}p_i(y) = 0\right) \wedge \left(\bigwedge_{j<s}q_j(y) > 0\right)
 			\]
 			where the $p_i$'s and $q_j$'s are polynomials in $y$ with coefficients in $A$.
 			
 			Suppose $\varphi$ contains some nontrivial equation $p_i(y) = 0$.  The witness $b$ is algebraic over $A$. This implies that $A$ already contains $b$ (\textcolor{red}{why?}) so we are done. Suppose thus that $r=0$.
 			
 			There are only finitely many points $c_0,\ldots,c_{n-1}$ in $A$ with $q_{j}(c_i)=0$ for one or more $j$ (a nonzero polynomial over a field can only have finitely many roots). Since RCFs have the IVT for polynomials the $q_j$'s can only change signs at one of these points.
 			
 			As $\mathbb{R}$ is dense, it follows that RCFs are also. Hence, as the $c_i$'s are in $A$, we have an element of $A$ between any pair of distinct $c_i$'s. Note that the $c_i$'s partition $B$ into finitely many intervals. Pick $a\in A$ such that it is in the same interval that $b$. It follows that $A\models \varphi(\bar{a},a)$ so we are done proving (a) of Corollary \ref{coro:qf_iim}.
 			
 			Now we need to prove the initial intermediate model property. Let $A$ be a real-closed field and let $C\subseteq A$ be a substructure. Clearly $C$ must be an ordered integral domain. The field of fractions $F(C)$ is naturally an ordered field by setting $\frac{a}{b}>0$ iff $ab>0$. 
 			
 			The embedding of $C$ into $A$ is an injective order-preserving ring homomorphism into an ordered field. By the universal property of $F(C)$ there's a unique order-preserving homomorphism $F(C) \to A$ extending $C \to A$.
 			
 			Let $A'$ be the real closure of $F(C)$. By the universal property of $A'$ there is an embedding $A' \to A$ extending the other embeddings. Further, $A'$ is the initial intermediate model by combining its two universal properties.  			
 		\end{eg}
 		\begin{coro}[Hilbert's Nullstellensatz]
 			Let $k$ be an algebraically closed field, and $I$ be a proper ideal of $k[x_1,\ldots,x_n]$. Then there's $\bar{a}\in k^n$ such that $f(\bar{a}) = 0$ for all $f\in I$.
 		\end{coro}
 		\begin{proof}
 			By Zorn's Lemma, we may assume that $I$ is a maximal ideal. Let $L\coloneqq \frac{k[x_1,\ldots,x_n]}{I}$ be its residue field; and $\bar{L}$ its algebraic closure. Choose a finite set of generators $I = (f_1,\ldots, f_r)$ for $I$ (this can be done by Hibert's basis theorem).
 			
 			Note that $\bar{L} \models \exists \bar{x}. f_1(\bar{x}) = 0 \wedge \cdots \wedge f_r(\bar{x}) = 0$ \textcolor{red}{(I have the same confusion as in ES2 Q4. Why is $f_i$ a term in this language?)}. We have embeddings $k\subseteq L \subseteq \bar{L}$ where $k$ and $\bar{L}$ are algebraically closed. But ACF is model-complete by quantifier elimination, and the result follows.
 		\end{proof}
 		Let $X$ be a topological space. Inductively, say that open sets are constructible and constructible sets are closed under finite Boolean operations. Below we consider $X= k^n$ with the Zariski topology.
 		\begin{coro}[Chevalley's Theorem]
 			Let $k$ be an algebraically closed field. Then the image of a constructible subset of $k^n$ under a polynomial map is constructible.
 		\end{coro}
 		\begin{proof}
			Note that, as ACF has quantifier elimination, the definable subsets of $k^n$ are finite Boolean combinations of Zariski-closed subsets, which implies that a subset of $k^n$ is constructible if and only if it is definable.
			
			Let $X\subseteq k^n$ be definable, and $p$ a polynomial map, then $p(X) =\{y\in k^m \mid \exists x. p(x) = y\}$ \textcolor{red}{(Again, why on Earth are polynomials suddenly terms?)}. This is still definable in the same language.
 		\end{proof}
 		\section{Ultraproducts}
 		\begin{defn}[Product]
 			Let $\{M_i\colon i \in I\}$ be a set of $\mathcal{L}$-structures. The \emph{product} $\prod_{i\in I}M_i$ of this family is the $\mathcal{L}$-structure whose carrier set is $\prod_{i\in I}M_i$ as a Cartesian product with the following interpretations.
 			\begin{itemize}
 				\item An $n$-ary function symbol $f$ is interpreted as a function
 				 \[
 				 	f^{\prod_{i\in I}M_i} \colon \left(\prod_{i\in I}M_i\right)^n \to \prod_{i\in I}M_i
 				 \]
 				 given by
 				 \[
 				 	(\alpha_1,\ldots,\alpha_n) \mapsto (i\mapsto f^{M_i}(\alpha_1(i),\ldots,\alpha_n(i))).
 				 \]
 				 \item An $n$-ary relation symbol $R$ as the set
 				 \[
 				 	R^{\prod_{i\in I}M_i} \coloneqq \left\{(\alpha_1,\ldots,\alpha_n)\in \left({\prod_{i\in I}M_i}\right)^n \colon (\alpha_1(i),\ldots,\alpha_n(i))\in R^{M_i} \text{ for all }i\right\}
 				 \]
 			\end{itemize}
 		\end{defn}
 		\begin{defn}[Lattices and Boolean algebras]
 			A \emph{lattice} is a set $L$ equipped with binary commutative an associative operations $\wedge$ and $\vee$ that satisfy the absorption law
 			\[
 			a= a\vee (a \wedge b)  = a\wedge (a\vee b)
 			\]
 			for all $a,b\in L$.	The lattice is called
 			\begin{itemize}
 				\item Distributive iff $a \wedge (b \vee c) = (a\wedge b) \vee (a \wedge c)$ for all $a,b,c\in L$.
 				\item Bounded iff there are elements $\top,\bot \in L$ such that $a =a \vee \bot  = a \wedge \top $ for all $a\in L$
 				\item Complemented iff it is bounded and for every element $a\in L$ there is some $a^{\ast}\in L$ such that $a\vee a^* = \top$ and $a\wedge a^* = \bot$.
 			\end{itemize}
 			A \emph{Boolean algebra} is a complemented distributive lattice.
 		\end{defn}
 		Every lattice has an partial ordering induced by the algebraic structure: define $a\leq b$ to mean $a\wedge b = a$.
 		\begin{eg}
 			\begin{enumerate}[label=\arabic*.]
 				\item For every set $I$ the power set $\mathcal{P}(I)$ is the prototypical Boolean algebra with $\wedge \coloneqq \cap$ and $\vee \coloneqq \cup$. More generally, the clopen subsets of a topological space form a Boolean algebra (and, in a sense, all Boolean algebras arise in this way. This result is called Stone's representation theorem).
 				\item For any $\mathcal{L}$-structure $M$ and subset $B \subseteq M$, the set $\{\varphi(M) \coloneqq \varphi(x)\in \mathcal{L}_B\}$ of definable subsets of $M$ with parameters in $B$ form a Boolean algebra.
 			\end{enumerate}
 		\end{eg}
 			\begin{defn}[Filters]
 				Let $X$ be a lattice. A \emph{filter} $\mathcal{F}$ on $X$ is a subset of $X$ with the following properties.
 				\begin{enumerate}[label= \arabic*.]
 					\item $\mathcal{F} \neq \emptyset$
 					\item $\mathcal{F}$ is a terminal segment of $X$, i.e., if $f\leq x$ and $f\in \mathcal{F}$ then $x\in \mathcal{F}$. 
 					\item $\mathcal{F}$ is closed under finite meets, i.e., if $x,y\in \mathcal{F}$ then $x\wedge y \in \mathcal{F}$.
 				\end{enumerate}
 			\end{defn}
 			\begin{eg}
 				\begin{enumerate}[label=\arabic*.]
 					\item Given an element $j\in I$ the family $\mathcal{F}_j$ of all subsets of $I$ containing $j$ is a \emph{principal} filter. Filters that aren't principal are called \emph{free}.
 					\item The family of all cofinite subsets of $I$ (given that $I$ is infinite) is a filter on $\mathcal{P}(I)$, called the \emph{Frechét filter}.
 					\item The family of all measurable subsets of $[0,1]$ with Lebesgue measure 1 is a filter.
 				\end{enumerate}
 			\end{eg}
 			A filter $\mathcal{F}$ is \emph{proper} if it is not the whole lattice. A maximal (under inclusion) proper filter is called an \emph{ultrafilter}. Ultrafilters on $\mathcal{P}(I)$ are exactly those filters $\mathcal{F}$ such that for all subsets $U$ of $I$ we have $U \in \mathcal{F}$ or $I\setminus U \in \mathcal{F}$. 
 			\begin{prop}[Ultrafilter principle]
 				Given a set $I$, every proper filter on $\mathcal{P}(I)$ can be extended to an ultrafilter.
 			\end{prop}
 			\begin{proof}
 				Suppose $\mathcal{F}$ is a proper filter on $\mathcal{P}(I)$. Let $\Phi$ be the set of all proper filters on $\mathcal{P(I)}$ extending $\mathcal{F}$. We show that $\Phi$ contains a maximal element under inclusion using Zorn's Lemma.
 				
 				First, note that $\mathcal{F}\in \Phi$ so $F$ is nonempty. Suppose $T\subseteq \Phi$ is a chain of proper filters all containing $\mathcal{F}$, and let $U$ be the union of all elements of $T$. We show that $U$ is still a proper filter on $\mathcal{P}(I)$ that contains $\mathcal{F}$.
 				
 				Clearly $\mathcal{F}\subseteq U$ and $U \neq \emptyset$. Let $X\in U$ and suppose $Y\supseteq X$. Then $X$ belongs to some filter in $T$, so $Y$ belongs to that filter and thus to $U$. Hence $U$ is closed under supersets. Similarly, if $X,Y\in U$ then there is a single filter in $T$ that contains both $X$ and $Y$ and thus contains $X\cap Y$. This all shows that $U$ is a filter. Now the filter $U$ is proper iff $\emptyset \notin U$. As $\emptyset \notin t$ for all $t\in T$ we see that $U$ is a proper filter containing $\mathcal{F}$.
 				
 				Zorn's Lemma applies and gives us some proper filter $\mathcal{U}$ maximal among proper filters on $\mathcal{P}(I)$ containing $\mathcal{F}$. Let $A$ be a subset of $I$. Suppose that both $\mathcal{U}\cap A$ and $\mathcal{U}\cap A^c$ don't have the finite intersection property (Example Sheet 2 Question 5). Then there is $U,U'\in \mathcal{U}$ with $A\cap U = A^c \cap U' = \emptyset$. But $U\cap U' \in \mathcal{U}$ and 
 				\[
 					\emptyset = ((U\cap U') \cap A) \cup ((U\cap U') \cap A') = U\cap U'\neq \emptyset
 				\]
 				a contradiction. Therefore one of $\mathcal{U}\cap A$ and $\mathcal{U}\cap A^c$ has the finite intersection property. By Example Sheet 2 Question 5 we can extend the one with the property to a proper filter. But by maximality of $\mathcal{U}$ the extension must be trivial, i.e., one of $A$ or $A^c$ is in $\mathcal{U}$ already. This shows that $\mathcal{U}$ is an ultrafilter, as desired.
 			\end{proof}
 			We introduce some notation. For $\bar{\alpha}\in \prod_{i\in I}M_i$ and $\varphi(\bar{x})$ an $\mathcal{L}$-formula we write
 			\[
 				[\varphi(\bar{\alpha})] \coloneqq \{i\in I \colon M_i \models \varphi(\bar{\alpha}(i))\}.
 			\]
 			\begin{defn}[Reduced product]
 				Let $I$ be a set, $\mathcal{F}$ a filter on $\mathcal{P}(I)$, and $\mathcal{M}_i$ and $\mathcal{L}$-structure for all $i\in I$.
 				
 				The carrier set for the \emph{reduced product} $\sfrac{\prod_{i\in I}{M_i}}{\mathcal{F}}$ is the quotient of the Cartesian product $\prod_{i\in I}{M_i}$ by the equivalence relation defined by 
 				\[
 					\alpha \sim \beta \iff [\alpha = \beta]\in \mathcal{F}.
 				\]
 				Write $\langle \alpha \rangle$ for the class represented by $\alpha$. The interpretation of an n-ary function symbol $f$ is given by
 				\[
 					f^{\sfrac{\prod_{i\in I}M_i}{\mathcal{F}}} (\langle \alpha_1\rangle, \ldots, \langle \alpha_n\rangle) \coloneqq \langle i \mapsto f^{M_i}(\alpha_1(i), \ldots,\alpha_n(i))\rangle.
 				\] 
 				An $n$-ary relation symbol $R$ is interpreted by the rule
 				\[
 					(\langle \alpha_1\rangle, \ldots, \langle \alpha_n\rangle) \in R^{\sfrac{\prod_{i\in I} M_i}{\mathcal{F}}}  \iff [R(\alpha_1,\ldots,\alpha_n)]\in \mathcal{F}.
 				\]
 				These turn out to be well-defined, so the reduced product is an $\mathcal{L}$-structure. If $\mathcal{F}$ is an ultrafilter we call the reduced product an \emph{ultraproduct}. If, in addition, all the $M_i$'s are equal, we call it an \emph{ultrapower}.
 			\end{defn}
 			Note that if the filter is principal then the reduced product is basically just the distinguished structure.
 			\begin{thm}[\L o\'s's Theorem]
 				Let $\{M_i\colon i\in I\}$ be a family of structures and $\mathcal{U}$ an ultrafilter on $\mathcal{P}(I)$. Then for every $(\langle \alpha_1\rangle, \ldots, \langle \alpha_n\rangle)\in \left(\sfrac{\prod_{i\in I}M_i}{\mathcal{U}}\right)^n$ and $\mathcal{L}$-formula $\varphi(x_1,\ldots,x_n)$ we have that
 				\[
 					\sfrac{\prod_{i\in I}M_i}{\mathcal{U}} \models \varphi(\langle \alpha_1\rangle, \ldots, \langle \alpha_n\rangle)
 				\]
 				if and only if
 				\[
 					[\varphi(\langle \alpha_1\rangle, \ldots, \langle \alpha_n\rangle)] \in \mathcal{U}.
 				\]
 				In particular, if, for all $i$, $M_i$ is the model of a theory $\mathcal{T}$ then $\sfrac{\prod_{i\in I}M_i}{\mathcal{U}} \models \mathcal{T}$.
 			\end{thm}
 			\begin{proof}
 					By induction on the structure of $\phi$. The case for atomic formulae follows from the definition.
 					
 					Suppose the claim holds for $\psi(\bar{x})$ and $\varphi(\bar{x}) = \neg\psi(\bar{x})$. Then 
 					\begin{align*}
 						\sfrac{\prod_{i\in I}M_i}{\mathcal{U}} \models \neg\psi(\langle \alpha_1\rangle, \ldots, \langle \alpha_n\rangle) \,\,\, &\text{ iff }\,\,\, \sfrac{\prod_{i\in I}M_i}{\mathcal{U}} \nvDash \psi(\langle \alpha_1\rangle, \ldots, \langle \alpha_n\rangle)\\
 						&\text{ iff }\,\,\,  [\psi(\langle \alpha_1\rangle, \ldots, \langle \alpha_n\rangle)] \notin \mathcal{U}\\
 						&\text{ iff }\,\,\, I\setminus  [\psi(\langle \alpha_1\rangle, \ldots, \langle \alpha_n\rangle)] \in \mathcal{U}\\
 						&\text{ iff }\,\,\, [\neg\psi(\langle \alpha_1\rangle, \ldots, \langle \alpha_n\rangle)] \in \mathcal{U}\\
 					\end{align*}
 				where we have used the fact that, since $\mathcal{U}$ is an ultrafilter, it either contains a given subset of $I$ or its complement. Now assume the claim holds for $\psi_1(\bar{x})$ and $\psi_2(\bar{x})$ and suppose $\varphi(\bar{x}) = \psi_1(\bar{x})\vee \psi_2(\bar{x})$.
 				\begin{align*}
 					\sfrac{\prod_{i\in I}M_i}{\mathcal{U}} \models \varphi(\langle \alpha_1\rangle, \ldots, \langle \alpha_n\rangle) \,\,\,&\text{ iff }\,\,\, \sfrac{\prod_{i\in I}M_i}{\mathcal{U}} \models \psi_1(\langle \alpha_1\rangle, \ldots, \langle \alpha_n\rangle) \vee \psi_2(\langle \alpha_1\rangle, \ldots, \langle \alpha_n\rangle)\\
 					&\text{ iff }\,\,\, \sfrac{\prod_{i\in I}M_i}{\mathcal{U}} \models \psi_1(\langle \alpha_1\rangle, \ldots, \langle \alpha_n\rangle)\text{ or }\sfrac{\prod_{i\in I}M_i}{\mathcal{U}} \models \psi_2(\langle \alpha_1\rangle, \ldots, \langle \alpha_n\rangle)\\
 					&\text{ iff }\,\,\,  [\psi_1(\langle \alpha_1\rangle, \ldots, \langle \alpha_n\rangle)] \in \mathcal{U} \text{ or }[\psi_2(\langle \alpha_1\rangle, \ldots, \langle \alpha_n\rangle)] \in \mathcal{U}\\
 					&\text{ iff }\,\,\, [\psi_1(\langle \alpha_1\rangle, \ldots, \langle \alpha_n\rangle)] \cup [\psi_2(\langle \alpha_1\rangle, \ldots, \langle \alpha_n\rangle)]  \in \mathcal{U}\\
 					&\text{ iff }\,\,\, [\psi_1(\langle \alpha_1\rangle, \ldots, \langle \alpha_n\rangle)\vee \psi_2(\langle \alpha_1\rangle, \ldots, \langle \alpha_n\rangle)] \in \mathcal{U}\\
 				\end{align*}
 				where we have used the fact that in an ultrafilter $\mathcal{U}$ we have $A\cup B \in \mathcal{U}$ iff $A \in \mathcal{U}$ or $B\in \mathcal{U}$. 
 				
 				Finally, suppose the claim holds for $\psi(\bar{x},y)$ and $\varphi(\bar{x}) = \exists y. \psi(\bar{x},y)$. If we have 
 				\[
 					\sfrac{\prod_{i\in I}M_i}{\mathcal{U}} \models \psi(\langle \alpha_1\rangle, \ldots, \langle \alpha_n\rangle, \langle \alpha \rangle)
 				\]
 				for some $\alpha$ then, by inductive hypothesis, we have $[\psi(\langle \alpha_1\rangle, \ldots, \langle \alpha_n\rangle, \langle \alpha \rangle)] \in \mathcal{U}$. But
 				\[
 					[\psi(\langle \alpha_1\rangle, \ldots, \langle \alpha_n\rangle, \langle \alpha \rangle)] \subseteq [\exists y. \psi(\langle \alpha_1\rangle, \ldots, \langle \alpha_n\rangle, y)]
 				\]	
 				so, as $\mathcal{U}$ is closed under supersets, we see that $[\varphi(\langle \alpha_1\rangle, \ldots, \langle \alpha_n\rangle)]\in\mathcal{U}$ and we have one direction.
 				
 				Conversely, suppose that $[\exists y. \psi(\langle \alpha_1\rangle, \ldots, \langle \alpha_n\rangle, y)]\in \mathcal{U}$. Using the axiom of choice, we can pick a witness $\alpha(i)\in M_i$ to the formula for each $i\in \mathcal{U}$; if $i\neq \mathcal{U}$ let $\alpha(i)$ be any element of $M_i$. Hence we have constructed an element $\langle \alpha\rangle\in \sfrac{\prod_{i\in I}M_i}{\mathcal{U}}$ that serves as a witness to
 				\[
 					\sfrac{\prod_{i\in I}M_i}{\mathcal{U}} \models \exists y. \psi(\langle \alpha_1\rangle, \ldots, \langle \alpha_n\rangle, y)
 				\] 
 			\end{proof}
 			\begin{eg}
 				The class of torsion abelian groups is not first-order axiomatizable (in the language with signature (+,0)). Suppose, for the sake of contradiction, that $\mathcal{T}$ axiomatized this class.
 				
 				For each $i\in\mathbb{Z}^+$ let $C_i$ denote the cyclic group of order $i$. Clearly these are all models of $\mathcal{T}$. Fix some free ultrafilter $\mathcal{U}$ on $\mathbb{Z}^+$. Form the ultraproduct
 				\[
 					G\models \sfrac{\prod_{i \in \mathbb{Z}^+} C_i}{\mathcal{U}}.
 				\]
 				If we write $g_i$ for the generator of $C_i$ we can define $g\coloneqq \langle i \mapsto g_i\rangle \in G$. We ask whether $g$ has finite order. This happens if there is some positive integer $n$ such that $[ng = 0]\in \mathcal{U}$.
 				
 				But for any fixed $n$ we have $ng_i = 0$ for finitely many $i$ and any free ultrafilter on an infinite set contains the Fréchet filter, and thus doesn't contain any finite set. Hence $g$ has infinite order, i.e. $G$ is not torsion even though $G\models \mathcal{T}$.
 			\end{eg}
 			\begin{eg}
 				Fix a free ultrafilter $\mathcal{U}$ on $\omega$ and consider the ultrapower
 				\[
 					\mathbb{N}^{\mathcal{U}} \coloneqq \sfrac{\prod_{i < \omega} \mathbb{N}}{\mathcal{U}}.
 				\]
 				Its elements are equivalence classes of sequences of natural umbers. If $\mathbb{N}$ carries its usual structure for the language of arithmetic then $\mathbb{N}^{\mathcal{U}}$ is a nonstandard model of arithmetic, by \L o\'s's Theorem.
 			\end{eg}
 			\begin{eg}
 				Similarly, $\mathbb{R}^{\mathcal{U}}$ is an elementary extension of $\mathbb{R}$. It includes `large numbers' bigger than any standard real (e.g., $\omega = \langle (n)_{n<\omega}\rangle$) and also infinitesimal numbers such as $\frac{1}{\omega}$. Infinitesimals are positive but smaller than any positive standard real number.
 			\end{eg}
 			We can now give a proof of the Compactness Theorem that doesn't rely on the Completeness Theorem.
 			\begin{coro}[Compactness]
 				Let $\mathcal{T}$ be a first-order theory such that every finite subset of $\mathcal{T}$ has a model. Then $\mathcal{T}$ itself has a model.
 			\end{coro}
 			\begin{proof}
 				Assume $\mathcal{T}$ is infinite since otherwise the claim is trivial. Let $I$ be the set of all finite subtheories of $\mathcal{T}$. For each axiom $\varphi$ in $\mathcal{T}$ let $X_{\varphi}\coloneqq \{\Delta \in I \mid \varphi \in \Delta\}$ be the set of all finite subtheories of $\mathcal{T}$ containing $\varphi$. 
 				
 				Let $\varphi_1,\ldots,\varphi_n\in \mathcal{T}$ and note $\bigcap_{i = 1}^n X_{\varphi_i}$ is nonempty since it contains $\{\varphi_1,\ldots,\varphi_{n}\}\in I$. Thus the set $X$ of all sets of the form $X_{\varphi}$ has the finite intersection property. By Example Sheet 2 Question 5 (a), there is a proper filter $D$ on $\mathcal{P}(I)$ containing $X$. By the Ultrafilter Principle, we can extend $D$ to an ultrafilter $\mathcal{U}$.
 				
 				Use the Axiom of Choice to pick a model $M_{\Delta}$ for each $\Delta\in I$, which we can do by hypothesis. For any $\varphi\in\mathcal{T}$ the and $\Delta \in I$ so that $\varphi \in \Delta$ we obviously have that $M_{\Delta} \models \varphi$. This shows that $X_{\varphi} \subseteq \{\Delta \in I \mid M_{\Delta} \models \varphi\}$ and, as $\mathcal{U}$ is closed under supersets, we have $\{\Delta \in I  \mid M_{\Delta} \models \varphi\} \in \mathcal{U}$ for all $\varphi\in \mathcal{T}$. \L o\'s's Theorem then implies that for all $\varphi\in\mathcal{T}$ we have
 				\[
 					\sfrac{\prod_{\Delta \in I} M_{\Delta}}{\mathcal{U}} \models \varphi.
 				\]
 				In other words, $\sfrac{\prod_{\Delta \in I} M_{\Delta}}{\mathcal{U}}$ is a model of $\mathcal{T}$. 				
 			\end{proof}
 	\section{Types}
 	Given a theory we would like to investigate its models; in particular, we would like to describe the sort of elements the models can and cannot posses. This leads to the notion of types
 	\begin{defn}[Definable sets]
 		Let $M$ be an $\mathcal{L}$-structure, $X\subseteq M^n$, and $P \subseteq M$ be a subset. We say that $X$ is \emph{definable with parameters in }$P$ if there is $\bar{p}\in P$ and some $\mathcal{L}_P$-formula $\varphi(\bar{x},\bar{y})$ such that
 		\[
 			X = \varphi(M,\bar{p}) \coloneqq \{\bar{m}\in M^n \colon M \models \varphi(\bar{m},\bar{p})\}.
 		\]
 	\end{defn}
 	\begin{eg}
 		Consider the natural numbers $\mathbb{N}$ as a structure for the language $\mathcal{L}$ with signature $(+,\cdot, 0,1)$. There is an $\mathcal{L}$-formula $T(e,x,s)$ such that $\mathbb{N}\models T(e,x,s)$ if and only if the Turing machine of code $e$ halts on input $x$ in at most $s$ steps. This implies that the set of all halting programmes is definable by $\exists s. T(e,x,s)$ (although not computable).
 	\end{eg}
 	Let $M$ be an $\mathcal{L}$-structure. We would like $M$ to have an element with particular properties. For example $\mathbb{R}$ to have infinitesimals (resulting in hyperreal numbers) or $\mathbb{N}$ to have infinitely large numbers. Though these structures don't themselves contain such elements, we would like to construct an extension of them that do. We can do this using Compactness. For example, an infinitesimal in $\mathbb{R}$ would have to satisfy the formulae
 	\[
 		0< x < 1,\,\,\, 0<x<\frac{1}{1+1},\,\,\, 0<x<\frac{1}{1+ 1 +1},\ldots
 	\]
 	which are expressible in the language of ordered fields. The above aren't sentences but if we ignore this fact for a moment we could take the collection of these formulae together with some suitable sentences to build a theory that will be consistent by Compactness ($\mathbb{R}$ would be a model for any finite subset of the theory). Hence we would obtain the desired element in some extension of $\mathbb{R}$.
 	
 	The idea is to first describe an element by a set $S$ of formulae just as we did above. These formulae generate a filter as follows. First, close $S$ under conjunction: if two formulae are satisfied by this ideal element then their conjunction is satisfied by the element too. Second, close $S$ under implication: if $\varphi(x)$ is satisfied by the ideal element and $\varphi(x)\implies\psi(x)$ then $\psi(x)$ is also satisfied by the ideal element. Hence we obtain a filter. Only proper filters are interesting since we don't want $S$ to contain or imply $\bot$. We might as well quotient out by formulae which the structure think are equivalent.
 	\begin{defn}[Lindenbaum-Tarski algebra and types]
 		Let $\mathcal{T}$ be a theory and $n\in \mathbb{N}$. We obtain an equivalence relation $\sim$ on the set $\mathcal{L}(\bar{x})$ or $\mathcal{L}$ formulae with $n$ free variables $\bar{x}$ by setting
 		\[
 			\varphi(\bar{x}) \sim \psi(\bar{x})\,\,\, \text{ iff }\,\,\, \mathcal{T} \vdash \forall x. (\varphi(\bar{x})\iff \psi(\bar{x}))
 		\]
 		The quotient $B_n(\mathcal{T}) \coloneqq \sfrac{\mathcal{L}(\bar{x})}{\sim}$ becomes a Boolean algebra by setting 
 		\begin{itemize}
 			\item $[\varphi] \wedge [\psi] \coloneqq [\varphi \wedge \psi]$
 			\item $[\varphi]\vee [\psi] \coloneqq [\varphi \vee \psi]$
 			\item $[\varphi]^* \coloneqq [\neg \varphi]$.
 		\end{itemize}
 		We call $B_n(\mathcal{T})$ the \emph{Lindenbaum-Tarski algebra of $\mathcal{T}$ on variables $\bar{x}$}.
 	\end{defn}
 	Now let $M$ be an $\mathcal{L}$-structure and $A\subseteq M$ a subset. In the special case where $\mathcal{T} \coloneqq \textnormal{Th}_A(M)$ is collection of all sentences with parameters in $A$ satisfied by $M$ we call the proper filters on $B_n(\mathcal{T})$ the $n$-\emph{types} of $M$ over $A$. We can alternatively define types as follows.
 	\begin{defn}[Types]
 		Let $M$ be an $\mathcal{L}$-structure and $A \subseteq M$. A set $p$ of $\mathcal{L}_A$ formulae with $n$ free variables $\bar{x}$ is an $n$\emph{-type of $M$} if, for all finite subsets $\Psi$ of $p$ the theory
 		\[
 			\Th_{A}(M)\cup \left\{\exists \bar{x}. \bigwedge_{\psi\in \Psi} \psi\right\}
 		\] 
 		is consistent (and thus has a model). More generally, we can replace $\Th_A(M)$ with a theory $\mathcal{T}$ above to define $n$-types of $\mathcal{T}$.
 		
 		An $n$-type $p$ is called \emph{complete} if it is maximal among the collection of $n$-types, in the sense that for every $\mathcal{L}$-formulae $\varphi(\bar{x})$ either $\varphi\in p$ or $\neg\varphi\in p$. We denote the set of complete $n$-types by $S_n(\mathcal{T})$, or $S_n^M(A)$ if $T = \textnormal{Th}_A(M)$. An element $\bar{m}\in M^n$ \emph{realises} a an $n$-type $p$ in $M$ if $M\models \varphi(\bar{m})$ for all $\varphi(\bar{x})\in p$. If no elements realize a type we say that the type is omitted in $M$.
 	\end{defn}
 	
 	To give another alternative definition, say we have an $\mathcal{L}$-theory $\mathcal{T}$ and a set $p$ of $\mathcal{L}$-formulae with $n$ free variables $\bar{x}$. Then $p$ is an $n$-type of $\mathcal{T}$ iff in the expanded language $\mathcal{L}_{\bar{x}}$ the theory $\mathcal{T} \cup p$ is consistent.
 	
 	Complete types correspond to ultrafilters under the old definition. Recall that any proper filter can be extended to an ultrafilter. Complete types are named so because they give a complete description of the element.
 	\begin{eg}
 		\begin{enumerate}
 		\item Let $M= (\mathbb{Q},<)$ and consider the formulae $n<x$ for each natural number $n$. This collection of formulae is a 1-type of $M$ over $\mathbb{N}$ as every finite subset of it is consistent. This type is omitted in $M$ but, as we will see, it is realized in an elementary extension of $\mathbb{Q}$.
 		\item Similarly the formulae $0<x<1/n$ form a 1-type of $\mathbb{R}$ over $\mathbb{N}$. This type is omitted in $\mathbb{R}$ but realized in, for example, an ultrapower with respect to a free ultrafilter.
 		\item For any $\mathcal{L}$-structure $M$, subset $A\subseteq M$, and tuple $\bar{m}\in M$, we can form the $n$-type of all $\mathcal{L}_A$-formulae that hold in $M$ of $\bar{m}$.
 		\[
 			\text{tp}^M(\bar{m}/A)\coloneqq \{\varphi(\bar{x})\in \mathcal{L}_A\colon M \models \varphi(\bar{m})\}.
 		\]
 		This is a complete $n$-type called the \emph{type of $\bar{m}$ over $A$}.
 	\end{enumerate}
 	\end{eg}
 	\begin{prop}\label{prop:type_real_ext}
 		Let $M$ be an $\mathcal{L}$-structure with $A \subseteq M$ and let $p$ be an $n$-type of $M$ over $A$. Then there is an elementary extension $N$ of $M$ that realizes $p$.
 	\end{prop}
 	\begin{proof}
 		Let $\bar{c}$ be an $n$-tuple of constants (disjoint from $M$) and consider the language $\mathcal{L}^+\coloneqq \mathcal{L}_{M\cup \bar{c}}$. It suffices to show that the $\mathcal{L}^+$-theory
 		\[
 			\Gamma \coloneqq \Diag_{\text{el}}(M) \cup p
 		\]
 		is consistent, where $p$ is interpreted as an $\mathcal{L}_{\bar{c}}$-theory. Let $\Delta$ be a finite subset of $\Gamma$. Define $\phi(\bar{m},\bar{a})$ to be the conjunctions of all sentences in $\Delta \cap \Diag_{\text{el}}(M)$, where $\bar{m}\in M\setminus A$, and similarly define $\psi(\bar{c},\bar{a}')$ to be the conjunction of all formulae in $\Delta \cap p$. We are done if we can find a model of $\phi \wedge \psi$.
 		
 		By definition of type, there is an $\mathcal{L}_A$-structure $N$ satisfying $\Th_{A}(M)\cup p$, so $N\models \psi$. By definition we have $M\models \phi(\bar{m},\bar{a})$, so in particular we have $M\models \exists \bar{y}. \phi(\bar{y},\bar{a})$, which is an $\mathcal{L}_A$ sentence. It follows that $N \models \exists \bar{y}. \phi(\bar{y},\bar{a})$, so we can interpret $\bar{m}$ in $N$ as a witness to this statement (and interpret all other constants as whatever). Hence $N$ is an $\mathcal{L}^+$ structure with $N\models \phi\wedge \psi$.
 	\end{proof}
 	\begin{coro}
 		An $n$-type $p$ of $M$ over $A\subseteq M$ is complete if and only if there is an elementary extension $N$ of $M$ and some $\bar{a}\in N$ such that $p = \text{tp}^N(\bar{a}/A)$
 	\end{coro}
 	\begin{proof}
 		Let $N$ be an elementary extension of $M$, and $\bar{a}\in N$ be such that $p = \text{tp}^N(\bar{a}/A)$. If $\phi(\bar{x})$ is an $\mathcal{L}_A$-formula then either $N \models \phi(\bar{a})$, in which case $\phi(\bar{x})\in p$, or $N\models \neg\phi(\bar{a})$, in which case $\phi(\bar{x})\notin p$, so $p$ is complete.
 		
 		Conversely, suppose $p$ is a complete $n$-type. By Proposition \ref{prop:type_real_ext} there is an extension $N$ of $M$ realizing $p$; let $\bar{a}\in N$ realize $p$, i.e., $p\subseteq \text{tp}^N(\bar{a}/A)$. Let $\phi(\bar{x})$ be an $\mathcal{L}_A$-formula with $N\models \phi(\bar{a})$ . Then, as we cannot have $\neg\phi \in p$, we must have $\phi\in p$ since $p$ is complete; hence $\text{tp}^N(\bar{a}/A)\subseteq p$. 
 	\end{proof}
 	Let $M$ be an $\mathcal{L}$-structure and $A \subseteq M$. As we've seen, complete types of $M$ over $A$ can be thought of as elements in extensions of $M$. Hence the set of all types over $M$ forms a kind of space, populated by these ideal elements.
 	
 	This idea can be made more precise, since we can topologize this set. Two types are thought to be close to each other when their realizations satisfy some of the same $\mathcal{L}_A$-formulae. This motivates the following definition.
 	
 	For each $\mathcal{L}_A$ formula $\varphi(x_1,\ldots,x_n)$ we consider the set
 	\[
 		\llbracket \varphi \rrbracket\coloneqq \{p \in S_n^M(A) \colon \varphi\in p\}.
 	\]
 	
 	Note that $\llbracket \varphi \vee \psi\rrbracket = \llbracket \varphi \rrbracket \cup \llbracket \psi \rrbracket$ and similarly $\wedge$ corresponds to $\cap$. Hence these sets form a basis for a topology on $S_n^M(A)$. Moreover each $\llbracket \varphi \rrbracket = S_n^M(A) \setminus \llbracket \neg \varphi\rrbracket$ is closed, so the topology is generated by clopen sets.
 	
 	A \emph{Stone space} is a compact, totally disconnected, topological space. Turns out that $S_n^M(A)$ is always a Stone space.
 	\begin{eg}
 		Let $F\models \text{ACF}$ and $k$ be a subfield of $F$. The complete $n$-types $p\in S_n^{F}(k)$ are determined by the prime ideals of $k[x_1,\ldots,x_n]$. For such a type $p$ define
 		\[
 			I_p \coloneqq \{f\in k[x_1,\ldots,x_n] \colon (f(\bar{x}) = 0)\in p\}.
 		\]
 		Each $I_p$ is a prime ideal and in fact every prime ideal is of this form (\textcolor{red}{why?}). The map $p\mapsto I_p$ is a continuous bijection (but not a homeomorphism) $S_n^F(k) \to \text{Spec}(k[x_1,\ldots,x_n])$ (\textcolor{red}{why?}), where the latter is the set of prime ideals of $k[x_1,\ldots,x_n]$ with the Zariski topology. Also we have a bound (\textcolor{red}{why?})
 		\[
 			|S_n^F(k)|\leq |k| + \aleph_0.
 		\]
 	\end{eg}
 		Recall that a point $p$ of a topological space is \emph{isolated} if $\{p\}$ is an open set. In the case $p\in S_n^F(k)$ we must have
 		\[
 			\{p\} = \bigcup_{i\in I}\llbracket\varphi_i\rrbracket
 		\]
 		for some $I$. It follows that $\{p\} = \llbracket \varphi\rrbracket$ for some $\varphi$. We say that $\varphi$ \emph{isolates} $p$. We generalize this to more general types as follows.
 		\begin{defn}
 			Let $\mathcal{T}$ be an $\mathcal{L}$-theory. A formula $\varphi(x_1,\ldots,x_n)$ \emph{isolates} an $n$-type $p$ if $\mathcal{T}\cup \{\varphi\}$ is satisfiable and
 			\[
 				\mathcal{T}\models \forall \bar{x}.(\varphi(\bar{x}) \Rightarrow \psi(\bar{x}))
 			\]
 			for all $\psi\in p$.
 		\end{defn}
 		\begin{prop}
 			If $\varphi$ isolates an $n$-type $p$, then $p$ is realized in any model of $\mathcal{T}\cup \{\exists \bar{x}. \varphi(\bar{x})\}$. In particular, if $\mathcal{T}$ is a complete theory, then every isolated type is realized in all models of $\mathcal{T}$.
 		\end{prop}
 		\begin{proof}
	 		Let $M$ be a model of $\mathcal{T}\cup \{\exists \bar{x}. \varphi(\bar{x})\}$, and $\bar{m}\in M$ be such that $M\models \varphi(\bar{m})$. Then $M\models \psi(\bar{m})$ for all $\psi\in p$; hence $\bar{m}$ realizes $p$.
	 		
	 		Now assume that $\mathcal{T}$ is complete. Either $\exists \bar{x}. \varphi(\bar{x})$ or its negation is implied by $\mathcal{T}$. But $\mathcal{T}\cup\{\exists \bar{x}. \varphi(\bar{x})\}$ is consistent by definition. The claim follows.
 		\end{proof}
 		\begin{thm}
 			Let $\mathcal{L}$ be a countable language, $\mathcal{T}$ an $\mathcal{L}$-theory, and $p$ a non-isolated $n$-type of $\mathcal{T}$. Then there is a countable model $M\models \mathcal{T}$ that omits $p$.
 		\end{thm}
 		\begin{proof}
 			As the theory $\mathcal{T}$ has a type it is, in particular, consistent. Let $C=\{c_0,c_1,\ldots\}$ be a countable set of new constants. We expand $\mathcal{T}$ to a consistent $\mathcal{L}_C$-theory $\mathcal{T}^*$ by adding recursively defined sentences $\theta_0,\theta_1,\ldots$. This will be done in such a way that if $s<t$, then $\theta_t$ implies $\theta_s$.
 			
 			Enumerate the $n$-tuples $C^n = \{\bar{d}_0,\bar{d}_1,\ldots\}$ as well as the $\mathcal{L}_C$-sentences $\varphi_0,\varphi_1,\ldots$. Start with $\theta_0\coloneqq \forall x. x=x$ and suppose we already constructed $\theta_s$ such that $\mathcal{T}\cup\{\theta_s\}$ is consistent. We define $\theta_{s+1}$ as follows.
 			\begin{itemize}
 				\item If $s = 2i$ is even then we will design $\theta_{s+1}$ to turn $C$ into (the domain of) an elementary substructure of a model of $\mathcal{T}$.
 				
 				Suppose $\varphi_i$ is existential so that $\varphi_i = \exists x. \psi(x)$, and $\mathcal{T}\vdash \theta_{s} \Rightarrow \varphi_i$. As only finitely many constants from $C$ have been used thus far in the construction of the $\theta$'s we can find some unused $c\in C$. Let $\theta_{s+1}\coloneqq \theta_{s} \wedge \psi(c)$.
 				
 				We know that $\mathcal{T}\cup \{\theta_s\}$ is consistent, so let $N$ be a model for it. As $\mathcal{T}\vdash \theta_{s} \Rightarrow \varphi_i$ there is a witness $n\in N$ to $\psi$. By interpreting $c$ in $N$ as that witness we have $N\models \Theta_{s+1}$. So $\mathcal{T}\cup \{\theta_{s+1}\}$ is consistent.
 				
 				If $\varphi_i$ is not existential or $\mathcal{T}\nvdash \theta_{s} \Rightarrow \varphi_i$ then set $\theta_{s+1}\coloneqq \theta_{s}$.
 				
 				\item If $s= 2i + 1$ is odd we will design $\theta_{s+1}$ to guarantee that $C$ omits $p$. Write $\bar{d}_i = (e_1,\ldots, e_n)$. Remove every occurrence of $e_j$ from $\theta_s$ by replacing it with the variable $x_j$, and every occurrence other constants, say $c\in C$, by replacing them with a fresh variable $x_c$ together with a $\exists x_c$ in front of the formula.
 				
 				This yields a new formula $\psi(x_1,\ldots,x_n)$. As $p$ is not isolated there must be some $\varphi(\bar{x})$ that is not implied by $\psi(\bar{x})$ according to $\mathcal{T}$. Define $\theta_{s+1} \coloneqq \theta_s \wedge \neg\varphi(\bar{d}_i)$. Then $\mathcal{T}\cup\{\theta_{s+1}\}$ is consistent since $\mathcal{T}\cup\{\theta_{s}\}$ is consistent and $\exists \bar{x}. \neg\varphi(\bar{x})$ is independent of it.
 			\end{itemize}
 				
 			Let $\mathcal{T}^*\coloneqq \mathcal{T}\cup \{\theta_0,\theta_1,\theta_2,\ldots\}$. By Compactness, it follows that $\mathcal{T}^*$ is consistent. Let $M$ be a model for this theory. 
 				
 				It follows that $M$ has a countable subset $C$, and the construction of $\Theta_{2i+1}$ ensures that if $\varphi_i$ is an existential statement holding in $M$ then it holds in $C$. Thus $C$ is an elementary substructure of $M$ by the Tarski-Vaught test.
 				
 				Moreover, if $\bar{c}\in \mathcal{T}^*$ then $\bar{c} = \bar{d}_i$ for some $i$. As $C\models \theta_{2i + 2}$ we have that $\neg\varphi(\bar{c})$ for some $\varphi\in p$. So $\bar{c}$ cannot realise $p$ in $C$.	As $C\models \mathcal{T}$, we are done.
 			
 		\end{proof}
 		\section{Indiscernibles}
 		Given a linear order $\eta$ we write $[\eta]^k$ to be the set of ordered $k$-tuples of $\eta$, i.e.,
 		\[
 			[\eta]^k \coloneqq \{\bar{a}\in \eta^k \colon a_0 < a_1<\cdots < a_{k-1}\}.
 		\]
 		\begin{defn}[Indiscernibles]
 			Let $M$ be an $\mathcal{L}$-structure, $\eta$ a chain of elements of $M$, and $\Phi$ be a set of $\mathcal{L}$-formulae. We say $\eta$ is $\Phi$\emph{-indiscernible} in $M$ if
 			\[
 				M \models \varphi(\bar{a}) \iff \varphi(\bar{b})
 			\]
 			for all $\bar{a},\bar{b}\in [\eta]^k$ and $\varphi \in \Phi$. If $\Phi$ is the set of all $\mathcal{L}$-formulae, we simply say that $\eta$ is a sequence of \emph{indiscernibles}.
 		\end{defn}
 		\begin{eg}
 			\begin{enumerate}
 				\item Any ordered basis $B$ of a vector space provides a sequence of indiscernibles: given $\bar{a},\bar{b}\in [B]^k$ there is an automorphism mapping $\bar{a}$ to $\bar{b}$.
 				\item Similarly, any chain of algebraically independent elements of some algebraically closed field of characteristic zero is a sequence of indiscernibles.
 				\item If $R$ is a ring then the variables $x_1,x_2,\ldots,x_n$ form a set of indiscernibles of $R[x_1,\ldots,x_n]$. 
 			\end{enumerate}
 		\end{eg}
 		\begin{defn}[Ehrenfeucht–Mostowski functors]
 			An \emph{Ehrenfeucht–Mostowski functor} (henceforth abbreviated E.M. functor) is a functor $F$ from the category of linear orders and order embeddings to the category of $\mathcal{L}$-structures with structure embeddings which satisfies the following conditions.
 			\begin{itemize}
 				\item For all linear orders $\eta$ the structure $F(\eta)$ has (a copy of) $\eta$ as a subset and $F(\eta) = \langle \eta\rangle_{F(\eta)}$.
 				\item For each embedding $\eta \hookrightarrow \epsilon$ of linear orders, the corresponding embedding $F(\eta) \hookrightarrow F(\epsilon)$ extends it.
 			\end{itemize}
 		\end{defn}
 		This can be rephrased in purely categorical terms. Let $\mathcal{C}$ denote the category of linear order embeddings, $\mathcal{D}$ the category of $\mathcal{L}$-structure embeddings, and $\mathcal{E}$ the category of set injections. 
 		
 		There are obvious forgetful functors $U_\mathcal{C}\colon \mathcal{C}\to \mathcal{E}$ and $U_\mathcal{D}\colon\mathcal{D}\to \mathcal{E}$. Let $F \colon \mathcal{C} \to \mathcal{D}$. Then specifying a natural transformation $\alpha \colon U_\mathcal{C} \to U_{\mathcal{D}}F$ is the same as specifying an embedding of $\eta$ to $F(\eta)$ as sets for all $\eta$, and naturality is precisely the statement that embeddings $\eta \hookrightarrow \epsilon$ are extended as embeddings $F(\eta)\hookrightarrow F(\epsilon)$.
 		
 		Now for an $\mathcal{L}$-structure $M$ we consider the comma category $\mathcal{D}_M$ of structure-embeddings into $M$. Similarly we have the category $\mathcal{E}_{U_{\mathcal{D}}(M)}$ of set-injections into $U_{\mathcal{D}}(M)$; by an abuse of notation we denote this latter category by $\mathcal{E}_M$. The forgetful functor $U_{\mathcal{D}}\colon \mathcal{D}\to\mathcal{E}$ clearly restricts (or rather induces) a functor $U_{\mathcal{D},M}\colon \mathcal{D}_M\to\mathcal{E}_M$. This functor has a left adjoint $\langle - \rangle_M\colon \mathcal{E}_M \to \mathcal{D}_M$ (and what it does is obvious by the name).
 		
 		The natural transformation $\alpha$ gives for each linear order $\eta$ an injection $\alpha_\eta\colon U_{\mathcal{C}}(\eta) \to U_{\mathcal{D}}F(\eta)$. Taking $M = F(\eta)$ we see that this is an element of $\mathcal{E}_M$ and thus we can require that $\langle - \rangle_M(\alpha_{\eta})$ is an isomorphism (rather than only an embedding). This guarantees that $\eta$ generates $F(\eta)$. 
 		
 		To summarize, an E.M. functor is a functor $F\colon \mathcal{C}\to \mathcal{D}$ together with a natural transformation $\alpha \colon U_\mathcal{C} \to U_{\mathcal{D}}F$ such that for all linear orders $\eta$ the map $\langle - \rangle_{F(\eta)} (\alpha_\eta)$ is an isomorphism.
 		
 		\begin{prop}[Slicing]
 			Let $F$ be an E.M. functor, $\eta,\epsilon$ linear orders, and $\bar{a}\in[\eta]^k$,$\bar{b}\in[\epsilon]^k$. For every quantifier-free formula $\varphi(x_1,\ldots,x_k)$ we have
 			\[
 				F(\eta) \models \varphi \,\,\,\text{ iff } \,\,\, F(\epsilon) \models \varphi(\bar{b}).
 			\]
 		\end{prop}
 		\begin{proof}
 			\textcolor{red}{TODO}
 		\end{proof}
 		
 		We see that the chain $\eta \subseteq F(\eta)$ is indiscernible for quantifier-free formulae. Suppose we are given an $\mathcal{L}$-structure $M$ that contains a linear order $\eta\subseteq M$. The \emph{theory of} $\eta$ \emph{in} $M$, denoted $\textnormal{Th}(M,\eta)$ is the set of all $\mathcal{L}$-formulae $\varphi(\bar{x})$ so that 
 		\[
 			M \models \varphi(\bar{a})
 		\]
 		for all $\bar{a}\in [n]^k$.  Similarly, the theory of an E.M. functor $F$ is the set of all $\mathcal{L}$-formulae $\varphi(\bar{x})$ such that $F(\eta) \models \varphi(\bar{a})$ for all linear orders $\eta$ and $\bar{a}\in [n]^k$.
 		\begin{lem}(Stretching)
 			Let $M$ be an $\mathcal{L}$-structure that contains the linear order $\omega$ as a generating set, and suppose $\omega$ is indiscernible in $M$ by quantifier free formulae. Then there is an E.M. functor $F$ such that $M = F(\omega)$. This functor is unique up to natural isomorphism.
 		\end{lem}
 		\begin{proof}
 			\textcolor{red}{TODO}
 		\end{proof}
 		If $F$ is an E.M. functor and $\mathcal{T}$ is an $\mathcal{L}$-theory, then the models of $\mathcal{T}$ of the form $F(\eta)$ are called \emph{E.M. models} of $\mathcal{T}$. 
 		
 		Before moving on, we need a theorem from (infinitary) combinatorics.
 		\begin{thm}[Ramsey]
 			Let $X$ be a countable linear order, and $k$ and $n$ be positive integers. For every $f\colon [X]^k \to [n]$ there is an infinite subset $Y\subseteq X$ such that $f$ is constant on $[Y]^k$.
 		\end{thm}
 		\begin{proof}
 			\textcolor{red}{TODO}
 		\end{proof}
 		We will use Ramsey's Theorem to show that E.M. models of Skolem Theories with infinite models always exist.
 		\begin{lem}
 			Let $F$ be an E.M. functor such that $\textnormal{Th}(F(\omega))$ is Skolem. Then $\textnormal{Th}(F)$ includes $\varphi(\bar{x})$ or $\neg\varphi(\bar{x})$ for every $\mathcal{L}$-formula $\varphi(\bar{x})$. In particular all the structures $F(\eta)$ are elementarily equivalent and each linear order $\eta$ is indiscernible in $F(\eta)$.
 		\end{lem}
 		\begin{proof}
 			\textcolor{red}{TODO}
 		\end{proof}
 		\begin{thm}[Ehrenfeucht-Mostowski]
 			Let $M$ be an $\mathcal{L}$-structure whose theory is Skolem. If $\eta$ is an infinite linear order contained in $M$ as a set, then there is an E.M. functor $F$ whose theory expands $\textnormal{Th}(M,\eta)$.
 		\end{thm}
 		\begin{proof}
 			\textcolor{red}{TODO}
 		\end{proof}
 		
 		\chapter{Non-classical Logic}
 		\section{Intuitionistic Logic}
 		We will now redefine all the logic we've been currently using. In particular, we will redefine syntax and semantics of first-order logic. Consider the following statements.
 		\begin{itemize}
 			\item $\bot$ has no proof.
 			\item To prove $\varphi \wedge \psi$ is to give a proof of $\varphi$ together with a proof of $\psi$.
 			\item To prove $\varphi \Rightarrow \psi$ is to give a procedure that converts a proof of $\varphi$ into a proof of $\psi$.
 			\item To prove $\varphi \vee \psi$ is to specify a proof of $\varphi$ or a proof of $\psi$.
 		\end{itemize}
 		These are not exactly the rules we have been working with thus far. For example, previously we had the law of excluded middle, i.e., that
 		\[
 			\varphi \vee \neg \varphi
 		\]
 		is always provable. However, this law is clearly in contradiction with our new rule for disjunction. So this is a genuinely different logical system.
 		\begin{itemize}
 			\item To prove $\exists x.\varphi(x)$ is to give a term $t$ with a proof of $\varphi(t)$.
 			\item To prove $\forall x. \varphi(x)$ is to give a procedure that takes each term $t$ to a proof of $\varphi(t)$.
 		\end{itemize}
 		Weird things can happen with intuitionistic logic. Though we haven't given a formal description of this logic just yet, we can still understand the following theorem (and its proof)
 		\begin{thm}[Diaconescu]
 			The law of excluded middle can be intuitionistically deduced from the Axiom of Choice.  
 		\end{thm}
 		\begin{proof}
 			\textcolor{red}{TODO}
 		\end{proof}
 		Why bother with intuitionistic logic?
 		\begin{itemize}
 			\item We assume less than classical logic, so we develop a more general theory.
 			\item Notions that are classically conflated can be genuinely different in intuitionistic mathematics; there's a greater variety of concepts. For example, there are at least four non-intuitionistically-equivalent definitions of a finite set.
 			\item Constructive proofs have a computable content that is often missing from classical proofs. 
 			\item Intuitionistic logic is the internal logic of an arbitrary topos.
 		\end{itemize}
 		We will develop the Brouwer-Heyting-Kolmogorov interpretation of intuitionistic logic.
\end{document}