\documentclass{article}
%Setting margins
%\usepackage[margin = 1.75in]{geometry}
%Basic Maths
\usepackage{amsmath}
\usepackage{amssymb}
\usepackage{mathdots}
\usepackage{mathtools}
\usepackage{gensymb}
%For definining theorem-like environments
\usepackage{amsthm}
%For beautiful letters (e.g. for a partition, see $\mathscr{P}$)
\usepackage{mathrsfs}
%For importing the solution file
%\usepackage{import}
%For drawing commutative diagrams
\usepackage{quiver}
%For pretty colours
\usepackage{xcolor}
%For scaling some relations, for instance see https://tex.stackexchange.com/a/108482
\usepackage{mleftright}
%Set paragraph spacing. I believe this is close to what is used in the book.
%\usepackage[skip=.3\baselineskip, indent = 15pt]{parskip}
%To customize lists
\usepackage{enumitem}
%To strikethrough terms in equations
\usepackage{cancel}
%For bibliography
\usepackage[backend=biber]{biblatex}
%For pictures
\usepackage{tikz}
\usetikzlibrary{calc,positioning}

\usepackage[hidelinks]{hyperref}
\usepackage{soul}
\usepackage[normalem]{ulem}
\usepackage{lipsum}
\usepackage{breqn}

%\addbibresource{main.bib}


\newcommand{\myhy}[2]{\href{#1}{\color{blue}\setulcolor{blue}\ul{#2}}}

%Fix section numbering to match the book's convention
\renewcommand\thesection{\arabic{section}}

%Displays "Exercises". To put after each section.
\newcommand{\extitle}{\subsection*{Exercises}}

%For personal notes
\newcommand{\note}[1]
{\smallskip {\noindent\textbf{Note} #1}}

%Roman numerals!
\newcommand{\RNo}[1]{%
	\textup{\uppercase\expandafter{\romannumeral#1}}%
}

%San-serif for names of categories
\newcommand{\serif}[1]{{\fontfamily{cmss}\selectfont #1}}
\newcommand{\srf}{\textsf}

%Shorthands for common sets
\newcommand{\N}{\mathbb{N}}
\newcommand{\Z}{\mathbb{Z}}
\newcommand{\Q}{\mathbb{Q}}
\newcommand{\R}{\mathbb{R}}
\newcommand{\C}{\mathbb{C}}
\newcommand{\zmod}[1]{\bZ/#1\bZ}

%Miscellaneous commands
\newcommand{\defeq}{\coloneqq}
\newcommand{\divides}{\mid}
\newcommand{\legendre}[2]{\ensuremath{\left( \frac{#1}{#2} \right) }}
\newcommand{\Mod}[1]{\ (\mathrm{mod}\ #1)}
\newcommand{\mbold}[1]{\mathrm{\mathbf{#1}}}


%Useful operations and delimiters
\DeclareMathOperator{\Hom}{Hom}
\DeclareMathOperator{\End}{End}
\DeclareMathOperator{\Aut}{Aut}
\DeclareMathOperator{\Obj}{Obj}
\DeclareMathOperator{\id}{id}
\DeclareMathOperator{\lcm}{lcm}
\DeclareMathOperator{\GL}{GL}
\DeclareMathOperator{\SO}{SO}
\DeclareMathOperator{\SL}{SL}
\DeclareMathOperator{\U}{U}
\DeclareMathOperator{\SU}{SU}
\DeclareMathOperator{\Inn}{Inn}
\DeclareMathOperator{\PSL}{PSL}
\DeclareMathOperator{\im}{im}
\DeclareMathOperator{\coker}{coker}
\DeclareMathOperator{\rot}{rot}
\DeclareMathOperator{\rf}{ref}
\DeclareMathOperator{\Symm}{Symm}
\DeclareMathOperator{\vspan}{span}
\DeclareMathOperator{\ev}{ev}
\DeclareMathOperator{\Gal}{Gal}
\DeclareMathOperator{\ob}{ob}
\DeclareMathOperator{\mor}{mor}
\DeclareMathOperator{\dom}{dom}
\DeclareMathOperator{\cod}{cod}
\DeclareMathOperator{\Cone}{Cone}
\DeclarePairedDelimiter\abs{\lvert}{\rvert}%
\DeclarePairedDelimiter\norm{\lVert}{\rVert}%
\DeclarePairedDelimiter\innprod{\langle}{\rangle}%
\DeclarePairedDelimiter\ceil{\lceil}{\rceil}
\DeclarePairedDelimiter\floor{\lfloor}{\rfloor}
\newcommand\binuparrow{\mathbin{\uparrow}}
%Claim environment
\newtheorem{claim}{Claim}


%Exercise environment
\theoremstyle{definition}
\newtheorem{ex}{Exercise}

%Standard theorem-like environment
\theoremstyle{plain}
\newtheorem{thm}{Theorem}[section]

\newtheorem{prop}[thm]{Proposition}
\newtheorem{lem}[thm]{Lemma}
\newtheorem{coro}[thm]{Corollary}
\newtheorem{prob}{Problem}
\newtheorem{conj}{Conjecture}


\theoremstyle{definition}
\newtheorem{defn}[thm]{Definition}
\newtheorem{rem}[thm]{Remark}
\newtheorem{eg}[thm]{Example}
\newtheorem{egs}[thm]{Examples}
\newtheorem{fact}[thm]{Fact}
\newtheorem{task}{Task}



%Solution environment
\newenvironment{solution}
{\begin{proof}[Solution]}
	{\end{proof}}

%Function restrictions
% From https://tex.stackexchange.com/a/22255
\newcommand\restr[2]{{% we make the whole thing an ordinary symbol
		\left.\kern-\nulldelimiterspace % automatically resize the bar with \right
		#1 % the function
		\vphantom{\big|} % pretend it's a little taller at normal size
		\right|_{#2} % this is the delimiter
}}

%\newcommand\nvdash{\mkern-2mu\not\mkern2mu\vdash}

\makeatother
\setlist{parsep=0pt,listparindent=\parindent}
\begin{document}
	\title{Geometric Group Theory\\ ES1 Solutions}
	\date{}
	\maketitle
	\begin{enumerate}
		\item Torus and Klein bottle respectively,
		\item \begin{enumerate}
			\item It suffices to show that $z$ commutes with the generators, but this is immediate.
			\item The centre of $A_5$ is trivial so $z$ must be send to the identity (here we use surjectivity). It remains to find elements of order 2 and 3 in $A_5$ such that their product has order 5 and they generate the whole group. The permutations $(1\,2)(3\,4)$ and $(1\,3\,5)$ satisfy these conditions.
		\end{enumerate}
		\item Define $\varphi(k)\coloneqq ba^kb^{-1}a^{-2^k}$ for $k\geq 1$. Check by induction that
		\[
			\varphi(k) =\prod_{r=0}^{k-1}a^{2^{r}}(bab^{-1}a^{-2})a^{-2^r}
		\]
		in $F\coloneqq F(\{a,b\})$. It follows that $\varphi(k)$ is trivial in $BS(1,2)$ for all $k\geq 1$. Now define $\Phi(n,k)\coloneqq b^{n-1}\varphi(k)b^{-(n-1)}$ for all $n,k\geq 1$. Obviously $\Phi(n,k)$ becomes trivial in $BS(1,2)$. An induction on $m$ shows that for all $1\leq m\leq n$ we have the following identity in $F$
		\[
			 \prod_{r=0}^{m-1} \Phi(n-r,2^r)= b^{n}ab^{-m}a^{-(2\binuparrow\binuparrow m)}b^{-(n-m)},
		\]
		where $2\binuparrow\binuparrow m$ refers to the number $2^{2^{2^{\iddots}}}$ where $2$ appears $m$ times (this is Knuth's arrow notation). In particular, for $m=n$ we have
		\[
			\Psi(n)\coloneqq \prod_{r=0}^{n-1} \Phi(n-r,2^r)= b^nab^{-n}a^{-(2\binuparrow\binuparrow n)} .
		\]
		Hence
		\begin{align*}
			\Phi(n)a\Phi^{-1}(n)a^{-1} &= b^nab^{-n}a^{-(2\binuparrow\binuparrow n)} a(a^{2\binuparrow\binuparrow n}b^na^{-1}b^{-n})a^{-1}\\
			&= b^nab^{-n}(a^{-(2\binuparrow\binuparrow n)} aa^{2\binuparrow\binuparrow n})(b^na^{-1}b^{-n})a^{-1}\\
			&= b^nab^{-n}ab^na^{-1}b^{-n}a^{-1}\\
			&= w_n.
		\end{align*}
		We can then calculate an upper bound for the area of $w_n$. Clearly the area of $\phi(k)$ is at most $k$, and hence the same goes for $\Phi(n,k)$, independently of $n$. It then follows that $\Psi(n)$ has area at most 
		\[
			1 + 2 + 4+ \ldots + 2^{n-1} = 2^n -1.
		\]
		Thus the area of $w_n$ is at most $(2^n - 1) + (2^n -1) = 2^{n+1} - 2$.
		\item Let $X$ be the presentation complex of $\mathbb{Z}^2 \cong \langle a,b \mid [a,b]\rangle$. Let $D\hookrightarrow X$ be a van-Kampen diagram for $w_n$ of minimal area. The covering map $\tilde{X}\to X$ induces a map $\tilde{X}\to D$ $\ldots$
		\item \begin{enumerate}
			\item Isometries are affine maps with determinant $\pm 1$. In $\mathbb{R}$, this means that all isometries are either translations or multiplication by $-1$ followed by translations. It immediately follows that $B$ generates $D_{\infty}$ and since $sr= t$ we get that $A$ is also a generating set.
			\item Omitted 
		\end{enumerate}
		\item Suppose $\langle A\mid R\rangle$ has a solvable word problem. Given some $n$, it is easy to find an algorithm that will construct the closed ball of radius $n$ around 1 of $\text{Cay}_A(F(A))$; call this graph $G_n$. As $G_n$ is finite, we can check, using the algorithm for the word problem, which vertices should be identified in $G_n$ according to $R$. The result will be $B_n$.
		
		Conversely, suppose we can construct $B_n$ for any $n$. Given a word of length $n$, simply check whether the corresponding path in $B_n$ ends at the identity or not.
		\item 
	\end{enumerate}
\end{document}