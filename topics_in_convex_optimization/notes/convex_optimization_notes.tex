\documentclass{report}
%Setting margins
%\usepackage[margin = 1.75in]{geometry}
%Basic Maths
\usepackage{amsmath}
\usepackage{amssymb}
\usepackage{mathtools}
\usepackage{gensymb}
%For definining theorem-like environments
\usepackage{amsthm}
%For beautiful letters (e.g. for a partition, see $\mathscr{P}$)
\usepackage{mathrsfs}
%For importing the solution file
%\usepackage{import}
%For drawing commutative diagrams
\usepackage{quiver}
%For pretty colours
\usepackage{xcolor}
%For scaling some relations, for instance see https://tex.stackexchange.com/a/108482
\usepackage{mleftright}
%Set paragraph spacing. I believe this is close to what is used in the book.
%\usepackage[skip=.3\baselineskip, indent = 15pt]{parskip}
%To customize lists
\usepackage{enumitem}
%To strikethrough terms in equations
\usepackage{cancel}
%For bibliography
\usepackage[backend=biber]{biblatex}
%For pictures
\usepackage{tikz}
\usetikzlibrary{calc,positioning}

\usepackage[hidelinks]{hyperref}
\usepackage{soul}
\usepackage[normalem]{ulem}
\usepackage{lipsum}
\usepackage{breqn}

%\addbibresource{main.bib}


\newcommand{\myhy}[2]{\href{#1}{\color{blue}\setulcolor{blue}\ul{#2}}}

%Fix section numbering to match the book's convention
\renewcommand\thesection{\arabic{section}}

%Displays "Exercises". To put after each section.
\newcommand{\extitle}{\subsection*{Exercises}}

%For personal notes
\newcommand{\note}[1]
{\smallskip {\noindent\textbf{Note} #1}}

%Roman numerals!
\newcommand{\RNo}[1]{%
	\textup{\uppercase\expandafter{\romannumeral#1}}%
}

%San-serif for names of categories
\newcommand{\serif}[1]{{\fontfamily{cmss}\selectfont #1}}
\newcommand{\srf}{\textsf}

%Shorthands for common sets
\newcommand{\N}{\mathbb{N}}
\newcommand{\Z}{\mathbb{Z}}
\newcommand{\Q}{\mathbb{Q}}
\newcommand{\R}{\mathbb{R}}
\newcommand{\C}{\mathbb{C}}
\newcommand{\zmod}[1]{\bZ/#1\bZ}

%Miscellaneous commands
\newcommand{\defeq}{\coloneqq}
\newcommand{\divides}{\mid}
\newcommand{\legendre}[2]{\ensuremath{\left( \frac{#1}{#2} \right) }}
\newcommand{\Mod}[1]{\ (\mathrm{mod}\ #1)}
\newcommand{\mbold}[1]{\mathrm{\mathbf{#1}}}


%Useful operations and delimiters
\DeclareMathOperator{\Hom}{Hom}
\DeclareMathOperator{\End}{End}
\DeclareMathOperator{\Aut}{Aut}
\DeclareMathOperator{\Obj}{Obj}
\DeclareMathOperator{\id}{id}
\DeclareMathOperator{\lcm}{lcm}
\DeclareMathOperator{\GL}{GL}
\DeclareMathOperator{\SO}{SO}
\DeclareMathOperator{\SL}{SL}
\DeclareMathOperator{\U}{U}
\DeclareMathOperator{\SU}{SU}
\DeclareMathOperator{\Inn}{Inn}
\DeclareMathOperator{\PSL}{PSL}
\DeclareMathOperator{\im}{im}
\DeclareMathOperator{\coker}{coker}
\DeclareMathOperator{\rot}{rot}
\DeclareMathOperator{\rf}{ref}
\DeclareMathOperator{\Symm}{Symm}
\DeclareMathOperator{\vspan}{span}
\DeclareMathOperator{\ev}{ev}
\DeclareMathOperator{\Gal}{Gal}
\DeclareMathOperator{\ob}{ob}
\DeclareMathOperator{\mor}{mor}
\DeclareMathOperator{\dom}{dom}
\DeclareMathOperator{\cod}{cod}
\DeclareMathOperator{\Cone}{Cone}
\DeclarePairedDelimiter\abs{\lvert}{\rvert}%
\DeclarePairedDelimiter\norm{\lVert}{\rVert}%
\DeclarePairedDelimiter\innprod{\langle}{\rangle}%
\DeclarePairedDelimiter\ceil{\lceil}{\rceil}
\DeclarePairedDelimiter\floor{\lfloor}{\rfloor}
%Claim environment
\newtheorem{claim}{Claim}


%Exercise environment
\theoremstyle{definition}
\newtheorem{ex}{Exercise}

%Standard theorem-like environment
\theoremstyle{plain}
\newtheorem{thm}{Theorem}[section]

\newtheorem{prop}[thm]{Proposition}
\newtheorem{lem}[thm]{Lemma}
\newtheorem{coro}[thm]{Corollary}
\newtheorem{prob}{Problem}
\newtheorem{conj}{Conjecture}


\theoremstyle{definition}
\newtheorem{defn}[thm]{Definition}
\newtheorem{rem}[thm]{Remark}
\newtheorem{eg}[thm]{Example}
\newtheorem{egs}[thm]{Examples}
\newtheorem{fact}[thm]{Fact}
\newtheorem{task}{Task}



%Solution environment
\newenvironment{solution}
{\begin{proof}[Solution]}
	{\end{proof}}

%Function restrictions
% From https://tex.stackexchange.com/a/22255
\newcommand\restr[2]{{% we make the whole thing an ordinary symbol
		\left.\kern-\nulldelimiterspace % automatically resize the bar with \right
		#1 % the function
		\vphantom{\big|} % pretend it's a little taller at normal size
		\right|_{#2} % this is the delimiter
}}

%\newcommand\nvdash{\mkern-2mu\not\mkern2mu\vdash}

\makeatother
\begin{document}
	\title{Topics in Convex Optimization\\ Things stated without proof}
	\section{Lecture 1}
	\section{Lecture 2}	
	\begin{thm}[Projection theorem]
		Let $C\subseteq \mathbb{R}^n$ be a closed convex set. For all points $y\in\mathbb{R}^n$ there is a unique $p_C(y)\in C$ so that $||y - x|| \geq ||y - p_C(y)||$ for all $x\in C$.
 	\end{thm}
 	\begin{proof}
 		This is a special case of Hilbert's projection theorem, which has an elementary proof on Wikipedia.
 	\end{proof}
	\begin{prop}[Obtuse angle criterion]
		Let $C\subseteq \mathbb{R}^n$ be a closed convex set and $y\in \mathbb{R}^n$. Then, for all $x\in C$,
		\[
			\langle y - p_C(y), x- p_C(y)\rangle \leq 0.
		\] 
	\end{prop}
	\begin{proof}
		Let $\lambda \in (0,1)$. As $x$ and $p_C(y)$ are in $C$, so is any convex combination. By definition of $p_C(y)$ we must have
		\begin{align*}
			||p_C(y) - y||^2 &\leq ||\lambda x + (1-\lambda)p_C(y) - y||^2\\
			&= ||\lambda (x - p_C(y)) -(y-p_C(y))||^2\\
			&= \lambda^2||x-p_C(y)||^2 -2\lambda \langle x - p_C(y), y-p_C(y)\rangle + ||p_C(y) - y||^2.
		\end{align*}
		It follows by cancelling and rearranging that 
		\[
			\langle y - p_C(y), x- p_C(y)\rangle  \leq \frac{\lambda}{2}||x-p_C(y)||^2.
		\]
		As $\lambda$ can be made arbitrarily small, we are done.
	\end{proof}
	Next we prove the Separating hyperplane theorem. First, a lemma.
	\begin{lem}
		Let $C\subseteq \mathbb{R}^n$ be a convex set. The function $y\mapsto p_C(y)$ is continuous.
	\end{lem}
	\begin{proof}
		Let $\varepsilon>0$ be arbitrary. Suppose $y,y'\in\mathbb{R}^n$ are such that $||y-y'||\leq \delta$ where $\delta\coloneqq \varepsilon$. Then, by the Obtuse angle criterion, 
		\begin{align*}
			0&\geq \langle y - p_C(y), p_C(y') - p_C(y)\rangle\\
			&= \langle (y-p_C(y')) +(p_C(y') - p_C(y)), p_C(y') - p_C(y)\rangle\\
			&= \langle y-p_C(y'), p_C(y') - p_C(y)\rangle + ||p_C(y') - p_C(y)||^2.
		\end{align*}
		It follows that
		\begin{align*}
			||p_C(y') - p_C(y)||^2 &\leq \langle p_C(y') - y, p_C(y') - p_C(y)\rangle\\
			&= \langle (p_C(y') - y') +(y'- y), p_C(y') - p_C(y)\rangle\\
			&= \langle p_C(y') - y', p_C(y') - p_C(y)\rangle + \langle y'- y, p_C(y') - p_C(y)\rangle\\
			&\leq \langle y'- y, p_C(y') - p_C(y)\rangle,
		\end{align*}
		where we have used the Obtuse angle criterion in the last inequality. By the Cauchy-Schwarz inequality, we finally obtain
		\[
			||p_C(y') - p_C(y)||^2 \leq ||y'- y|| \cdot ||p_C(y') - p_C(y)||,
		\]
		from which it follows that $||p_C(y') - p_C(y)|| \leq \varepsilon$ as desired.
	\end{proof}
	\begin{thm}[Separating hyperplane theorem]
		Let $C \subseteq \mathbb{R}^n$ be a convex set and let $y\notin C$. Then there is $a\in\mathbb{R}^n\setminus\{0\}$ and $b\in \mathbb{R}$ such that for all $x\in C$ we have
		\[
			\langle a, x\rangle \leq b  \,\,\,\text{ and }\,\,\, \langle a,y\rangle\geq b.
		\]
	\end{thm}
	\begin{proof}
		The proof for $C$ closed was given in lectures. If $C$ is not closed, take the closure $\bar{C}$ of $C$ and obtain $a$ and $b$ as required. The only case where this does not work is if $y \in \partial C$, so we assume $y\in \partial C$.
		
		Recall that $p\in \partial C$ if and only if every neighbourhood of $p$ contains a point in $C$ and a point not in $C$. In particular, there is a sequence of points $(y_k)_{k=0}^{\infty}$ converging to $y$ such that $y_k\in \mathbb{R}^n\setminus C$ for all $k$.
	\end{proof}
\end{document}