\documentclass{article}
\usepackage{amsmath}
\usepackage{amsthm}
\usepackage{enumitem}
\usepackage{amssymb}
\usepackage{mathtools}
\usepackage{url}
\usepackage{xcolor}

\theoremstyle{theorem}
\newtheorem{claim}{Claim}

\begin{document}
	\title{Model Theory and Non-Classical Logic\\ Example Sheet 2 Solutions}
	\author{Hernán Ibarra Mejia}
	\maketitle
	\begin{enumerate}[leftmargin=*]
		\item \begin{enumerate}
			\item We use syntactic quantifier elimination. Note that the only terms are variables, and the atomic formulae are either equalities or of the form $x\sim y$. Hence it suffices to eliminate quantifiers from formulae of the form
			\begin{multline*}
				\exists y. \left(\bigwedge_{i\in I}y = x_i\right)\wedge\left(\bigwedge_{i\in J}y \neq x_i\right)\wedge \left(\bigwedge_{i,j\in K}x_i = x_j\right)\wedge\left(\bigwedge_{i,j\in L}x_i \neq x_j\right)\\
				\wedge\left(\bigwedge_{i\in I'}y \sim x_i\right)\wedge\left(\bigwedge_{i\in J'}y \not\sim x_i\right)\wedge \left(\bigwedge_{i,j\in K'}x_i \sim x_j\right)\wedge\left(\bigwedge_{i,j\in L'}x_i \not\sim x_j\right).
			\end{multline*}
			It's easy to eliminate the quantifier in the first conjunct so we assume $I$ is empty. By the same argument, we may assume that $I'$ is empty. We may also push the quantifier inside conjucts where it actually appears, so we are left with
			\[
				\exists y.\left( \bigwedge_{i\in J} y \neq x_i \right) \wedge \left(\bigwedge_{i\in J'} y\not\sim x_i\right) \wedge \varphi
			\]
			where $\varphi$ is quantifier-free. Since there are infinitely many equivalence classes, we can pick one that does not contain any $x_i\in J'$. Inside this class (since classes are infinite) we can pick $y$ distinct from all $x_i\in J$. We are done.
		\item We know that theories with quantifier elimination are model-complete. However, $(0,1) \subseteq [0,1]$ is an embedding of models but $\exists y. \forall x. x \leq y$ is satisfied in $[0,1]$ but not in $(0,1)$; hence the embedding is not elementary.
		\item We use the Corollary that says that if $A$ is a finite $\mathcal{L}$-structure then $\text{Th}(A)$ has quantifier elimination iff every isomorphism between finitely generated substructures of $A$ can be extended to an automorphism of $A$. 
		
		Let $F$ is a finite field. Then a substructure of $F$ is the same as a subring. But subrings of finite fields are fields. Indeed, if $R$ is a subring of $F$ the multiplication map $a \mapsto \lambda a$ for a fixed nonzero $\lambda \in R$ is bijective as a function $F \to F$. It remains injective when restricted to a map $R \to R$. But, since $R$ is finite, the map is also surjective. Hence, all non-zero $\lambda \in R$ have inverses in $R$.
		
		Let $F_1,F_2$ be subfields of $F$ with an isomorphism $f\colon F_1 \to F_2$. We show that $F_1 = F_2$. Clearly $n\coloneqq |F_1| = |F_2|$. Consider the groups of units of $F_1$ and $F_2$. Their elements clearly satisfy $x^{n-1} = 1$. But, as $F$ is a field, this equation has at most $n-1$ solutions. It follows that the two groups consist of the same elements, i.e. they are equal. Thus $F_1 = F_2$.
		
		Note that automorphisms of finite fields are generated by the Frobenius automorphism so we can extend the automorphism of the smaller field to the bigger field (\textcolor{red}{this requires some algebra I don't want to get into}). We are done.
		\item By syntactic quantifier elimination; basically the same argument as in (a) but noting that terms are just linear combinations of variables.
		\end{enumerate}
		\item Suppose $M$ is elementarily equivalent to $\mathbb{R}$. The sentence
		\[
			\forall x. (x\geq 0 \implies \exists y. y^2 = x)
		\]
		is satisfied in $\mathbb{R}$ so it must be satisfied in $M$. Similarly, for any odd number $n$ the sentence
		\[
			\forall a_0,a_1,\ldots,a_{n}. \,\,\,(a_n \neq 0) \implies \exists x. (a_0 + a_1x +\cdots +a_nx^n = 0)
		\]
		is satisfied in $\mathbb{R}$ and hence in $M$.
		
		Conversely, we show that any two structures satisfying RCF are elementarily equivalent. We claim that the theory RCF is complete, which immediately implies the result.
		
		As RCF has quantifier elimination, we only need to show that each quantifier-free sentence is decidable from RCF. But these sentences are all equivalent to Boolean combinations of sentences of the form $(q = 0)$ or $q>0$ where $q\in \mathbb{Z}$ and, since any ordered field contains a copy of $\mathbb{Z}$, we are done.
		
		Again, as RCF as quantifier elimination (and we are not allowing parameters), the definable subsets of $\mathbb{R}$ are just the subsets definable by quantifier free formulae. A quantifier-free formula with exactly one free variable is a Boolean combination of sentences of the form $(p(x) = 0)$ or $p(x) > 0$ where $p(x)$ is a polynomial on $x$ with integer coefficients. It is easy to see that this results in unions of intervals whose endpoints are either infinite or algebraic.
		\item We use Theorem 1.2.18 (b). Let $A,B$ be models of the theory, and $\bar{a}\in A$, $\bar{b}\in B$ be tuples of the same size such that $(A,\bar{a})\equiv_0 (B,\bar{b})$. Let $\varphi(\bar{x},y)$ be a quantifier-free sentence and suppose that $A\models \exists y. \varphi(\bar{a},y)$, and further let $c\in A$ be such that $A\models \varphi(\bar{a},c)$. We need to show that $B\models \exists y. \varphi(\bar{b},y)$.
		
		As $(A,\bar{a})\equiv_0 (B,\bar{b})$ there is an isomorphism $f\colon \langle \bar{a}\rangle_A \stackrel{\cong}{\to} \langle \bar{b}\rangle_B$. Elementarily extend $A,B$ to structures $A',B'$ as in the premise. Then, by the ``forth'' property, there is some $g$ extending $f$ such that $g$ includes $c$ in its domain. Clearly the domain of $g$ satisfies $\varphi(\bar{a},c)$ so the codomain of $g$ satisfies $\varphi(\bar{b},g(c))$. It follows that $B\models \varphi(\bar{b},g(c))$ and we are done. \textcolor{red}{We didn't seem to need the ``back'' property. Also I don't know how to do the second part.}
		\item Let $F$ be a field and $f\in F$ be such that both $f$ and $-1$ cannot be written as a sum of squares. We show that there is an ordering on $F$ that makes $F$ an ordered field with $f<0$.
		
		Call a subset $P$ of $F$ a \emph{prepositive cone} if $P$ is closed under addition and multiplication, $P$ contains all squares, and $-1\notin P$. Note that any such $P$ is closed under inverses too: if $p\in P$ is nonzero then $p\cdot (p^{-1})^2 = p^{-1}\in P$. Clearly the smallest possible $P$ is the set of all sums of squares, which does not contain -1 nor $f$ by hypothesis. Then the set
		\[
			\Pi \coloneqq \{P\subseteq F\mid P \text{ is a prepositive cone and }f\notin P\}.
		\]	
		is nonempty. We use Zorn's Lemma to find a maximal element of $\Pi$ under inclusion. Let $T$ be a chain of elements of $\Pi$ and let $U$ be the union of all elements of $T$. We need to show that $U\in \Pi$. Since $f$ is not in any elements of $T$ it is not in $U$ either; thus we only show that $U$ is a prepositive cone.
		
		But any two elements of $U$ can be made to belong to the same prepositive cone in $T$ and thus their sum and product are also in $U$. Clearly $U$ contains all squares, and $-1\notin U$. By Zorn's Lemma, $\Pi$ has a maximal element $P^*$.
		
		We claim that $F = P^* \cup (-P^*)$. First we show that $-f\in P^{*}$. Suppose, for the sake of contradiction, that $f,-f\notin P^{*}$. Consider the set $L \coloneqq \{q - pf \mid p,q\in P^*\}$. This set is closed under addition, and it is also closed under multiplication since for $p,q,p',q'\in P^*$ we have
		\[
			(q-pf)(q'-p'f) = (f^2pp' + qq') -f(pq' + p'q) \in L.
		\]
		Also, if $f\in L$ we would have $f = q -pf$ and thus $q = (1+p)f$. As $-1\notin P^*$ we have $p+1\neq 0$ and thus $f= q(1+p)^{-1}\in P^*$ which we know is impossible. Thus $f\notin L$ and so, if we could show that $-1\notin L$ we would have $L\in \Pi$. But $P^{*}\subsetneq L$ so by maximality this cannot happen. Hence $-1\in L$, i.e. $-1 = q - pf$ where clearly $p\neq 0$, which implies $f= (q+1)p^{-1}\in P^*$, another contradiction. Thus our initial assumption that $-f\notin P^*$ must be wrong, i.e. $-f\in P^*$.
		
		Now we consider the more general claim that $F = P^* \cup (-P^*)$. Let $a\in F$ and suppose $a\notin P^*$. Consider the set $S \coloneqq \{ap + q \mid p,q\in P\}$. For the same reasons as above, $S$ is closed under addition and multiplication. 
		
		Clearly $P^* \subsetneq S$ and so, if $-1\notin S$ and $f\notin S$ then $S\in \Pi$. Maximality of $P^*$ wouldn't allow this so either $-1\in S$ or $f\in S$. In the first case we would have $ap+q = -1$ with $p,q\in P^*$ and $p\neq 0$ and thus $-a = p^{-1}(1+q)$, which implies $-a \in P^{*}$ as desired. In the second case we have $ap+q = f$ which still implies $-a = p^{-1}(q - f)\in P^*$ as $-f\in P^*$. This shows that $F = P^* \cup (-P^*)$. We can then define an ordering on $F$ by declaring that $a\leq b$ iff $b-a\in P^*$. The properties of $P^*$ implies that this ordering turn $F$ into an ordered field in which $f < 0$.
		
		\textcolor{red}{This is as far as I got, what follows is a spelled-out version of the proof found in Marker's book.} Back to the problem, we take $F\coloneqq \mathbb{R}(x_1,\ldots,x_n)$ and thus, by assuming $f$ is not a sum of squares, get an ordering on $F$ where $f<0$.
		
		Embed $F$ into its real closure $R$ that extends its ordering. Note that we had the field-embedding $\mathbb{R} \hookrightarrow F$ by sending each real to its associated constant polynomial. Hence we have a field-embedding $e\colon \mathbb{R}\hookrightarrow R$ which must be order preserving since if $a\leq b$ are real numbers we have $a = b + c^2$ for some $c\in\mathbb{R}$ and thus $e(a) = e(b) + e(c)^2$, which means $e(a)\leq e(b)$ as desired. It follows that $e$ is a structure embedding and, by model-completeness of RCF, it must be an elementary embedding. Without loss of generality, assume $e$ is an inclusion.
		
		Let $f = c_nx^n + \cdots c_1x + c_0$ where $c_i\in \mathbb{R}$ for all $i$. Consider the $\mathcal{L}$-term
		\[
		p(\bar{y},v) \coloneqq y_nv^n + \cdots + y_{1}v + v.
		\]
		It follows from $f<0$ in $R$ that
		\[
			R \models \exists \bar{v}. p(\bar{c},\bar{v}) <0.
		\]
		Indeed, setting $\bar{v}$ to be the single-variable polynomials $x_1,\ldots, x_n$ does the job. As the embedding was elementary,
		\[
			\mathbb{R} \models \exists \bar{v}. p(\bar{c},\bar{v}) <0
		\]	
		giving our contradiction.
		\item[$(\star)$] \textcolor{red}{I think this is a cool result but it requires: finding the axioms for the language (here \url{https://en.wikipedia.org/wiki/Tarski\%27s_axioms#}), proving that the theory is complete, proving that every model of this theory is some $R^2$ where $R$ is a real-closed field, and finally using quantifier-elimination of RCF to obtain the decision algorithm. I know this is an starred question but it's just not doable if you don't already know the answer (happy to be proven wrong though! there may be simplifications/approaches that I'm missing so let me know if you have any insights).}
		\item 
		\begin{enumerate}
			\item Let $\mathcal{F}$ be a proper filter containing $\mathcal{B}$; so $\emptyset\notin \mathcal{F}$. For every $B_1,\ldots B_n\in \mathcal{B}$ we have $B_1\cap \cdots \cap B_n\in \mathcal{F}$. It follows that $B_1\cap \cdots \cap B_n \neq \emptyset$.
			
			Conversely, suppose $\mathcal{B}$ has the finite intersection property. Define
			\[
				\mathcal{F} \coloneqq \{A \subseteq I \mid A \supseteq B_1 \cap \cdots \cap B_n \text{ for some }B_i\in \mathcal{B}\}.
			\]
			Clearly $\mathcal{F}$ is closed under supersets, and a little reflections shows that it is also closed under intersections. And by the assumption on $\mathcal{B}$ we see that $\emptyset\notin \mathcal{F}$ so we are done.
			
			\item Let $\mathcal{U}$ be a ultrafilter on $\mathcal{P}(I)$, where $I$ is infinite. First we show that if $\mathcal{U}$ does not contain the Fréchet filter then it must be principal. If $\mathcal{U}$ does not contain the Fréchet filter then there must be some cofinite subset of $I$ not in $\mathcal{U}$. As $\mathcal{U}$ is an ultrafilter, this means that there is a finite set in $\mathcal{U}$.
			
			Suppose there is some finite $A= \{a_1,\ldots,a_n\}\in \mathcal{U}$. We claim that there is some $a_i$ with $\{a_i\}\in \mathcal{U}$. Suppose not. Then, as $\mathcal{U}$ is an ultrafilter we must have 
			\[
				I\setminus\{a_1\},\ldots,I\setminus \{a_n\}\in \mathcal{U}.
			\]
			The intersection of these sets is in $\mathcal{U}$ since $\mathcal{U}$ is a filter. But then
			\[
				\mathcal{U} \ni \bigcap_{i=1}^n I \setminus\{a_i\} = I\setminus \bigcup_{i=1}^n \{a_i\} = I\setminus A \notin \mathcal{U},
			\]
			a contradiction. So there must be some $a\in I$ with $\{a\}\in \mathcal{U}$. Any set containing $a$ must be in $\mathcal{U}$ by closure under supersets. Furthermore, if $U\in \mathcal{U}$ we must have $U\cap\{a\} \neq \emptyset$ as $\mathcal{U}$ is proper. This shows that $\mathcal{U}$ is principal generated by $a$, as desired.
			
			Conversely, suppose $\mathcal{U}$ contains the Fréchet filter. If $\mathcal{U}$ were principal then, in particular, there would be some $a\in I$ with $\{a\}\in\mathcal{U}$. But then, as $I\setminus\{a\}$ is cofinite,
			\[
				\emptyset= \{a\} \cap (I \setminus\{a\}) \in\mathcal{U},
			\]
			contradicting the fact that $\mathcal{U}$ is proper.
		\end{enumerate}
		\item We need two claims.
		\begin{claim}
			Suppose $t(\bar{x})$ is a term and $\bar{\alpha} = (\alpha_1,\ldots,\alpha_n)\in \prod_{i\in I} A_i$. Then for all $i\in I$
			\[
				\left(t^{\prod A_i}(\bar{\alpha})\right)(i) = t^{A_i}(\alpha_1(i),\ldots,\alpha_n(i)).
			\]
		\end{claim}
		\begin{proof}
			Induction on $t$. If $t$ is a variable $x_j$ for $1\leq j\leq n$ we have for all $i\in I$,
			\[
				\left((x_j)^{\prod A_i}(\bar{\alpha})\right)(i) =\alpha_j(i) = (x_j)^{M_i}(\alpha_1(i),\ldots,\alpha_n(i)), 
			\]
			so the base case holds. For the inductive hypothesis, let $f$ be a function symbol of arity $m$ and let $t_1(\bar{x}),\ldots, t_m(\bar{x})$ be terms for which the claim holds. Then if $t(\bar{x}) = f(t_1(\bar{x}),\ldots, t_m(\bar{x}))$ we have for all $i\in I$
			\begin{align*}
				\left(t^{\prod A_i}(\bar{\alpha})\right)(i) &= (f(t_1(\bar{x}),\ldots,t_m(\bar{x})))^{\prod A_i}(\bar{\alpha})(i) \\
				&= f^{\prod A_i}(t_1^{\prod A_i}(\bar{\alpha}), \ldots, t_m^{\prod A_i}(\bar{\alpha}))(i)\\
				&= f^{A_i}(t_1^{\prod A_i}(\bar{\alpha})(i), \ldots, t_m^{\prod A_i}(\bar{\alpha})(i))\\
				&= f^{A_i}(t_1^{A_i}(\alpha_1(i),\ldots, \alpha_n(i)),\ldots, t_m^{A_i}(\alpha_1(i),\ldots, \alpha_n(i)))\\
				&= (f(t_1(\bar{x}),\ldots,t_m(\bar{x})))^{A_i}(\alpha_1(i),\ldots, \alpha_n(i))\\
				&= t^{A_i}(\alpha_1(i),\ldots, \alpha_n(i))
			\end{align*}
			as desired (in the third equality we used the definition of product). The induction is thus complete.
		\end{proof}
		\begin{claim}
			Suppose $\phi(\bar{x})$ is an atomic formula and $\bar{\alpha} = (\alpha_1,\ldots,\alpha_n)\in \prod_{i\in I} A_i$. Then $\prod_{i\in I} A_i\models \phi(\bar{\alpha})$ if and only if for all $i\in I$ we have $A_i \models \phi(\alpha_1(i), \ldots,\alpha_n(i))$.  
		\end{claim}
		\begin{proof}
			By induction on $\phi(\bar{x})$. First suppose $\phi(\bar{x})$ is of the form $(t(\bar{x}) = s(\bar{x}))$ for terms $t,s$. But
			\begin{equation*}
				\prod_{i\in I} A_i \models (t(\bar{\alpha}) = s(\bar{\alpha})) \,\,\,\text{ iff }\,\,\, t^{\prod A_i}(\bar{\alpha}) = s^{\prod A_i}(\bar{\alpha}).
			\end{equation*}
			and so, for all $i\in I$, we have by Claim 1 that the above happens iff
			\[
				t^{A_i}(\alpha_1(i),\ldots,\alpha_n(i)) = s^{A_i}(\alpha_1(i),\ldots,\alpha_n(i)).
			\]
			This is equivalent to saying that $A_i \models \phi(\alpha_1(i),\ldots,\alpha_n(i))$ for all $i\in I$ as desired.
			
			Now let $R$ be a relation symbol of arity $m$ and let $t_1(\bar{x}),\ldots, t_m(\bar{x})$ be terms. Suppose that $\phi(\bar{x}) = R(t_1(\bar{x}),\ldots, t_m(\bar{x}))$. Then $\prod_{i\in I}A_i\models \phi(\bar{\alpha})$ means that 
			\[
				\left(t_1^{\prod A_i}(\bar{\alpha}),\ldots, t_m^{\prod A_i}(\bar{\alpha})\right) \in R^{\prod A_i}.
			\]
			By definition of product, this is equivalent to saying that for all $i\in I$ we have 
			\[
				\left(t_1^{\prod A_i}(\bar{\alpha})(i),\ldots, t_m^{\prod A_i}(\bar{\alpha})(i)\right) \in R^{A_i}.
			\]
			But by Claim 1 this is the same as saying that for all $i\in I$ we have
			\[
				\left(t_1^{A_i}(\alpha_1(i),\ldots,\alpha_n(i)), \ldots, t_m^{A_i}(\alpha_1(i),\ldots,\alpha_n(i))\right) \in R^{A_i},
			\]	
			and this latter is exactly saying that $A_i \models \phi(\alpha_1(i),\ldots,\alpha_n(i))$ for all $i\in I$.			
			\end{proof}
			Back to the problem, suppose we have for all $i \in I$ that 
			\[A_i \models \forall \bar{x}.\left(\bigwedge_{j\in J}\phi_j(\bar{x}) \Rightarrow \psi(\bar{x})\right)\]
			We need to show that 
			\[
				\prod_{i\in I}A_i \models \forall \bar{x}.\left(\bigwedge_{j\in J}\phi_j(\bar{x}) \Rightarrow \psi(\bar{x})\right).
			\]
			Let $\bar{\alpha}\in \prod_{i\in I}A_i$ and suppose that we had $\prod_{i\in I}A_i \models \phi_j(\bar{\alpha})$ for all $j\in J$. It then follows from Claim 2 that for all $i\in I$ and $j\in J$ we have $A_i\models \phi_j(\alpha_1(i),\ldots, \alpha_n(i))$. By assumption, this means that $A_i\models \psi(\alpha_1(i),\ldots, \alpha_n(i))$ for all $i$. Again by Claim 2 this implies that $\prod_{i\in I}A_i\models \psi(\bar{\alpha})$ and we are done.
			
			Note that in a language with empty signature there is a sentence saying ``I am a set with exactly $2$ elements''. Then any sets with 2 elements model this sentence but the product of more than one of these sets will not.
			
			\textcolor{red}{I don't know whether a sentence preserved under products must also be preserved under reduced products. My gut says that it must.}
			\item 
			\begin{enumerate}
				\item We show existence of $\text{sh}(b)$ and afterwards uniqueness. Let $r,s$ be standard reals so that $r < b <s$. Partition $(r,s)$ into two intervals of the same length, then into three intervals, and so on. As ${}^*\mathbb{R}$ is an ordered field we have that at each stage $b$ belongs to exactly one of these intervals. Using the axiom of choice we can pick a sequence of standard reals $(a_n)_{n\in \mathbb{N}}$ so that $|a_n - b| < \frac{s-r}{n}$ for all $n\in\mathbb{N}$. Clearly this sequence is Cauchy, hence convergent; call its limit (which is a standard real) $a$.
				
				For any positive standard real $\varepsilon > 0$ we have $|a_n - a|< \frac{\varepsilon}{2}$ for all $n\in\mathbb{N}$ greater than some fixed $N_1\in\mathbb{N}$.  Similarly, there is some $N_2\in\mathbb{N}$ such that $|a_{n} - b| < \frac{\varepsilon}{2}$ for all $n\geq N_2$. Note that the triangle inequality is a first-order property:
				\[
					\forall x,y,z. |x - z| \leq |x - y| + |y - z| .
				\]
				So, if $N\coloneqq \max\{N_1,N_2\}$, we have
				\[
					|a - b| \leq |a - a_N| + |a_N - b| < \frac{\varepsilon}{2} + \frac{\varepsilon}{2} = \varepsilon.
				\]
				This shows that either $b =a$ or $b-a$ is infinitesimal.
				
				Now we prove uniqueness. Suppose $a,a'$ are standard reals so that $|b-a|$ and $|b-a'|$ are infinitesimal. The triangle inequality implies that $|a - a'|$ is either 0 or infinitesimal, but as both $a$ and $a'$ are standard their difference has to be standard, hence $a = a'$.
				
				\item As functions of arity 0 are constants, we see that the language contains a constant for each real number. This will be relevant later.
							
				Coming back to the problem, suppose $s\colon \mathbb{N}\to \mathbb{R}$ converges to $L\in\mathbb{R}$. Let $n\in{}^*\mathbb{N}$ be unlimited. We claim that this means that $n$ is greater than any standard  natural number. Indeed, the only other alternative is that $n$ is smaller than all standard natural numbers. But (as 0 is a constant) we have that
				\[
					\mathbb{R}\models \forall x. (x\in \mathbb{N}) \Rightarrow (0 \leq x)
				\] 
				(here we use the convention that $x\in A$ in a formula where $A\subseteq \mathbb{R}$ means that $x$ satisfies the relation induced by $A$). It follows that 
				\[
					{}^*\mathbb{R}\models \forall x. (x\in {}^*\mathbb{N}) \Rightarrow (0 \leq x).
				\]
				Hence $n$ can only be unlimited and positive.
				
				Note that for any standard $\varepsilon >0$ there is some standard natural number $N$ so that 
				\[
					\mathbb{R} \models \forall x. (x\in \mathbb{N}) \wedge (x \geq N) \Rightarrow |s_x - L| < \varepsilon
				\]
				and hence
				\[
					{}^*\mathbb{R} \models \forall x. (x\in {}^*\mathbb{N}) \wedge (x \geq N) \Rightarrow |{}^*s_x - L| < \varepsilon
				\]
				so, as $n\geq N$, we have that $|{}^*s_n - L| < \varepsilon$ which proves that ${}^*s_n$ is infinitely close to $L$.
				
				Conversely, suppose that $s$ does not converge to $L$. Then there is some standard $\varepsilon >0$ such that for all standard $N\in\mathbb{N}$ there is a standard $n\in\mathbb{N}$ with $n\geq N$ such that $|s_n - L| \geq \varepsilon$. Using the axiom of choice we can construct a (standard) sequence of standard naturals $i_1,i_2,\ldots$ such that $i_j \geq j$ and $|s_{i_j} - L| \geq \varepsilon$ for all standard $j$. Define $b = \langle i_1, i_2,\ldots \rangle\in {}^*\mathbb{R}$.
				
				Clearly $b$ is unlimited and it is a hypernatural since for all $j$ we have $i_j\in\mathbb{N}$. As we have $|s_{i_j} - L| \geq \varepsilon$ we must have $|{}^*s_b - L| \geq \varepsilon$ so ${}^*s_b$ is not infinitely close to $L$, as desired.
				
				Now, let $s\colon \mathbb{N} \to \mathbb{R}$ be a sequence of non-decreasing reals that is bounded above by some standard $M \in\mathbb{R}$. Then
				\[
					\mathbb{R} \models \forall x. (x\in \mathbb{N}) \Rightarrow s_0 \leq s_{x} \leq M
				\]
				so we have
				\[
					{}^*\mathbb{R} \models \forall x. (x\in {}^*\mathbb{N}) \Rightarrow s_0 \leq {}^*s_{x} \leq M.
				\]
				This shows that the hypersequence bounded. Then, by (a), for every unlimited $n\in{}^*\mathbb{N}$ we have that ${}^*s_{n}$ has a shadow to which it is infinitely close. So, by the previous result, we will be done if we can show that for any two unlimited $n,n'\in{}^*\mathbb{N}$ their shadows, say $L$ and $L'$, are equal.
				
				Without loss of generality, assume $L' \leq L$, and for the sake of contradiction, we can assume $L' < L$. Let $\varepsilon >0$ be standard. Note that
				\[
					{}^*\mathbb{R} \models \exists x. (x\in {}^*\mathbb{N}) \wedge |{}^*s_x- L| < \varepsilon/4
				\]
				since $n$ is a witness to this. Hence,
				\[
					\mathbb{R} \models \exists x. (x\in \mathbb{N}) \wedge |s_x- L| < \varepsilon/4.
				\]
				Let $k\in\mathbb{N}$ be a witness to the above. We have that $k$ is smaller than both $n$ and $n'$ since the latter two are unlimited. The hypersequence is easily seen to be non-decreasing, so $s_k\leq s_n$ and $s_k\leq s_{n'}$.
				
				We claim that $s_{n'} \leq L$. Indeed, if this were false, we would have $L' < L < s_{n'}$ and so $0< L-L' < s_{n'} - L'$, which, as $L-L'$ is standard, contradicts the fact that $s_{n'}$ is infinitely close to $L'$. Hence $s_{n'} \leq L$.
				
				Now we have $s_k\leq s_{n'} \leq L$. It follows that
				\begin{equation*}
					|s_{n'} - s_k| = (L - s_k) - (L-s_{n'}) \leq |L-s_k|\leq .
				\end{equation*}
				Finally, note that $|s_{n'} - L'| \leq \varepsilon/2$ so
				\begin{align*}
					|L-L'| &\leq |L- s_k| + |s_k - L'|\\
					&\leq |L-s_k| + |s_k - s_{n'}| + |s_{n'} - L'|\\
					&\leq 2|L-s_k| + |s_{n'} - L'| = 2\frac{\varepsilon}{4} + \frac{\varepsilon}{2} = \varepsilon.
				\end{align*}
				As $\varepsilon$ was arbitrary, we see that $L$ and $L'$ are infinitely close. But they are both standard so $L=L'$.
			\end{enumerate}
			\item\leavevmode
			\begin{itemize}
				\item[$(a)\Rightarrow (b)$:] Let $\mathcal{T} \coloneqq \text{Th}(M)$ be the collection of all $\mathcal{L}$-sentences $M$ satisfies. Obviously $\mathcal{T}$ is consistent but we will apply the Compactness Theorem anyway since our new proof of Compactness shows that the resulting model is an ultraproduct of the models of finite subsets of $\mathcal{T}$.
				
				Suppose we have a finite subset of $\mathcal{T}$. Take the conjunction of all these sentences and call it $\varphi$. Then $M_{\varphi}$ is a finite model of the finite subset. By Compactness there is a model $\text{Th}(M)$ that is an ultraproduct of finite structures.
				\item[$(b)\Rightarrow (c)$:] If a sentence satisfied by all finite structures then it must be satisfied by the ultraproduct and therefore by $M$.
				\item[$(c)\Rightarrow (a)$:] We prove the contrapositive. Let $\varphi$ be a sentence \emph{not} satisfied by any finite $\mathcal{L}$-structure. Then $\neg \varphi$ is satisfied by every finite $\mathcal{L}$-structure and hence $M\models \neg \varphi$ by hypothesis. We conclude that $M\nvDash \varphi$ as desired.
			\end{itemize}
			\item
			\begin{enumerate}
				\item Define 
				\[
				T \coloneqq \{J \subseteq B \mid J \text{ is an ideal of $B$ containing $I$ and }J\cap F =\emptyset\}.
				\]
				We have $I\in T$ so $T$ is nonempty. We use Zorn's Lemma to show that $T$ has a maximal element. Let $C$ be any chain in $T$ and let $U$ be the union of all elements of $C$. Clearly $U$ is an ideal containing $I$ and disjoint from $F$, i.e. an upper bound for $C$.
				
				Zorn's Lemma applies and we have some maximal $M\in T$. We claim that $M$ is prime. Suppose, for the sake of contradiction, that there are some $a,b\in B$ with $a\wedge b \in M$ but $a,b\notin M$.
				
				Consider the set 
				\[
					N\coloneqq \{m \vee (x\wedge a) \mid m\in M,\, x\in B\};
				\]
				we claim that $N$ is an ideal of $B$. Let $m\in M$ and $x\in B$ and suppose $y\in B$ is such that $y\leq m \vee (x\wedge a)$. We have then that
				\begin{align*}
					y &= y \wedge (m\vee (x\wedge a))\\
					&= (y\wedge m) \vee (y\wedge x \wedge a)\\
					&= (m\wedge y) \vee ((y\wedge x)\wedge a)\in N
				\end{align*}
				since $m\wedge y\leq m$ and $M$ is an initial segment. This shows that $N$ is an initial segment.
				
				Now let $m,n\in M$ and $x,y\in B$. Then
				\begin{align*}
					(m\vee (x\wedge a)) \vee (n \vee (y\wedge a)) &= (m\vee n) \vee (x\wedge a)\vee (y\wedge a)\\
					&= (m\vee n) \vee (((x\vee y) \wedge (x\vee a) \wedge (a\vee y)) \wedge a)\in N
				\end{align*}
				since $M$ is closed under binary joins, so $m\vee n\in M$. This shows that $N$ is closed under binary joins. Hence $N$ is an ideal. Similarly, if we define
				\[
					N'\coloneqq \{n \vee (y \wedge b) \mid n\in M,\, y\in B\}
				\]
				we have that $N'$ is an ideal. As both $N$ and $N'$ contain $M$ (here we use the fact that $B$ is Boolean since we note that $m = m\vee (a^* \wedge a)$) we have, by maximality of $M$, that $N$ and $N'$ intersect $F$. Let $m,n\in M$ and $x,y\in B$ be such that $m\vee (x\wedge a)$ and $n\vee (y\wedge b)$ are in $F$. Then their meet is in $F$, but we have
				\begin{equation*}
					(m\vee (x\wedge a)) \wedge (n\vee (y\wedge b)) = (m\wedge n) \vee (m\wedge (x\wedge b)) \vee (n\wedge (x\wedge a)) \vee (x\wedge (a\wedge b))
				\end{equation*}
				which is clearly in $M$, contradicting $M\cap F = \emptyset$. Hence $M$ is a prime ideal, as desired.
				\item We would like to show that the Boolean Prime Ideal Theorem (BPIT) implies the Ultrafilter Principle (UP) in ZF. We show that this suffices. Let $(M_{\alpha})_{\alpha \in I}$ be a family of nonempty sets, we need to show that the product is nonempty. If $I$ is finite this is trivial so suppose $I$ is infinite.
				
				Let $\mathcal{L}$ be the language generated by the empty signature and let $\mathcal{T}$ be the theory of nonempty sets (i.e. containing the single sentence $\exists x. x=x$). Then $(M_{\alpha})_{\alpha \in I}$ is a family of $\mathcal{L}$-structures that model $\mathcal{T}$. Let $\mathcal{U}$ be an ultrafilter on $\mathcal{P}(I)$ containing the Fréchet filter, which we can have by UP. By \L o\'s's Theorem, the ultraproduct $\frac{\prod_{i\in I}M_i}{\mathcal{U}}$ is also a model of $\mathcal{T}$, from which it follows that $\prod_{i\in I}M_i$ is nonempty and we are done
				
				Now we prove UP from BPIT. Let $F$ be a proper filter on $\mathcal{P}(I)$. Note that $\{\emptyset\}$ is an ideal of $\mathcal{P}(I)$ and it doesn't intersect $F$ precisely because $F$ is proper. By BPIT there is a prime ideal $P$ of $\mathcal{P}(I)$ that is disjoint from $F$. 
				
				Define
				\[
					\mathcal{U} \coloneqq \{A \subseteq I \mid A \notin P\}.
				\]
				We claim that $\mathcal{U}$ is an ultrafilter extending $F$. Clearly it does contain $F$ since if $A\in F$ then $A\notin P$.
				
				Note that for all $A\subseteq I$ we have either $A\in P$ or $I\setminus A\in P$. Indeed, $A \cap (I\setminus A) = \emptyset\in P$ and $P$ is prime. In fact, we have exactly one of the scenarios, because if both $A\in P$ and $I\setminus A\in P$ we would have $A\cup I\setminus A = I \in P$ and, as $P$ is an initial segment, this implies that $P=\mathcal{P}(I)$ contradicting the fact that $P$ is disjoint from $F$.
				
				Hence, we must have that for all $A\subseteq \mathcal{U}$ exactly one of $A\in \mathcal{U}$ and $I\setminus A \in \mathcal{U}$ is true. Now, if $A,B\in \mathcal{U}$ we have that $I\setminus A$ and $I \setminus B$ are in $P$, and thus so is their union $I \setminus (A\cap B)$, which implies that $A\cap B \in \mathcal{U}$. Finally, suppose that $A\subseteq B\subseteq I$ and $A\in \mathcal{U}$. We have that $I\setminus A\in P$ and, since $P$ is an initial segment, that $I\setminus B$ is in $P$, which means $B\in \mathcal{U}$. So, $\mathcal{U}$ is an ultrafilter extending $F$.
			\end{enumerate}
			\item Clearly the cardinality of $\mathbb{C}$ is $2^{\aleph_0}$ (since that is the cardinality of $\mathbb{R}$), so it suffices to show that $\prod_{\mathcal{P}}\tilde{\mathbb{F}}_p/\mathcal{U}$ has characteristic 0 and also has cardinality $2^{\aleph_0}$. Note that $\tilde{\mathbb{F}}_p$ is countable for all $p$ (say, by Example Sheet 1 Question 10) so
			\[
				\left|\prod_{\mathcal{P}}\tilde{\mathbb{F}}_p \right| = \aleph_0^{\aleph_0} = 2^{\aleph_0}.
			\]
			As $\left|\prod_{\mathcal{P}}\tilde{\mathbb{F}}_p \right| \geq \left|\prod_{\mathcal{P}}\tilde{\mathbb{F}}_p/\mathcal{U}\right|$ we have that $\prod_{\mathcal{P}}\tilde{\mathbb{F}}_p/\mathcal{U}$ has cardinality at most $2^{\aleph_0}$. (\textcolor{red}{Insert here a proof of a lower bound for the cardinality}). Hence $\left|\prod_{\mathcal{P}}\tilde{\mathbb{F}}_p/\mathcal{U}\right| = 2^{\aleph_0}$. But for any finite number of 1's we have that $[1 + 1 +\cdots + 1 \neq 0]$ is cofinite and hence in $\mathcal{U}$. It follows that $\prod_{\mathcal{P}}\tilde{\mathbb{F}}_p/\mathcal{U}$ cannot have positive characteristic and hence it must be isomorphic to $\mathbb{C}$ as desired.
			
			Let $\phi$ be a sentence in the language of rings. Suppose $\tilde{\mathbb{F}}_p\models \phi$ for all but finitely many $p\in\mathcal{P}$. Then $[\phi]$ is cofinite and hence in $\mathcal{U}$, from which it follows that $\prod_{\mathcal{P}}\tilde{\mathbb{F}}_p/\mathcal{U}\models \phi$ and so $\mathbb{C}\models \phi$. Conversely, suppose $\mathbb{C}\models \phi$. For the sake of contradiction, assume that there is an infinite sequence of primes $p_0<p_1 < \cdots$ such that $\tilde{\mathbb{F}}_{p_i}\models \neg \phi$ for all $i$. Let $\mathcal{U}'$ be a free ultrafilter on the set of natural numbers and we can run the same construction as before to show that $\prod_{\mathbb{N}}\tilde{\mathbb{F}}_{p_i}/\mathcal{U'}$ is isomorphic to $\mathbb{C}$; however this new structure does not believe in $\phi$ while $\mathbb{C}$ does---a contradiction. Hence there can only be finitely many primes not satisfying $\phi$.
			
			\textcolor{red}{Don't know how to do the last part.}
			\item[(+)] \textcolor{red}{TODO}
			
	\end{enumerate}
\end{document}

suppose that $[\phi]$ is not cofinite, i.e., $[\neg\phi]$ is infinite. If $[\neg\phi]$ is cofinite then $[\neg\phi]\in\mathcal{U}$ it soon follows that $\mathbb{C}\models \neg\phi$.

If $[\neg\phi]$ is not cofinite then $\mathcal{P}\setminus [\neg\phi]$ is countable and so we can run the above construction to conclude that $\prod_{\mathcal{P}\setminus[\neg\phi]}\tilde{\mathbb{F}}_p/\mathcal{U}'$ is isomorphic to $\mathbb{C}$, where $\mathcal{U}'$ is a free ultrafilter on $\mathcal{P}\setminus[\neg\phi]$. In this new structure we have $[\neg\phi]$
