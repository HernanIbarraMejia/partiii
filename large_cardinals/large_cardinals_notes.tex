\documentclass{report}
%Setting margins
%\usepackage[margin = 1.75in]{geometry}
%Basic Maths
\usepackage{amsmath}
\usepackage{amssymb}
\usepackage{mathtools}
\usepackage{gensymb}
%For definining theorem-like environments
\usepackage{amsthm}
%For beautiful letters (e.g. for a partition, see $\mathscr{P}$)
\usepackage{mathrsfs}
%For importing the solution file
%\usepackage{import}
%For drawing commutative diagrams
\usepackage{quiver}
%For pretty colours
\usepackage{xcolor}
%For scaling some relations, for instance see https://tex.stackexchange.com/a/108482
\usepackage{mleftright}
%Set paragraph spacing. I believe this is close to what is used in the book.
%\usepackage[skip=.3\baselineskip, indent = 15pt]{parskip}
%To customize lists
\usepackage{enumitem}
%To strikethrough terms in equations
\usepackage{cancel}
%For bibliography
\usepackage[backend=biber]{biblatex}
%For pictures
\usepackage{tikz}
\usetikzlibrary{calc,positioning}

\usepackage[hidelinks]{hyperref}
\usepackage{soul}
\usepackage[normalem]{ulem}
\usepackage{lipsum}
\usepackage{breqn}

%\addbibresource{main.bib}


\newcommand{\myhy}[2]{\href{#1}{\color{blue}\setulcolor{blue}\ul{#2}}}

%Fix section numbering to match the book's convention
\renewcommand\thesection{\arabic{section}}

%Displays "Exercises". To put after each section.
\newcommand{\extitle}{\subsection*{Exercises}}

%For personal notes
\newcommand{\note}[1]
{\smallskip {\noindent\textbf{Note} #1}}

%Roman numerals!
\newcommand{\RNo}[1]{%
	\textup{\uppercase\expandafter{\romannumeral#1}}%
}

%San-serif for names of categories
\newcommand{\serif}[1]{{\fontfamily{cmss}\selectfont #1}}
\newcommand{\srf}{\textsf}

%Shorthands for common sets
\newcommand{\N}{\mathbb{N}}
\newcommand{\Z}{\mathbb{Z}}
\newcommand{\Q}{\mathbb{Q}}
\newcommand{\R}{\mathbb{R}}
\newcommand{\C}{\mathbb{C}}
\newcommand{\zmod}[1]{\bZ/#1\bZ}

%Miscellaneous commands
\newcommand{\defeq}{\coloneqq}
\newcommand{\divides}{\mid}
\newcommand{\legendre}[2]{\ensuremath{\left( \frac{#1}{#2} \right) }}
\newcommand{\Mod}[1]{\ (\mathrm{mod}\ #1)}
\newcommand{\mbold}[1]{\mathrm{\mathbf{#1}}}


%Useful operations and delimiters
\DeclareMathOperator{\Hom}{Hom}
\DeclareMathOperator{\End}{End}
\DeclareMathOperator{\Aut}{Aut}
\DeclareMathOperator{\Obj}{Obj}
\DeclareMathOperator{\id}{id}
\DeclareMathOperator{\lcm}{lcm}
\DeclareMathOperator{\GL}{GL}
\DeclareMathOperator{\SO}{SO}
\DeclareMathOperator{\SL}{SL}
\DeclareMathOperator{\U}{U}
\DeclareMathOperator{\SU}{SU}
\DeclareMathOperator{\Inn}{Inn}
\DeclareMathOperator{\PSL}{PSL}
\DeclareMathOperator{\im}{im}
\DeclareMathOperator{\coker}{coker}
\DeclareMathOperator{\rot}{rot}
\DeclareMathOperator{\rf}{ref}
\DeclareMathOperator{\Symm}{Symm}
\DeclareMathOperator{\vspan}{span}
\DeclareMathOperator{\ev}{ev}
\DeclareMathOperator{\Gal}{Gal}
\DeclareMathOperator{\ob}{ob}
\DeclareMathOperator{\mor}{mor}
\DeclareMathOperator{\dom}{dom}
\DeclareMathOperator{\cod}{cod}
\DeclareMathOperator{\Cone}{Cone}
\DeclareMathOperator{\cf}{cf}
\DeclarePairedDelimiter\abs{\lvert}{\rvert}%
\DeclarePairedDelimiter\norm{\lVert}{\rVert}%
\DeclarePairedDelimiter\innprod{\langle}{\rangle}%
\DeclarePairedDelimiter\ceil{\lceil}{\rceil}
\DeclarePairedDelimiter\floor{\lfloor}{\rfloor}
%Claim environment
\newtheorem{claim}{Claim}


%Exercise environment
\theoremstyle{definition}
\newtheorem{ex}{Exercise}

%Standard theorem-like environment
\theoremstyle{plain}
\newtheorem{thm}{Theorem}[section]

\newtheorem{prop}[thm]{Proposition}
\newtheorem{lem}[thm]{Lemma}
\newtheorem{coro}[thm]{Corollary}
\newtheorem{prob}{Problem}
\newtheorem{conj}{Conjecture}


\theoremstyle{definition}
\newtheorem{defn}[thm]{Definition}
\newtheorem{nondefn}[thm]{Non-definition}
\newtheorem{rem}[thm]{Remark}
\newtheorem{eg}[thm]{Example}
\newtheorem{noneg}[thm]{Non-example}
\newtheorem{egs}[thm]{Examples}
\newtheorem{fact}[thm]{Fact}
\newtheorem{task}{Task}



%Solution environment
\newenvironment{solution}
{\begin{proof}[Solution]}
	{\end{proof}}

%Function restrictions
% From https://tex.stackexchange.com/a/22255
\newcommand\restr[2]{{% we make the whole thing an ordinary symbol
		\left.\kern-\nulldelimiterspace % automatically resize the bar with \right
		#1 % the function
		\vphantom{\big|} % pretend it's a little taller at normal size
		\right|_{#2} % this is the delimiter
}}

%\newcommand\nvdash{\mkern-2mu\not\mkern2mu\vdash}

\makeatother
\begin{document}
	\title{Large Cardinals}
	\maketitle
	\section{Introduction}
	Modern set theory is all about understanding the consequences of Gödel's incompleteness theorems, which say, roughly, that any reasonable set theory $T$ is incomplete, and furthermore it cannot prove its own consistency.
	
	Hence there are sentences $\varphi$ such that
	\[
		\{\psi \mid T\vdash \psi\} \subsetneq \{\psi \mid T \cup \{\varphi\} \vdash \psi\}.
	\]
	We express this by saying that $T$ has strictly fewer consequences than $T+\varphi$, i.e., $T<_{\text{consequence}} T+\varphi$. Set theory is concerned about understanding the ordering $\leq_{\text{consequence}}$ and other similar relations. Large cardinal axioms form the most natural hierarchy to form the basis of measuring the strengths of different set theories.
	
	The irony is that we will not be formally defining what a large cardinal is.
	\begin{nondefn}[Large Cardinal]
		A \emph{large cardinal property} is a formula $\Phi(x)$ in the language of set theory such that $\Phi(\kappa)$ implies that 
		\begin{itemize}
			\item $\kappa$ is a very large cardinal; and
			\item $\kappa$ is so large that its existence cannot be proved in ZFC.
		\end{itemize}
		By a \emph{large cardinal axiom} we mean a sentence $\Phi C$ of the form $\exists \kappa. \Phi(\kappa)$ with $\Phi$ as above.
	\end{nondefn}
	\begin{noneg}
		A cardinal $\kappa$ is called an $\aleph$ \emph{fixed point} if $\kappa = \aleph_\kappa$. These are \emph{very} large. Obviously $n \neq  \aleph_{n}$ for finite $n$. But not even $\aleph_{\omega}$ is an $\aleph$ fixed point.
		
		However we have the following result. Say a class function $F\colon \text{Ord} \to \text{Ord}$ is \emph{normal} if it satisfies the following conditions.
		\begin{itemize}
			\item If $\alpha \leq \beta$ then $F(\alpha) \leq F(\beta)$.
			\item If $\lambda$ is a limit ordinal then $F(\lambda) = \bigcup_{\alpha < \lambda}F(\alpha)$.
		\end{itemize}
		It is a theorem that all normal functions have arbitrarily large fixed points (this will be in ES1). As the map $\alpha \mapsto \aleph_\alpha$ is normal, we have that ZFC proves the existence of $\aleph$ fixed points.
	\end{noneg}
	\begin{noneg}
		Let $\Phi$ be the formula $(\kappa =  \aleph_{\kappa}) \wedge \text{Con(ZFC)}$ where Con(ZFC) is the formula expressing that ZFC is consistent (Gödel showed that there is such a thing). By the second incompleteness theorem we see that $\text{ZFC}\nvdash \Phi C$. 
		
		While this technically satisfies our non-definition, it is cheating. We would like the size $\kappa$ to be the reason we cannot prove $\Phi C$.
	\end{noneg}
	In order to find a genuine example of a large cardinal axiom, we make an analogy with $\omega$. Note that $\omega$ is far bigger that any $n<\omega$. So, we will try to express this in mathematical properties and use them as a large cardinal property.
	\begin{enumerate}
		\item If $n<\omega$ then $n^+ < \omega$ where $n^+$ denotes the cardinal successor of $n$ (i.e. the smallest cardinal bigger than $n$). So, define $\Lambda(\kappa)$ to be the formula $\forall \alpha. (\alpha < \kappa \Rightarrow \alpha^+ < \kappa)$. If $\Lambda(\kappa)$ is true we call $\kappa$ a \emph{limit cardinal}. For example, $\aleph_{\omega}$ is a limit cardinal. Hence limit cardinals are neither very large nor their existence unprovable.
		\item If $n<\omega$ then $2^n< \omega$. So, define $\Sigma(\kappa)$ to be the formula $\forall \alpha.(\alpha < \kappa \Rightarrow 2^{\alpha} < \kappa)$. We say that $\kappa$ is a \emph{strong limit cardinal} if $\Sigma(\kappa)$ is true.
		
		Similarly to the $\aleph$-hierarchy we can define the $\beth$ operator as follows
		\begin{align*}
			\beth_0 &\coloneqq \aleph_0\\
			\beth_{\alpha^+} &\coloneqq 2^{\beth_{\alpha}}\\
			\beth_{\lambda} &\coloneqq \sup\{\beth_\alpha \mid \alpha < \lambda\}. 
		\end{align*}
		By induction $\aleph_{\alpha} \leq \beth_{\alpha}$. Clearly $\kappa$ is a strong limit cardinal iff $\kappa = \beth_{\lambda}$ for a limit cardinal $\lambda$ (\textcolor{red}{why? see \url{https://math.stackexchange.com/questions/434205/are-the-strong-limit-cardinals-precisely-those-of-the-form-beth-lambda-wher}}). Thus $\text{ZFC} \vdash \Sigma C$.
		
		\item If $s\colon n \to \omega$ is a function then $\sup s < \omega$ for all $n<\omega$.
		\begin{defn}[Cofinality]
			Let $\lambda$ be a limit ordinal. We say $C\subseteq \lambda$ is \emph{cofinal} or \emph{unbounded} if $\bigcup C = \lambda$. The \emph{cofinality} of $\lambda$ is
			\[
				\cf(\lambda)\coloneqq \min\{|C| \,\mid \, C \text{ is cofinal in }\lambda\}.
			\]
		\end{defn}
		Clearly $\cf(\lambda)\leq |\lambda|$.  So if $\lambda$ is a cardinal we have $\cf(\lambda)\leq \lambda$. Note that the following definition makes sense since all infinite cardinals are limit ordinals.
		\begin{defn}[Regular cardinals]
			An infinite cardinal $\kappa$ is called \emph{regular} if $\cf(\kappa) = \kappa$, and \emph{singular} if $\cf(\kappa) < \kappa$.
		\end{defn}
		\begin{egs}
			$\aleph_0$ is regular. Moreover, $\aleph_1$ is regular since union of countable sets are countable. In general $\aleph_{\alpha+1}$ is always regular (ES1).
			
			Note that $\aleph_{\omega} = \bigcup\{\aleph_{n} \mid n\in \mathbb{N}\}$. But then $\{\aleph_{n} \mid n\in \mathbb{N}\}$ is a cofinal set in $\aleph_{\omega}$ of countable cardinality. Hence $\aleph_{\omega}$ is singular. Limit cardinals tend to be singular. 
		\end{egs}
	\end{enumerate}
	\begin{defn}[Inaccessible cardinals]
		A cardinal $\kappa$ is called \emph{weakly inaccessible} if it's an uncountable regular limit cardinal. It is called \emph{strongly inaccessible} (or just inaccessible) if it is, in addition, a strong limit cardinal.
	\end{defn}
	Let $\text{WI}(\kappa)$ be the formula expressing that $\kappa$ is weakly inaccessible and $\text{I}(\kappa)$ to be the one expressing that $\kappa$ is strongly inaccessible. Our goal is to argue that WI and I are large cardinal properties in the sense of our non-definition. First we show they are very large indeed.
	\begin{prop}
		Weakly inaccessible cardinals are $\aleph$ fixed points.
	\end{prop}
	\begin{proof}
		Suppose $\kappa$ is not an $\aleph$ fixed point. Then $\kappa < \aleph_k$. It follows that $\kappa = \aleph_{\alpha}$ for some ordinal $\alpha < \kappa$. If $\alpha$ is a successor ordinal $\alpha = \beta + 1$ then $\kappa$ is the successor cardinal of $\aleph_\beta$; hence $\kappa$ is not weakly inaccessible. On the other hand, if $\alpha$ is a limit ordinal we have
		\[
			\kappa = \aleph_{\alpha} = \bigcup_{\beta < \alpha}\aleph_{\beta}
		\]
		and so $\cf(\kappa) \leq |\alpha| < \kappa$ and hence $\kappa$ is singular. In particular, $\kappa$ is not weakly inaccessible.
	\end{proof}
\end{document}