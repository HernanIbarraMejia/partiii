\documentclass{article}
\usepackage{amsmath}
\usepackage{amsthm}
\usepackage{enumitem}
\usepackage{amssymb}
\usepackage{mathtools}
\usepackage{url}
\usepackage{xcolor}
\usepackage{stmaryrd}
\usepackage{quiver}

\theoremstyle{theorem}
\newtheorem{claim}{Claim}
\newtheorem{prop}{Proposition}

\DeclareMathOperator{\tp}{tp}
\DeclareMathOperator{\Diag}{Diag}
\DeclareMathOperator{\Th}{Th}
\DeclareMathOperator{\dom}{dom}
\DeclareMathOperator{\cod}{cod}
\DeclareMathOperator{\im}{im}



\begin{document}
	\title{Model Theory and Non-Classical Logic\\ Example Sheet 3 Solutions}
	\author{Hernán Ibarra Mejia}
	\maketitle
	A partial function $f\colon M \to N$ is called \emph{partial elementary} if for all $\varphi(\bar{x})\in\mathcal{L}$ and $\bar{d}\in \dom(f)$
	\[
	M \models \varphi(\bar{d}) \,\,\,\iff\,\,\, N \models \varphi(f(\bar{d})).
	\]
	In particular, these functions are injective (consider $\varphi(x,y)$ to be the formula $(x=y)$). If $f\colon M \to N$ is partial elementary and $A \coloneqq \dom f$ then we call the map $f\colon A \to N$ just elementary.
	
	We also use the following strengthening of elementary amalgamation (which is proved in the same way).
	\begin{prop}
		Let $M$ be a structure. Suppose $A$ is a subset of $M$ and $f\colon A \to M$ is an elementary map. Then there exists an elementary extension $N$ of $M$ and an elementary embedding $g\colon M \to N$ such that the following diagram commutes.
		% https://q.uiver.app/#q=WzAsNCxbMSwyLCJBIl0sWzAsMSwiTSJdLFsyLDEsIk0iXSxbMSwwLCJOIl0sWzAsMSwiIiwwLHsic3R5bGUiOnsidGFpbCI6eyJuYW1lIjoiaG9vayIsInNpZGUiOiJ0b3AifX19XSxbMCwyLCJmIiwyXSxbMSwzLCJnIl0sWzIsMywiIiwyLHsic3R5bGUiOnsidGFpbCI6eyJuYW1lIjoiaG9vayIsInNpZGUiOiJib3R0b20ifX19XV0=
		\[\begin{tikzcd}[cramped]
			& N \\
			M && M \\
			& A
			\arrow["g", from=2-1, to=1-2]
			\arrow[hook', from=2-3, to=1-2]
			\arrow[hook, from=3-2, to=2-1]
			\arrow["f"', from=3-2, to=2-3]
		\end{tikzcd}\]
	\end{prop}
	\begin{proof}
		
	\end{proof}
	\begin{enumerate}[leftmargin=*]
		\item Let $p$ be a complete 1-type. Then for all $a\in \mathbb{Q}$ exactly one of $x<a$, $x=a$, and $x>a$ is in $p$.
		
		Suppose that $(x=a)\in p$ for some $a\in\mathbb{Q}$. We claim that $p = \tp^\mathbb{Q}(a/\mathbb{Q})$. By Corollary 1.4.7, there is an elementary extension $M$ of $\mathbb{Q}$ such that $p = \tp^{M}(a/\mathbb{Q})$, where we used the fact that $(x=a)\in p$. But $\tp^{M}(a/\mathbb{Q})=\tp^\mathbb{Q}(a/\mathbb{Q})$ as the extension is elementary, so the claim follows.
		
		Now assume that $(x=a)\notin p$ for all $a\in\mathbb{Q}$. Then the sets
		\[
			U\coloneqq \{a\in\mathbb{Q}\mid (x<a)\in p\} \,\,\,\text{ and }\,\,\, L\coloneqq \{b\in\mathbb{Q} \mid (b<x)\in p\}
		\]
		partition $\mathbb{Q}$. Note that if $a\in U$ and $b\in L$, then $b<a$. Indeed, the sets are disjoint, so the only alternative is that $a<b$. But from this it follows that in any realization of the type $p$, say by an element $c$, we must have $c<a<b<c$ and thus $c<c$. As $\mathbb{Q}\models \forall y. \neg(y<y)$ this is a contradiction. Hence the complete type $p$ gives rise to a partition $U,L$ of $\mathbb{Q}$ such that $L<U$ in the above sense.
		
		Conversely, any such partition can be extended to a complete type by the Ultrafilter Principle, and it is clear that this correspondence is bijective.
		
		\item We prove a more general result. Let $M$ be a structure and $S$ a subset of $M$ together with an elementary map $f\colon S \to M$. Then there is an elementary extension $N$ of $M$ and an automorphism of $N$ extending $f$.
		
		To see that the result immediately solves the problem, observe the following. As $\tp^M(\bar{a}/A) = \tp^M(\bar{b}/B)$ it follows that the map $f\colon A\cup \{\bar{a}\} \to M$ given by fixing $A$ and sending $\bar{a}\mapsto \bar{b}$ is elementary.
		
		Now we prove the result. We construct a chain of structures and maps $(M_i,f_i)_{i<\omega}$ satisfying the following properties for all $i\geq 0$.
		\begin{itemize}
			\item $M_i\preccurlyeq M_{i+1}$.
			\item $\im f_i \subseteq \dom f_{i+1}$ and $f_{i+1}$ extends $f_i$.
			\item If $i$ is even $\dom(f_{i + 1}) = M_{i}$.
			\item If $i$ is odd $\im(f_{i + 1}) = M_{i}$.
		\end{itemize}
		Furthermore we will have $M_0 \coloneqq M$ and $f_0 =f$.
		
		Suppose, for a moment, that we have finished the construction. Then let $N \coloneqq \bigcup_{i<\omega}M_i$, which is an elementary extension of $M$. Also let $\sigma \coloneqq \bigcup_{i<\omega} f_i\colon N \to N$, which is clearly an automorphism since $\dom(\sigma)=\im(\sigma)= N$. Also $\sigma$ extends $f_0 = f$ so we will be done.
		
		We continue with the construction. Use elementary amalgamation to have the following diagram.
		% https://q.uiver.app/#q=WzAsNCxbMSwyLCJTIl0sWzAsMSwiTV8wIl0sWzIsMSwiTV8wIl0sWzEsMCwiTV8xIl0sWzAsMSwiIiwwLHsic3R5bGUiOnsidGFpbCI6eyJuYW1lIjoiaG9vayIsInNpZGUiOiJ0b3AifX19XSxbMCwyLCJmPWZfMCIsMl0sWzEsMywiZl8xIl0sWzIsMywiIiwyLHsic3R5bGUiOnsidGFpbCI6eyJuYW1lIjoiaG9vayIsInNpZGUiOiJib3R0b20ifX19XV0=
		\[\begin{tikzcd}[cramped]
			& {M_1} \\
			{M_0} && {M_0} \\
			& S
			\arrow["{f_1}", from=2-1, to=1-2]
			\arrow[hook', from=2-3, to=1-2]
			\arrow[hook, from=3-2, to=2-1]
			\arrow["{f=f_0}"', from=3-2, to=2-3]
		\end{tikzcd}\]
		It is easy to check that this satisfies the requirements. We can apply amalgamation to $f_1^{-1}$ to obtain the following diagram
		% https://q.uiver.app/#q=WzAsNSxbMSwzLCJcXGltKGZfMSkiXSxbMCwyLCJNXzAiXSxbMCwxLCJNXzEiXSxbMiwxLCJNXzEiXSxbMSwwLCJNXzIiXSxbMCwxLCJmXzFeey0xfSJdLFswLDMsIiIsMix7InN0eWxlIjp7InRhaWwiOnsibmFtZSI6Imhvb2siLCJzaWRlIjoidG9wIn19fV0sWzEsMiwiIiwwLHsic3R5bGUiOnsidGFpbCI6eyJuYW1lIjoiaG9vayIsInNpZGUiOiJ0b3AifX19XSxbMiw0LCIiLDAseyJzdHlsZSI6eyJ0YWlsIjp7Im5hbWUiOiJob29rIiwic2lkZSI6InRvcCJ9fX1dLFszLDQsImciLDJdXQ==
		\[\begin{tikzcd}[cramped]
			& {M_2} \\
			{M_1} && {M_1} \\
			{M_0} \\
			& {\im(f_1)}
			\arrow[hook, from=2-1, to=1-2]
			\arrow["g"', from=2-3, to=1-2]
			\arrow[hook, from=3-1, to=2-1]
			\arrow[hook, from=4-2, to=2-3]
			\arrow["{f_1^{-1}}", from=4-2, to=3-1]
		\end{tikzcd}\]
		Define $f_2\colon \im g \to M_1$ to be $g^{-1}$. It follows from the diagram that $g\circ f_1(x) = x$ for $x\in M_0$. Hence we have $f_1(x) = f_2(x)$ for all $x\in M_0$ and so $f_2$ extends $f_1$. Furthermore it is easy to see that $f_2$ is an isomorphism $\im g \to M_1$ so in particular it is surjective and so $\im(f_2) = M_1$. Hence all conditions are satisfied. Keep going on in this way (apply amalgamation to $f_i$ when $i$ is even and to $f_{i}^{-1}$ when $i$ is odd). 
		\item First we show that, given a finite number of complete types $p_1,\ldots,p_k\in S_{n}^M(M)$ there is an elementary extension $N$ of $M$ realizing all of them. For $k=0$ this is trivial. Now suppose there is an elementarily extension $N'$ of $M$ realizing $p_1,\ldots,p_{k-1}$. Note that $S_{n}^M(M)= S_n^{N'}(M)$ since the extension is elementary, so in particular $p_k\in S_n^{N'}(M)$ and by Proposition 1.4.6 there is an elementary extension $N$ of $N'$ realizing $p_k$. Obviously $N$ is an elementary extension of $M$ realizing $p_1,\ldots, p_k$ so we are done by induction.
		
		Back to the main problem. To the language $\mathcal{L}$ we add a constant for each element of $M$ and we add $n$ constants $c_1^p,\ldots, c_n^p$ for every $p\in S_n^M(M)$. In the expanded language, consider the theory
		\[
			\left(\bigcup S_{n}^M(M)\right) \cup \Diag_{\text{el}}(M)  
		\]
		where  each $\varphi(\bar{x})\in p\in S_n^{N}$ is replaced by $\varphi(\bar{c}^p)$. Clearly if this theory is consistent then we are done. But every finite subset of this theory is satisfied by an elementary extension of $M$ that has to realize only finitely many types, so we are done by our previous result.
		\item \leavevmode
		\begin{enumerate}
			\item Let $p,q$ be distinct types. Without loss of generality, we assume that there is a formula $\varphi(\bar{x})$ such that $\varphi\in p$ but $\varphi \notin q$. Then $\llbracket \varphi \rrbracket$ is a clopen set containing $p$ but not $q$. This shows that $S_n^M(A)$ is totally disconnected.
			
			For the second part, we need a claim.
			\begin{claim}
				Let $F$ be a set of $\mathcal{L}_A$ formulae with $n$ variables. Add $n$ new constants $\bar{c}$ to the language. Then the set $\mathcal{C}\coloneqq \{\,\llbracket\varphi\rrbracket \mid \varphi(\bar{x}) \in F\,\}$ covers $S_{n}^M(A)$ if and only if the theory
				\[
					\mathcal{T}\coloneqq \Th_A(M)\cup \{\,\neg \varphi(\bar{c}) \mid \varphi(\bar{x})\in F\,\}
				\]
				is inconsistent.
			\end{claim}
			\begin{proof}
				Suppose $\mathcal{T}$ were consistent. Then $\neg F$ is an $n$-type, which, by the Ultrafilter Principle, can be extended to a complete $n$-type $q\in S_n^M(A)$. For all $\varphi\in F$ we must have $\neg\varphi \in q$, which means $\varphi\notin q$; thus $\mathcal{C}$ does not cover $q$.
				
				Conversely, suppose that there is some $q\in S_n^M(A)$ such that $\varphi\notin q$ for all $\varphi\in F$. That means that $\neg\varphi\in q$ for all $\varphi \in F$ since $q$ is complete. By definition of type, we have that $\Th_A(M)\cup q$ is consistent when we replace the variables $\bar{x}$ in $q$ by the constants $\bar{c}$. It follows that $\mathcal{T}$ is consistent.
			\end{proof}
			Back to the problem, let $\mathcal{C}$ be an open cover of $S_n^M(A)$. As open sets are unions of basis elements, we can assume that $\mathcal{C}$ is of the form $\{\,\llbracket\varphi\rrbracket \mid \varphi(\bar{x}) \in F\,\}$ for some set of $\mathcal{L}_A$-formulae $F$.
			
			Now we know that $\mathcal{T}$ is inconsistent, where $\mathcal{T}$ is as in Claim 1. By the Compactness Theorem (for first-order logic) there is a finite subset $\mathcal{T}'$ of $\mathcal{T}$ that is inconsistent. Hence there is a finite subset $F'$ of $F$ such that $\Th_A(M)\cup \{\neg\varphi(\bar{c}) \mid \varphi(\bar{x})\in F'\}$ is inconsistent. Again by Claim 1, the set $\mathcal{C}'\coloneqq \{\,\llbracket\varphi\rrbracket \mid \varphi(\bar{x}) \in F'\,\}$, which is a finite subset of $\mathcal{C}$, covers $S_n^M(A)$.
			\item \textcolor{red}{I think that he meant to define $f^*(p)\coloneqq \{\phi(\bar{x},f(\bar{a}))\mid \phi(\bar{x},\bar{a})\in p\}$, and that we need to show that $f^*(p)\in S_n^N(f(A))$.}
			
			To show that $f^*(p)\in S_n^N(f(A))$ first we need to show that 
			\[
				\Th_{f(A)}(N) \cup f^*(p)
			\]
			is satisfiable. By assumption $p$ is an $n$-type, so there is an elementary extension $X$ of $M$ and a tuple $\bar{r}\in X$ with $\phi(\bar{r},\bar{a})$ for all $\phi(\bar{x},\bar{a})\in p$. Note that $X$ can also be interpreted as an $\mathcal{L}_{f(A)}$-structure. As $f$ is elementary it is clear that $X\models \Th_{f(A)}(N)$ and is immediate $X\models \phi(\bar{r},f(\bar{a}))$. This all shows that $f^*(p)$ is an $n$-type; and it is complete since $p$ is complete.
			
			Now we show that $f^*$ is continuous. By general topology, it suffices to show that for each basis element $\llbracket \varphi\rrbracket\subseteq S_n^{N}(f(A))$ the set $(f^*)^{-1}(\llbracket \varphi\rrbracket)$ is open in $S_n^M(A)$.
			
			So, let $\varphi(\bar{x},f(\bar{a}))$ be an $\mathcal{L}_{f(A)}$-formula. Then
			\begin{align*}
				(f^*)^{-1}(\llbracket \varphi\rrbracket) &= \{p\in S_n^M \mid f^*(p)\in \llbracket \varphi\rrbracket\}\\
				&= \{p\in S_n^M \mid \varphi\in f^*(p)\}\\
				&= \{p\in S_n^M \mid \varphi(\bar{x},f(\bar{a}))= \psi(\bar{x}, f(\bar{a})) \text{ for some }\psi(\bar{x},\bar{a})\in p\}.
			\end{align*}
			But notice that, as $f$ is injective, $\varphi(\bar{x},f(\bar{a}))= \psi(\bar{x}, f(\bar{a}))$ implies that $\psi = \phi$. Thus,
			\[
				(f^*)^{-1}(\llbracket \varphi\rrbracket) = \{p\in S_n^M \mid \varphi(\bar{x},\bar{a})\in p\} = \llbracket \varphi(\bar{x},\bar{a})\rrbracket
			\] 
			which is open in $S_n^M(A)$.
		\end{enumerate}
		\item In the proof of Theorem 1.4.11 we make a small modification. Let $f\colon \omega \to \omega \times \omega$ be a bijection. When we define $\theta_s$ for odd $s= 2i + 1$ then if $f(i) = (j,k)$ instead of taking $\bar{d}_i$ we take $\bar{d}_j$, run the same process to get $\psi$ and we let $\varphi$ be a formula in $p_k$ that is not implied by $\psi$. The rest of the proof is the same, except that at the very end we notice that if $\bar{c}\in C^n$ then $\bar{c}= \bar{d}_j$ for some $j$ and that if $k< \omega$ then we can define $i\coloneqq f^{-1}(j,k)$; it then follows that $\theta_{2i+2}$ implies that $\bar{c}$ does not realize $p_k$. As $\bar{c}$ and $k$ were arbitrary, we are done.
		
		\item Let $\mathcal{L}$ be the language which has as signature $\aleph_2$ constants, say $\{c_\alpha\}_{\alpha < \aleph_2}$. Let $\mathcal{T}$ be the theory of uncountable sets, i.e.
		\[
			\mathcal{T} \coloneqq \{c_\alpha \neq c_{\alpha'} \mid \alpha<\alpha' <\aleph_1\}.
		\]
		Define
		\[
			p \coloneqq \{x \neq c_{\alpha} \mid \alpha < \aleph_2\}.
		\]
		It is easy to check that this is a 1-type of $\mathcal{T}$ (any set of cardinality bigger than $\aleph_2$ is a model of $\mathcal{T}$ realizing $p$). 
		
		For the sake of contradiction, suppose this type is isolated by a formula $\varphi(x)$. This means that $\mathcal{T}\cup \{\varphi(x)\}$ is satisfiable and
		\[
			\mathcal{T} \models \forall x. (\varphi(x) \rightarrow x \neq c_{\alpha})
		\]
		for all $\alpha < \aleph_2$. Choose $\alpha$ such that $c_{\alpha}$ does not appear in $\mathcal{T}$ nor in $\varphi(x)$; this is possible because $\mathcal{T}$ mentions only $\aleph_1$-many constants and $\varphi(x)$ only finitely many. Hence by generalization we have
		\[
		\mathcal{T} \models \forall x,y. (\varphi(x) \rightarrow x \neq y).
		\]
		In particular,
		\[
		\mathcal{T} \models \forall x. (\varphi(x) \rightarrow x \neq x).
		\]
		contradicting the fact that $\mathcal{T} \cup{\varphi(x)}$ is satisfiable. Thus $p$ is not isolated. However there can be no countable model of $\mathcal{T}$ ommiting $p$ since $\mathcal{T}$ has no countable models!
		
		\item Let $\mathcal{M}$ be a countable model of PA. If $\mathcal{L}$ denotes the language of arithmetic, we add a constant $c$ to the language and let $\mathcal{L}^+ \coloneqq \mathcal{L}_{\{c\}\cup \mathcal{M}}$. We define the $\mathcal{L}^+$-theory
		\[
			\mathcal{T} \coloneqq \Diag_{\text{el}}(M) \cup \{c > m \mid m\in \mathcal{M}\}.
		\]
		This theory is clearly consistent by Compactness. 
		
		Say an element $m\in \mathcal{M}$ is \emph{natural} if the interval $[0,m]$ is finite. If $m$ is natural then there are $m_1,\ldots,m_k\in \mathcal{M}$ such that
		\[
			\mathcal{M} \models \forall x. \left(\bigwedge_{i=1}^k x \neq m_i \rightarrow x > m\right)
		\]
		
		If $m_0$ is not natural we say it is \emph{unnatural}. For every unnatural $m_0\in \mathcal{M}$ we define 
		\[
			p_{m_0} \coloneqq \{x \neq  m \mid m\in \mathcal{M}\} \cup\{x < m_0\}.
		\]
		We claim that $p_{m_0}$ is a non-isolated type over $\mathcal{T}$ for all unnatural $m_0\in\mathcal{M}$. Firstly, if $p\subseteq p_{m_0}$ is finite then $\mathcal{T}\cup p$ is satisfiable precisely because $[0,m_0]$ is infinite, so $p_{m_0}$ is a 1-type over $\mathcal{T}$. 
		
		Suppose, for the sake of contradiction, that $p_{m_0}$ is isolated by an $\mathcal{L}^+$-formula $\varphi(x)$. Write $\varphi(x) = \psi(x,c)$ where $\psi$ is an $\mathcal{L}_{\mathcal{M}}$-formula. As $\varphi$ is an isolating formula we have that $\mathcal{T}\cup\{\varphi(x)\}$ is satisfiable. So let $\mathcal{M}'$ be a satisfying structure. Clearly $\mathcal{M}'$ is an elementary extension of $\mathcal{M}$ that includes two constants $c^{\mathcal{M}'},d^{\mathcal{M'}}$ such that $\mathcal{M'} \models c^{\mathcal{M}'} > m$ for all $m\in\mathcal{M}$ and $\mathcal{M'} \models \psi(d^{\mathcal{M'}},c^{\mathcal{M'}})$. As $\varphi(x)$ isolates $p_{m_0}$ we have in particular that $\psi(x,c)$ implies that $x < m_0$. Thus
		\[
			\mathcal{M}' \models \psi(d^{\mathcal{M'}},c^{\mathcal{M'}}) \wedge d^{\mathcal{M'}} < m_0.
		\]
		It follows that 
		\[
			\mathcal{M}' \models \exists z < m_0. \psi(z,c^{\mathcal{M'}}).
		\]
		Furthermore, for all $m\in \mathcal{M}$ we have that 
		\[
			\mathcal{M}' \models \exists y > m \exists z < m_0. \psi(z,y).
		\]
		Thus, as $\mathcal{M}$ is an elementary substructure of $\mathcal{M}'$ we have that 
		\[
			\mathcal{M} \models \exists y > m. \exists z < m_0. \psi(z,y),
		\]
		for all $m\in\mathcal{M}$. Hence,
		\[
		\mathcal{M} \models \forall x. \exists y > x.\exists z < m_0. \psi(z,y).
		\]
		Now, PA proves the pigeonhole principle. Think of $z$ as the pigeonholes, $y$ as the pigeons, and $\psi(z,y)$ as the statement ``pigeon $y$ is assigned pigeonhole $z$''; then it is clear that, as we are trying to fit infinitely many pigeons in finitely many pigeonholes there is a pigeonhole with arbitrarily many pigeoholes (of course $[0,m_0]$ is not actually finite but PA proves the pigeonhole principle nevertheless). Thus,
		\[
		\mathcal{M} \models \exists z < m_0.\forall x. \exists y > x. \psi(z,y).
		\]
		In particular, there is some $m_1\in \mathcal{M}$ such that $m_1< m_0$ and 
		\[
		\mathcal{M} \models \forall x. \exists y > x. \psi(m_1,y).
		\]
		We claim that $\mathcal{T}\cup\{\psi(m_1,c)\}$ is consistent. If it were inconsistent then, by Compactness, there is $n\in \mathcal{M}$ such that
		\[
			\Diag_{\text{el}}(M) \cup \{c> n\} \models \neg\psi(m_1,c),
		\]
		and thus $\Diag_{\text{el}}(M) \models c > n \rightarrow \neg\psi(m_1,c)$. As $\Diag_{\text{el}}(M)$ does not mention $c$ we have, by generalization, 
		\[
			\Diag_{\text{el}}(M) \models \forall y> n. \neg\psi(m_1,c),
		\]
		contradicting the defining property of $m_1$. We conclude that $\mathcal{T}\cup\{\psi(m_1,c)\}$ is consistent. As $\varphi(x)$ isolates $p_{m_0}$ we also must have
		\[
			\mathcal{T} \models \psi(m_1,c) \rightarrow m_1 \neg m_1,
		\]
		a contradiction. Thus none of the types $p_{m_0}$ is isolated. By Q5, and the fact that $\mathcal{L}^+$ is countable, there is a countable model $\mathcal{N}$ of $\mathcal{T}$ omitting all $p_{m_0}$ for $m_0\in \mathcal{M}$ unnatural.
		
		We claim that $\mathcal{N}$ is a proper end extension. It is clearly an (elementary) extension, and it is proper since $c^{\mathcal{N}}$ is greater than all elements of $\mathcal{M}$. If it weren't an end extension then there must be some $n\in\mathcal{N}\setminus\mathcal{M}$ and $m_0\in\mathcal{M}$ such that $n<m_0$. Clearly this cannot happen for $m_0$ unnatural since $\mathcal{N}$ omits $p_{m_0}$. So suppose $m_0$ is natural. But then there are $m_1,\ldots,m_k\in \mathcal{M}$ such that
		\[
		\mathcal{M} \models \forall x. \left(\bigwedge_{i=1}^k x \neq m_i \rightarrow x > m_0\right).
		\] 
		As $\mathcal{N}$ is an elementary extension, we have that $\mathcal{N}$ also models this sentence, and thus $n > m_0$ as well as $n < m_0$, a contradiction. Thus $\mathcal{N}$ is a proper end extension of $\mathcal{M}$.
		\item Let $M \coloneqq \{m_1, m_2, \ldots\}$ and $N \coloneqq \{n_1,n_2,\ldots\}$ be two countable $\omega$-saturated, elementarily equivalent $\mathcal{L}$-structures. We construct a sequence $f_0,f_1,\ldots$ such that for all $i\in \mathbb{N}$:
		\begin{itemize}
			\item $f_i$ is an elementary partial function $M\to N$;
			\item $f_{i+1}$ extends $f_i$;
			\item $\dom(f_i)$ (and hence $\cod(f_i)$) is finite;
			\item $\{m_1,\ldots,m_i\}\subseteq \dom(f_i)$ and $\{n_1,\ldots,n_i\}\subseteq \cod(f_i)$.
		\end{itemize}
		
		Define $f_0$ to be the empty function, which is elementary since $M$ and $N$ are elementarily equivalent. Suppose $f_i$ has been defined, and let $D\coloneqq \dom(f_i)$ and $C\coloneqq \cod(f_i)$ be finite. Consider the complete 1-type
		\[
			\tp^M(m_{i+1}/D).
		\]
		Using the notation of Question 4 (b), we note that $f_i^*(p)$ is a complete 1-type of $N$ by basically the same argument as in Q4 and the fact that $f_i$ is elementary. Since $N$ is $\omega$-saturated, it follows that there is some $n\in N$ realizing this type. Let $g\colon D\cup\{m_{i+1}\} \to C\cup\{n\}$ be the extension of $f_i$ that sends $m_{i+1}\mapsto n$ (if $m_{i+1}\in D$ then just let $g=f_i$). Thus $g$ is elementary by construction.
		
		Similarly, consider the complete 1-type $p\coloneqq \tp^N(n_{i+1}/C\cup\{n\})$. As $g$ is elementary it follows that 
		\[\{\varphi(x,\bar{d}) \colon \varphi(x,g(\bar{d}))\in p\text{ for some $\bar{d}\in D\cup\{m_{i+1}\}$}\}\]
		is a 1-type for $M$ so it has a realization $m\in M$. Finally, we let $f_{i+1} \colon D\cup\{m,m_{i+1}\}\to C\cup\{n,n_{i+1}\}$ be the extension of $g$ mapping $m\mapsto n_{i+1}$. For the same reasons as before, $f_{i+1}$ is elementary. This finishes the construction.
		
		Now let $f\colon M \to N$ be the union of all $f_i$. By construction, $f$ is defined everywhere, elementary, and surjective. It is also injective since $N\models f(m) = f(m')$ will imply $M\models m = m'$. The fact that $f$ is a homomorphism can be similarly verified.
		
		\item We assume $\mathcal{T}$ is, in addition, consistent, since otherwise the result is trivial.
		
		Suppose there are finitely many equivalence classes for formulae (of a given arity). We claim that countable $\mathcal{T}$-models are $\omega$-saturated. Indeed, suppose $M\models\mathcal{T}$ is a model and let $\bar{a}\subseteq M$ be a finite tuple. Let $p\in S_n^M(\bar{a})$; we have to show that $M$ realizes $p$. 
		
		By assumption we can consider $p$ as a finite set of formulae to be satisfied, since $M\models \mathcal{T}$ and $\mathcal{T}$ has finitely many equivalence classes for formulae of a given arity, and the arity of formulae in $p$ is bounded by $n + |\bar{a}|$. By taking the conjunction of all these formulae, we are left with a single formulae $\varphi(\bar{x}, \bar{a})$ and we have to show that $M \models \exists \bar{x}. \varphi(\bar{x}, \bar{a})$. But as $p$ is a type, we know it is realized in an elementary extension $N$ of $M$, and so $N \models\exists \bar{x}. \varphi(\bar{x}, \bar{a})$. As the extension is elementary we get that $M$ realizes $p$. This shows that all models of $\mathcal{T}$ are $\omega$-saturated.
		
		Now, any two countable models of $\mathcal{T}$ are elementarily equivalent (since $\mathcal{T}$ is complete) and $\omega$-saturated, thus isomorphic by Q8. Hence $\mathcal{T}$ is $\aleph_0$-categorical, as desired.
		
		Conversely, suppose there is some $n$ such that are infinitely many $\mathcal{T}$-equivalence classes of formulae with $n$ variables $\bar{x} = (x_1,\ldots,x_n)$. We claim the following.
		\begin{claim}
			All models of $\mathcal{T}$ are infinite.
		\end{claim}
		\begin{proof}
			Suppose $\mathcal{T}$ has a finite model $M$. Then $M$ thinks there are only finitely many equivalence classes of formulae with $n$-variables. Indeed, the equivalence class of a formulae $\varphi(\bar{x})$ is determined by its truth value on its inputs, i.e. we can see the formula as a Boolean function $M^n \to \{0,1\}$, and there are only finitely many of those since $M$ is finite. Hence there are formulae $\varphi_1(\bar{x}),\ldots, \varphi_k(\bar{x})$ such that
			for all formulae $\varphi(\bar{x})$ there is some $i\leq k$ with
			\[
				M \models \forall x. (\varphi(\bar{x}) \leftrightarrow \varphi_i(\bar{x})).
			\]
			As $\mathcal{T}$ is complete we see that the same is true if we replace $M$ by $\mathcal{T}$ in the above, and thus there are only finitely many $\mathcal{T}$-equivalence classes of formulae with $n$ variables, a contradiction.
		\end{proof}
		
		Now, the condition on the equivalence classes can be equivalently stated to say that the Lindenbaum-Tarski algebra $\mathcal{B}_n(\mathcal{T})$ is infinite. Hence the Fréchet filter on $\mathcal{B}_n(\mathcal{T})$ is proper and can be extended to a free ultrafilter on $\mathcal{B}_n(\mathcal{T})$ (this is by ES2 Q5(b)). This ultrafilter corresponds to a complete type $p$, and the fact that is free means that $p$ is not isolated (!).
		
		By the ommitting types theorem (here we use the countability of the language), there is a countable model $N\models \mathcal{T}$ that omits $p$. By definition of types there is a model $M \models \mathcal{T}$ that realizes $p$, say with a tuple $\bar{m}\in M$. By our claim $M$ is infinite. Using the downwards Löwenheim–Skolem theorem (again using $|\mathcal{L}| = \aleph_0$) we can find an elementary substructure $M'$ of $M$ such that $M'$ is countable and $\bar{m}\in M'$. As the substructure is elementary, we have $M'\models \mathcal{T}$ and that $\bar{m}$ realizes $p$ in $M'$. 
		
		But then $M'$ and $N$ are countable models of $\mathcal{T}$, and one of the realizes $p$ while the other ones omits $p$. We conclude that $\mathcal{T}$ is \emph{not} $\aleph_0$-categorical.
		
		Consider a theory $\mathcal{S}$ with quantifier elimination in a language whose signature has finitely many relation symbols (of arity at least 1) and no function symbols. The language clearly is countable. We claim that $\mathcal{S}$ is complete. Indeed, it suffices to decide quantifier-free sentences, but these are only Boolean combinations of $\bot$ by the signature, so this is trivial. 
		
		Again, by quantifier-elimination, each formula is equivalent to a quantifier-free formula, and using the disjunctive normal form and by analysing all the possible literals we can reduce everything to finitely many formulae.
		
		\item Let $\varepsilon$ be an infinite linear order contained in an infinite model of $\mathcal{T}$. It follows from the Ehrenfeucht–Mostowski theorem (by Skolemizing the language first) that there is an Ehrenfeucht–Mostowski functor $F$ such that $\Th(F)$ expands $\Th(M,\eta)$. As a particular case of this, the sentences satisfied in $M$ are also satisfied in $\Th(F)$. More particularly, $F$ takes values in $\mathcal{T}$-models.
		
		Note that there is a homomorphism $G\to F(\eta)$ given by $g\mapsto F(g)$; that this is a homomorphism follows from functoriality of $F$. But if $F(g) = F(h)$ then, as $F$ extends maps, we have that $g=h$, and so this homomorphism identifies $G$ with a subgroup of $\text{Aut}(F(\eta))$.
		
		\item Let $T$ be the set of all closed terms in $\mathcal{L}$, and define an equivalence relation on $T$ by saying that for all $s,t\in T$
		\[
			s\sim t \iff s=t \in \Gamma.
		\]
		We claim that this is an equivalence relation. It is reflexive by the first condition of $=$-closed. Suppose $s=t\in \Gamma$. If we define $\varphi(x)\coloneqq (x=s)$ then, as $\varphi(s) \in \Gamma$ we have $\varphi(t) = (t=s)\in \Gamma$; hence the relation is symmetric. Finally, suppose $s=t$ and $t = u$ are in $\Gamma$. Let $\psi(x)\coloneqq x = u$. Then as $s = t$ and $\psi(t)\in \Gamma$ it follows that $\psi(s)\in \Gamma$, and so $s=u$ is in $\Gamma$; this shows that the relation is transitive.
		
		Define $M\coloneqq T/{\sim}$, and if $t\in T$ we denote by $[t]$ the corresponding equivalence class. We make $M$ into an $\mathcal{L}$-structure as follows. For a constant $c$ we define $c^M \coloneqq [c]$. For a function symbol $f$ of arity $n$ we inductively define
		\[	
			f([t_1],\ldots,[t_n]) = [f(t_1,\ldots,t_n)].
		\]
		We need to check that this assignment is well-defined. Suppose $t_1,\ldots,t_n,t_1',\ldots,t_n'$ are such that $t_i\sim t_i'$ for all $i$. It can be proved by induction on $n$ that 
		\[
			f(t_1,\ldots,t_n) \sim f(t_1',\ldots,t_n').
		\]	
		For relation symbols $R$ we define
		\[
			([t_1],\ldots,[t_n])\in R^M  \iff R(t_1,\ldots,t_n)\in T.
		\] 
		It can be checked that this is well-defined and thus $M$ becomes an $\mathcal{L}$-structure. The rest can be easily shown by induction over the structure of formulae.
		
		\item \leavevmode
		\begin{enumerate}
			\item Clearly the formula $x=x$ is contained in $\Th(M,\omega)$ so $S(\eta)$ contains $t=t$ for every closed $\mathcal{L}_{\eta}$-term. Suppose $S(\eta)$ contains $\phi(s(\bar{c}), \bar{c})$ and $s(\bar{c}) = t(\bar{c})$, where $s,t$ are $\mathcal{L}$-terms, $\phi$ is an atomic $\mathcal{L}$-formula, and $\bar{c}\in [\eta]^k$. It follows that if $\bar{d} \in [\omega]^k$ then 
			\[
				M \models \phi(s(\bar{d}),\bar{d}) \wedge s(\bar{d}) = t(\bar{d}),
			\]
			so $M\models \phi(t(\bar{d}),\bar{d})$ and hence $\phi(t(\bar{x}), \bar{x})$ is in $\Th(M,\omega)$, from which the claim follows.
			\item Use Q11 on $S(\eta)$ to obtain an $\mathcal{L}$-structure $F(\eta)$ such that the set of atomic $\mathcal{L}_\eta$sentences true in $F(\eta)$ is exactly $S(\eta)$, and whose every element is the interpretation of some closed term in $\mathcal{L}_{\eta}$.
			
			It follows that $F(\eta)$ contains a copy of $\eta$ by its interpretation of constants, and this is a faithful copy since $x =y$ is obviously not in $\Th(M,\omega)$. Furthermore for every closed $\mathcal{L}$-term $t$ there is an $\mathcal{L}$-term $s$ and some $\bar{d}\in [\omega]^k$ such that
			\[
				M \models t = s(\bar{d})
			\]
			since $M$ is generated by $\omega$; it follows readily that $F(\eta)$ must be generated by $\eta$.
			
			If $g\colon \eta \to \varepsilon$ is an order embedding then for every atomic formula $\phi$ (and hence for every quantifier free formula by induction) if $\bar{c}\in [\eta]^k$ then $F(\eta) \models \phi(\bar{c})$ implies $\phi(\bar{x})\in \Th(M,\omega)$ which in turn implies that $F(\varepsilon) \models \phi(g(\bar{c}))$ where we crucially use the fact that $g$ respects ordering. Using the method of diagrams we get some embedding $F(\eta) \to F(\varepsilon)$ which extends (and indeed it is determined by) $g$. This makes sure that the assignment is functorial so this is indeed an EM functor. Furthermore, $F(\omega)$ is generated by $\omega$,
			 from which it follows that $F(\omega) = M$.
			 \item By the Lemma right after the definition of $\Th(F)$, we have that all quantifier-free sentences in $\Th(M,\omega)$ are contained in both $\Th(F)$ and $\Th(G)$. But then for every quantifier-free formula $\varphi(\bar{x})$ and $\bar{c}\in[\eta]^k$ we get that
			 \[
				 F(\eta)\models \varphi(\bar{a}) \iff G(\eta) \models \varphi(\bar{a}).
			 \]
			 As $\eta$ is a generating set, we get again by the method of diagrams an isomorphism $F(\eta) \to G(\eta)$ fixing $\eta$.
		\end{enumerate}
		\end{enumerate}
\end{document}
