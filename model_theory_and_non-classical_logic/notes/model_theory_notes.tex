\documentclass{report}
%Setting margins
%\usepackage[margin = 1.75in]{geometry}
%Basic Maths
\usepackage{amsmath}
\usepackage{amssymb}
\usepackage{mathtools}
\usepackage{gensymb}
%For definining theorem-like environments
\usepackage{amsthm}
%For beautiful letters (e.g. for a partition, see $\mathscr{P}$)
\usepackage{mathrsfs}
%For importing the solution file
%\usepackage{import}
%For drawing commutative diagrams
\usepackage{quiver}
%For pretty colours
\usepackage{xcolor}
%For scaling some relations, for instance see https://tex.stackexchange.com/a/108482
\usepackage{mleftright}
%Set paragraph spacing. I believe this is close to what is used in the book.
%\usepackage[skip=.3\baselineskip, indent = 15pt]{parskip}
%To customize lists
\usepackage{enumitem}
%To strikethrough terms in equations
\usepackage{cancel}
%For bibliography
\usepackage[backend=biber]{biblatex}
%For pictures
\usepackage{tikz}
\usetikzlibrary{calc,positioning}

\usepackage{hyperref}
\usepackage{soul}

\usepackage{lipsum}

%\addbibresource{main.bib}


\newcommand{\myhy}[2]{\href{#1}{\color{blue}\setulcolor{blue}\ul{#2}}}

%Fix section numbering to match the book's convention
\renewcommand\thesection{\arabic{section}}

%Displays "Exercises". To put after each section.
\newcommand{\extitle}{\subsection*{Exercises}}

%For personal notes
\newcommand{\note}[1]
{\smallskip {\noindent\textbf{Note} #1}}

%Roman numerals!
\newcommand{\RNo}[1]{%
	\textup{\uppercase\expandafter{\romannumeral#1}}%
}

%San-serif for names of categories
\newcommand{\serif}[1]{{\fontfamily{cmss}\selectfont #1}}
\newcommand{\srf}{\textsf}

%Shorthands for common sets
\newcommand{\N}{\mathbb{N}}
\newcommand{\Z}{\mathbb{Z}}
\newcommand{\Q}{\mathbb{Q}}
\newcommand{\R}{\mathbb{R}}
\newcommand{\C}{\mathbb{C}}
\newcommand{\zmod}[1]{\bZ/#1\bZ}

%Miscellaneous commands
\newcommand{\defeq}{\coloneqq}
\newcommand{\divides}{\mid}
\newcommand{\legendre}[2]{\ensuremath{\left( \frac{#1}{#2} \right) }}
\newcommand{\Mod}[1]{\ (\mathrm{mod}\ #1)}


%Useful operations and delimiters
\DeclareMathOperator{\Hom}{Hom}
\DeclareMathOperator{\End}{End}
\DeclareMathOperator{\Aut}{Aut}
\DeclareMathOperator{\Obj}{Obj}
\DeclareMathOperator{\id}{id}
\DeclareMathOperator{\lcm}{lcm}
\DeclareMathOperator{\GL}{GL}
\DeclareMathOperator{\SO}{SO}
\DeclareMathOperator{\SL}{SL}
\DeclareMathOperator{\U}{U}
\DeclareMathOperator{\SU}{SU}
\DeclareMathOperator{\Inn}{Inn}
\DeclareMathOperator{\PSL}{PSL}
\DeclareMathOperator{\im}{im}
\DeclareMathOperator{\coker}{coker}
\DeclareMathOperator{\rot}{rot}
\DeclareMathOperator{\rf}{ref}
\DeclareMathOperator{\Symm}{Symm}
\DeclareMathOperator{\vspan}{span}
\DeclareMathOperator{\ev}{ev}
\DeclareMathOperator{\Gal}{Gal}
\DeclareMathOperator{\Diag}{Diag}
\DeclarePairedDelimiter\abs{\lvert}{\rvert}%
\DeclarePairedDelimiter\norm{\lVert}{\rVert}%
\DeclarePairedDelimiter\innprod{\langle}{\rangle}%
\DeclarePairedDelimiter\ceil{\lceil}{\rceil}
\DeclarePairedDelimiter\floor{\lfloor}{\rfloor}
%Claim environment
\newtheorem{claim}{Claim}


%Exercise environment
\theoremstyle{definition}
\newtheorem{ex}{Exercise}

%Standard theorem-like environment
\theoremstyle{plain}
\newtheorem{thm}{Theorem}[section]

\newtheorem{prop}[thm]{Proposition}
\newtheorem{lem}[thm]{Lemma}
\newtheorem{coro}[thm]{Corollary}
\newtheorem{prob}{Problem}
\newtheorem{conj}{Conjecture}


\theoremstyle{definition}
\newtheorem{defn}[thm]{Definition}
\newtheorem{rem}[thm]{Remark}
\newtheorem*{rem*}{Remark}
\newtheorem{eg}[thm]{Example}
\newtheorem{egs}[thm]{Examples}
\newtheorem{fact}[thm]{Fact}
\newtheorem{task}{Task}



%Solution environment
\newenvironment{solution}
{\begin{proof}[Solution]}
	{\end{proof}}

%Function restrictions
% From https://tex.stackexchange.com/a/22255
\newcommand\restr[2]{{% we make the whole thing an ordinary symbol
		\left.\kern-\nulldelimiterspace % automatically resize the bar with \right
		#1 % the function
		\vphantom{\big|} % pretend it's a little taller at normal size
		\right|_{#2} % this is the delimiter
}}


\makeatother

\setenumerate[1]{label=(\alph*)}	
\renewcommand{\thesection}{\thechapter.\arabic{section}}
\begin{document}
	\title{Model Theory and Non-Classical Logic}
	\author{Hernán Ibarra Mejia}
	\maketitle
	This is a set of lecture notes taken by me from the Part III course ``Model Theory and Non-Classical Logic'', lectured by Dr J. Siqueira in Michaelmas, 2023. I take full responsibility for any mistakes in these notes. Chapter 0 is my summary/expansion of \cite{NOLAST}
	
	
	\setcounter{chapter}{-1}
	\chapter{Logic Background (INCOMPLETE)}
	\begin{defn}[Signature]
		A \emph{signature} $\Sigma$ is a triplet $(\Omega, \Pi, \alpha)$, where $\Omega$ and $\Pi$ are disjoint sets and $\alpha\colon \Omega \cup \Pi \to \mathbb{N}$. We call the elements of $\Omega$ \emph{function symbols}, those of $\Pi$ we call \emph{predicate symbols}, and if $s\in \Omega \cup \Pi$ we call $\alpha(s)$ the \emph{arity} of $s$.
	\end{defn}
	For the rest of this chapter, assume $\Sigma = (\Omega, \Pi, \alpha)$ is an arbitrary signature and that we are given a countable set $X=\{x_1,x_2\ldots\}$, which we call the set of \emph{variables}. This set does not contain any symbols in our signature (nor in the set of strings on our signature, see below).
	\section{Terms, formulae, and structures}
	\begin{defn}[Terms]
		The set of $\Sigma$-\emph{terms} is a subset of the set of strings on $\Omega \cup X$, defined inductively as follows. 
		\begin{enumerate}
			\item If $x\in X$ then $x$ is a term
			\item If $t_1,\ldots t_n$ are terms, and $\omega\in \Omega$ with $\alpha(\omega) = n$ then $\omega (t_1,\ldots, t_n)$ is a term.
			\item That is all.
		\end{enumerate}
	\end{defn}
	\begin{rem}
		Now assume that $X\cup \Omega \cup \Pi$ do not contain the symbols `$=$', `(',`)', `$\bot$', `$\forall$' nor `$\Rightarrow$' (nor commas).
	\end{rem} 
	\begin{defn}[Atomic formulae]
		Let $T$ be the set of $\Sigma$-terms. We define the \emph{atomic formulae} of $\Sigma$ as certain strings on $T \cup \Pi \cup \{(, ), =, ,\}$ (note that the last comma is not a typo) according to the following rules.
		\begin{enumerate}
			\item If $s$ and $t$ are terms then $(s = t)$ is an atomic formula.
			\item If $\phi \in \Pi$, $\alpha(\phi) = n$ and $t_1, \ldots, t_n$ are terms then $\phi(t_1,\ldots, t_n)$ is an atomic formula.
			\item That is all.
		\end{enumerate}
	\end{defn}
	\begin{defn}[Pre-formulae]
		Let $T$ be the set of terms of $\Sigma$. We inductively define the set of $\Sigma$-\emph{pre-formulae} as a subset of the set of strings on $T \cup \Pi\cup \{=,\bot, \forall, \Rightarrow, (, )\}$ satisfying the following.
		\begin{enumerate}
			\item Atomic formulae are pre-formulae
			\item $\bot$ is a pre-formula.
			\item If $p$ and $q$ are pre-formulae then so is $(p\Rightarrow q)$.
			\item If $p$ is a pre-formula and $x\in X$ is a variable then $(\forall x)p$ is a pre-formula.
			\item That's all.
		\end{enumerate}
	\end{defn}
	
	Now we can define a function $\textnormal{PFV}$ (for pre-free variables) on the set of terms union with the set of pre-formulae by the following rules
	\begin{align*}
		&\textnormal{PFV}(x) = \{x\}\\
		&\textnormal{PFV}(\omega t_1 \cdots t_n) = \bigcup_{i = 1}^{n}\textnormal{PFV}(t_i)\\
		&\textnormal{PFV}(s=t) = \textnormal{PFV}(s) \cup \textnormal{PFV}(t)\\
		&\textnormal{PFV}(\phi(t_1, \ldots, t_n)) = \bigcup_{i = 1}^{n}\textnormal{PFV}(t_i)\\
		&\textnormal{PFV}(\bot) = \emptyset\\
		&\textnormal{PFV}(p \Rightarrow q) = \textnormal{PFV}(p) \cup \textnormal{PFV}(q)\\
		&\textnormal{PFV}((\forall x)p) = \textnormal{PFV}(p)\setminus\{x\} 
	\end{align*}
	
	Finally, we can define $\Sigma$-\emph{formulae} to be all pre-formulae of $\Sigma$ except those of the form $(\forall x)p$ where $x\notin \textnormal{PFV}(p)$. Define $\textnormal{FV}$ to be the restriction of $\textnormal{PFV}$ so that it only applies to terms and formulae.
	
	By the \emph{language} $\mathcal{L}$ of a signature $\Sigma$ we mean the set of all terms and formulae of $\Sigma$. Instead of saying $\Sigma$-terms and $\Sigma$-formulae we say $\mathcal{L}$-terms and  $\mathcal{L}$-formulae to mean the same thing.
	\begin{defn}[Language structures]
		An $\mathcal{L}$-\emph{structure} is a set $A$ together with functions $\omega_A\colon A^{\alpha(w)} \to A$ for each $\omega \in \Omega$ and relations $\phi_A\subseteq A^{\alpha(\phi)}$ for each $\phi \in \Pi$. We use the convention that $S^0$ is a singleton set (say $\{0\}$) for all sets $S$.
	\end{defn}
	\section{Derived symbols}
	\begin{defn}[Derived operations]
		Let $A$ be an $\mathcal{L}$-structure and $t$ a term. In addition, suppose $n$ is an integer with $\textnormal{FV}(t) \subseteq \{x_1,\ldots,x_n\}$. We define, $t_A(n)$ to be a function $A^{n} \to A$ as follows.
		\begin{enumerate}
			\item If $t\in X$ then $t = x_i$ for some $i\leq n$. Let $t_A\colon A^n \to A$ be the $i$-th projection function.
			\item Suppose $t = \omega t_1\ldots t_m$ where $\omega \in \Omega$ with $\alpha(\omega) = m$, and the $t_i$'s are terms for which we have defined $(t_i)_A(n)$. Then $t_A$ is the composite 
			% https://q.uiver.app/#q=WzAsMyxbMywwLCJBXm0iXSxbNCwwLCJBIl0sWzAsMCwiQV5uIl0sWzAsMSwiXFxvbWVnYV9BIl0sWzIsMCwiKCh0XzEpX0EobiksKHRfMilfQShuKSxcXGxkb3RzLCAodF9tKV9BKG4pKSJdXQ==
			\[\begin{tikzcd}[column sep=huge]
				{A^n} &&& {A^m} & A
				\arrow["{\omega_A}", from=1-4, to=1-5]
				\arrow["{((t_1)_A(n),(t_2)_A(n),\ldots, (t_m)_A(n))}", from=1-1, to=1-4]
			\end{tikzcd}\]		
		\end{enumerate}
	\end{defn}
	Note that in the empty structure all derived operations are the empty function (there can't be any constant symbols).
	\begin{lem}[Variable redundancy in terms]\label{lem:redun_terms}
		Let $A$ be a structure and $t$ a term with $\textnormal{FV}(t) \subseteq \{x_1,\ldots,x_n\}$ for some $n$. Suppose we have two sequences of elements of $A^n$, say $a$ and $b$, such that $a$ and $b$ agree on free variables, i.e. $a_k = b_k$ whenever $x_k\in \textnormal{FV}(t)$. Then $t_A(n)(a) = t_A(n)(b)$.
	\end{lem}
	\begin{proof}
		Induction on $t$. Suppose $t = x_i\in X$. Then $t_A(n)(a)$ and $t_A(n)(b)$ are $a_i$ and $b_i$ respectively. We have assumed these are the same, as $x_i\in \textnormal{FV}(t)$. Thus $t_A(n)(a)= t_A(n)(b)$.
		
		Now suppose $t = \omega t_1 \cdots t_m$ where $\omega \in \Omega$ with $\alpha(\omega) = m$, and the $t_i$'s are terms. By inductive hypothesis, we may assume that for all $1\leq i \leq n$ we have
		\[
		(t_i)_A(n) (a) =  (t_i)_A(n) (b). 
		\]
		It follows that 
		\[
		((t_1)_A(n),\ldots, (t_m)_A(m))(a) = ((t_1)_A(n),\ldots, (t_m)_A(n))(b).
		\]
		Applying $\omega_A$ to both sides gives the result.
	\end{proof}
	\begin{defn}(Derived formulae)
		Let $A$ be an $\mathcal{L}$-structure and $p$ a formula. In addition, suppose $n$ is an integer with $\textnormal{FV}(p) \subseteq \{x_1,\ldots,x_n\}$. We define $p_A(n)$ to be subset of $A^{n}$, or equivalently a function $A^{n} \to 2$, as follows.
		\begin{enumerate}
			\item If $p$ is the formula $(s=t)$ for terms $s$ and $t$ then
			%
			\[
				p_A(n) \coloneqq \{\, a \in A^{n} \mid s_A(n)(a) = t_A(n)(a)\,\}
			\]
			\item Suppose $p = \phi(t_1,\ldots, t_m)$ for some $\phi \in \Pi$, with  $\alpha(\phi) = m$, and terms $t_1,\ldots, t_m$. Then (the characteristic function of) $p_A(n)$ is defined by
			% https://q.uiver.app/#q=WzAsMyxbMywwLCJBXm0iXSxbNCwwLCIyIl0sWzAsMCwiQV5uIl0sWzAsMSwiXFxwaGlfQSJdLFsyLDAsIigodF8xKV9BKG4pLCh0XzIpX0EobiksXFxsZG90cywgKHRfbSlfQShuKSkiXV0=
			\[\begin{tikzcd}[column sep=huge]
				{A^n} &&& {A^m} & 2
				\arrow["{\phi_A}", from=1-4, to=1-5]
				\arrow["{((t_1)_A(n),(t_2)_A(n),\ldots, (t_m)_A(n))}", from=1-1, to=1-4]
			\end{tikzcd}\]
			\item If $p = \bot$ then $p_A(n)$ is the empty set (i.e. its characteristic function is constant with value zero).
			\item Suppose $p$ is $(q \Rightarrow r)$ for formulas $q$ and $r$, where $q_A(n)$ and $r_A(n)$ have already been defined. Then we define $p_A$ by the composition
			% https://q.uiver.app/#q=WzAsMyxbMiwwLCIyXFx0aW1lcyAyIl0sWzMsMCwiMiJdLFswLDAsIkFebiJdLFswLDEsIlxcUmlnaHRhcnJvd18yIl0sWzIsMCwiKHFfQShuKSxyX0EobikpIl1d
			\[\begin{tikzcd}[column sep=huge]
				{A^n} && {2\times 2} & 2
				\arrow["{\Rightarrow_2}", from=1-3, to=1-4]
				\arrow["{(q_A(n),r_A(n))}", from=1-1, to=1-3]
			\end{tikzcd}\]
			\item Suppose $p = (\forall x_m)q$ for some formula $q$ with $x_m \in \textnormal{FV}(q)$. Define $N\coloneqq \max(n,m)$. We can assume that $q_A(N)$ is defined. Let $a\in A^{n}$. We say that $a\in p_A(n)$ if and only if for all $a'\in A^N$ so that $a'$ agrees with $a$ in the first $n$ terms---except possibly on the $m$-th term---we have that $a'\in q_{A}(N)$. 
		\end{enumerate}
	\end{defn}
	\begin{lem}[Variable redundancy in formulae]\label{lem:redun_form}
		Let $A$ be a structure and $p$ a formula with $\textnormal{FV}(p) \subseteq \{x_1,\ldots,x_n\}$ for some $n$. Suppose we have two elements of $A^n$, say $a$ and $b$ such that $a$ and $b$ agree on free variables, i.e. $a_k = b_k$ whenever $x_k\in \textnormal{FV}(t)$. Then $a\in p_A(n)$ if and only if $b\in p_A(n)$.
	\end{lem}
	\begin{proof}
		By induction on $p$. If $p$ is the formula $(s=t)$ for terms $s$ and $t$ we have
		\begin{align*}
			a\in p_A(n) &\iff s_A(n)(a) = t_A(n)(a)\\
			&\iff s_A(n)(b) = t_A(n)(b)\\
			&\iff b\in p_A(n),
		\end{align*}
		where we have used Lemma \ref{lem:redun_terms}.
		
		Now suppose $p = \phi (t_1,\ldots,t_m)$ for some $\phi \in \Pi$ with $\alpha(\phi) = m$, and terms $t_1,\ldots, t_m$. Again by Lemma \ref{lem:redun_terms} we have
		\[
		((t_1)_A(n),\ldots, (t_m)_A(m))(a) = ((t_1)_A(n),\ldots, (t_m)_A(n))(b).
		\]
		Applying $\phi_A$ to both sides gives the result. This finishes the induction in the case that $p$ is an atomic formula.
		
		If $p = \bot$ then $a,b\notin p_A(n)$ so the claim holds. Now let $q,r$ be formulae with $p$ being $(q\Rightarrow r)$. By inductive hypothesis
		\[
		(q_A(n),r_A(n))(a) = (q_A(n),r_A(n))(b),
		\]
		and applying $\Rightarrow_2$ to both sides gives the result.
		
		Finally, we consider the case where $p = (\forall x_m)q$ for some formula $q$ and some variable $x_m\in \textnormal{FV}(q)$. Write $a = (a_1,\ldots,a_n)$ and $b= (b_1,\ldots,b_n)$. By symmetry, we only need to show that $a\in p_A(n)$ implies $b\in p_A(n)$. So, suppose $a \in p_A$. 
		
		Let $N= \max(n,m)$ and let $b'\in A^N$ be a sequence agreeing with $b$ in the first value except possibly on the $m$-th value, and call this value $c_m$. If we show that $b' \in q_A(N)$ then we are done.
		
		Define $a'\in A^N$ to be the sequence $b'$ but with its first $n$ values replaced by $a$ \emph{except} the $m$-th value, which remains $c_m$. As $a\in p_A(n)$ it is clear by definition that $a'\in q_A(N)$. But $a'$ and $b'$ agree on free variables of $q$, which are $\textnormal{FV}(p)\cup\{x_m\}$: the first $n$ variables are just $a$ and $b$, which agree on $\textnormal{FV}(p)$, only that we have specified that they agree on the $m$-th value $c_m$, and otherwise $a'$ and $b'$ are identical. Hence, by the inductive hypothesis, $b'\in q_A(N)$.
	\end{proof}
	Variable redundancy implies that free variables are the only thing that affects the values of $t_A(n)$ and $p_A(n)$. Hence we will write $t_A$ and $p_A$, without specifying $n$, we take $n$ to be the minimum so that $\textnormal{FV}(t)$ (resp. $\textnormal{FV}(p)$) is a subset of $\{x_1,\ldots,x_n\}$. And in any case, this only function only requires inputs $a_k$ where $x_k$ is a free variable of $t$ (resp. $p$).
	\section{First-order theories}
	\begin{defn}[Satisfying a formula]
		Let $A$ be an $\mathcal{L}$-structure, and let $p$ be a formula. We say that $p$ is \emph{satisfied} in $A$ if the indicator function of $p_A$ is constant with value 1. In this case we write $A\models p$. 
	\end{defn}
	\begin{defn}[Sentences and universal closure]
		Let $p$ be a formula. We say that $p$ is a \emph{sentence} if it has no free variables. (In this case $p_A$ is a constant function since $n=0$). In any case, we can obtain a sentence $\bar{p}$, called the \emph{universal closure} of $p$, by prefixing $p$ with universal quantifiers for each of the free variables of $p$ (say, in decreasing order of subscripts).
	\end{defn}
	We remark that if $A$ is empty and $p$ is not a sentence then the indicator $p_\emptyset \colon \emptyset^{n} \to 2$ is constant with value 1 since it sends all elements of $\emptyset^{\mathbb{N}}$ (which there are none) to 1 (this is a vacuous truth).
	\begin{prop}
		For all $\mathcal{L}$-structures $A$ and formulas $p$,
		\[
		A \models p \,\,\,\textnormal{ if and only if }\,\,\, A\models \bar{p}.
		\]
	\end{prop}
	\begin{proof}
		We prove a weaker statement first. Let $q$ be a formula with a free variable $x_m$. Then we claim that $A \models q$ if and only if $A\models (\forall x_m) q$. Indeed, $A\models \forall x_n q$ if and only if for all $a\in A^{n}$ (where $n$ is the minimum so that $\textnormal{FV}((\forall x_m) q)\subseteq \{x_1,\ldots,x_n\}$) we have $a\in ((\forall x_m) q)_A(n)$. And this happens iff for all $a\in A^n$ and for all $a'\in A^{N}$ (where $N\coloneqq \max(n,m)$) that agrees with $a$ on the first $n$ values except possibly in the $m$-th value we have $a'\in q_A(N)$.
		
		This is clearly equivalent to saying that for all $a' \in A^N$ we have $a \in q_A(N)$, i.e. $A \models q_A$.
		
		Now if $p$ is a formula with $k$ free variables we can use induction on $k$, together with the above result, to deduce the claim about $\bar{p}$. 
	\end{proof}
	\begin{defn}[First-order theory]
		Let $T$ be a set of $\mathcal{L}$-formulae and $A$ a structure. We write $A\models T$ if $A\models p$ for all $p\in T$.
		
		In the special case where $T$ is a set of sentences we call it a \emph{first-order theory}, and its formulae are called \emph{axioms}. If $A\models T$ in this case we would say that $A$ \emph{models} the theory $T$.
	\end{defn}
	\section{Semantics and syntax}
	Recall that we had an arbitrary signature $\Sigma$ that generated a language $\mathcal{L}$. We would like to add things to the signature (which will generate a difference language) from time to time. Here I will give some notation for a typical situation. Let $S$ be a set. we denote by $\Sigma_S$ the signature $\Sigma$ but with $|S|$ new constant symbols (i.e. function symbols of arity zero) added. Similarly, we denote by $\mathcal{L}_S$ the language generated by $\Sigma_S$. Note that in the special case that $S$ is an $\mathcal{L}$-structure we clearly have that $S$ is an $\mathcal{L}_S$-structure: just interpret the new constant symbols as the elements of $S$. 
	\begin{defn}[Semantic entailment]
		 Let $T$ be a theory and let $p$ be a sentence. We say that $T$ \emph{semantically entails} $p$, written as $T\models p$, to mean that every model of $T$ satisfies $p$.
		 
		 In the case where $T$ and $p$ are not sentences simply consider the language 
		 \[\mathcal{L}'\coloneqq \mathcal{L}_{\textnormal{FV}(T)\cup\textnormal{FV}(p)},\]
		 where $\textnormal{FV}(T)$ is just the set of all free variables appearing in a formulae of $T$. Let $T'$ and $p'$ be the same formulae but with free variables replaced by the corresponding constants in $\mathcal{L}'$. Then $T'\cup\{p'\}$ is just a set of sentences in $\mathcal{L}'$, so declare that $T\models p$ in $\mathcal{L}$ if and only if $T'\models p'$ in $\mathcal{L}'$.
	\end{defn}
	This seems roundabout: why not define semantic entailment $T\models p$ as ``for all $A\models T$ we have $A\models p$''? This certainly makes sense when $T\cup\{p\}$ is not a set of sentences. The problem is that if we adopted this alternate definition we would have undesired consequences with the empty structure. For example, if $T=\{\neg(x_1 = x_1)\}$ and $p = \{\bot\}$ then the only model for $T$ is the empty structure but $\emptyset \nvDash p$. However, since $T$ is clearly a contradictory statement we would like to have $T\models p$ in this case, which is guaranteed by the real definition since the addition of constants invalidate the empty structure.
	
	Now we turn to our system of deduction. If $w$ is a formula, $t$ is a term and $x$ a variable, we define $w[t/x]$ to be the formula obtained from $w$ on replacing each free occurrence of $x$ by $t$, \emph{provided} no free variable of $t$ occurs bound in $w$. More formally, we define
	\begin{align*}
		&y[t/x] = \begin{cases}
			y & \textnormal{ if }x \neq y\\
			t & \textnormal{ if }x = y
		\end{cases}\\[6pt]
		&(\omega t_1 \ldots t_n)[t/x] = \omega (t_1[t/x]) \cdots (t_n[t/x])\\[6pt]
		&(s=s')[t/x] = (s[t/x] = s'[t/x])\\[6pt]
		&\phi(t_1, \ldots ,t_n)[t/x] = \phi((t_1[t/x]), \ldots ,(t_n[t/x]))\\[6pt]
		&\bot[t/x] = \bot\\[6pt]
		&(p \Rightarrow q)[t/x] = (p[t/x] \Rightarrow q[t/x])\\[6pt]
		&((\forall y)p)[t/x]= \begin{cases}
			(\forall y) (p[t/x]) & \textnormal{ if } x \neq y\\
			(\forall y)p & \textnormal{ if }x =y.
		\end{cases} 
	\end{align*}
	\begin{lem}\label{lem:subs_op}
		Let $w$ be a term or a formula, let $x_m$ be a variable and let $t$ be a term such that all free variables in $t$ do not appear bound in $w$. Suppose $A$ is an $\mathcal{L}$-structure and $a\in A^{n}$ where $n$ is the minimum nonnegative integer so that $\textnormal{FV}(w)\subseteq \{x_1,\ldots,x_n\}$. Denote by $a'$ the sequence $a$ but with the $m$-th value replaced by $t_A(a)$. Then we have
		\[
		(w[t/x_n])_A(a) = w_A(a'). 
		\]
		If $A = \emptyset$ then $(w[t/x_n])_A = w_A$
	\end{lem}
	\begin{proof}
		Suppose $A = \emptyset$. If $w$ does not have $x_n$ as a free variable then $w[t/x_n] = w$. It follows that $(w[t/x_n])_A = w_A$.  
		
		Now assume that $x_n$ is a free variable of $w$. Clearly $w$ is not a sentence. We claim that neither is $w[t/x_n]$. This is easily seen from the fact that $t$ is not a constant (since otherwise the empty set could not be an $\mathcal{L}$-structure) and thus has free variables and \emph{in addition} we assumed that no free variables of $t$ are being bound in $w$. Thus, as neither $w$ nor $w[t/x_n]$ are sentences, they are indicators $\emptyset^{\mathbb{N}} \to 2$ and thus equal. This proves the claim for the empty structure, so from now on assume $A\neq \emptyset$.
		
		First suppose that $w$ is a term. We use induction, so assume $w = x_m$ for some $m$. If $m \neq n$ then $w[t/x_n] = w$ and so we only need to show that $w_A(a) = w_A(a')$. This is immediate by variable redundancy: $a$ and $a'$ agree on the free variable $x_m$. Now suppose $m = n$. Then $w[t/x_n] = t$ and we need to show that $t_A(a) = (x_n)_A(a')$. Again, this is obvious: the right-hand side of the equation is the $n$-th value of $a'$, which we assumed is $t_A(a)$. This closes the base case.
		
		Now, suppose $w = \omega t_1 t_2\cdots t_m$ for some $\omega \in \Omega$ with $\alpha(\omega) = m$ and where the $t_i$'s are terms. Clearly 
		\[
		w[t/x_n] = \omega (t_1[t/x_n]) (t_2[t/x_n])\cdots (t_m[t/x_n]).
		\]
		It follows that
		\begin{align*}
			(w[t/x_n])_A(a) &= \omega_A ((t_1[t/x_n])_A(a),(t_2[t/x_n])_A(a),\ldots,(t_m[t/x_n])_A(a))\\
			&= \omega_A((t_1)_A(a'), (t_2)_A(a'),\ldots, (t_m)_A(a'))\\
			&= w_A(a'),
		\end{align*}
		where we have used the inductive hypothesis. This closes the induction and proves the statement when $w$ is a term.
		
		Suppose now that $w$ is a formula. We again use induction. If $w$ is the formula $(s = s')$ for terms $s$ and $s'$ we have that 
		\[
		w[t/x_n] = (s[t/x_n] = s'[t/x_n]).
		\]
		Then,
		\begin{align*}
			a \in (s[t/x_n] = s'[t/x_n])_A &\iff (s[t/x_n])_A(a) = (s'[t/x_n])_A(a)\\
			&\iff s_A(a') = t_A(a')\\
			&\iff a'\in w_A,
		\end{align*}
		where we used the result for terms. Now suppose $w$ is $\phi (t_1,\ldots,t_m)$ for some $\phi \in \Pi$ with $\alpha(\phi) = m$, and terms $t_1,\ldots, t_m$. Again, using the claim for terms, we have 
		\[
		((t_1[t/x_n])_A(a),\ldots,(t_m[t/x_n])_A(a)) = ((t_1)_A(a'),\ldots,(t_m)_A(a'))
		\]
		and applying $\phi_A$ to both sides gives the result. This closes the base case, i.e. the case where $w$ is an atomic formula.
		
		Clearly $(\bot[t/x_n])_A(a) = \bot_A (a) = 0 = \bot_A(a')$. Now, if $w$ is $(p\Rightarrow q)$ then, by the inductive hypothesis
		\[
		((p[t/x_n])_A(a), (q[t/x_n])_A(a)) = (p_A(a'),q_A(a')).
		\]
		Applying $\Rightarrow_2$ to both sides gives the result.
		
		Finally, suppose $w = (\forall x_m) p$. Then we have two cases. If $m = n$ then $w[t/x_n] = w$ and so we need to show that $w_A(a) = w_A(a')$. But in this case clearly $x_n$ is not a free variable of $w$, so $a$ and $a'$ agree on free variables, and the claim follows by variable redundancy.
		
		Now assume $m\neq n$. Then $w[t/x_n] = (\forall x_m) (p[t/x_n])$. First, note that that 
		
		\[a = (a_1,a_2,\ldots)\in ((\forall x_m) (p[t/x_n]))_A\]
		if and only if
		\[
		(a_1,\ldots, a_{m-1}, c,a_{m+1},\ldots)\in (p[t/x_n])_A \textnormal{ for all }c\in A. 
		\]
		For $c\in A$ let $\alpha(c)$ be the sequence above, i.e. $a$ but the $m$-th value replaced by $c$. Similarly, let $\alpha'(c)$ be the sequence $a'$ but replacing the $m$-th value with $c$. Finally, let $\alpha^*(c)$ be the sequence $\alpha(c)$ but with the $n$-th value replaced by $t_A(\alpha(c))$ Then we can reformulate our statement as so:
		\[
		\alpha(c) \in (p[t/x_n])_A \textnormal{ for all }c\in A.
		\]
		By the inductive hypothesis, this new statement is equivalent to
		\[
		\alpha^*(c) \in p_A \textnormal{ for all }c\in A.
		\]
		Note that, for all $c$, we have that $\alpha^*(c)$ and $\alpha'(c)$ agree on all values (including the $m$-th) except possibly on the $n$-th value, where we have $t_A(\alpha(c))$ and $t_A(a)$ fro $\alpha^*(c)$ and $\alpha'(c)$ respectively. We claim that in fact \emph{they do} agree on the $n$-th value, i.e. $t_A(\alpha(c)) = t_A(a)$.
		
		Indeed, by definition, $a$ and $\alpha(c)$ agree on all values except possibly on the $m$-th value. However, we assumed (and this is the first and only time we use the assumption when $A\neq \emptyset$) that the free variables of $t$ do not appear bound in $w$. Clearly $x_m$ is bound in $w$ (recall that we insist that variables that are being bound appear in the formula). Thus $x_m$ cannot be a free variable of $t$, which implies that $a$ and $\alpha(c)$ agree on free variables; hence $t_A(\alpha(c)) = t_A(a)$ by variable redundancy. Thus $\alpha^*(c)$ is the same sequence as $\alpha'(c)$. Therefore we can, once again, reformulate our statement:
		\[
		\alpha'(c) \in p_A \textnormal{ for all }c\in A.
		\]
		This is manifestly equivalent to $a'\in((\forall x_m)p)_A$, as desired.
	\end{proof}
	
	We now postulate our axioms to be substitution instances of these propositions.
	\begin{enumerate}
		\item $(p\Rightarrow (q\Rightarrow p))$
		\item $((p\Rightarrow (q\Rightarrow r)) \Rightarrow ((p \Rightarrow q) \Rightarrow (p\Rightarrow r)))$
		\item $(\neg\neg p \Rightarrow p)$\\ (here $p,q,r$ may be any formulae of $\mathcal{L}$)
		\item $((\forall x)p \Rightarrow p[t/x])$\\ (here $p$ is any formula with $x\in \textnormal{FV}(p)$, $t$ any term whose free variables don't occur bound in $p$)
		\item $((\forall x) (p\Rightarrow q) \Rightarrow (p \Rightarrow (\forall x)q))$\\ ($p,q$ formulae, $x\notin \textnormal{FV}(p)$)
		\item $(\forall x)(x = x)$
		\item $(\forall x,y)((x = y) \Rightarrow (p \Rightarrow p[y/x]))$\\ ($p$ any formula with $x\in \textnormal{FV}(p)$, $y$ not bound in $p$ and distinct from $x$)
	\end{enumerate}
	\begin{prop}
		All the axioms above are tautologies.
	\end{prop}
	\begin{proof}
		Let $p,q,r$ be formulae in $\mathcal{L}$ and let $A$ be an $\mathcal{L}$-structure.
		\begin{enumerate}
			\item First suppose that $A=\emptyset$. Then, if there are free variables in $p$ or $q$ then it is clear that $\emptyset\models (p\Rightarrow (q\Rightarrow p))$. Otherwise, $p$ and $q$ are sentences and so they have a truth value. Case-by-case analysis reveals that $\emptyset\models (p\Rightarrow (q\Rightarrow p))$. Now assume that $A$ is nonempty.
			
			Note that for all $a\in A^{\mathbb{N}}$ we have
			\[
			(p\Rightarrow (q\Rightarrow p))_A(a) = (\Rightarrow_2) (p_A(a), (\Rightarrow_2)(p_A(a),q_A(a)))
			\]
			as elements of $2 = \{0,1\}$. Plugging in the possible values for $p_A(a)$ and $q_A(a)$ we conclude that in all cases $(p\Rightarrow (q\Rightarrow p))_A(a) = 1$.
			\item Similar to (a).
			\item Similar to (a).
			\item Suppose $x\in \textnormal{FV}(p)$ and $t$ is any term whose free variables don't occur bound in $p$. It is easy to see that the axiom is never a sentence, so $\emptyset$ models it. Assume now that $A\neq \emptyset$.
			
			Let $a =(a_1,a_2,\ldots)\in A^{\mathbb{N}}$ and consider 
			\[
			(\Rightarrow_2)(((\forall x)p)_A(a), p[t/x]_A(a))
			\]
			If $((\forall x)p)_A(a) = 0$ then the above equals 1, clearly. Now suppose $((\forall x)p)_A(a) = 1$ and let $x = x_n$ for some $n$. This means that, for all $a_n'\in A$ we have
			\[
			(a_1,\ldots, a_{n-1},a_n',a_{n+1},\ldots)\in p_A.
			\]
			Set $a_n' \coloneqq t_A(a)$. By Lemma \ref{lem:subs_op}, the above implies that $p[t/x]_A(a) = 1$, as desired.
			\item Let $x\notin \textnormal{FV}(p)$. If the axiom is not a sentence then it has $\emptyset$ as a model. Suppose now that the axiom is a sentence; this is easily seen to imply that $q$ has $x$ as its only free variable. Clearly $(\forall x)(p\Rightarrow q)$ is satisfied in $\emptyset$. Note that $(\forall x) q$ is also a satisfied sentence in $\emptyset$. Therefore the whole axiom is seen to be satisfied in $\emptyset$. Now assume $A\neq \emptyset$.
			
			Let $a =(a_1,a_2,\ldots)\in A^{\mathbb{N}}$. If $((\forall x) (p \Rightarrow q))_A(a) = 0$ then the formula is true for $a$. So, assume that $a\in ((\forall x) (p \Rightarrow q))_A$. Let $x = x_n$ for some $n$. We have that, for all $a_n'\in A$:
			\[
			(a_1,\ldots, a_{n-1},a_n',a_{n+1},\ldots) \in (p\Rightarrow q)_A.
			\]
			In other words, for all $a_n'\in A$:
			\[
			(\Rightarrow_2)(p_A,q_A)(a_1,\ldots, a_{n-1},a_n',a_{n+1},\ldots) = 1
			\]
			But, as $x_n\notin \textnormal{FV}(p)$, the value $p_A(a_1,\ldots, a_{n-1},a_n',a_{n+1},\ldots)$ does not depend on $a_n'$ by variable redundancy. Thus we conclude that for all $a_n'\in A$.
			\[
			(\Rightarrow_2)(p_A(a),q_A((a_1,\ldots, a_{n-1},a_n',a_{n+1},\ldots))) = 1.
			\]
			From this, it is easy to deduce that $a\in (p \Rightarrow (\forall x)q)_A$, as desired.
			
			\item The empty set is easily seen to model this axiom. Let $a =(a_1,a_2,\ldots)\in A^{\mathbb{N}}$ and let $x = x_n$l Then $a\in ((\forall x)(x=x))_A$ iff for all $a_n'\in A$ we have
			\[
			(a_1, \ldots, a_{n-1}, a_{n}',a_{n+1},\ldots) \in (x=x)_A.
			\]	
			This happens iff for all $a_n'$ we have
			\[
			x_A(a_1, \ldots, a_{n-1}, a_{n}',a_{n+1},\ldots) = x_A(a_1, \ldots, a_{n-1}, a_{n}',a_{n+1},\ldots),
			\]
			which is manifestly true. 
			\item The empty set is easily seen to model this axiom. Let $x = x_n\in \textnormal{FV}(p)$ and $y = x_m$ be not bound in $p$ with $n\neq m$. Have some $a =(a_1,a_2,\ldots)\in A^{\mathbb{N}}$. We need to show that 
			\[
			a\in ((\forall x_n)(\forall x_m) ((x_n= x_m) \Rightarrow (p \Rightarrow p[x_m/x_n])))_A
			\]
			For $c_n,c_m \in A$ define $\alpha(c_n,c_m)$ to be the sequence $a$ but with the $i$-th value replaced by $c_i$ for $i\in \{n,m\}$ (recall that $n\neq m$). Then the above proposition is equivalent to
			\[
			\alpha(c_n,c_m) \in ((x_n= x_m) \Rightarrow (p \Rightarrow p[x_m/x_n]))_A \textnormal{ for all }c_n,c_m\in A
			\]
			We need to prove the above. To that end, let $c_n,c_m\in A$ be arbitrary. If $\alpha(c_n, c_m)\notin (x_n = x_m)_A$ then we do have the inclusion above. So, assume $\alpha(c_n,c_m)\in (x_n = x_m)_A$; this clearly implies that $c \coloneqq c_n = c_m$. Now, we need to show that 
			\[
			\alpha(c,c) \in (p \Rightarrow p[x_m/x_n])_A.
			\]
			If $\alpha(c,c)\notin p_A$ then the above is true. Therefore we can suppose $\alpha(c,c)\in p_A$. We want to prove that $\alpha(c,c)\in (p[x_m/x_n])_A$. As $x_m$ is not bound in $p$ we can apply Lemma \ref{lem:subs_op} which tells us that it suffices to show that $\alpha(c,c)' \in p_A$, where $\alpha(c,c)'$ denotes the sequence $\alpha(c,c)$ but replacing the $n$-th value by $(x_m)_A(\alpha(c,c)) = c$. Clearly $\alpha(c,c)' = \alpha(c,c)$ and we supposed at the start that $\alpha(c,c)\in A$. Thus we are done.
		\end{enumerate}
	\end{proof}
	To our deductive system we add the following rules of inference.
	\begin{enumerate}
		\item[(MP)] From $p$ and $(p\Rightarrow q)$, we may infer $q$, \emph{provided} either $q$ has a free variable or $p$ is a sentence.
		\item [(Gen)] From $p$ we may infer $(\forall x) p$, \emph{provided} $x$ does not occur free in any premiss which has been used in the proof of $p$ (but is a free variable of $p$).
	\end{enumerate}
	Formally, we define our concept of deduction as follows.
	\begin{defn}[Deduction sequence]
		Let $S$ be a set of formulae. A \emph{deduction sequence} on $S$ is a finite sequence on the set of formulae of $\mathcal{L}$, defined inductively below.
		\begin{enumerate}[label=(\roman*)]
			\item The empty sequence is a deduction sequence.
			\item If $(p_1,\ldots,p_n)$ is a deduction sequence and $p$ is an axiom or an element of $S$, then $(p_1,\ldots,p_n,p)$ is a deduction sequence.
			\item Let $(p_1,\ldots,p_n)$ be a deduction sequence. Suppose there are $1\leq i,j\leq n$ so that $p_j$ is the formula $(p_i\Rightarrow p)$ for some $p$, and, in addition either $p_i$ is a sentence or $p$ has a free variable. Then $(p_1,\ldots,p_n,p)$ is a deduction sequence.
			\item Suppose $(p_1,\ldots, p_n)$ is a deduction sequence so that $p_n$ has a free variable $x$ but $x$ is not a free variable of $p_i$ for $i<n$, and $p_n\notin S$. Then, if $p = (\forall x) p_n$, we have that  $(p_1,\ldots, p_n,p)$ is a deduction sequence.
			\item That is all.
		\end{enumerate}
	\end{defn}
	\begin{defn}[Syntactic entailment]
		Let $S$ be a set of formulae and $p$ a formula. We say that $S$ \emph{syntactically entails} $p$, written as $S\vdash p$, if there is a deduction sequence terminating at $p$.
	\end{defn}
	\section{Properties of first-order languages}
	Again, we fix a language $\mathcal{L}$ with a set of variables $X=\{x_1,x_2,\ldots\}$
	\section{Completeness}
	The aim of this section is to prove the Completeness theorem. Before that, we need to prove the Soundness theorem. First, a couple of lemmata.
	\begin{lem}\label{lem:sem_mp}
		Let $S$ be a set of formulae and let $p$ and $q$ be formulae so that either $q$ has a free variable or $p$ is a sentence. If $S\models p$ and $S\models (p\Rightarrow q)$, then $S\models q$.
	\end{lem}
	\begin{proof}
		Let $A$ be an $\mathcal{L}$-structure. We want to show that one of the two following statements holds:
		\begin{enumerate}
			\item[(I)] $A$ is empty and there is a non-sentence in $S\cup \{q\}$ .
			\item[(II)] $\bigcap_{r\in S} r_A \subseteq q_A$.
		\end{enumerate}
		We know that one of these two statements holds:
		\begin{enumerate}
			\item[(a)] $A$ is empty and there is a non-sentence in $S\cup \{p\}$ .
			\item[(b)] $\bigcap_{r\in S} r_A \subseteq p_A$.
		\end{enumerate}
		Suppose (a) holds. If there is a non-sentence in $S$ then there is a non-sentence in $S\cup \{q\}$ and $A$ is empty, i.e. (I) holds. Otherwise, if $p$ is a non-sentence, then, by the premiss of the statement $q$ is a non-sentence and again (I) holds. So, from now on, assume (b) holds	
		
		Similarly, we also know that one of the two statements below holds:
		\begin{enumerate}
			\item[(a')] $A$ is empty and there is a non-sentence in $S\cup \{p,q\}$ .
			\item[(b')] $\bigcap_{r\in S} r_A \subseteq (p\Rightarrow q)_A$.
		\end{enumerate}
		Suppose (a') holds. If $q$ is a non-sentence then (I) holds, and if instead the non-sentence is in $S\cup \{p\}$ we have reduced to case (a). So, we can assume (b') holds. But (b) and (b') are easily seen to imply (II), even when $A$ is empty.	
	\end{proof}
	\begin{lem}\label{lem:sem_gen}
		Let $S$ be a set of formulae, $p$ a formula, and $x$ a variable so that $x$ does not occur free in any formulae of $S$. If $S\models p$ then $S\models (\forall x)p$.
	\end{lem}
	\begin{proof}
		Let $A$ be an $\mathcal{L}$-structure. As before, we want to show that one of the two following statements holds:
		\begin{enumerate}
			\item[(I)] $A$ is empty and there is a non-sentence in $S\cup \{(\forall x) p\}$ .
			\item[(II)] $\bigcap_{r\in S} r_A \subseteq ((\forall x)p)_A$.
		\end{enumerate}
		The hypothesis is that one of the two following statements holds. (We know that $p$ is a non-sentence already)
		\begin{enumerate}
			\item[(a)] $A$ is empty
			\item[(b)] $\bigcap_{r\in S} r_A \subseteq p_A$.
		\end{enumerate}
		Suppose (a) is true. If there is a non-sentence in $S\cup \{(\forall x) p\}$ then (I) holds, so assume that $S$ is a set of sentences, and that $(\forall x) p$ is a sentence. But then, as $A$ is empty, we have that the indicator of $((\forall x) p)_A$ is constant with value 1, implying that (II) holds.
		
		Now suppose (a) is not true. Then $A$ is nonempty and (b) holds. Let $a= (a_1,a_2,\ldots)\in r_A\subseteq A^{\mathbb{N}}$ for all $r\in S$.  We would like to show that $a\in ((\forall x)p)_A$, which, if $x= x_n$, is equivalent to the statement
		\[
		(a_1,\ldots, a_{n-1},a_n',a_{n+1},\ldots) \in p_A \textnormal{ for all }a_n'\in A.
		\]
		So, fix some $a_n'\in A$ and set $a' \coloneqq (a_1,\ldots, a_{n-1},a_n',a_{n+1},\ldots)$. By assumption $x_n$ is not a free variable of $r$ for all $r\in S$. By variable redundancy, we conclude that $a'\in r_A$ for all $r\in S$. Finally, (b) implies that $a'\in p_A$, as desired. 
	\end{proof}
	\begin{prop}[the Soundness Theorem]
		Let $S$ be a set of formulae and $p$ a formula. If $S\vdash p$ then $S\models p$.
	\end{prop}
	\begin{proof}
		It is enough to show that, for all deduction sequences $\sigma$, all formulae of $\sigma$ are semantically entailed by $S$. We use induction on the set of deduction sequences.
		
		The claim is vacuously true when $\sigma$ is the empty sequence. Suppose $\sigma = (p_1,\ldots, p_n,p)$, where $S\models p_i$ for all $i$, and $p$ is an axiom or an element of $S$. It easily follows that $S\models p$ (recall axioms are tautologies).
		
		Now suppose that $\sigma = (p_1,\ldots,p_n,p)$, where $S\models p_i$ for all $i$, and there are $1\leq i,j, \leq n$ so that $p_j$ is the formula $(p_i \Rightarrow p)$, and, in addition, either $p_i$ is a sentence or $p$ has a free variable. Then Lemma \ref{lem:sem_mp} says that $S\models p$.
		
		Finally, suppose that $\sigma = (p_1,\ldots,p_n,p)$, where $S\models p_i$ for all $i$, and that $p_n\notin S$ has a free variable $x$ but $x$ is not a free variable of $p_i$ for $i<n$. In addition, we suppose $p = (\forall x)p_n$. Let $S' = \{p_1,\ldots, p_{n-1}\}$. We claim that $S'\models p_n$	
	\end{proof}
	\chapter{Model Theory}
	\section{Substructures and diagrams}
	\begin{defn}[$\mathcal{L}$-homomorphism]
		Let $M$ and $N$ be $\mathcal{L}$-structures. An $\mathcal{L}$-\emph{homomorphism} is a map $\eta\colon M \to N$ such that given $\bar{a} = (a_1,\ldots,a_n)\in M^n$:
		\begin{itemize}
			\item for all function symbols $f$ of arity $n$ we have that 
			\[
				\eta(f^{M}(\bar{a})) = f^N(\eta^n(\bar{a})),
			\]
			in other words the diagram
			% https://q.uiver.app/#q=WzAsNCxbMCwwLCJNXm4iXSxbMCwxLCJNIl0sWzEsMSwiTiJdLFsxLDAsIk5ebiJdLFsxLDIsIlxcZXRhIiwyXSxbMCwxLCJmXk0iLDJdLFszLDIsImZeTiJdLFswLDMsIlxcZXRhXm4iXV0=
			\[\begin{tikzcd}
				{M^n} & {N^n} \\
				M & N
				\arrow["\eta"', from=2-1, to=2-2]
				\arrow["{f^M}"', from=1-1, to=2-1]
				\arrow["{f^N}", from=1-2, to=2-2]
				\arrow["{\eta^n}", from=1-1, to=1-2]
			\end{tikzcd}\]
			commutes;
			\item for all relation symbols $R$ of arity $n$ we have that
			\[
				\bar{a}\in R^M \text{ iff }\eta^n(\bar{a})\in  R^N.
			\]
		\end{itemize}
		An injective $\mathcal{L}$-homomorphism is an $\mathcal{L}$-\emph{embedding} and an invertible one is an $\mathcal{L}$-\emph{isomorphism}. If $M$ and $N$ are isomorphic we write $M \cong N$. If $M\subseteq N$ and the inclusion map is an $\mathcal{L}$-homomorphism we say that $M$ is a \emph{substructure} of $N$, and $N$ is an \emph{extension} of $M$.
	\end{defn}
	We are going to stop writing $\bar{m}\in M^n$ where $n$ is the length of $\bar{m}$ and just write $\bar{m}\in M$ when $n$ can be inferred or its unimportant.
	\begin{egs}\leavevmode
		\begin{enumerate}
			\item Let $\mathcal{L}$ be the language of groups. Then $(\mathbb{N},+,0)$ is a subset of the the integers $(\mathbb{Z},+,0)$, but it is not a substructure.
			\item If $M$ is an $\mathcal{L}$-structure and $X\subseteq M$ then $X$ is the domain of a substructure of $M$ iff it is closed under the interpretation of all function symbols.
			
			Indeed, the inclusion $\iota\colon X \to M$ clearly preserves relations. But if it is not closed under some function $f^M$ then there is no way to interpret $f^X$. 
			\item It follows from the previous point that the intersection of a family of substructures is a substructure: indeed, applying a function $f^M$ to anything in the intersection will land on all substructures (since these are closed under function symbols) and thus in the intersection.
			
			The substructure generated by $X\subseteq M$ is defined to be the intersection of all substructures of $M$ containing $X$; it is denoted by $\langle X\rangle_{M}$. Again, by the previous point, $\langle X\rangle_M$ is also the intersection of all subsets of $M$ that are closed under function symbols.
			
			Hence
			\[
				\langle X\rangle_{M} = X \cup \{t^M(\bar{m})\mid t\text{ a term and }\bar{m}\in X\}.
			\]
			Indeed, the RHS is obviously closed under function symbols and no strict subset of it could possibly be. Therefore $|\langle X \rangle_{M}| \leq |X| + |\mathcal{L}|$.
			
			We say a structure $M$ is \emph{finitely generated} if $M = \langle X \rangle_M$ for some finite $X\subseteq M$.
		\end{enumerate}
	\end{egs}
	What kind of sentence is preserved under substructures?
	\begin{prop}\label{prop:qf_preser_substr}
		Let $\varphi(\bar{x})$ be a quantifier-free $\mathcal{L}$-formula with $n$ variables, $M$ be an $\mathcal{L}$-structure and $\bar{a}\in M$. For every extension $N$ of $M$ we have $M \models \varphi(\bar{a})$ iff $N\models \varphi(\bar{a})$.
	\end{prop}
	\begin{proof}
		First we show that if $t(\bar{x})$ is a term with $k$ free variables then $t^M(\bar{b}) = t^N(\bar{b})$ for all $\bar{b}\in M$.
		
		This is clearly the case if $t = x_i$ is a variable since then $t^M(\bar{b}) = b_i = t^N(\bar{b})$. Now suppose $t = f(q_1,\ldots, q_l)$ for a function symbol $f$ of arity $l$ and the $q_i$'s are terms. By the inductive hypothesis we can assume $q_i^M(\bar{b}) = q_i^N(\bar{b})$ for all $i$. Then,
		\[
			t^M(\bar{b}) = f^M(q_1^M(\bar{b}),\ldots, q_l^M(\bar{b})) = f^N(q_1^N(\bar{b}), \ldots, q_l^N(\bar{b}))) = t^N(\bar{b})
		\]
		where we have used the fact that $M$ is a substructure of $N$.
		
		Now onto the main result. Let $t_1$ and $t_2$ be terms with at most $n$ free variables. Then
		\[
			M\models (t_1(\bar{a}) = t_2(\bar{a}))
		\] 
		if and only if $t^{M}(\bar{a}) = t^M(\bar{a})$. But this happens iff $t^{N}(\bar{a}) = t^N(\bar{a})$, and this is equivalent to $N \models (t_1(\bar{a}) = t_2(\bar{a}))$.
		
		Next, let $R(t_1,\ldots,t_l)$ be an $l$-ary relation, where all the $t_i$'s have at most $n$ free variables. We have that $M\models R(t_1(\bar{a}),\ldots,t_l(\bar{a}))$ iff $(t_1^M(\bar{a}),\ldots,t_l^M(\bar{a}))\in R^M$. As $N$ is an extension (and by the result for terms) the latter happens iff $(t_1^N(\bar{a}),\ldots, t_l^N(\bar{a}))\in R^N$. And this is of course the same as $N \models R(t_1(\bar{a}),\ldots,t_l(\bar{a}))$.
		
		This finishes the induction for atomic formulae. Now if $\varphi$ is a formula satisfying the claim then clearly 
		\[
			M\models \neg \varphi(\bar{a})\,\,\, \text{ iff }\,\,\, M \nvDash \varphi(\bar{a}) \,\,\,\text{ iff }\,\,\, N \nvDash\varphi(\bar{a})\,\,\, \text{ iff }\,\,\, N \models \varphi(\bar{a}). 
		\]
		If the proposition is true for $\varphi$ and $\psi$ then $M \models \varphi \vee \psi$ iff $M \models \varphi$ or $M\models \psi$, which happens iff $N \models \varphi$ or $N \models \psi$, which is clearly equivalent to $N \models \varphi \vee \psi$.
	\end{proof}
	A \emph{universal formula} is one of the form $\forall \bar{x} \varphi(\bar{x},\bar{y})$ where $\varphi$ is quantifier free. A \emph{universal theory} is one whose axioms are universal sentences.
	\begin{defn}
		Structures $M$ and $N$ are \emph{elementary equivalent} if for every $\mathcal{L}$-sentence $\varphi$ we have $M\models \varphi$ iff $N\models \varphi$. 
		
		A homomorphism $f\colon M \to N$ is an \emph{elementary embedding} if it is injective and for all $\mathcal{L}$-formulae $\varphi(x_1,\ldots,x_n)$ and elements $m_1,\ldots,m_n\in M$ we have 
		\[
			M \models \varphi(m_1,\ldots,m_n) \,\,\,\text{ iff }\,\,\, N \models \varphi(f(m_1),\ldots, f(m_n)).
		\] 
		
		We denote `$M$ and $N$ are elementary equivalent' by $M \equiv N$.
	\end{defn}
	\begin{rem*}
		If $M$ and $N$ are $\mathcal{L}$-structures and we have some tuples of the same size $\bar{m}\in M$ and $\bar{n}\in N$ then by $(M,\bar{m}) \equiv (N,\bar{n})$ we mean that the expanded structures are elementary equivalent as $\mathcal{L}_{\bar{c}}$-structures, where $\bar{c}$ has the same size as $\bar{m}$ and $\bar{n}$.
	\end{rem*}
	\begin{prop}\label{prop:iso_imp_el_equiv}
		If $M\cong N$ then $M \equiv N$.
	\end{prop}
	\begin{proof}
		Let $F\colon M \to N$ be an isomorphism. By symmetry, we only need to show that all formulaes modelled by $M$ are also modelled by $N$.
		
		
		First we would like to show that if $t$ is a term then for all $\bar{m} \in M$ we have $t^M(\bar{m}) = t^N(F(\bar{m}))$.
		
		We use induction over the structure of formulae.
		
		If $t_1(\bar{x})$ and $t_2(\bar{x})$ are terms with the same free variables, then $M \models (t_1(\bar{x}) = t_2(\bar{x}))$ iff for all $\bar{m}\in M$ we have $t_1^M(\bar{m}) = t_2^M(\bar{m})$. But then for $\bar{n}\in N$ we have that
		\begin{align*}
			t_1^N(\bar{n}) &= t_1^N(FF^{-1}\bar{n})\\
			&= F(t_1^M(F^{-1}(\bar{n})))\\
			&= F(t_2^M(F^{-1}(\bar{n})))\\
			&= t_2^N(FF^{-1}\bar{n})\\
			&= t_2^N(\bar{n}).
		\end{align*}
		Now suppose that $R(x_1,\ldots,x_l)$ is a relation so that $M \models R(x_1,\ldots,x_l)$. Then $\bar{m}\in R^M$ for all $\bar{m}\in M$. But then if $\bar{n}\in N$ we have that $\bar{n}\in R^N$ iff $F^{-1}(n)\in R^M$ which is clearly true.
		
		Let $\varphi$ be a formulae for which this holds. Then $M\models \neg\varphi$ iff $M \nvDash \varphi$ iff $N \nvDash \varphi$ iff $N \models \neg\varphi$. Similarly for disjunction.
		
		Now suppose $M \models (\forall x) \varphi(x,\bar{y})$. That means that for all $\bar{m},m'\in M$ we have that  $\varphi^M(m',\bar{m})$ is true which means that $M\models \varphi(x,\bar{y})$. Apply inductive hypothesis and we are done.
		\end{proof}
		Recall that a theory $\mathcal{T}$ is \emph{complete} if $T\models \varphi$ or $T \models \neg \varphi$ for every sentence $\varphi$. Any two models of the same complete theory are elementary equivalent: indeed the formulae satisfied by them are completely determined by the theory. However, the models can have different cardinalities (and hence be non-isomorphic), see Examples \ref{egs:vaught_test}.
		
		\begin{defn}
			A substructure $M\subseteq N$ is an \emph{elementary substructure} if the inclusion map $M \hookrightarrow N$ is an elementary embedding.
		\end{defn}
		\begin{defn}
			A theory $\mathcal{T}$ is \emph{model-complete} if every embedding between models of $\mathcal{T}$ is elementary.
		\end{defn}
		\begin{defn}
			Let $\kappa$ be an infinite cardinal. We say that a theory $\mathcal{T}$ is $\kappa$-categorical if all models of $\mathcal{T}$ of cardinality $\kappa$ are isomorphic.
		\end{defn}
		\begin{prop}[Vaught's Test]
			Let $\mathcal{T}$ be a consistent $\mathcal{L}$-theory with no finite models. If $\mathcal{T}$ is $\kappa$-categorical for some infinite $\kappa\geq |\mathcal{L}|$ then $\mathcal{T}$ is a complete theory. 
		\end{prop}
		\begin{proof}
			For the sake of contradiction, suppose $\mathcal{T}$ is not complete, so that there is a sentence $\varphi$ with $T \nvDash \varphi$ and $T\nvDash \neg\varphi$. It follows (by the Deduction Theorem) that $T\cup\{\varphi\}$ and $T\cup\{\neg\varphi\}$ are consistent, i.e. they have a model.
			These two models cannot be finite since then they would be finite models of $\mathcal{T}$. Thus they are infinite.
			
			By the Upwards L{\"o}wenheim-Skolem Theorem (together with the Downwards version if necessary) we get models of $T\cup\{\varphi\}$ and $T\cup\{\neg\varphi\}$ of cardinality $\kappa$. In particular, these are models of $\mathcal{T}$ so, as $\mathcal{T}$ is $\kappa$-categorical, they must be isomorphic. But they disagree on the valuation of $\varphi$, contradicting Proposition \ref{prop:iso_imp_el_equiv}.
 		\end{proof}
 		\begin{egs}\label{egs:vaught_test}\leavevmode
 			\begin{enumerate}
 				\item Any two countable dense linear orders without endpoints are isomorphic to $\mathbb{Q}$. So the theory is $\aleph_0$-categorical and hence complete.
 				\item For every field $F$ the theory of infinite $F$-vector spaces is $\kappa$-categorical for some $\kappa > |F|$ ({\color{red} Exercise}), hence complete.
 			\end{enumerate}
 		\end{egs}
 		\begin{prop}[Tarski-Vaught test]
 			Let $N$ be an $\mathcal{L}$-structure and $M\subseteq N$. Then $M$ is the domain of an elementary substructure of $N$ if and only if the following condition is satisfied.
 			\begin{itemize}
 				\item Let $\varphi(x,\bar{t})$ be a formula and $\bar{m}\in M$. If there is some $n\in N$ such that $N\models \varphi(n,\bar{m})$, then there is $\hat{m}\in M$ such that $N \models \varphi(\hat{m},\bar{m})$.
 			\end{itemize}
 		\end{prop}
 		\begin{proof}
 			Suppose $M$ is an elementary substructure of $N$. Let $\varphi(x,\bar{t})$ be a formula and $\bar{m}\in M$. Furthermore, assume that there is some $n\in N$ so that $N\models \varphi(n,\bar{m})$. It follows that $N\models (\exists x)\varphi(x,\bar{m})$. Thus, as $M$ is an elementary substructure, we have $M \models (\exists x)\varphi(x,\bar{m})$. Hence, there is some $\hat{m}\in M$ such that $M \models \varphi(\hat{m},\bar{m})$. But then clearly $N \models \varphi(\hat{m},\bar{m})$.
 			
 			Conversely, assume that the property holds. Consider the formulae $\varphi_f(x,\bar{t}) \coloneqq (x = f(\bar{t}))$ for each function symbol $f$ in $\mathcal{L}$.
 			
 			For any $\bar{m}\in M$ there is $n\in N$ such that $N\models (n = f(\bar{m}))$ (say $n\coloneqq f^N(\bar{m})$). By hypothesis there is $\hat{m}$ such that $N \models (\hat{m} = f(\bar{m}))$. It follows that $M$ is closed under function symbols. Interpreting relation symbols as $R^M \coloneqq R^N \cap M$ we turn $M$ into an $\mathcal{L}$-structure that is clearly a substructure of $N$. 			
 			
 			We need to show that $N$ and $M$ satisfy the same formulae when the parameters are in $M$. Let $\varphi(\bar{x})$ be a formula and $\bar{m}\in M$. Note that if $N \models \varphi(\bar{m})$ iff $M \models \varphi(\bar{m})$ whenever $\varphi$ is quantifier free by Proposition \ref{prop:qf_preser_substr}. So, by induction, we only need to check that the case when $\varphi$ is an existential formula, say $\varphi(\bar{x}) = (\exists t)\psi(t, \bar{x})$.
 			
 			If $M \models (\exists t)\psi(t, \bar{m})$ then clearly $N \models (\exists t)\psi(t, \bar{x})$ since a witness in $M$ is also a witness in $M$. Conversely, if $N \models (\exists t)\psi(t, \bar{m})$ then there is some $n\in N$ so that $N\models \psi(n,\bar{m})$. By the condition, there is some $\hat{m}$ such that $N \models \psi(\hat{m},\bar{m})$. Using the inductive hypothesis, we get that $M \models \psi(\hat{m},\bar{m})$ and thus $M \models \varphi(\bar{m})$ as desired. Hence $M$ is an elementary substructure of $N$.
 		\end{proof}
 		\begin{defn}
 			Let $N$ be an $\mathcal{L}$-structure. We define the \emph{diagram} of $N$ as 
 			\[
 				\Diag(N) \coloneqq \{\varphi(n_1,\ldots,n_k) \mid \varphi \text{ is a q.f. $\mathcal{L}_{N}$-formula and }N \models \varphi(n_1,\ldots,n_k)\}.
 			\]
 			The \emph{elementary diagram} of $N$ is defined as 
 			\[
 				\Diag_{\text{el}}(N) \coloneqq \{\varphi(n_1,\ldots,n_k) \mid \varphi \text{ is an $\mathcal{L}_{N}$-formula and }N \models \varphi(n_1,\ldots,n_k)\}
 			\]
 		\end{defn}
 		Basically, diagrams are all (q.f.) formulas the structure believes in. For the difference between diagrams of structures and the theory of a structure (the latter we'll discuss later) see \href{https://math.stackexchange.com/a/3644376}{this link}.
 		
 		The most important thing about diagrams is their models. Indeed, let $M$ be a model of $\Diag(N)$. In particular, $M$ is an $\mathcal{L}$-structure with a constant $m_n\in M$ for each element $n\in N$. Hence we have a function $\eta \colon N \to M$ given by $\eta(n) \coloneqq m_n$. But $N\models \neg(n = n')$ for any distinct $n,n'\in N$. Thus, $M\models \neg(m_n = m_{n'})$ for distinct $n,n'\in N$, i.e. $\eta$ is injective. 
 		
 		Let $f$ be a function symbol. Then (bare with me on this) for all $\bar{n}\in N$ we have $N \models (f(\bar{n}) = f^{N}(\bar{n}))$ where the LHS is the function symbol $f$ applied to the constants $\bar{n}$ and the RHS is just the constant $f^{N}(\bar{n})$. Hence, $M\models (f(\eta(\bar{n})) = \eta(f^{N}(\bar{n})))$, i.e. we have
 		\[
 			f^M(\eta(\bar{n})) = \eta(f^N(\bar{n})).
 		\]
 		Also, if $R$ is a relation symbol, then for all $\bar{n}\in N$, if we have  $N \models R(\bar{n})$ then $M \models R(\eta(\bar{n}))$ then we have $N \models R(\bar{n})$. Conversely, if $N \nvDash R(\bar{n})$ then $N\models \neg R(\bar{n})$ and we apply the same argument. This all shows that $f$ is an injective homomorphism, so we can see $M$ as an extension of $N$. In summary, models of $\Diag(N)$ are just extensions of $N$.
 		
 		Similarly, models $\Diag_{\text{el}}(N)$ are elementary extensions of $N$. Indeed, if $M$ is such a model then $M$ is an extension of $N$ by the previous argument. Let $\varphi(\bar{x})$ be a formula and  let $\bar{m}\in M$. If $N \models \varphi(\bar{m})$ then $M \models \varphi(\bar{m})$ by definition of $\Diag_{\text{el}}(N)$. Conversely, if $M \models \varphi(\bar{m})$ we have that $N \models \varphi(\bar{m})$ by Proposition \ref{prop:qf_preser_substr} if $\varphi$ is quantifier free, and otherwise it is obvious for universal formulae.
 		
 		\begin{lem}\label{lem:uni_consis_diag}
 			Let $\mathcal{T}$ be a consistent theory and let $\mathcal{T}_{\forall}$ be the theory of all universal sentences that follow from $\mathcal{T}$. If $N$ is a model of $\mathcal{T}_{\forall}$, then $\mathcal{T}\cup \Diag(N)$ is consistent.
 		\end{lem}
 		\begin{proof}
 			For the sake of contradiction, suppose it is inconsistent. By Compactness, there is some finite subset of $\mathcal{T}\cup \Diag(N)$ that is inconsistent. As $\mathcal{T}$ is consistent, there must be a finite subset of $\Diag(N)$ that is inconsistent with $\mathcal{T}$.
 			
 			Take the conjunction of all of these sentences and call it $\varphi(\bar{n})$. Then $\mathcal{T}\cup\{\varphi(\bar{n})\}$ is inconsistent, i.e. $\mathcal{T}\vdash \neg \varphi(\bar{n})$. But $\mathcal{T}$ is an $\mathcal{L}$-theory, so it contains none of the constants $\bar{n}$. Hence, by generalization, $\mathcal{T} \vdash (\forall x)\neg \varphi(\bar{x})$. However $N$ is a model of $\mathcal{T}_{\forall}$ so $N \models (\forall x)\neg \varphi(\bar{x})$, and thus $N \models \neg \varphi(\bar{n})$, a contradiction.
 		\end{proof}
 		We say a theory $\mathcal{T}$ has an \emph{universal axiomatization} if there is an universal theory that has exactly the same models as $\mathcal{T}$
 		\begin{thm}[Tarski, \L o\'s]
 			An $\mathcal{L}$-theory $\mathcal{T}$ has a universal axiomatization iff whenever $N$ is a substructure of $M$ and $M \models T$, then $N \models T$.
 		\end{thm}
 		\begin{proof}
 			One direction is obvious (see Proposition \ref{prop:qf_preser_substr}). For the converse, suppose $\mathcal{T}$ is preserved under taking substructures. We would like to say that $\mathcal{T}_{\forall}$ is an universal axiomatization of $\mathcal{T}$. To prove this we need to show that if $N \models T_{\forall}$ then $N \models\mathcal{T}$ (since the converse is obvious).
 			
 			By Lemma \ref{lem:uni_consis_diag}, we have that $\mathcal{T} \cup \Diag(N)$ is consistent, so let $M$ be a model for it. Then $M$ is a model of $\Diag(N)$ and hence an extension of $N$. But $M$ is a model of $\mathcal{T}$ so taking substructures we conclude that $N \models \mathcal{T}$.
 		\end{proof}
 		The method of diagrams is powerful. We can show much more with the same method.
 		\subsubsection*{1. Finding a common elementary extension to given structures}
 		\begin{thm}[Elementary amalgamation]
 			Let $M$ and $N$ be $\mathcal{L}$-structures, $\bar{m}\in M$ and $\bar{n}\in N$ be of the same length such that $(M, \bar{m}) \equiv (N, \bar{n})$.  Then there is an $\mathcal{L}$-structure $K$ and elementary embeddings $g\colon N \hookrightarrow K$ and $h\colon M \hookrightarrow K$ such that $g(\bar{n}) = h(\bar{m})$.
 		\end{thm}
 		\begin{proof}
 			Form the disjoint union of $M$ and $N$ and quotient it out by the smallest equivalence relation containing $(m_i,n_i)$ for all $i$. Then the resulting set is basically copies of the two sets $M$ and $N$ that only intersect at $\bar{m} = \bar{n}$. Hence we may assume, without loss of generality that $\bar{m} = \bar{n}$ and otherwise $M$ and $N$ are disjoint.
 			
 			We would like to show that $\mathcal{T}\coloneqq \Diag(N)_{\text{el}} \cup \Diag_{\text{el}}(M)$ is consistent; of course, we do so by Compactness. Let $\Phi$ be a finite subset of $\mathcal{T}$, which of course contains only finitely many sentences of $\Diag_{\text{el}}(N)$.
 			
 			Let $\varphi(\bar{n},\bar{k})$ be the conjunction of all these sentences, where $\bar{k}$ does not contain any elements of $\bar{n}$ and its elements are pairwise disjoint. Define $\varphi(\bar{x},\bar{y})$ to be the corresponding $\mathcal{L}_N$ formula. If $\Phi$ is inconsistent then $\Diag_{\text{el}} \vdash \neg \varphi(\bar{m},\bar{k})$. Since the elements in $\bar{k}$  are distinct and not in $M$ we in fact have $\Diag_{\text{el}}(M)\vdash (\forall \bar{y}) \neg \varphi(\bar{m},\bar{y})$ by generalization.
 			
 			In particular, $(M,\bar{m})\models (\forall \bar{y}) \neg \varphi(\bar{m},\bar{y})$ and so by hypothesis  $(N,\bar{n})\models (\forall \bar{y}) \neg \varphi(\bar{m},\bar{y})$, contradicting the fact that $\varphi(\bar{m},\bar{k})\in \Diag_{\text{el}}(N)$.
 			
 			By compactness $T$ must be consistent and a model for it would be an elementary extension of both $N$ and $M$.
 		\end{proof}
 		\subsubsection*{2. Controlling the size of a model}
 		\begin{thm}[Löwenheim–Skolem]
 			Let $M$ be an infinite $\mathcal{L}$-structure and $\kappa \geq |\mathcal{L}|$ be an infinite cardinal.
 			\begin{enumerate}
 				\item[(\textdownarrow)] If $\kappa < |M|$ then $M$ admits an elementary substructure of size $\kappa$.
 				\item[(\textuparrow)] If $\kappa > |M|$ then $M$ admits an elementary extension of size $\kappa$.
 			\end{enumerate}
 		\end{thm}
 		\begin{proof}[Proof of (\textuparrow)]
 			Expand the language $\mathcal{L}$ by adding one constant symbol for each $m\in M$ and $c\in \kappa$. Let $\mathcal{T} \coloneqq \Diag_{\text{el}}(M)\cup \{\neg(c = c')\}_{c,c'\in \kappa, c\neq c'}$ be a theory in such a language. By compactness $\mathcal{T}$ has a model, i.e. an elementary extension of size at least $\kappa$. By (\textdownarrow), there must be one of size exactly $\kappa$.
 		\end{proof}
 		\section{Existentially closed structures and quantifier-elimination}
 		\begin{defn}
 			Let $\mathcal{T}$ be an $\mathcal{L}$-theory and $\varphi(\bar{x},y)$ be an $\mathcal{L}$-formula with $\bar{x}$ nonempty. A \emph{Skolem function} for $\varphi$ is an $\mathcal{L}$-term $t(\bar{x})$ such that
 			\[
 				\mathcal{T} \models \forall \bar{x}:((\exists y: \varphi(\bar{x}, y)) \Rightarrow \varphi(\bar{x},t(\bar{x}))).
 			\]
 			A \emph{skolemization} of an $\mathcal{L}$-theory $\mathcal{T}$ is a language $\mathcal{L}^+\supseteq \mathcal{L}$ together with an $\mathcal{L}^+$-theory $\mathcal{T}^+\supseteq \mathcal{T}$ such that :
 			\begin{enumerate}[label=(\arabic*)]
 				\item Every $\mathcal{L}$-structure that models $\mathcal{T}$ can be expanded to a model of $\mathcal{T}^+$.
 				\item The $\mathcal{L}^+$-theory $\mathcal{T}^+$ admits Skolem functions for every $\mathcal{L}^+$-formula $\varphi(\bar{x},y)$ with $\bar{x}\neq \emptyset$.
 			\end{enumerate}
 			Finally, a theory $\mathcal{T}$ is a \emph{Skolem theory} if it is a skolemization of itself
 		\end{defn}
 		We say two $\mathcal{L}$-formulae $\varphi$ and $\psi$ are \emph{equivalent modulo} $\mathcal{T}$, where $\mathcal{T}$ is an $\mathcal{L}$-theory, iff $T \vdash (\varphi \iff \psi)$.
 		\begin{prop}\label{prop:fam_eq_mod}
 			Let $\mathcal{T}$ be an $\mathcal{L}$-theory and $F$ be a collection of $\mathcal{L}$-formulae that includes all atomic formulae and is closed under Boolean combinations. If for every formula $\psi(\bar{x},y)$ in $F$ we have $\varphi(\bar{x})$ in $F$ such that 
 			\[
 				\mathcal{T} \vdash \forall \bar{x}: ((\exists y: \psi(\bar{x}, y))\iff \varphi(\bar{x})),
 			\]
 			then every $\mathcal{L}$-formula is equivalent modulo $\mathcal{T}$ to one in $F$ with the same free variables.
 		\end{prop}
 		\begin{proof}
 			By induction on the structure of formulae. Atomic formulae are in $F$ by hypothesis and $F$ is closed under Boolean combinations, so we only need to check the case for existential statements. But that's exactly the hypothesis.
 		\end{proof}
 		If $M$ is an elementary substructure of $N$ we denote this by $M \preccurlyeq N$.
 		\begin{prop}
 			Let $\mathcal{T}$ be an Skolem $\mathcal{L}$-theory. Then
 			\begin{enumerate}[label=(\arabic*)]
 				\item Every $\mathcal{L}$-formula $\varphi(\bar{x})$ with $\bar{x}\neq \emptyset$ is equivalent to some quantifier-free $\psi(\bar{x})$ modulo $\mathcal{T}$.
 				\item If $N \models \mathcal{T}$ and $X$ is a subset of $N$, then either $\langle X \rangle_N = \emptyset$ or $\langle X \rangle_N\preccurlyeq N$.
 			\end{enumerate}
 		\end{prop}
 		\begin{proof}\leavevmode
 			\begin{enumerate}[label=(\arabic*)]
 				\item This just follows from Proposition \ref{prop:fam_eq_mod} by taking $F$ to be the set of all quantifier-free formulae, and by the definition of Skolem theory.
 				\item Assume $X$ is nonempty (otherwise this is trivial). Let $M \coloneqq \langle X \rangle_N$; we will use the Tarski-Vaught test. Let $\varphi(x,\bar{y})$ be an $\mathcal{L}$-formula and $\bar{m}\in M$. Suppose that there is some $n\in N$ with $N \models \varphi(n,\bar{m})$. Then $N \models (\exists y)(\varphi(y,\bar{m}))$. Hence there is a Skolem function $t(\bar{x})$ such that $N \models \varphi(t(\bar{m}), \bar{m})$. As $M$ is a substructure of $N$ it is closed under interpretations  of terms, hence $t^N(\bar{m}) \in M$. We are done.\qedhere
 			\end{enumerate}
 		\end{proof}
 		\begin{thm}[Skolemization Theorem]
 			Every (first-order) language $\mathcal{L}$ can be expanded to some $\mathcal{L}^+ \supseteq \mathcal{L}$ that includes an $\mathcal{L}^+$-theory $\Sigma$ such that
 			\begin{enumerate}[label=(\arabic*)]
 				\item $\Sigma$ is a Skolem $\mathcal{L}^+$-theory.
 				\item Every $\mathcal{L}$-structure can be expanded to an $\mathcal{L}^+$-structure that models $\Sigma$.
 				\item $|\mathcal{L}| = |\mathcal{L}^+|$.
 			\end{enumerate}
 		\end{thm}
 		\begin{proof}
 			For an arbitrary language $\mathcal{L}$ and for every $\mathcal{L}$-formula of the form $\chi(\bar{x},y)$ with $\bar{x}\neq \emptyset$ we create an function symbol $F_{\chi}$ of arity $|\bar{x}|$. By adding all such function symbols to $\mathcal{L}$ we get a new language $\mathcal{L}^*$. We define $\Sigma(\mathcal{L})$ to be the $\mathcal{L}^*$-theory:
 			\[
 				\Sigma(\mathcal{L}) \coloneqq \{\forall \bar{x} ((\exists y: \chi(\bar{x}, y))\Rightarrow \chi(\bar{x}, F_{\chi}(\bar{x}))) \mid \chi(\bar{x},y) \text{ is an $\mathcal{L}$-formula and $\bar{x}\neq\emptyset$}\}.
 			\]
 			Intuitively, $\Sigma$ says that $\mathcal{L}^*$ has Skolem functions \emph{for all $\mathcal{L}$-formulae}. We would be basically done if this included $\mathcal{L}^*$ formulae too. The way to fix this is to iterate the construction.
 			
 			Let $\mathcal{L}$ be a language. Define a sequence of languages and theories as follows. Start with $\mathcal{L}_0 \coloneqq \mathcal{L}$ and $\Sigma_0 \coloneqq \emptyset$. For $n\geq 1$ define
 			\[
 				\mathcal{L}_{n} \coloneqq \mathcal{L}_{n-1}^* \,\,\,\text{ and }\,\,\,\Sigma_n \coloneqq \Sigma(\mathcal{L}_{n-1}) \cup \Sigma_{n-1}.
 			\]
 			Set $\mathcal{L}^+ \coloneqq \bigcup_{n \in \mathbb{N}} \mathcal{L}_n$ and $\Sigma \coloneqq \bigcup_{n\in \mathbb{N}} \Sigma_n$.
 			
 			It is easy to see that $\Sigma$ is an $\mathcal{L}^+$-theory. Furthermore, it is Skolem since every $\mathcal{L}^+$-formula of the required form is in some $\mathcal{L}_n$ and thus there is an Skolem function for it by $\Sigma_{n+1}$. Also $|\mathcal{L}| = |\mathcal{L}^+|$ by cardinal multiplication (countable cardinalities are absorbed).
 					
 			We now check the structure expansion property. We first check it step-by-step down the chain. Let $M$ be a nonempty $\mathcal{L}$-structure. Say we have $\chi(\bar{x},y)$ with $\bar{x}\neq\emptyset$ and a tuple $\bar{m}\in M$. If there is a $b$ such that $M \models \chi(\bar{m}, b)$, choose one and define $F_{\chi}^M(\bar{m}) \coloneqq b$. If there is no such $b$ interpret it as $m_0$ (or whatever).
 		\end{proof}
\end{document}